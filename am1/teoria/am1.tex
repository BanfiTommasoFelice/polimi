\documentclass[a4paper,12pt,twoside]{article}

\usepackage[italian]{babel}
\usepackage[a4paper, left=18mm, right=18mm, top=25mm, bottom=20mm]{geometry}
\usepackage{amssymb}
\usepackage{mathtools}
\usepackage{interval}
\usepackage{amsthm}
\usepackage{thmtools}
\usepackage{cancel}
\usepackage{hyperref}
\usepackage{tikz}
\usepackage{pgfplots}
\usepackage{nicefrac}
\usepackage{enumitem}
\usepackage{verbatim}
\usepackage{tabularray}
\usepackage{bold-extra}
\usepackage{colonequals}
\usepackage{titling}
\usepackage{fancyhdr}
\usepackage{ifthen}
\usepackage{accents}
\usepackage{xcolor}
\usepackage[many]{tcolorbox}

\hypersetup{
  colorlinks=true,
  linkcolor=black,
    filecolor=magenta,      
    urlcolor=cyan,
    pdftitle={Analisi matematica 1},
    % bookmarks=true,
    bookmarksopen=true,
    pdfpagemode=UseOutlines,
    pdfauthor={Amato Michele Pasquale},
}

\title{\huge Analisi matematica 1}
\author{Amato Michele Pasquale}
\date{\today}

\pgfplotsset{compat = newest}
\makeatletter
\renewcommand\l@subsection{\@dottedtocline{2}{1.5em}{3em}}
\makeatother
\setitemize{noitemsep,topsep=3pt,parsep=0pt,partopsep=0pt}
\setenumerate{noitemsep,topsep=3pt,parsep=0pt,partopsep=0pt}

\renewcommand{\sectionmark}[1]{\markboth{\textbf{#1}}{}}
\renewcommand{\subsectionmark}[1]{\markright{\thesubsection\ \ \emph{#1}}}
\setlength{\headheight}{15pt}

\tcbuselibrary{breakable}
\tcbuselibrary{external}
\tcbEXTERNALIZE
\tcbset{
    arc=0pt,
    outer arc=0pt,
    breakable=true,
}
\newtcolorbox{obsbox}{
    colback=white,
    colframe=black,
    boxrule=1pt,
    left=5pt,
    right=5pt,
    top=5pt,
    bottom=5pt,
}
\definecolor{98fb98}{HTML}{98fb98}
\newtcolorbox{exbox}{
    colback=98fb98!40,
    colframe=white,
    boxrule=1pt,
    left=5pt,
    right=5pt,
    top=5pt,
    bottom=5pt,
    arc=5pt,
}

\newcommand{\abs}[1]{\left\lvert #1 \right\rvert}
\newcommand{\ceil}[1]{\left\lceil #1 \right\rceil}
\newtheorem{theorem}{Teorema}
\newtheorem{definition}{Definizione}
\newtheorem{lemma}{Lemma}
\newtheorem{axiom}{Assioma}
\newtheorem{corollary}{Corollario}
\renewcommand\qedsymbol{$\blacksquare$}
\newcommand{\asin}{\arcsin}
\newcommand{\acos}{\arccos}
\newcommand{\atan}{\arctan}
\newcommand{\impl}{\Rightarrow}
\newcommand{\rimpl}{\Leftarrow}
\setitemize{noitemsep,topsep=3pt,parsep=0pt,partopsep=0pt}
\setenumerate{noitemsep,topsep=3pt,parsep=0pt,partopsep=0pt}
\newcommand{\triang}[1]{\overset{\triangle}{#1}}
\renewcommand{\a}{\alpha}
\renewcommand{\b}{\beta}
\renewcommand{\c}{\gamma}
\renewcommand{\(}{\left(}
\renewcommand{\)}{\right)}
\renewcommand{\mod}[1]{\ \( \mathrm{mod}\ #1 \)}
\renewcommand{\emptyset}{\varnothing}
\newcommand{\walrus}{\coloneqq}
\renewcommand{\epsilon}{\varepsilon}
\renewcommand{\l}{\ell}
\newcommand{\reals}{\mathbb{R}}
\newcommand{\complex}{\mathbb{C}}
\DeclareMathOperator{\re}{Re}
\DeclareMathOperator{\im}{Im}
\renewcommand{\r}{\rho}
\renewcommand{\t}{\theta}
\DeclareMathOperator{\sgn}{sgn}
\renewcommand{\phi}{\varphi}
\renewcommand{\underbar}[1]{\underaccent{\bar}{#1}}
\makeatletter
\let\oldint\int
\renewcommand{\int}{
  \if@display \oldint\limits
  \else \oldint 
  \fi
}
\makeatother

\intervalconfig{soft open fences}
\newcommand{\intv}[2]{\interval[scaled]{#1}{#2}}
\newcommand{\ointv}[2]{\ointerval[scaled]{#1}{#2}}
\newcommand{\lintv}[2]{\linterval[scaled]{#1}{#2}}
\newcommand{\rintv}[2]{\rinterval[scaled]{#1}{#2}}

\newenvironment{example}
{\begin{exbox}\par\noindent\textbf{Esempio}\hspace{0.5em}}
{\hfill\end{exbox}}
\newenvironment{observation}
{\begin{obsbox}\par\noindent\textbf{Osservazione}\hspace{0.5em}}
{\hfill\end{obsbox}}

\usetikzlibrary{external}

\allowdisplaybreaks

\begin{document}
\pagenumbering{gobble}
\maketitle
\vfill
\begin{center}
  \emph{}
\end{center}
\newpage
\newpage
\
\newpage
\pagenumbering{roman}
\tableofcontents
\newpage
\pagenumbering{gobble}
\newpage
\
\newpage
\pagenumbering{arabic}

\pagestyle{fancy}
\fancyhead{}
\fancyfoot{}
\fancyhead[RO,LE]{\thepage}
\fancyhead[LO,RE]{\leftmark}
\fancyfoot[LO,RE]{\footnotesize\theauthor\ \textsuperscript{\copyright}}

% \tikzexternalize[prefix=figs/]

% lezione del 12/09/2022

Il problema che ci si pone è trovare un modo opportuno per rappresentare all'interno di un sistema di calcolo le informazioni in modo efficiente, rispetto alla realtà fisica del sistema e alla loro manipolazione.

\begin{definition}[Alfabeto]
  Si definisce \textbf{alfabeto} un insieme di simboli utilizzabili e, pertanto, distinguibili tra loro.
\end{definition}

\begin{definition}[Codice]
  Si definisce \textbf{codice} l'insieme delle sequenze di simboli o delle regole per definire le combinazioni ammissibili.
\end{definition}

Dati l'insieme degli elementi da rappresentare e l'insieme delle configurazioni ammissibili, il codice ne definisce la relazione biunivoca.
Le configurazioni ammissibili hanno tutte egual dimensione. Tale dimensione dipende sia dall'alfabeto, sia dalla quantità di elementi da rappresentare: siano $S$ l'alfabeto di riferimento e $\abs{S}$ la sua cardinalità (ossia il numero di simboli che lo compone), se si vogliono rappresentare $n$ elementi, ogni elemento avrà dimensione:
$$k=\ceil{\log_{\abs{S}} n}$$
Al contrario, se gli elementi di un codice hanno lunghezza $k$, le combinazioni ammissibili sono:
$$n=\abs{S}^k$$

\section{Rappresentazione binaria}

I componenti elettronici che costituiscono il sistema di calcolo sono caratterizzati da una realtà costituita da due stati (condensatore carico/scarico, tensione alta/bassa, etc...). Si effettua, quindi, una mappatura diretta con un sistema costituito da \textbf{due simboli} che, pertanto, si chiama \textbf{binario}. L'alfabeto di riferimento diventa $\left\{ 0,1 \right\}$.

La cifra della codifica (0 o 1) prende il nome di \textbf{bit}, dall'inglese \emph{\textbf{bi}nary digi\textbf{t}}. L'insieme ordinato di 8 bit prende il nome di \textbf{byte}. Come per le cifre decimali, si ha una nomenclatura per le potenze della base:
\begin{center}
  \begin{tblr}{colspec={c|c}, cells={c,m}, columns={20mm}}
    \textbf{Nome} & \textbf{Quantità} \\
    \hline
    KB            & $2^{10}$          \\
    MB            & $2^{20}$          \\
    GB            & $2^{30}$          \\
    TB            & $2^{40}$          \\
  \end{tblr}
\end{center}

\begin{example}
  Se si vogliono rappresentare i giorni della settimana usando l'alfabeto binario, si calcola la dimensione del singolo elemento:
  $$k=\ceil{\log_27}=3$$
  e si assegna ad ogni combinazione di 3 bit un giorno della settimana distinto:
  \begin{center}
    \begin{tblr}{colspec={c|c|c|c|c|c|c}, cells={c,m}, columns={18mm}}
      Lunedì & Martedì & Mercoledì & Giovedì & Venerdì & Sabato & Domenica \\
      \hline
      000    & 001     & 010       & 011     & 100     & 101    & 110      
    \end{tblr}
  \end{center}
  Da notare che non viene utilizzata la combinazione 111, in quanto le combinazioni ammissibili sono $2^3=8$ ma per le necessità del caso ne servono solo 7.
\end{example}

Nella scelta della codifica da adottare, bisogna tenere a mente alcuni aspetti:
\begin{itemize}
  \item l'insieme degli elementi da rappresentare;
  \item il grado di semplificazione delle operazioni più eseguite;
  \item il grado di conservazione delle proprietà dell'insieme originale.
\end{itemize}

L'informazione può essere, per comodità, suddivisa in aree:

\begin{center}
  \begin{tblr}{colspec={c|c|c|c|c|c}, cells={c,m}, columns={20mm}}
    \SetCell[c=6]{c} Informazione                                                                       \\ \hline
    \SetCell[c=3]{c} Numerica &          &           & \SetCell[c=3]{c} Non numerica                    \\ \hline
    Naturali                  & Relativi & Razionali & Testi                         & Audio & Immagini 
  \end{tblr}
\end{center}

% lezione del 13/09/2022

La notazione che si usa in base 10 è una \textbf{notazione posizionale pesata}, vale a dire che ogni cifra vale in base alla posizione che essa occupa all'interno del numero. Il valore del numero, infatti, è dato da
$$\text{valore}=\sum_{0}^nc_ib^i$$
dove $c_i$ è la cifra in posizione $i$ e $b$ è la base di riferimento.

\begin{example}
  $$315_{10}=3\cdot10^2+1\cdot10^1+5\cdot10^0$$
\end{example}

Nella notazione in base 2, si definisce un codice che associa al valore numerico una configurazione, in cui la cifra più a destra è la cifra meno significativa (\textbf{LSB}, Least Significant Bit), mentre quella più a sinistra è quella più significativa (\textbf{MSB}, Most Significant Bit).

\section{Conversione di base}
Idealmente, ogni qualvolta si vuole rappresentare una quantità in una certa base, si vogliono implicitamente rappresentare in un'altra base tutte le quantità minori uguali a quella di partenza, non avrebbe senso altrimenti\footnote{se me ne serve solo uno, uso il primo valore disponibile}.

Se $n$ è il valore da rappresentare, significa che in base 2 si avrà bisogno di $k$ bits:
$$k=\ceil{\log_2\(n+1\)}$$
\paragraph*{Osservazione} L'argomento del logaritmo è $n+1$ in quanto bisogna anche considerare lo 0.

Per passare da una base $a$ ad una base $b$, con $a<b$, si procede nel seguente modo: si moltiplica ogni coefficiente per la base elevata alla sua posizione e poi si sommano tutti i prodotti ottenuti.

\begin{example}
  $$1010_2=1\cdot2^3+0\cdot2^2+1\cdot2^1+0\cdot2^0=8+0+2+0=10_{10}$$
\end{example}

Per passare da una base $a$ ad una base $b$, con $a>b$, si procede nel seguente modo:

\begin{enumerate}
  \item si divide il numero per la base;
  \item si prende il resto;
  \item si ripete il processo con il quoziente ottenuto.
\end{enumerate}

Il processo si conclude quando il quoziente diventa 0. Il risultato non è altro che la sequenza ordinata dei resti ottenuti, letta al contrario. 

\begin{example}
  \begin{center}
    \begin{tblr}{c|c}
      17 & 1 \\
      8  & 0 \\
      4  & 0 \\
      2  & 0 \\
      1  & 1 
    \end{tblr}
    $$\impl 17_{10}=10001_2$$
  \end{center}
\end{example}

Le modalità con cui si trasformano le quantità fra le diverse basi sono interscambiabili, pertanto si dicono metodi.

\section{Modulo e segno}

Nella notazione decimale si utilizza la forma ``modulo e segno'' per esprimere i numeri relativi:
$$-11\;\;+4$$

Tale ragionamento non si può estendere alla rappresentazione binaria, perché richiederebbe l'introduzione un nuovo simbolo per `$+$' o `$-$'. Pertanto, per convenzione, il primo bit indica il segno, che può essere 0 se è positivo, 1 se è negativo:
$$+17_{10\text{MS}}=010001_{2\text{MS}}$$
$$-17_{10\text{MS}}=110001_{2\text{MS}}$$
$$-23_{10\text{MS}}=110111_{2\text{MS}}$$

Si devono eseguire le operazioni aritmetiche usando la notazione modulo e segno.
\begin{center}
  \begin{tblr}{c|c}
    $
      \begin{aligned}
        x & =+11_{10\text{MS}} \\
        y & =+8_{10\text{MS}}  \\
        z & =-7_{10\text{MS}}  \\
      \end{aligned}
    $
     & 
    $
      \begin{aligned}
        x+y & =+19_{10\text{MS}} \\
        x+z & =+4_{10\text{MS}}  \\
      \end{aligned}
    $
  \end{tblr}
\end{center}

Nella rappresentazione decimale si eseguono i passaggi descritti nel seguente algoritmo:

\begin{algorithm}[H]
  \caption{Somma di due valori in notazione modulo e segno}\label{algo:sommams}
  \begin{algorithmic}[1]
    \If{$\sgn x= \sgn y$}
    \State $\abs{n}=\abs{x}+\abs{y}$
    \State $\sgn n=\sgn x$
    \Else
    \State $a = \max \(\abs{x},\abs{y}\)$
    \State $b = \min \(\abs{x},\abs{y}\)$
    \State $\abs{n}=a-b$
    \If{$\abs{x}=a$}
    \State $\sgn n=\sgn a$
    \Else
    \State $\sgn n=\sgn b$
    \EndIf
    \EndIf
  \end{algorithmic}
\end{algorithm}

Tuttavia, per un calcolatore un tal numero di operazioni renderebbe il calcolo proibitivamente lento. Per questo, si introduce una nuova notazione, più efficiente di quella di modulo e segno, incentrata sull'agilità dell'elaborazione delle informazioni.

Ovviamente, la finestra dei valori che si vuole rappresentare si estende anche al negativo, per cui se si vuole rappresentare il $17$, si vuole rappresentare tutti i valori da $-17$ a $17$.

Si descrivono le possibili combinazioni avendo a disposizione 4 bit:
\begin{center}
  \begin{tblr}{c|c||c|c}
    $0000_2$ & $0_{10}$ & $1000_2$ & $-0_{10}$ \\
    $0001_2$ & $1_{10}$ & $1001_2$ & $-1_{10}$ \\
    $0010_2$ & $2_{10}$ & $1010_2$ & $-2_{10}$ \\
    $0011_2$ & $3_{10}$ & $1011_2$ & $-3_{10}$ \\ \hline
    $0100_2$ & $4_{10}$ & $1100_2$ & $-4_{10}$ \\
    $0101_2$ & $5_{10}$ & $1101_2$ & $-5_{10}$ \\
    $0110_2$ & $6_{10}$ & $1110_2$ & $-6_{10}$ \\
    $0111_2$ & $7_{10}$ & $1111_2$ & $-7_{10}$ \\
  \end{tblr}
\end{center}

Si nota subito che lo 0 ha una codifica ridondante:
$$0000_{2\text{MS}}=1000_{2\text{MS}}=0_{10}$$

Si pone l'attenzione sull'identità:
$$x+\(-x\)=0$$

Volendo rispettare l'identità con un valore $x=3_{10}$ si ha:

\begin{center}
  \begin{tblr}{ccccc}
    $^10$ & $^10$ & $^11$ & $1$ & $+$ \\
    $1$   & $1$   & $0$   & $1$ & $=$ \\
    \hline
    $0$   & $0$   & $0$   & $0$       
  \end{tblr}
\end{center}

quindi risulta che $1101_2$ è la codifica di $0011_2$ al negativo. Da questo ragionamento, esteso agli altri numeri risulta che:
\begin{center}
  \begin{tblr}{c|c||c|c}
    $0000_2$ & $0_{10}$ & $1000_2$ & $\pm8_{10}$ \\
    $0001_2$ & $1_{10}$ & $1001_2$ & $-7_{10}$   \\
    $0010_2$ & $2_{10}$ & $1010_2$ & $-6_{10}$   \\
    $0011_2$ & $3_{10}$ & $1011_2$ & $-5_{10}$   \\ \hline
    $0100_2$ & $4_{10}$ & $1100_2$ & $-4_{10}$   \\
    $0101_2$ & $5_{10}$ & $1101_2$ & $-3_{10}$   \\
    $0110_2$ & $6_{10}$ & $1110_2$ & $-2_{10}$   \\
    $0111_2$ & $7_{10}$ & $1111_2$ & $-1_{10}$   \\
  \end{tblr}
\end{center}

Esiste anche un metodo algebrico per trovare l'opposto:
$$-x=\sim x+1$$
oppure un metodo grafico:
\emph{tutti i bit che dall'LSB al primo bit 1 rimangono invariati, mentre tutti gli altri si cambiano}.

L'efficacia di questa notazione sta nella facilità con cui si eseguono le operazioni:
$$2_{10}+4_{10}=0010_2+0100_2=0110_2=6_{10}$$
$$-3_{10}+\(-1_{10}\)=1101_2+1111_2=\cancel{1}1100_2=-4_{10}$$
$$5_{10}+\(-3_{10}\)=0101_2+1101_2=\cancel{1}0010_2=2_{10}$$

Questa notazione prende il nome di complemento in base 2.
Si chiama complemento in quanto i valori negativi si completano a quelli positivi.

Per passare dalla base 10 alla base 2 in complemento a 2, si effettuano i seguenti passaggi:
\begin{algorithm}[H]
  \caption{Conversione da 10MS a 2C2}\label{algo:10to2c2}
  \begin{algorithmic}[1]
    \State $n=\abs{x}$
    \If{$x<0$}
    \State $n=-n$
    \EndIf
  \end{algorithmic}
\end{algorithm}
\begin{example}
  $$x=-13_{10}$$
  $$\abs{x}=13_{10}=1101_2=01101_{2\text{MS}}\equiv 01101_{2\text{C}2}$$
  $$x=-\abs{x}=10011_{2\text{C}2}$$
\end{example}

Si presentano casi ambigui in cui apparentemente sembra che l'aritmetica non funzioni:
$$5_{10}+3_{10}=0101_2+0011_2=1000_2=-8_{10}$$
$$-2_{10}+\(-6_{10}\)=1110_2+1010_2=\cancel{1}1000_2=-8_{10}$$
Si nota che tutti i numeri positivi cominciano con 0 mentre quelli negativi cominciano con 1.
Convenzionalmente, in virtù di quanto appena detto, la combinazione $1000_2$ assume il valore $-8_{10}$.

Usando la notazione modulo e segno il range dei valori ammissibili è $\interval{-7}{7}$, mentre in complemento a 2 il range è $\interval{-8}{7}$.

Considerando che la dimensione dell'informazione è sempre fissa e determinata, ci si ritrova in casi particolari detti di \textbf{overflow} (\emph{traboccamento}):
$$6_{10}+4_{10}=0110_2+0100_2=1010_2=-6_{10}$$
L'overflow è facilmente risolvibile con un aumento dello spazio a disposizione:
$$6_{10}+4_{10}=00110_2+00100_2=01010_2=10_{10}$$
Si ha overflow quando due valori concordi generano un valore discorde dai primi due.
Quando gli operandi sono discordi è impossibile generare overflow.

\paragraph*{Osservazione} L'estensione di un valore in complemento a 2 si esegue ripetendo il MSB quante volte se ne ha bisogno:

\begin{example}
  $$5_{10}=0101_{2\text{C}2}=000000101_{2\text{C}2}$$
  $$-5_{10}=1011_{2\text{C}2}=111111011_{2\text{C}2}$$
\end{example}

\begin{example}
  $$x=+12_{10\text{MS}}=01100_{2\text{C}2}$$
  $$y=-3_{10\text{MS}}=101_{2\text{C}2}=11101_{2\text{C}2}$$
  $$x+y=01100_2+11101_2=\cancel{1}01001_2$$
  $$x-y=x+\(-y\)=01100_2+00011_2=01111_2$$
\end{example}

% lezione del 15/09/2022

Per passare dalla base 2 in complemento a 2 alla base 10, si possono verificare due scenari:
\begin{enumerate}
  \item il numero è positivo, pertanto lo si converte usando la formula dei pesi;
  \item il numero è negativo: in questo caso si calcola l'opposto, lo si converte e gli si cambia il segno.
\end{enumerate}
\begin{example}
  $$010110_{2\text{C}2}=22_{10\text{MS}}$$
  $$1011101_{2\text{C}2}=-35_{10\text{MS}}$$
\end{example}

\section{Rappresentazione esadecimale}
Il problema principale con la rappresentazione binaria è l'enorme spazio richiesto. Inoltre, in virtù dell'uso sui calcolatori, la base da cercare deve essere una potenza di 2.
Per anni è stata utilizzata la base 8, ma ben presto è stata resa obsoleta, in favore della base 16.

Il codice esadecimale si crea associando le cifre esadecimali e combinazioni di 4 bit del codice binario.
\begin{center}
  \begin{tblr}{c|c|c}
    \textbf{esadecimale} & \textbf{binario} & \textbf{decimale} \\\hline
    0                    & 0000             & 0                 \\
    1                    & 0001             & 1                 \\
    2                    & 0010             & 2                 \\
    3                    & 0011             & 3                 \\
    4                    & 0100             & 4                 \\
    5                    & 0101             & 5                 \\
    6                    & 0110             & 6                 \\
    7                    & 0111             & 7                 \\
    8                    & 1000             & 8                 \\
    9                    & 1001             & 9                 \\
    A                    & 1010             & 10                \\
    B                    & 1011             & 11                \\
    C                    & 1100             & 12                \\
    D                    & 1101             & 13                \\
    E                    & 1110             & 14                \\
    F                    & 1111             & 15                \\
  \end{tblr}
\end{center}

Per effettuare la conversione dalla base 2 alla base 16 basta considerare ogni quadrupla e convertirla in loco:
\begin{center}
  \begin{tblr}{cccc}
    0101 & 0101 & 1101 & 1010 \\ \hline
    5    & 5    & D    & A    
  \end{tblr}
\end{center}
Banalmente, la stessa cosa avviene per il processo inverso.

Al fine di rappresentare attraverso la notazione modulo e segno un numero esadecimale, gli si prepone il segno, facendo la conversione dal binario.

\section{Numeri razionali}

Si vuole rappresentare i valori espressi da:
$$\frac{m}{n},\; m\in\mathbb{Z},n\in\mathbb{N}/\left\{ 0 \right\}$$

Nella notazione decimale si usa scrivere prima la parte intera, un separatore decimale e poi la parte frazionaria.
In base 2 si può fare la stessa cosa:
$$101.01_2=2^2+2^0+2^{-2}=5.25_{10}$$
Avendo a disposizione una certa quantità bit, si sceglie la parte di essi che contiene la parte frazionaria e quella che contiene la parte intera. 

\begin{example}
  Si considerano 3 bit per la parte intera e 2 per quella frazionaria. Si ha che il range massimo esprimibile è $\interval{0}{7.75}$, con salti di $0.25$.
  
  Tuttavia, se si vuole rappresentare il valore $6.3$, non si può. Al limite, si può esprimere una sua approssimazione, che in questo caso è $6.25$.
\end{example}

Il fatto che non si possa rappresentare un valore con \textbf{precisione} non è dovuto ad una cattiva gestione dei bit, ma bensì al loro stato di finitezza.

La notazione utilizza si chiama a virgola fissa: viene stabilita la dimensione a priori sia per la parte intera sia per la parte frazionaria.
In una notazione del genere l'\textbf{errore assoluto} $\epsilon_A$ è costante: si sbaglia sempre della stessa quantità.

Esiste, tuttavia, una notazione alternativa che consente di variare l'errore assoluto e mantenere costante l'errore relativo: la notazione in \textbf{virgola mobile} (\emph{floating point} in inglese).
L'errore relativo $$\epsilon_R=\frac{\epsilon_A}{\text{valore}}$$

Per rappresentare un numero razionale in base 2, si procede come al solito per la parte intera, ovvero si divide per 2 segnando il resto, ma al contrario per la parte frazionaria, ovvero si moltiplica per 2 segnando l'unità. Un'altra differenza fondamentale è il verso di lettura delle cifre: mentre per la parte intera si procede dal basso verso l'alto, per la parte frazionaria si procede dall'alto verso il basso.

\begin{example}
  Si vuole rappresentare il numero $13.75$.
  \begin{center}
    \begin{minipage}{0.2\linewidth}
      \begin{center}
        \begin{tblr}{c|c}
          13 & 1 \\
          6  & 0 \\
          3  & 1 \\
          1  & 1 \\
          0  &   
        \end{tblr}
      \end{center}
    \end{minipage}
    \begin{minipage}{0.2\linewidth}
      \begin{center}
        \begin{tblr}{c|c}
          0.75 &   \\
          1.5  & 1 \\
          1.0  & 1 \\
          0    &   \\
        \end{tblr}
      \end{center}
    \end{minipage}
  \end{center}
  $$13.75_{10}=1101.11_2$$
\end{example}

\begin{example}
  Si vuole rappresentare il numero $7.32$.
  \begin{center}
    \begin{minipage}{0.2\linewidth}
      \begin{center}
        \begin{tblr}{c|c}
          7 & 1 \\
          3 & 1 \\
          1 & 1 \\
          0 &   
        \end{tblr}
      \end{center}
    \end{minipage}
    \begin{minipage}{0.2\linewidth}
      \begin{center}
        \begin{tblr}{c|c}
          0.32     &          \\
          0.64     & 0        \\
          1.28     & 1        \\
          0.56     & 0        \\
          1.12     & 1        \\
          0.24     & 0        \\
          0.48     & 0        \\
          0.96     & 0        \\
          1.92     & 1        \\
          1.84     & 1        \\
          $\cdots$ & $\cdots$ 
        \end{tblr}
      \end{center}
    \end{minipage}
  \end{center}
  $$7.32_{10}=111.010100011...$$
\end{example}

Se si considerano i razionali da una diversa prospettiva, si nota che:
$$13.75_{10}=1.375\cdot 10^1$$
$$7.32_{10}=7.32\cdot 10^0$$
Analogamente:
$$1101.11_2=1.10111\cdot2^3$$
$$111.010100011_2=1.11010100011\cdot2^2$$

La notazione in virgola mobile, quindi, ha:
\begin{itemize}
  \item un bit per il \textbf{segno};
  \item una parte per l'\textbf{esponente}, a cui viene aggiunto un numero tale che il più piccolo esponente possibile sia 0;
  \item una parte per la \textbf{mantissa}, ossia la parte frazionaria della notazione scientifica\footnote{si prende la parte frazionaria in quanto la parte intera sarà sempre 1, quindi è inutile sprecare un bit}.
\end{itemize}
Lo standard che regola la notazione in virgola mobile è lo \textbf{IEEE 754}. Lo standard prevede 3 versioni:
\begin{itemize}
  \item 32 bit: 1 per il segno, 8 per l'esponente, 23 per la mantissa;
  \item 64 bit: 1 per il segno, 11 per l'esponente, 52 per la mantissa;
  \item 128 bit: 1 per il segno, 15 per l'esponente, 112 per la mantissa.
\end{itemize}

Il valore di un numero espresso attraverso lo standard IEEE 754 è dato da:
$$\text{valore}=\(-1\)^S\(1+M\)\cdot2^E$$
dove $S$ è il segno, $M$ è la mantissa e $E$ è l'esponente

\paragraph*{Nota bene}
La costante che bisogna aggiungere all'esponente si chiama ``eccesso'' ed è:
\begin{itemize}
  \item 127 per la singola precisione;
  \item 1023 per la doppia precisione;
  \item 16383 per la quadrupla precisione.
\end{itemize}

\begin{example}
  $$+5.65_{10\text{MS}}$$
  $$+101.10{1001}_{2\text{MS}}=1.01101001\cdot2^2=01000000101101001100110011001100$$
  $$-0.028_{10\text{MS}}$$
  $$-0.00000111001010_2=-1.1100101\cdot2^{-6}=101001111110101...$$
\end{example}

Lo standard, tuttavia, presenta delle falle che vengono compensate dall'introduzione di combinazioni di bit speciali che, proprio per questo, vengono dette \textbf{denormarlizzate}.

\paragraph*{Forma denormarlizzata generale}
Quando tutti i bit dell'esponente sono 0, la mantissa non è sommata ad 1 ma bensì a 0.

\paragraph*{Zero}
Quando tutti i bit (indifferentemente dal primo) sono 0, il valore è 0.

\paragraph*{Infinito}
Quando tutti i bit dell'esponente sono 1 e quelli della mantissa sono tutti 0, il valore è infinito, che può essere sia negativo che positivo.

\paragraph*{NaN}
Quando tutti i bit dell'esponente sono 1 e almeno uno di quelli della mantissa è 1, il valore non è un numero (\textbf{NaN}, \emph{Not a Number}).

% lezione del 16/09/2022

\section{L'informazione non numerica}

Per ovvi motivi, non si può rappresentare solo l'informazione numerica. L'obiettivo è, quindi, cercare di definire una codifica per ogni possibile carattere rappresentabile. Inizialmente si hanno:
\begin{itemize}
  \item alfabeto base (a...z, A...Z);
  \item caratteri numerici (0...9);
  \item caratteri di interpunzione (.,;:);
  \item caratteri speciali.
\end{itemize}

Tutti questi elementi sono circa 120, per cui sono necessari 7 bit.
Per la rappresentazione di tali simboli si utilizza il \textbf{codice ASCII} (\emph{American Standard Code for Information Interchange}): il codice dispone di 7 bit, per cui ha 128 combinazioni.
Per far fronte alle necessità sorte nel tempo, si è aggiunto un bit al codice ASCII, creando il codice \textbf{ASCII esteso}: esso dispone di 8 bit, e comprende caratteri nazionali, simboli e cornici.


\part{Elaborazione dell'informazione}

\section{Algoritmi}

Un algoritmo è un procedimento, ovvero una sequenza di passi, che a partire da un insieme di dati (\emph{input}), attraverso un numero finito di passi, genera un insieme di dati (\emph{output}).

Ogni algoritmo ha un punto d'inizio e un punto di fine.
Un algoritmo che non accetta dati di input produce sempre lo stesso output.

Ogni passo dell'algoritmo si serve di operatori, ossia simboli che indicano un'operazione da effettuare:
\begin{itemize}
  \item operatore assegnamento \verb|=|: è un operatore binario e tutto ciò che è alla sua destra, una volta calcolato, viene assegnato alla variabile che si trova alla sua sinistra;
  \item operatori aritmetici \verb|+ - * / %|: sono operatori binari che effettuano le principali operazioni aritmetiche tra due operandi (rispettivamente somma, differenza, prodotto, quoziente, resto), che possono essere variabili o costanti;
  \item operatori logici \verb|&&|, \verb!||!, \verb|!|: sono operatori binari, tranne l'ultimo che è unario, che esprimono rispettivamente le verità date dalle proposizioni: $a\wedge b$, $a\vee b$, $\lnot a$;
  \item operatori relazionali \verb|>|, \verb|<|, \verb|>=|, \verb|<=|, \verb|==|, \verb|!=|: sono operatori binari che esprimono l'uguaglianza e la disuguaglianza degli operandi. 
\end{itemize}

\begin{example}
  Algoritmo che calcola e visualizza, dato un numero strettamente positivo, l'area e il perimetro di un cerchio avente come raggio il numero dato.
  \begin{center}
    \begin{tikzpicture}[node distance=1.5cm]
      \node (start) [startstop] {\scshape Inizio};
      \node (in1) [input, below of=start] {\verb|r|};
      \node (pro1) [process, below of=in1] {\verb|p = 2 * PI * r|};
      \node (pro2) [process, below of=pro1] {\verb|a = PI * r * r|};
      \node (out1) [output, below of=pro2] {\verb|p, a|};
      \node (stop) [startstop, below of=out1] {\scshape Fine};
      \draw [arrow] (start) -- (in1);
      \draw [arrow] (in1) -- (pro1);
      \draw [arrow] (pro1) -- (pro2);
      \draw [arrow] (pro2) -- (out1);
      \draw [arrow] (out1) -- (stop);
    \end{tikzpicture}
  \end{center}
\end{example}

\begin{example}
  Algoritmo che, dato un numero $s$ intero positivo di secondi, calcola e visualizza le ore, i minuti e i secondi che $s$ quantifica.
  \begin{center}
    \begin{tikzpicture}[node distance=1.5cm]
      \node (start) [startstop] {\scshape Inizio};
      \node (pro1) [process, below of=start] {\verb|SECS_PER_MIN = 60|};
      \node (pro2) [process, below of=pro1] {\verb|SECS_PER_HOUR = SECS_PER_MIN * 60|};
      \node (in) [input, below of=pro2] {\verb|s|};
      \node (pro3) [process, below of=in] {\verb|h = s / SECS_PER_HOUR|};
      \node (pro4) [process, below of=pro3] {\verb|s = s % SECS_PER_HOUR|};
      \node (pro5) [process, below of=pro4] {\verb|m = s / SECS_PER_MIN|};
      \node (pro6) [process, below of=pro5] {\verb|s = s % SECS_PER_MIN|};
      \node (out1) [output, below of=pro6] {\verb|h, m, s|};
      \node (stop) [startstop, below of=out1] {\scshape Fine};
      \draw [arrow] (start) -- (pro1);
      \draw [arrow] (pro1) -- (pro2);
      \draw [arrow] (pro2) -- (in);
      \draw [arrow] (in) -- (pro3);
      \draw [arrow] (pro3) -- (pro4);
      \draw [arrow] (pro4) -- (pro5);
      \draw [arrow] (pro5) -- (pro6);
      \draw [arrow] (pro6) -- (out1);
      \draw [arrow] (out1) -- (stop);
    \end{tikzpicture}
  \end{center}
\end{example}

\begin{example}
  Algoritmo che, dato un numero $x$, calcola e visualizza il suo valore assoluto.
  \begin{center}
    \begin{tikzpicture}[node distance=1.8cm]
      \node (start) [startstop] {\scshape Inizio};
      \node (in) [input, below of=start] {\verb|x|};
      \node (d1) [decision, below of=in] {\verb|x < 0|};
      \node (p1) [process, left of=d1, xshift=-2cm] {\verb|y = -x|};
      \node (p2) [process, right of=d1, xshift=2cm] {\verb|y = x|};
      \node (out) [output, below of=d1] {\verb|y|};
      \node (stop) [startstop, below of=out] {\scshape Fine};
      \draw [arrow] (d1) -- node[anchor=north] {sì} (p1);
      \draw [arrow] (d1) -- node[anchor=north] {no} (p2);
      \draw [arrow] (start) -- (in);
      \draw [arrow] (in) -- (d1);
      \draw [arrow] (out) -- (stop);
      \draw [arrow] (p1) |- (out);
      \draw [arrow] (p2) |- (out);
    \end{tikzpicture}
  \end{center}
\end{example}

\begin{example}
  Algoritmo che, dato un numero $r$ come raggio di una circonferenza, ne calcola l'area e il perimetro, solo se $r$ è positivo, altrimenti attende un nuovo valore.
  \begin{center}
    \begin{tikzpicture}[node distance=1.8cm]
      \node (start) [startstop] {\scshape Inizio};
      \node (in) [input, below of=start] {\verb|r|};
      \node (d1) [decision, below of=in] {\verb|r < 0|};
      \node (p1) [process, below of=d1] {\verb|p = 2 * PI * r|};
      \node (p2) [process, below of=p1] {\verb|a = PI * r * r|};
      \node (o1) [output, below of=p2] {\verb|p, a|};
      \node (stop) [startstop, below of=o1] {\scshape Fine};
      \draw [arrow] (start) -- (in);
      \draw [arrow] (in) -- (d1);
      \draw [arrow] (d1) -- node[anchor=east] {no} (p1);
      \draw [arrow] (d1.west) |- node[anchor=east] {sì} (in);
      \draw [arrow] (p1) -- (p2);
      \draw [arrow] (p2) -- (o1);
      \draw [arrow] (o1) -- (stop);
    \end{tikzpicture}
  \end{center}
\end{example}

\begin{example}
  Algoritmo che, dati tre numeri, visualizza 1 se i tre numeri compongono una terna pitagorica, 0 altrimenti.
  \begin{center}
    \begin{tikzpicture}[node distance=2cm]
      \node (start) [startstop] {\scshape Inizio};
      \node (in) [input, below of=start] {\verb|a, b, c|};
      \node (d1) [decision, below of=in] {\verb|a > b|};
      \node (d2) [decision, below left of=d1, xshift=-2cm] {\verb|a > c|};
      \node (d3) [decision, below right of=d1, xshift=2cm] {\verb|b > c|};
      \node (p1) [process, below of=d2] {\verb|swap(a, c)|};
      \node (p2) [process, below of=d3] {\verb|swap(b, c)|};
      \node (d4) [decision, below of=d1, yshift=-3.5cm] {\verb|a * a + b * b == c * c|};
      \node (o1) [output, below left of=d4, xshift=-2cm, yshift=-1cm] {\verb|1|};
      \node (o2) [output, below right of=d4, xshift=2cm, yshift=-1cm] {\verb|0|};
      \node (stop) [startstop, below of=d4, yshift=-2cm] {\scshape Fine};
      \draw [arrow] (start) -- (in);
      \draw [arrow] (in) -- (d1);
      \draw [arrow] (d1) -| node[anchor=east] {sì} (d2);
      \draw [arrow] (d1) -| node[anchor=west] {no} (d3);
      \draw [arrow] (d2) -- node[anchor=east] {sì} (p1);
      \draw [arrow] (d2) -| node[anchor=south] {no} (d4);
      \draw [arrow] (d3) -- node[anchor=west] {sì} (p2);
      \draw [arrow] (d3) -| (d4);
      \draw [arrow] (p1.east) -| (d4.north);
      \draw [arrow] (p2.west) -| (d4.north);
      \draw [arrow] (d4.west) -| node[anchor=east] {sì} (o1);
      \draw [arrow] (d4.east) -| node[anchor=west] {no} (o2);
      \draw [arrow] (o1) |- (stop);
      \draw [arrow] (o2) |- (stop);
    \end{tikzpicture}
  \end{center}
\end{example}

\begin{example}
  Algoritmo che, dati 10 valori interi, calcola e visualizza il massimo. 
  \begin{center}
    \begin{tikzpicture}[node distance=1.5cm]
      \node (start) [startstop] {\scshape Inizio};
      \node (p1) [process, below of=start] {\verb|i = 0|};
      \node (p2) [process, below of=p1] {\verb|m = -INF|};
      \node (d1) [decision, below of=p2] {\verb|i < 10|};
      \node (in) [input, below left of=d1, xshift=-2cm] {\verb|x|};
      \node (d2) [decision, below of=in] {\verb|x > m|};
      \node (p3) [process, below of=d2] {\verb|m = x|};
      \node (p4) [process, below of=p3] {\verb|i = i + 1|};
      \node (f) [dummy, left of=p4, xshift=-1cm] {};
      \node (g) [dummy, right of=d2, xshift=1cm] {};
      \node (out) [output, below right of=d1, xshift=2cm] {\verb|m|};
      \node (stop) [startstop, below of=out] {\scshape Fine};
      \draw [arrow] (start) -- (p1);
      \draw [arrow] (p1) -- (p2);
      \draw [arrow] (p2) -- (d1);
      \draw [arrow] (d1) -| node[anchor=south] {sì} (in);
      \draw [arrow] (in) -- (d2);
      \draw [arrow] (d2) -- node[anchor=east] {sì} (p3);
      \draw [arrow] (d2) -- node[anchor=south] {no} (g) |- (p4);
      \draw [arrow] (p3) -- (p4);
      \draw [arrow] (d1) -| node[anchor=south] {no} (out);
      \draw [arrow] (out) -- (stop);
      \draw [arrow] (p4) -- (f) |- (d1.north);
    \end{tikzpicture}
  \end{center}
\end{example}

\begin{example}
  Algoritmo che, dati 10 valori interi, calcola e visualizza il massimo e il minimo. 
  \begin{center}
    \begin{tikzpicture}[node distance=1.5cm]
      \node (start) [startstop] {\scshape Inizio};
      \node (p1) [process, below of=start] {\verb|i = 0|};
      \node (p2) [process, below of=p1] {\verb|M = -INF|};
      \node (p3) [process, below of=p2] {\verb|m = INF|};
      \node (d1) [decision, below of=p3] {\verb|i < 10|};
      \node (in) [input, below left of=d1, xshift=-2cm] {\verb|x|};
      \node (d2) [decision, below of=in] {\verb|x < m|};
      \node (p4) [process, below of=d2] {\verb|m = x|};
      \node (d3) [decision, below of=p4, yshift=-1cm] {\verb|x > M|};
      \node (p5) [process, below of=d3] {\verb|M = x|};
      \node (p6) [process, below of=p5] {\verb|i = i + 1|};
      \node (f) [dummy, right of=d3, xshift=1cm] {};
      \node (j) [dummy, above of=d3] {};
      \node (g) [dummy, right of=d2, xshift=1.4cm] {};
      \node (h) [dummy, left of=p6, xshift=-1cm] {};
      \node (k) [dummy, right of=p4, xshift=1.2cm] {};
      \node (out) [output, below right of=d1, xshift=2cm] {\verb|M, m|};
      \node (stop) [startstop, below of=out] {\scshape Fine};
      \draw [arrow] (start) -- (p1);
      \draw [arrow] (p1) -- (p2);
      \draw [arrow] (p2) -- (p3);
      \draw [arrow] (p3) -- (d1);
      \draw [arrow] (d1) -| node[anchor=south] {sì} (in);
      \draw [arrow] (in) -- (d2);
      \draw [arrow] (d2) -- node[anchor=east] {sì} (p4);
      \draw [arrow] (d2) -- node[anchor=south] {no} (g) |- (j) -- (d3.north);
      \draw [arrow] (p4) -- (k) |- (p6);
      \draw [arrow] (d3) -- node[anchor=east] {sì} (p5);
      \draw [arrow] (p5) -- (p6);
      \draw [arrow] (p6) -- (h) |- (d1.north);
      \draw [arrow] (d3) -- node[anchor=south] {no} (f) |- (p6);
      \draw [arrow] (d1) -| node[anchor=south] {no} (out);
      \draw [arrow] (out) -- (stop);
    \end{tikzpicture}
  \end{center}
\end{example}

\newpage

\section{Il linguaggio C}

Ogni programma scritto in C, comincia la propria esecuzione dalla funzione \verb|main|:

\begin{verbatim}
  int main(int argc, char *argv[]) {
    // codice
    return 0;
  }
\end{verbatim}

Il linguaggio C è un {\bf linguaggio staticamente tipizzato}, ossia per ogni variabile deve essere esplicitamente indicato il tipo.
La ragione principale di questa scelta è stata la gestione della memoria.

La {\bf dichiarazione} di una variabile si compone di due parti: un tipo e un nome.
I tipi elementari del C sono:
\begin{itemize}
  \item \verb|char|: numero intero a 8 bit, normalmente rappresenta un carattere ASCII;
  \item \verb|short|: numero intero a 16 bit;
  \item \verb|int|: numero intero a 32 bit;
  \item \verb|long|: numero intero a 64 bit;
  \item \verb|float|: numero razionale a 32 bit;
  \item \verb|double|: numero razionale a 64 bit;
  \item \verb|long double|: numero razionale a 128 bit;
\end{itemize}
Ad ogni tipo intero può essere preposto la keyword \verb|unsigned|, che trasforma il tipo in naturale.

I nomi delle variabili seguono alcune regole di base:
\begin{itemize}
  \item non possono cominciare con una cifra;
  \item non possono contenere operatori aritmetici;
  \item non possono contenere spazi o simboli di interpunzione.
\end{itemize}

La dichiarazione di una variabile è un'{\bf istruzione}.
Ogni istruzione termina con il carattere `\verb|;|'. Difatti, questo metodo di separazione delle istruzioni consente di comprimere il codice elidendo gli spazi e gli a capo.
L'istruzione ``\verb|;|'' è un'istruzione valida, ma che non produce effetti.

Il gruppo di istruzioni:
\begin{verbatim}
  int a;
  int b;
\end{verbatim}
è equivalente all'unica istruzione:
\begin{verbatim}
  int a, b;
\end{verbatim}

L'istruzione di {\bf assegnamento} è composta da 3 parti: l'assegnando, l'operatore di assegnazione e l'espressione da assegnare
\begin{verbatim}
  a = 1;      // assegnamento di un valore costante
  a = b;      // assegnamento di una variabile
  a = b * 2;  // assegnamento di un'espressione
\end{verbatim}

In C, le operazioni tra {\bf variabili} e {\bf costanti} seguono alcune regole di valutazione:
\begin{verbatim}
  int   op int   -> int
  float op float -> float
  int   op float -> float
\end{verbatim}
In generale, qualora due operandi non siano omogenei, il risultato dell'operazione ha il tipo dell'operando che contiene più informazione.

L'operazione che cambia il tipo di una variabile si chiama \textbf{\emph{cast}} (promozione). Nel caso in cui il cast venga fatto automaticamente, viene detto \textbf{implicito}. Altrimenti, viene detto \textbf{esplicito}, e avviene mediante l'istruzione:
\begin{verbatim}
  (tipo)variabile
\end{verbatim}

L'istruzione che consente di acquisire da input i dati è
\begin{verbatim}
  scanf(formato, ...);
\end{verbatim}
il formato è una stringa che contiene le specifiche del formato di acquisizione ed è caratterizzato da sequenze specifiche di caratteri che indicano i formati e le posizioni:
\begin{itemize}
  \item \verb|%d|, \verb|%i|: \verb|int|;
  \item \verb|%f|, \verb|%g|: \verb|float|;
  \item \verb|%lf|: \verb|double|;
  \item \verb|%c|: \verb|char|;
  \item \verb|%s|: \verb|char*|;
\end{itemize}
I vari argomenti della funzione saranno le destinazioni dei vari elementi del formato.

\begin{example}
\begin{verbatim}
    int a, b;
    float c;
    scanf("%d", &a);
    scanf("%f", &c);
    scanf("%d %d", &a, &b);\end{verbatim}
\end{example}

L'istruzione che consente di visualizzare l'output è
\begin{verbatim}
  printf(formato, ...);
\end{verbatim}
Il formato e gli argomenti seguono le stesse regole della \verb|scanf|.

Le stringhe in C possono avere al loro interno caratteri ASCII non stampabili, come a capo o la tabulazione. Questi caratteri sono rappresentabili attraverso le formule di escape, e sono:
\begin{itemize}
  \item \verb|\n|: a capo;
  \item \verb|\t|: tabulazione;
  \item \verb|\a|: campanella;
  \item \verb|\r|: ritorno;
  \item \verb|\0|: terminatore;
\end{itemize}

\begin{example}
  \begin{verbatim}
    int a
    float b;
    printf("risultato: %d (%f%%)\n", a, b);\end{verbatim}
\end{example}


Nel linguaggio C esistono le {\bf direttive}, ossia delle istruzioni che non vengono eseguite, ma di cui il compilatore tiene conto. Tra queste le più importanti sono:
\begin{itemize}
  \item \verb|#include|: serve a inserire nel codice altro codice contenuto nel file indicato;
  \item \verb|#define|: consente di indicare degli alias per una sequenza di caratteri.
\end{itemize}

\begin{example}
  \begin{verbatim}
    #include "stdio.h"
    #include <stdlib.h>
    
    #define PI 3.14
    #define F(X) ((X) - 1)\end{verbatim}
\end{example}

\begin{example}
  Scrivere un programma che, dato in input un numero intero strettamente positivo, calcola e visualizza la circonferenza e l'area.
  \begin{verbatim}
    #include "stdio.h"
    #include "math.h"
    int main(int argc, char **argv) {
      unsigned int r;
      scanf("%u", &r);
      double p, a;
      p = 2 * M_PI * r;
      a = M_PI * r * r;
      printf("%lf %lf", p, a);
      return 0;
    }\end{verbatim}
\end{example}

\begin{example}
  Scrivere un programma che, dato in input un numero razionale, ne calcola e visualizza l'arrotondamento per difetto.
  \begin{verbatim}
    #include "stdio.h"
    int main(int argc, char **argv) {
      float x;
      scanf("%f", &x);
      int y;
      y = (int)x;
      printf("%d", y);
      return 0;
    }\end{verbatim}
\end{example}

Nei programmi C è solito utilizzare l'istruzione \verb|n = n + 1|. Questa operazione è equivalente all'istruzione \verb|n++|. Analogamente, l'istruzione \verb|n = n - 1| è equivalente a \verb|n--|.
Inoltre, le istruzioni del tipo \verb|n = n op x| sono equivalenti alle istruzioni \verb|n op= x|.

\begin{example}
  Scrivere un programma che, dato in input un numero intero, ne calcola e visualizza il valore assoluto.
  \begin{verbatim}
    #include <stdio.h>
    int main(int argc, char **argv) {
      int x;
      scanf("%d", &x);
      if (x < 0) 
        x *= -1;
      printf("%d", x);
      return 0;
    }\end{verbatim}
\end{example}

Il costrutto di selezione in C è nella forma:
\begin{verbatim}
  if (expr)
    code_true;
  else
    code_false;
\end{verbatim}

La parola chiave \verb|if| valuta l'espressione \verb|expr| e, se vera, esegue l'istruzione \verb|code_true|, altrimenti esegue l'istruzione \verb|code_false|.
Si può, altresì, eseguire più di un'istruzione, usando i blocchi, ossia sequenze ordinate di istruzioni, che si delimitano dalle parentesi graffe \verb|{ }|: agli occhi della \verb|if|, il blocco appare come un'unica istruzione.
Si precisa che la parte \verb|else ...| è opzionale, per cui si può omettere quando non serve.

Le espressioni che esprimono un valore di verità sono dette booleane. Esse si servono dei costrutti logici \verb|&&| (congiunzione), \verb!||! (disgiunzione) e \verb|!| (negazione).

\begin{example}
  Scrivere un programma che, dato in input un numero intero, visualizza `\verb|+|' se il valore è positivo, `\verb|-|' se è negativo e `\verb| |' se è nullo.
  \begin{verbatim}
    #include <stdio.h>

    #define POSITIVE '+'
    #define NEGATIVE '-'
    #define ZERO ' '

    int main(int argc, char **argv) {
      int x;
      char sign;
      scanf("%d", &x);
      if (x > 0) sign = POSITIVE;
      else if (x < 0) sign = NEGATIVE;
      else sign = ZERO;
      printf("%c", sign);
      return 0;
    }\end{verbatim}
\end{example}

In C, un'espressione è comunque booleana, ed è vera se il suo valore numerico è diverso da 0.
Alcuni esempi:
\begin{itemize}
  \item \verb|x|: \verb|x!=0|;
  \item \verb|!a|: \verb|a==0|;
  \item \verb|5 = a|: errore nella compilazione;
  \item \verb|a = 5|: sempre vero, dato che ad \verb|a| viene assegnato il valore 5 e il risultato di un'operazione di assegnamento è il valore assegnato.
\end{itemize}

Il costrutto di iterazione si presenta sia nella forma a condizione iniziale, sia nella forma a condizione finale:
\begin{verbatim}
  while (expr)
    code;
\end{verbatim}
La parola chiave \verb|while| valuta l'espressione \verb|expr| e, finché vera, esegue l'istruzione \verb|code|, reiterandone l'esecuzione.
\begin{verbatim}
  do
    code;
  while (expr);
\end{verbatim}
La parola chiave \verb|do| esegue l'istruzione \verb|code| e successivamente valuta l'espressione \verb|expr| attraverso \verb|while|: finché vera, l'esecuzione di \verb|code| è reiterata.

In entrambi i casi, analogamente al costrutto di selezione, \verb|code| può essere un blocco di istruzioni, consentendo l'esecuzione ripetuta di più istruzioni consecutive.

\begin{example}
  Scrivere un programma che, dati in input 20 numeri interi, ne calcola e visualizza il valore massimo.
  \begin{verbatim}
    #define N 20
    #include <stdio.h>
    int main(int argc, char** argv) {
      unsigned i = N;
      int n, m = 0x80000000;
      while (i--) {
        scanf("%d", &n);
        if (n > m) m = n;
      }
      printf("%d", m);
      return 0;
    }\end{verbatim}
\end{example}

\begin{example}
  Scrivere un programma che chiede un numero intero (finché esso non è positivo) e ne calcola e visualizza il numero di cifre.
  \begin{verbatim}
    #include <stdio.h>
    int main(int argc, char** argv) {
      int n, m, tmp;
      do scanf("%d", &n);
      while (n <= 0);
      m   = 0;
      tmp = n;
      while (tmp) {
        tmp /= 10;
        m++;
      }
      printf("%d", m);
      return 0;
    }\end{verbatim}
\end{example}

Il linguaggio C consente l'uso di una struttura dati particolare: l'array. L'array è una struttura dati che contiene dati omogenei, la cui cardinalità è costante e nota a priori. Lo si può immaginare come una cassettiera, dove ogni cassetto contiene un elemento.
La dichiarazione di un array è nella seguente forma:
\begin{verbatim}
  tipo nome[dimensione];
\end{verbatim}
Per accedere ad un elemento specifico dell'array, si scrive:
\begin{verbatim}
  nome[i]
\end{verbatim}
dove \verb|i| è l'indice, ossia un valore numerico che va da 0 a \verb|dimensione - 1|.
Ai fini della scrittura di codice, l'espressione \verb|nome[i]| è assimilabile ad una variabile e ne eredita tutte le proprietà.

\begin{example}
  Scrivere un programma che acquisisce 50 valori interi e visualizza i valori superiori al valor medio dei valori acquisiti.
  \begin{verbatim}
    #define N 50
    #include <stdio.h>
    int main(int argc, char** argv) {
      int v[N], s, i;
      float avg;
      i = 0;
      s = 0;
      while (i < N) {
        scanf("%d", &v[i]);
        s += v[i++];
      }
      avg = (float)s / N;
      i   = 0;
      while (i < N) {
        if (v[i] > avg) printf("%d ", v[i]);
        i++;
      }
      return 0;
    }\end{verbatim}
\end{example}

Strettamente legato agli array è il costrutto \verb|for|:
\begin{verbatim}
  for (pre; condition; post)
    code;
\end{verbatim}
Esso non è altro che l'abbreviazione di:
\begin{verbatim}
  pre;
  while (condition) {
    code;
    post;
  }
\end{verbatim}

Il linguaggio C consente di creare array di cardinalità $n$--dimensionale: monodimensionale è un vettore, bidimensionale è una matrice, e così via.
La dichiarazione di un array bidimensionale è nella forma:
\begin{verbatim}
  tipo nome[dim1][dim2];
\end{verbatim}
tenendo bene a mente che \verb|dim1| è il numero delle righe e \verb|dim2| è il numero delle colonne.
Valgono per essi le stesse proprietà degli array monodimensionale, tra cui l'accesso ad un singolo elemento che è:
\begin{verbatim}
  nome[i][j]
\end{verbatim}

\begin{example}
  Scrivere un programma che acquisisce i valori interi di una matrice $5\times 5$ e calcola e visualizza 1 se si tratta di una matrice identità, 0 altrimenti.
  \begin{verbatim}
    #define NR           5
    #define NC           5
    #define IDENTITY     1
    #define NON_IDENTITY 0
    int main(int argc, char** argv) {
      int m[NR][NC], s, i, j;
      for (i = 0; i < NR; i++)
        for (j = 0; j < NC; j++) scanf("%d", &m[i][j]);
      s = (NR == NC ? IDENTITY : NON_IDENTITY);
      for (i = 0; s == IDENTITY && i < NR; i++)
        for (j = 0; s == IDENTITY && j < NC; j++)
          if (i == j && m[i][j] != 1)
            s = NON_IDENTITY;
          else if (i != j && m[i][j] != 0)
            s = NON_IDENTITY;
      printf("%d", s);
      return 0;
    }\end{verbatim}
\end{example}

\paragraph*{Linearizzazione della memoria}

Essendo la memoria contigua, un array bidimensionale non può essere contenuto nella sua forma naturale in memoria. Tuttavia, il segmento di memoria occupato dall'array deve essere contiguo e perciò si ricorre alla linearizzazione dell'array, ossia dell'appiattimento virtuale che consente di considerare una matrice come un vettore. Così, una matrice di dimensione $n\times m$ diventa un vettore $n\cdot m$ e, all'elemento \verb|x[i][j]| corrisponderà l'elemento \verb|y[i * C + j]|, dove \verb|C| è il numero di colonne dichiarato.

Il linguaggio C prevede la dichiarazione di un nuovo tipo, contenente altri tipi di natura non necessariamente omogenea. Questa nuova tipologia di informazione è resa possibile dalla keyword \verb|struct|:
\begin{verbatim}
  struct nome_s {
    tipo v1;
    tipo v2;
  };
\end{verbatim}
Ogni qualvolta si vuole utilizzare il nuovo tipo, si procede nel seguente modo:
\begin{verbatim}
  struct nome_s nome;
\end{verbatim}
Per utilizzare i campi della nuova struttura si usa la notazione punto, ossia:
\begin{verbatim}
  nome.campo
\end{verbatim}
\begin{example}
  \begin{verbatim}
    struct data_s {
      int giorno;
      int mese;
      int anno;
    };
    ...
    struct data_s oggi;
    oggi.giorno = 21;
  \end{verbatim}
\end{example}

Per evitare l'uso continuo di \verb|struct|, si fa uso della keyword \verb|typedef|, che si usa nel seguente modo:
\begin{verbatim}
  typedef nome_s nome_t;
\end{verbatim}

\begin{example}
  \begin{verbatim}
    typedef struct data_s {
      int giorno;
      int mese;
      int anno;
    } data_t;
    ...
    data_t oggi;
    oggi.giorno = 21;
  \end{verbatim}
\end{example}

La libreria standard del C prevede una specializzazione degli array di caratteri: le stringhe. Esse sono array di caratteri, il cui contenuto è terminato dal carattere terminatore \verb|\0| (valore ASCII 0). Nelle funzioni di input/output, le stringhe hanno il segnaposto \verb|%s|:
\begin{verbatim}
  char s[32];
  scanf("%s", s); /* equivalente: scanf("%s", &s[0]) */
  printf("%s", s); /* equivalente: printf("%s", &s[0]) */
\end{verbatim}

La \verb|scanf|, però, ha un limite: non considera gli spazi come caratteri di una stringa. Pertanto, al fine di ottenere stringhe ``complete'', si usa la funzione \verb|gets| contenuta nell'header \verb|stdio.h|:
\begin{verbatim}
  #include <stdio.h>
  ...
  char s[32];
  gets(s);
\end{verbatim}

Ovviamente, il carattere terminatore deve essere in memoria e quindi deve occupare uno spazio all'interno dell'array. A tal fine, considerando \verb|n| come dimensione massima della stringa, in fase di dichiarazione la lunghezza da assegnare sarà \verb|n + 1|:
\begin{verbatim}
  #define N 50
  ...
  char s[N + 1];
\end{verbatim}
\section{Sistemi lineari}

Si considera il sistema di $m$ equazioni in $n$ incognite:
$$
  \begin{cases}
    a_{11}x_1+a_{12}x_2+\cdots +a_{1n}x_n=b_1 \\
    a_{21}x_1+a_{22}x_2+\cdots +a_{2n}x_n=b_2 \\
    \vdots                                    \\
    a_{m1}x_1+a_{m2}x_2+\cdots +a_{mn}x_n=b_m \\
  \end{cases}
$$

La matrice $A\walrus\(a_{ij}\)$ è detta \textbf{matrice dei coefficienti}, il vettore colonna $B\walrus\(b_{i}\)$ è detto \textbf{vettore dei termini noti} e il vettore $X\walrus\(x_i\)$ è detto \textbf{vettore delle incognite}.

Si osserva che il sistema è equivalente a:
$$AX=B$$

\subsection{Matrice completa}

\begin{definition}[Matrice completa]
  La \textbf{matrice completa} del sistema $AX=B$ è la matrice $\(A,B\)$.
\end{definition}

\begin{example}
  $$
    \begin{cases}
      2x+y=1 \\
      x-y=-1
    \end{cases}
    \mapsto
    \begin{pmatrix}
      2 & 1  & 1  \\
      1 & -1 & -1 
    \end{pmatrix}
  $$
\end{example}

\begin{definition}[Soluzione]
  Una \textbf{soluzione} di un sistema $AX=B$ è un $m$--vettore colonna $\bar{X}$ che soddisfa l'equazione, cioè $A\bar{X}=B$.
  L'insieme delle soluzioni del sistema si indica con $\sol\(A,B\)$.
\end{definition}

\begin{definition}[Sistema omogeneo]
  Un sistema $AX=B$ si dice \textbf{omogeneo} se $B=\vec{0}$.
\end{definition}

\begin{theorem}[Legge di sovrapposizione]
  Se $X_1,X_2$ sono soluzioni di un sistema omogeneo, anche $X_1+X_2$ lo è.  
  Se $X$ è soluzione di un sistema omogeneo, allora anche $tX$ lo è.
\end{theorem}
\begin{proof}
  $$A\(X_1+X_2\)=AX_1+AX_2=\vec{0}+\vec{0}=\vec{0}$$
  $$A\(tX\)=tAX=t\vec{0}=\vec{0}$$
\end{proof}

Si suppone che $AX=B$ abbia soluzione $X_P$. Allora tutte le soluzioni del sistema sono date da $X_P+Y\ \forall Y\in\sol\(A,0\)$. In simboli:
$$\sol\(A,B\)=X_P+\sol\(A,0\)$$

\begin{proof}
  Se $Y\in\sol\(A,0\)$ allora
  $$A\(X_P+Y\)=AX_P+AY=B+\vec{0}=B$$
  Se $X'\in\sol\(A,B\)$ sia $Y=X'-X_P$ cosicché $X'=X_P+Y$ si ha che
  $$AY=A\(X'-X_P\)=AX'-AX_P=B-B=\vec{0}$$
  quindi
  $$X'=X_P+Y,\ Y\in\sol\(A,0\)$$.
\end{proof}

\begin{example}
  $$
    \begin{cases}
      x+y=0 \\
      x-y=2 \\
    \end{cases}
    \mapsto
    \begin{pmatrix}
      1 & 1  & 0 \\
      1 & -1 & 2 \\
    \end{pmatrix}
  $$
  $$
    X_P=
    \begin{pmatrix}
      1  \\
      -1 \\
    \end{pmatrix}
  $$
  Ogni soluzione si scrive come $X_P+Y$ dove $Y$ è soluzione di 
  $$
    \begin{cases}
      x+y=0 \\
      x-y=0 \\
    \end{cases}
    \impl
    Y=
    \begin{pmatrix}
      0 \\
      0 \\
    \end{pmatrix}
  $$
  Pertanto $X_P$ è l'unica soluzione.
\end{example}

\begin{example}
  $$
    \begin{cases}
      x+y+z=1  \\
      2x+3z=2  \\
      x-y+2z=1 \\
    \end{cases}
    \impl
    \begin{cases}
      x+y+z=1 \\
      2x+3z=2 \\
      2x+3z=2 \\
    \end{cases}
    \impl
    \begin{cases}
      2x+2y+2z=2 \\
      2x+3z=2    \\
    \end{cases}
    \impl
  $$
  $$
    \impl
    \begin{cases}
      x+y+z=1 \\
      -2y+z=0 \\
    \end{cases}
    \impl
    \begin{cases}
      z=2y     \\
      x+y+2y=1 \\
    \end{cases}
    \impl
    \begin{cases}
      z=2y   \\
      x+3y=1 \\
    \end{cases}
    \impl
    \begin{cases}
      y=\frac{1}{2}z   \\
      x=1-\frac{3}{2}z \\
    \end{cases}
  $$
  Le soluzioni sono date da:
  $$
    \begin{pmatrix}
      1 \\
      0 \\
      0
    \end{pmatrix}
    +
    z
    \begin{pmatrix}
      -\nicefrac{3}{2} \\
      \nicefrac{1}{2}  \\
      1                \\
    \end{pmatrix}
  $$
\end{example}

\subsection{Metodo di Gauss--Jordan}

\begin{definition}[Pivot]
  In una matrice qualsiasi il primo elemento non nullo di una riga viene chiamato \textbf{pivot} della riga.
\end{definition}
  
\begin{definition}[Matrice a gradini]
  Una matrice si dice \textbf{a gradini} se il pivot di ogni riga è in una colonna successiva a quella del pivot della riga precedente.
  In particolare, in una matrice a gradini, se una riga è nulla allora tutte le righe successive devono essere nulle.
\end{definition}
  

Si definiscono operazioni elementari di riga, operanti su una matrice:
\begin{itemize}
  \item scambiare due righe;
  \item moltiplicare una riga per $n\in\mathbb{R}\setminus\left\{ 0 \right\}$;
  \item sommare ad una riga un multiplo di un'altra riga.
\end{itemize}
Si scrive $A\sim B$ se $B$ si ottiene a partire da $A$ mediante una sequenza di operazioni elementari di riga.

Il metodo di Gauss--Jordan consiste nel trasformare una matrice completa in una matrice a gradini.

\begin{example}
  $$
    \begin{pmatrix}
      1 & 1  & 1 & 1 \\
      2 & 0  & 3 & 2 \\
      1 & -1 & 2 & 1 \\
    \end{pmatrix}
    \sim
    \begin{pmatrix}
      1 & 1  & 1 & 1 \\
      1 & -1 & 2 & 1 \\
      2 & 0  & 3 & 2 \\
    \end{pmatrix}
    \sim
    \begin{pmatrix}
      1 & 1  & 1 & 1 \\
      0 & -2 & 1 & 0 \\
      2 & 0  & 3 & 2 \\
    \end{pmatrix}
    \sim
    \begin{pmatrix}
      1 & 1  & 1 & 1 \\
      0 & -2 & 1 & 0 \\
      0 & -2 & 1 & 0 \\
    \end{pmatrix}
    \sim
    \begin{pmatrix}
      1 & 1  & 1 & 1 \\
      0 & -2 & 1 & 0 \\
      0 & 0  & 0 & 0 \\
    \end{pmatrix}
  $$
  $$
    \begin{pmatrix}
      1 & 1  & 2 & 2 \\
      2 & 0  & 4 & 2 \\
      1 & -1 & 2 & 0 \\
      0 & 2  & 0 & 2 \\
    \end{pmatrix}
    \sim
    \begin{pmatrix}
      1 & 1  & 2 & 2  \\
      0 & -2 & 0 & -2 \\
      0 & -2 & 0 & -2 \\
      0 & 2  & 0 & 2  \\
    \end{pmatrix}
    \sim
    \begin{pmatrix}
      1 & 1  & 2 & 2  \\
      0 & -2 & 0 & -2 \\
      0 & 0  & 0 & 0  \\
      0 & 0  & 0 & 0  \\
    \end{pmatrix}
  $$
  $$
    \begin{pmatrix}
      0 & 2  & 1  & 0 & 1  \\
      0 & 3  & -1 & 1 & 1  \\
      1 & 1  & -2 & 1 & 0  \\
      1 & -1 & -3 & 1 & -1 \\
    \end{pmatrix}
    \sim
    \begin{pmatrix}
      1 & -1 & -3 & 1 & -1 \\
      1 & 1  & -2 & 1 & 0  \\
      0 & 3  & -1 & 1 & 1  \\
      0 & 2  & 1  & 0 & 1  \\
    \end{pmatrix}
    \sim
    \begin{pmatrix}
      1 & -1 & -3 & 1 & -1 \\
      0 & 2  & 1  & 0 & 1  \\
      0 & 3  & -1 & 1 & 1  \\
      0 & 2  & 1  & 0 & 1  \\
    \end{pmatrix}
    \sim
  $$
  $$
    \sim
    \begin{pmatrix}
      1 & -1 & -3           & 1 & -1           \\
      0 & 2  & 1            & 0 & 1            \\
      0 & 0  & -\frac{5}{2} & 1 & -\frac{1}{2} \\
      0 & 0  & 0            & 0 & 0            \\
    \end{pmatrix}
    \sim
    \begin{pmatrix}
      1 & -1 & -3 & 1 & -1 \\
      0 & 2  & 1  & 0 & 1  \\
      0 & 0  & -5 & 2 & -1 \\
      0 & 0  & 0  & 0 & 0  \\
    \end{pmatrix}
  $$  
\end{example}


Un sistema si dice a gradini se la sua matrice completa lo è. Dato che ogni matrice può essere trasformata mediante una sequenza di operazioni elementari di riga in una matrice a gradini, ogni sistema è equivalente ad una sistema a gradini: basta ridurre la sua matrice a gradini.

In un sistema a gradini le variabili che corrispondono ai pivot sono dette \textbf{variabili vincolate}, mentre le altre variabili sono dette \textbf{variabili libere}.
La soluzione del sistema si trova sostituendo i vari coefficienti con la matrice completa. In particolare, se l'ultimo pivot è nell'ultima colonna, il sistema non ha soluzione.


\section{Spazi vettoriali}

\subsection{Vettori liberi}

I vettori fisici sono quantità caratterizzate da un verso, una direzione e un modulo.
Un modo di rappresentare i vettori fisici è dato dall'uso dei segmenti orientati.

\begin{center}
  \begin{tikzpicture}[scale=1.5]
    \draw[-stealth] (0,0) node[label={$P$}] {} -- (3,1) node[label={$Q$}] {};
  \end{tikzpicture}
\end{center}
In un segmento orientato $PQ$:
\begin{itemize}
  \item la lunghezza (ossia la distanza fra $P$ e $Q$) è il modulo del vettore;
  \item la retta passante per i due estremi è la direzione;
  \item l'ordine dei due estremi indica il verso.
\end{itemize}
Il segmento $PP$ rappresenta il vettore nullo.

\begin{definition}[Equipollenza]
  Due segmenti si dicono \textbf{equipollenti} se hanno in comune lunghezza, direzione e verso.
\end{definition}

\begin{definition}[Vettore libero]
  Un \textbf{vettore libero} è la classe di equipollenza di un segmento orientato, ossia l'insieme dei segmenti orientati che sono equipollenti fra loro. Il vettore libero individuato dal segmento orientato $PQ$ si indica con $\vec{PQ}$. In particolare, il vettore nullo è $\vec{0}$.
\end{definition}

\begin{definition}[Legge di Galileo]
  La somma di due vettori liberi è congruente al segmento orientato che congiunge l'origine del primo vettore con la destinazione del secondo vettore:
  $$\vec{PQ}+\vec{QR}\walrus\vec{PR}$$
  \begin{center}
    \begin{tikzpicture}[scale=1.2]
      \draw[-stealth] (0,0) node[label={$P$}] {} -- (2,2) node[label={$Q$}] {};
      \draw[-stealth] (2,2) node[label={$Q$}] {} -- (5,0) node[label={$R$}] {};
      \draw[-stealth] (0,0) -- (5,0);
    \end{tikzpicture}
  \end{center}
\end{definition}

\begin{definition}[Prodotto scalare]
  Il prodotto di un vettore libero $\vec{v}$ per uno scalare $t\in\reals$ è un vettore libero che ha:
  \begin{itemize}
    \item modulo $t\dabs{\vec{v}}$;
    \item stessa direzione;
    \item stesso verso se $t>0$, verso opposto se $t<0$.
  \end{itemize}
  \begin{center}
    \begin{tikzpicture}[scale=1.2]
      \draw[-stealth] (0,0) -- (2,0) node[midway,above] {$\vec{v}$};
      \draw[-stealth] (0,-1) -- (4,-1) node[midway,above] {$2\vec{v}$};
      \draw[-stealth] (0,-2) -- (-1,-2) node[midway,above] {$-\frac{1}{2}\vec{v}$};
    \end{tikzpicture}
  \end{center}
\end{definition}

\begin{theorem}[Regola del parallelogramma]
  La somma di due vettori liberi $\vec{v}$ e $\vec{w}$ è la diagonale del parallelogramma che ha per lati i due vettori.
  \begin{center}
    \begin{tikzpicture}[scale=1.2]
      \draw[-stealth] (0,0) node[label={$P$}] {} -- (1,2) node[label={$Q$}] {};
      \draw[-stealth] (0,0) -- (4,0) node[label={$S$}] {};
      \draw[-stealth] (0,0) -- (5,2) node[label={$R$}] {};
      \draw[dashed] (4,0) -- (5,2);
      \draw[dashed] (1,2) -- (5,2);
    \end{tikzpicture}
  \end{center}
  $$\vec{PQ}+\vec{PS}=\vec{PR}$$
\end{theorem}
\begin{proof}
  Per la legge di Galileo, $\vec{PQ}+\vec{QR}=\vec{PR}$ e $\vec{PS}+\vec{SR}=\vec{PR}$. Poiché i lati opposti di un parallelogramma sono congruenti ($\vec{PQ}\cong\vec{SR}$ e $\vec{PS}\cong\vec{QR}$), si ha:
  $$\vec{PQ}+\vec{PS}=\vec{PR}$$
\end{proof}

\subsubsection*{Proprietà}

Le proprietà di cui godono i vettori liberi sono analoghe a quelle dei vettori, per cui:
\begin{itemize}
  \item $(\vec{v}+\vec{w})+\vec{u}=\vec{v}+(\vec{w}+\vec{u})$
  \item $\vec{v}+\vec{w}=\vec{w}+\vec{v}$
  \item $\vec{v}+\vec{0}=\vec{v}$
  \item $\vec{v}+(-1)\vec{v}=\vec{0}$
  \item $t(\vec{v}+\vec{w})=t\vec{v}+t\vec{w}$
  \item $(t+s)\vec{v}=t\vec{v}+s\vec{v}$
  \item $(ts)\vec{v}=t(s\vec{v})$
  \item $1\vec{v}=\vec{v}$
\end{itemize}

\begin{lemma}
  Sia $\vec{PQ}$ un qualsiasi vettore libero.
  $$\vec{QP}=-\vec{PQ}$$
\end{lemma}
\begin{proof}
  $$\vec{PQ}+\vec{QP}=\vec{PP}=\vec{0}\impl\vec{QP}=-\vec{PQ}$$
\end{proof}

\subsection{Sistema di assi cartesiano}

\subsubsection*{Nel piano}
Dare un sistema di assi cartesiani nel piano equivale a fissare un punto $O$ e i vettori $\vec{OU_1}$ e $\vec{OU_2}$, ortogonali fra loro, detti \textbf{versori degli assi cartesiani} e indicati rispettivamente con $\vec{i}$ e $\vec{j}$.

Il sistema di assi cartesiani individuato da $O$ e $\vec{i}$ e $\vec{j}$ si indica con $$S\walrus\(O,\left\{ \vec{i},\vec{j} \right\}\)$$

Se $\vec{v}$ è un vettore, allora si può scrivere $\vec{v}=\vec{OP}$. La coppia $\(x,y\)$ delle \textbf{coordinate} di $P$ è detta 2--vettore delle coordinate di $\vec{v}$ e si nota che $\vec{v}=x\vec{i}+y\vec{j}$.

Questo ha delle conseguenze non trascurabili.
Se $\(x_1,y_1\)$ è il 2--vettore delle coordinate di $\vec{v_1}$ e $\(x_2,y_2\)$ è il 2--vettore delle coordinate di $\vec{v_2}$, allora il 2--vettore delle coordinate di $\vec{v_1}+\vec{v_2}$ è $\(x_1+x_2,y_1+y_2\)$.
Se $\(x,y\)$ è il 2--vettore delle coordinate di $\vec{v}$, allora $\(tx,ty\)$ è il 2--vettore delle coordinate di $t\vec{v}$.

Siano $a$ il 2--vettore delle coordinate del vettore libero $\vec{v}$ e $b$ il 2--vettore delle coordinate del vettore libero $\vec{w}$. Allora, $\vec{v}+\vec{w}=a+b$ e $t\vec{v}=ta$.

\subsubsection*{Nello spazio}
Quanto visto per i vettori nel piano vale anche per i vettori nello spazio:
\begin{itemize}
  \item si hanno tre versori $\vec{i}$, $\vec{j}$ e $\vec{k}$;
  \item il 3--vettore delle coordinate di $\vec{v}=\vec{OP}$ è la tripla $\(x,y,z\)$ delle coordinate di $P$;
  \item si ha che $\vec{v}=x\vec{i}+y\vec{j}+z\vec{k}$.
\end{itemize}

\subsection{Spazi vettoriali}

\begin{definition}[Spazio vettoriale]
  Uno \textbf{spazio vettoriale} è un insieme $V$ su cui è definita:
  \begin{itemize}
    \item un'operazione di \textbf{somma} che associa a due elementi $v,w$ un elemento $v+w\in V$;
    \item un'operazione di \textbf{prodotto scalare} che associa ad un elemento $v$ un elemento $tv\in V$.
  \end{itemize}
\end{definition}
Inoltre, devono necessariamente valere le seguenti proprietà:
\begin{itemize}
  \item $(v+w)+u=v+(w+u)$
  \item $v+w=w+v$
  \item $v+0=v$
  \item $v+(-1)v=0$
  \item $t(v+w)=tv+tw$
  \item $(t+s)v=tv+sv$
  \item $(ts)v=t(sv)$
  \item $1v=v$
\end{itemize}

L'insieme di tutti gli $n$--vettori si indica con $\reals^n$.
L'insieme di tutte le matrici $n\times m$ si indica con $\reals^{n\times m}$.
L'insieme dei polinomi si indica con $\reals\left[ x \right]$.
L'insieme dei polinomi di grado minore o uguale a $n$ si indica con $\reals_n\left[ x \right]$.

\begin{definition}[Sottospazio]
  Un \textbf{sottospazio} di uno spazio vettoriale $V$ è un sottoinsieme non vuoto $W$ tale che $v,w\in W\impl v+w\in W\wedge v\in W,t\in \reals\impl tv\in W$.
\end{definition}

\begin{example}
  $R_n\left[ x \right]$ è un sottospazio di $R\left[ x \right]$.
\end{example}

\begin{example}
  Se $V$ è uno spazio vettoriale allora $V$ è un sottospazio di se stesso.
\end{example}

\begin{example}
  Se $AX=0$ è un sistema omogeneo di $n$ equazioni e $m$ incognite, allora l'insieme delle soluzioni $\sol \(A,0\)$ è un sottospazio di $\reals^m$, per la legge di sovrapposizione.
  \begin{proof}
    L'insieme delle soluzioni non è mai vuoto, perché $0$ è una soluzione.
    Per la legge di sovrapposizione:
    $$x_1,x_2 \text{ soluzioni}\impl x_1+x_2 \text{ soluzione}$$
    $$x\text{ soluzione}\impl tx \text{ soluzione}$$
  \end{proof}
\end{example}

\begin{theorem}
  $\vec{0}$ appartiene ad ogni sottospazio.  
\end{theorem}
\begin{proof}
  Se $W$ è un sottospazio, allora non è vuoto. Sia $w\in W$, allora $(-1)w\in W$ e quindi $w+(-1)w=\vec{0}\in W$.
\end{proof}
\begin{corollary}
  Se $\vec{0}$ non appartiene ad un insieme $W$, allora $W$ non può essere un sottospazio.
\end{corollary}

Se $W$ è un sottospazio di $V$, allora con la somma e il prodotto per uno scalare indotte da $V$, $W$ è uno spazio vettoriale. Ne segue che ogni proprietà che vale per gli spazi vettoriali, vale anche per tutti i sottospazi.

\begin{example}
  \begin{itemize}
    \item $\left\{ \(x,y\):2x-y=1 \right\}$ non è un sottospazio in quanto 0 non è una soluzione;
    \item $\left\{ \(x,y\):x^2+y^2=1 \right\}$ non è un sottospazio in quanto 0 non è una soluzione;
    \item $\left\{ \(x,y\):x^2-y^2=0 \right\}$ non è un sottospazio in quanto non è chiuso rispetto alla somma;
    \item $\left\{ \(x,y\):x^2+y^2=0 \right\}$ non è un sottospazio in quanto non è chiuso rispetto al prodotto scalare.
  \end{itemize}
\end{example}

\subsection{Spazi generati}

\begin{definition}[Combinazione lineare]
  In uno spazio vettoriale $V$, si dice che il vettore $v$ è una \textbf{combinazione lineare} dei vettori $v_1,v_2,...,v_k$ se
  $$v=a_1v_1+a_2v_2+\cdots+a_kv_k$$
  con $a_1,a_2,...,a_k\in\reals$ detti \textbf{coefficienti della combinazione lineare}.
\end{definition}

\begin{example}
  $\(1,3,-1,4\)$ è combinazione lineare di $\(1,1,0,1\)$, $\(1,0,1,1\)$, $\(0,0,0,1\)$, infatti
  $$\(1,3,-1,4\)=2\(1,1,0,1\)-\(1,0,1,1\)+3\(0,0,0,1\)$$
\end{example}

\begin{definition}[Vettori linearmente dipendenti]
  I vettori $v_1,v_2,...,v_n$ si dicono \textbf{linearmente dipendenti} se esiste una combinazione lineare di essi con coefficienti non tutti nulli.
\end{definition}
\begin{example}
  I vettori $\(1,0,1\), \(1,1,1\), \(-1,1,-1\)$ sono linearmente dipendenti:
  $$\(1,0,1\) -\frac{1}{2}\(1,1,1\) +\frac{1}{2}\(-1,1,-1\)=\(0,0,0\)$$
\end{example}
\begin{lemma}
  Due vettori sono linearmente dipendenti se e solo se uno è un multiplo dell'altro.
\end{lemma}
\begin{proof}
  Siano $v$ e $w$ due vettori. Perché essi siano linearmente dipendenti, si deve verificare:
  $$a,b\in\reals\setminus\left\{ 0 \right\}$$
  $$av+bw=\vec{0}$$
  $$av=-bw\impl v=-\frac{b}{a}w$$
\end{proof}

\begin{definition}[Vettori linearmente indipendenti]
  I vettori $v_1,v_2,...,v_n$ si dicono \textbf{linearmente indipendenti} se non sono linearmente dipendenti, vale a dire esiste una sola combinazione lineare ed è quella nulla.
  Per convenzione $\emptyset$ è linearmente indipendente.
\end{definition}
\begin{example}
  I vettori $\(1,1,1\), \(1,1,0\), \(1,0,0\)$ sono linearmente indipendenti:
  $$a\(1,1,1\) +b\(1,1,0\) +c\(1,0,0\)=\(0,0,0\)$$
  $$\(a,a,a\) +\(b,b,0\) +\(c,0,0\)=\(0,0,0\)$$
  $$
    \(a+b+c,a+b,a\)=\(0,0,0\)
    \iff
    \begin{cases}
      a+b+c=0 \\
      a+b=0   \\
      a=0     \\
    \end{cases}
    \iff
    \begin{cases}
      a=0 \\
      b=0 \\
      c=0 \\
    \end{cases}
  $$
\end{example}

\begin{definition}[Spazio generato]
  È detto \textbf{spazio generato} l'insieme di tutte le combinazioni lineari dei vettori $v_1,v_2,\dots,v_n$ è  e lo si indica con $L\(v_1,v_2,...,v_n\)$, oppure $\sp{v_1,v_2,...,v_n}$ oppure $\mathrm{span}\(v_1,v_2,...,v_n\)$. Per convenzione $\sp{\emptyset} = \left\{ 0 \right\}$.
\end{definition}
Lo spazio generato da $v_1,v_2,...,v_n$ è un sottospazio. Infatti:
\begin{itemize}
  \item non è vuoto: $\vec{0}\in\sp{v_1,v_2,...,v_n}$;
  \item è chiuso rispetto alla somma: $\(a_1v_1+a_2v_2+\cdots+a_nv_n\)+\(b_1v_1+b_2v_2+\cdots+b_nv_n\)=\(a_1+b_1\)v_1+\(a_2+b_2\)v_2+\cdots+\(a_n+b_n\)v_n\in \sp{v_1,v_2,...,v_n}$;
  \item è chiuso rispetto al prodotto per uno scalare: $t\(a_1v_1+a_2v_2+\cdots+a_nv_n\)=ta_1v_1+ta_2v_2+\cdots+ta_nv_n\in \sp{v_1,v_2,...,v_n}$.
\end{itemize}

Lo spazio generato da $v_1,v_2,...,v_n$ è il più piccolo sottospazio che contiene $v_1,v_2,...,v_n$, nel senso che $\sp{v_1,v_2,...,v_n}$ contiene $v_1,v_2,...,v_n$ e se $W$ è un sottospazio che contiene $v_1,v_2,...,v_n$, allora $\sp{v_1,v_2,...,v_n}$ è contenuto in $W$.

\begin{definition}[Insieme di generatori]
  Si dice che $\left\{ v_1,v_2,...,v_n \right\}$ è un \textbf{insieme di generatori} per lo spazio $V$ se $\sp{v_1,v_2,...,v_n}=V$: si dice che $V$ è generato da $v_1,v_2,...,v_n$ oppure che $v_1,v_2,...,v_n$ generano $V$.
\end{definition}

\begin{example}
  $\(1,0,0\),\(0,1,0\),\(0,0,1\)$ generano $\reals^3$, infatti se $\(x,y,z\)$ è un generico 3--vettore, allora:
  $$\(x,y,z\)=x\(1,0,0\)+y\(0,1,0\)+z\(0,0,1\)$$
\end{example}

\begin{definition}[Spazio finitamente generato]
  Uno spazio vettoriale $V$ si dice \textbf{finitamente generato} se esiste un insieme finito $\left\{ v_1,v_2,...,v_n \right\}$ di vettori che generano $V$.
\end{definition}

\begin{definition}[Base]
  Una \textbf{base} di uno spazio vettoriale $V$ finitamente generato è un insieme $\left\{ v_1,v_2,...,v_n \right\}$ di generatori di $V$ linearmente indipendente.
\end{definition}

\begin{theorem}[Teorema della base]
  Sia $V$ uno spazio finitamente generato. Allora:
  \begin{itemize}
    \item $V$ ha una base e tutte le basi di $V$ hanno lo stesso numero di elementi;
    \item ogni insieme linearmente indipendente in $V$ è contenuto in una base di $V$;
    \item ogni insieme di generatori di $V$ contiene una base.
  \end{itemize}
\end{theorem}

\begin{definition}[Dimensione]
  Il numero elementi di una qualsiasi base di uno spazio $V$ è detto \textbf{dimensione} di $V$ e lo si indica con $\dim V$.
\end{definition}

\begin{example}
  $$\dim \reals^3=3$$
  $$\dim \left\{ 0 \right\}=0$$
\end{example}
\paragraph*{Base canonica}
Sia $e_i\walrus\(0,0,\dots,0,0,1,0,0,\dots,0\)\in \reals^n$. È detta \textbf{base canonica} l'insieme $C^n\walrus\left\{ e_1,e_2,\dots,e_n \right\}$.

\begin{example}
  $$V=\sp{\(1,1,0\),\(1,2,-1\)}$$
  $\left\{ \(1,1,0\),\(1,2,-1\) \right\}$ è un insieme di generatori di $\left<\(1,1,0\),\(1,2,-1\)\right>$. 
  $$a\(1,1,0\)+b\(1,2,-1\)=\vec{0}\impl\(a+b,a+2b,-b\)=\vec{0}$$
  $$
    \begin{cases}
      a+b=0  \\
      a+2b=0 \\
      -b=0   \\
    \end{cases}
    \impl
    a=b=0
  $$
  Quindi l'insieme dei generatori $\left\{ \(1,1,0\),\(1,2,-1\) \right\}$ è linearmente indipendente e, pertanto, una base di $\sp{\(1,1,0\),\(1,2,-1\)}$.
\end{example}

\paragraph*{Assenza di una base per uno spazio infinito}
Si considera lo spazio $\reals\left[ n \right]$.
Per il principio di identità dei polinomi (che dice che due polinomi sono uguali se e solo se hanno coefficienti uguali) l'insieme $\left\{ x^0,x^1,x^2,\dots,x^n \right\}$ è linearmente indipendente $\forall n\in\mathbb{N}$.
$\vec{0}$ di $\reals\left[ x \right]$ è il polinomio nullo $P(x)=0$. Poiché $P(n)=\vec{0}\impl a_0=a_1=\cdots=a_n=0$ e poiché si può sempre aggiungere un elemento all'insieme del polinomio, l'insieme è sempre indipendente.
Per assurdo, se $\reals\left[ n \right]$ fosse di dimensione finita e $\dim \reals\left[ n \right]=k$, allora, per il teorema della base, $\left\{ x^0,x^1,x^2,\dots,x^n \right\}$ sarebbe contenuto in una base di e quindi $k\ge n\;\forall n$.


\begin{example}
  È $\left\{\(1,2\),\(2,1\)\right\}$ una base di $\reals^2$?
  
  \noindent Si verifica che l'insieme sia indipendente:
  $$a\(1,2\)+b\(2,1\)=0\impl\(a,2a\)+\(2b,b\)=0$$
  $$
    \begin{cases}
      a+2b=0 \\
      2a+b=0 \\
    \end{cases}
    \impl
    \begin{cases}
      2a=-4b  \\
      -4b+b=0 \\
    \end{cases}
    \impl
    \begin{cases}
      3b=0 \\
      2a=0
    \end{cases}
    \impl
    a=b=0
  $$
  Si verifica che l'insieme generi $\reals^2$:
  $$
    \begin{cases}
      a+2b=x \\
      2a+b=y \\
    \end{cases}
    \impl
    \begin{cases}
      2x-4b=2a  \\
      2x-4b+b=y \\
    \end{cases}
    \impl
    \begin{cases}
      x=a+2b  \\
      y=2x-3b \\
    \end{cases}
  $$
  Quindi l'insieme è un generatore ed è indipendente, il che vuol dire che è una base.
\end{example}

\begin{theorem}
  Sia $V$ uno spazio di dimensione finita. Se $W$ è un sottospazio di $V$ allora anche $W$ è di dimensione finita e $\dim W\le \dim V$.
\end{theorem}

\begin{theorem}
  Sia $V$ uno spazio di dimensione finita. Se $W$ è un sottospazio di $V$ e $\dim W=\dim V$ allora $V=W$.
\end{theorem}
\begin{proof}
  Sia $n=\dim V=\dim W$. Sia $w=\left\{ w_1,w_2,\dots,w_n \right\}$ una base di $W$. Siccome $w$ è linearmente indipendente allora, per il teorema della base, è contenuto in una base $b$ di $V$. Dato che $b$ ha $n$ elementi, si ha che $b=w$. In particolare, $W=\sp{w}=\sp{b}=V$.
\end{proof}

\begin{theorem}
  Sia $V$ uno spazio di dimensione finita. Se $v_1,v_2,\dots,v_k$ sono linearmente indipendenti in $V$, allora $k\le \dim V$.
\end{theorem}
\begin{proof}
  Siccome $v=\left\{ v_1,v_2,\dots,v_k \right\}$ è linearmente indipendente allora, per il teorema della base, è contenuto in una base $b$ di $V$. Si ha che $k\le \dim V$.
\end{proof}

\begin{theorem}
  Sia $V$ uno spazio di dimensione finita. Se $v_1,v_2,\dots,v_k$ generano $V$ allora $\dim V\le k$.
\end{theorem}
\begin{proof}
  Siccome $v=\left\{ v_1,v_2,\dots,v_k \right\}$ genera $V$ allora, per il teorema della base, contiene una base $b$ di $V$. Si ha che $k\ge \dim V$.
\end{proof}

\begin{theorem}
  Sia $V$ uno spazio di dimensione finita. Se $v_1,v_2,\dots,v_k$ sono linearmente indipendenti e $\dim V=k$ allora $v_1,v_2,\dots,v_k$ è una base di $V$.
\end{theorem}
\begin{proof}
  Siccome $v=\left\{ v_1,v_2,\dots,v_k \right\}$ è linearmente indipendente allora, per il teorema della base, è contenuto in una base $b$ di $V$. Dato che $b$ ha $k$ elementi, ne segue che $b=v$.
\end{proof}

\begin{theorem}
  Sia $V$ uno spazio di dimensione finita. Se $v_1,v_2,\dots,v_k$ generano $V$ e $\dim V=k$, allora $\left\{ v_1,v_2,\dots,v_k \right\}$ è una base di V.
\end{theorem}
\begin{proof}
  Siccome $v=\left\{ v_1,v_2,\dots,v_k \right\}$ è generatore di $V$ allora, per il teorema della base, è contiene una base $b$ di $V$. Dato che $b$ ha $k$ elementi, ne segue che $b=v$.
\end{proof}

\paragraph*{Notazione} Sia $A$ una matrice $n\times m$. Si indicano $A_i$ l'$i$--esima riga e $A^i$ l'$i$--esima colonna.

\begin{definition}[Spazio riga e colonna]
  Sia $A$ una matrice $n\times m$.
  Si definisce spazio riga $\sp{A_i}=\sp{A_1,A_2,\dots,A_n}$ di $A$ lo spazio generato dalle righe di $A$ e spazio colonna $\sp{A^i}=\sp{A^1,A^2,\dots,A^m}$ lo spazio generato dalle colonne di $A$.
\end{definition}

\begin{observation}
  Se $A$ è una matrice $n \times m$ e $X\walrus\(x_1,x_2,\dots,x_m\)'$, allora:
  $$AX=\sum_{i=1}^mx_iA^i$$
\end{observation}
\begin{proof}
  $$e_i\walrus\(0,0,\dots,0,0,1,0,0,\dots,0\)'\in \reals^n$$
  $$Ae_i=A^i$$
  $$AX=A\(\sum_{i=1}^mx_ie_i\)=\sum_{i=1}^mx_iAe_i=\sum_{i=1}^mx_iA^i$$
\end{proof}

\begin{observation}
  Se ad una matrice $A$ si applicano le operazioni elementari di riga, allora lo spazio riga non cambia (questo non accade però per lo spazio colonna, che invece cambia).
\end{observation}
\begin{proof}
  Le operazioni elementari di riga sono 3:
  \begin{enumerate}
    \item scambio di righe;
    \item prodotto di una riga per uno scalare;
    \item somma di una riga ad un'altra.
  \end{enumerate}
  Sia $A$ una matrice $n\times m$.
  Lo scambio di righe non produce alcun effetto sullo spazio riga:
  $$\sp{A_1,\dots,A_i,\dots,A_j,\dots,A_n}=\sp{A_1,\dots,A_j,\dots,A_i,\dots,A_n}$$
  Il prodotto di una riga per uno scalare non produce alcun effetto sullo spazio riga:
  $$\sp{A_1,\dots,A_i,\dots,A_n}=\sp{A_1,\dots,tA_i,\dots,A_n}$$
  La somma di una riga ad un'altra non produce alcun effetto sullo spazio riga:
  $$a_1A_1+\cdots+a_iA_i+\cdots+a_j\(tA_i+A_j\)+\cdots+a_nA_n=$$
  $$=a_1A_1+\cdots+\(a_i+a_jt\)A_i+\cdots+a_jA_j+\cdots+a_nA_n$$
  $$\sp{A_1,\dots,A_i,\dots,A_j,\dots,A_n}=\sp{A_1,\dots, A_i,\dots,tA_i+A_j,\dots,A_n}$$
\end{proof}

Quindi, ogni matrice $A$ può essere trasformata in una matrice a gradini $B$ attraverso una serie di operazioni di riga. Se una matrice $B$ è a gradini, allora le sue righe non nulle sono linearmente indipendenti e quindi sono una base del suo spazio riga.

Quindi, se $B$ è una riduzione a gradini di $A$, allora $\sp{A}=\sp{B}$. In particolare le righe non nulle di $B$ sono una base dello spazio riga di $A$ e la sua dimensione è il numero dei pivot di $B$.

\begin{example}
  $$
    \begin{pmatrix}
      1 & 1  & 2 & 2 \\
      2 & 0  & 4 & 2 \\
      1 & -1 & 2 & 0 \\
      0 & 2  & 0 & 2 \\
    \end{pmatrix}
    \sim
    \begin{pmatrix}
      1 & 1  & 2 & 2  \\
      0 & -2 & 0 & -2 \\
      0 & -2 & 0 & -2 \\
      0 & 2  & 0 & 2  \\
    \end{pmatrix}
    \sim
    \begin{pmatrix}
      1 & 1 & 2 & 2 \\
      0 & 1 & 0 & 1 \\
      0 & 0 & 0 & 0 \\
      0 & 0 & 0 & 0 \\
    \end{pmatrix}
  $$
  $$
    \mathfrak{B} =\left\{ \(1,1,2,2\),\(0,1,0,1\) \right\}
  $$
  $$
    \dim\mathfrak{B}=2
  $$
\end{example}

\begin{example}
  $$
    \begin{pmatrix}
      0 & 0 & 2 & 2 & -1 & 1 \\
      1 & 1 & 2 & 1 & 1  & 1 \\
      1 & 1 & 4 & 3 & 0  & 2 \\
    \end{pmatrix}
    \sim
    \begin{pmatrix}
      1 & 1 & 2 & 1 & 1  & 1 \\
      1 & 1 & 4 & 3 & 0  & 2 \\
      0 & 0 & 2 & 2 & -1 & 1 \\
    \end{pmatrix}
    \sim
  $$
  $$
    \sim
    \begin{pmatrix}
      1 & 1 & 2 & 1 & 1  & 1 \\
      0 & 0 & 2 & 2 & -1 & 1 \\
      0 & 0 & 2 & 2 & -1 & 1 \\
    \end{pmatrix}
    \sim
    \begin{pmatrix}
      1 & 1 & 2 & 1 & 1  & 1 \\
      0 & 0 & 2 & 2 & -1 & 1 \\
      0 & 0 & 0 & 0 & 0  & 0 \\
    \end{pmatrix}
  $$
  Siccome la dimesione dello spazio generato è 2, le righe della matrice non sono linearmente indipendenti.
\end{example}
\begin{example}
  $$
    \begin{pmatrix}
      1  & 2 & 2 & 2 \\
      1  & 0 & 1 & 1 \\
      1  & 2 & 3 & 0 \\
      -1 & 1 & 1 & 1 \\
    \end{pmatrix}
    \sim
    \begin{pmatrix}
      1  & 0 & 1 & 1 \\
      1  & 2 & 2 & 2 \\
      1  & 2 & 3 & 0 \\
      -1 & 1 & 1 & 1 \\
    \end{pmatrix}
    \sim
    \begin{pmatrix}
      1 & 0 & 1 & 1  \\
      0 & 2 & 1 & 1  \\
      0 & 2 & 2 & -1 \\
      0 & 1 & 2 & 2  \\
    \end{pmatrix}
    \sim
    \begin{pmatrix}
      1 & 0 & 1 & 1  \\
      0 & 2 & 1 & 1  \\
      0 & 0 & 1 & -2 \\
      0 & 0 & 3 & 3  \\
    \end{pmatrix}
    \sim
    \begin{pmatrix}
      1 & 0 & 1 & 1  \\
      0 & 2 & 1 & 1  \\
      0 & 0 & 1 & -2 \\
      0 & 0 & 0 & 3  \\
    \end{pmatrix}
  $$
  Siccome la dimesione dello spazio generato è 4, le righe della matrice sono linearmente indipendenti.
\end{example}

\subsection{Operazioni}

\begin{definition}[Spazio somma]
  Se $U,W$ sono sottospazi di uno stesso spazio $V$, allora lo \textbf{spazio somma} è:
  $$U+W\walrus \left\{ u+w:u\in U,w\in W \right\}$$
\end{definition}

\begin{definition}[Spazio intersezione]
  Se $U,W$ sono sottospazi di uno stesso spazio $V$, allora lo \textbf{spazio intersezione} è:
  $$U\cap W\walrus \left\{ v:v\in U\wedge v\in W \right\}$$
\end{definition}

\begin{observation}
  Se $U,W$ sono sottospazi di uno stesso spazio $V$, allora $U+W$ e $U\cap W$ sono sottospazi di $V$.
\end{observation}
\begin{proof}
  \hfill\break
  \noindent Spazio somma $U+W$:
  \begin{itemize}
    \item $0\in U+W$
    \item $v,z\in U+W\impl v+z\in U+W \because v+z=(u_1+w_1)+(u_2+w_2)=(u_1+u_2)+(w_1+w_2)$
    \item $v\in U+W\impl kv\in U+W\because kv=k(u+w)=ku+kw$
  \end{itemize}
  
  \noindent Spazio intersezione $U\cap W$:
  \begin{itemize}
    \item $0\in U\cap W$
    \item $v,z\in U\cap W\impl v+z\in U\cap W \because v+z=(u_1+w_1)+(u_2+w_2)=(u_1+u_2)+(w_1+w_2)$
    \item $v\in U\cap W\impl kv\in U\cap W\because kv=k(u+w)=ku+kw$
  \end{itemize}
\end{proof}

\begin{theorem}[Formula di Grassmann]
  $$\dim\(U+W\)+\dim \(U\cap W\)=\dim U+\dim W$$
\end{theorem}

\begin{definition}[Somma diretta]
  Si dice che uno spazio $V$ è \textbf{somma diretta} di due sottospazi $U,W$ se $V=U+W$ e $U\cap W=\left\{ 0 \right\}$. Si indica come:
  $$V=U \oplus W$$
\end{definition}

\begin{definition}[Complemento]
  Se $U$ è un sottospazio di $V$, un \textbf{complemento} di $U$ è un sottospazio $W$ tale che $V=U\oplus W$.
\end{definition}

\begin{example}
  $$U\walrus \left\{ (x,y):x=y \right\}$$
  $$W\walrus \left\{ (x,y):x=-y \right\}$$
  $W$ è complemento di $U$ in $\reals^2$.
\end{example}

\paragraph*{Osservazione} 
$V=U\oplus W$ se e solo se ogni vettore $v\in V$ può essere scritto univocamente come $v=u+w, u\in U, w\in W$.
\begin{proof}
  Sia $V=U\oplus W$, allora $v\in V$ può essere scritto come:
  $$v\in V=u+w, u\in U, w\in W$$
  Se, per assurdo, si suppone che tale scrittura non sia univoca, allora:
  $$v=u'+w'\impl u'+w'=u+w\impl u'-u=w'-w$$
  $$u'-u\in U,\ w'-w\in W$$
  $$u'-u\in U\cap W\impl u'-u=0\impl u=u'$$
  $$w'-w\in U\cap W\impl w'-w=0\impl w=w'$$
  
  Se $v\in V$ si scrive in modo univoco come $u+w,\ u\in U,\ w\in W$, allora
  $$V=U+W$$
  Inoltre, se $v\in U\cap W$:
  $$v\in U\wedge v\in W\impl v=0\impl V=U\oplus W$$
\end{proof}

\begin{definition}[Proiezione]
  Sia $V=U\oplus W$, dato $v\in V$, possiamo scrivere in modo univoco $v=u+w,\ u\in U,\ w\in W$.
  Il vettore $u$ è detto \textbf{proiezione} di $v$ su $U$, relativa alla somma diretta $V=U\oplus W$.
\end{definition}

\subsection{Rango e nullità}

\begin{definition}[Spazio nullo]
  Lo \textbf{spazio nullo} di una matrice $A$ è $\sol \(A,0\)$ e si indica con $N\(A\)$.
\end{definition}

\begin{definition}[Immagine]
  L'\textbf{immagine} di una matrice $A$ è lo spazio colonna della matrice e lo si indica con $R\(A\)$.
\end{definition}

\begin{definition}[Nullità]
  La dimensione di $N(A)$ è detta \textbf{nullità} di A e la si indica con $null(A)$:
  $$\nul A=\dim N(A)=\dim \sol \(A,0\)$$
\end{definition}

\begin{definition}[Rango]
  La dimensione di $R(A)$ è detta \textbf{rango} di $A$ e la si indica con $\rk A$, quindi se $A$ è una matrice $n\times m$:
  $$\rk A=\dim R(A)=\dim\sp{A^1,A^2,\dots,A^m}$$
  Si nota che $\rk A\le \min \left\{ n,m \right\}$.
\end{definition}

\begin{theorem}[Teorema del rango]
  Il rango di una matrice $A$ è uguale al rango di $A'$:
  $$\rk A=\rk A'$$
\end{theorem}
\begin{proof}
  Sia $A$ una matrice $n\times m$ e sia $r\walrus rk(A)$. Sia $B$ la base di $\sp{A^1,A^2,\dots,A^m}$ e $C$ la matrice avente per colonne gli elementi di $B$. Allora si ha che ogni colonna di $A$ è una combinazione lineare delle colonne di $C$:
  $$A^j=\sum_{i=1}^n k_{ij}C^i$$ % FIXME: forse l'indice deve arrivare ad m
  
  Sia $K=\(k_{ij}\)$. Siccome $A^j=CK^j$, si ha che $A=CK$ e quindi $A'=K'C'$. In particolare, le colonne di $A'$ sono combinazioni lineari delle colonne di $K'$ e quindi $\sp{A'^i}\subseteq \sp{K'^i}$.
  Pertanto $\rk A'\le \rk K'$ e dato che $K$ è $r\times m$, $\rk K'\le r$ e quindi $\rk A'\le r = \rk A$.
  Analogamente, scambiando i ruoli, si ha che $\rk A \le \rk A'$. Si conclude che $\rk A = \rk A'$.
\end{proof}

\paragraph*{Conseguenze}
Sia $r\walrus \rk A$:
\begin{itemize}
  \item la dimensione dello spazio riga è $r$;
  \item la dimesione dello spazio colonna è $r$;
  \item il massimo numero di colonne linearmente indipendenti è $r$;
  \item il massimo numero di righe linearmente indipendenti è $r$.
\end{itemize}

\begin{example}
  $$
    A=
    \begin{pmatrix}
      1 & 1  & 2 & 2 \\
      2 & 0  & 4 & 2 \\
      1 & -1 & 2 & 0 \\
      0 & 2  & 0 & 2 \\
    \end{pmatrix}
    \sim
    \begin{pmatrix}
      1 & 1  & 2 & 2  \\
      0 & -2 & 0 & -2 \\
      0 & -2 & 0 & -2 \\
      0 & 2  & 0 & 2  \\
    \end{pmatrix}
    \sim
    \begin{pmatrix}
      1 & 1 & 2 & 2 \\
      0 & 1 & 0 & 1 \\
      0 & 0 & 0 & 0 \\
      0 & 0 & 0 & 0 \\
    \end{pmatrix}
  $$
  $$\rk A = 2$$
\end{example}

\paragraph*{Metodo dei perni}
Il metodo dei perni consente di calcolare una base di uno spazio, partendo dalla matrice ridotta dei vettori colonna che generano tale spazio. Esso consiste nel trasformare la matrice dei vettori colonna che generano uno spazio in una matrice a gradini. La base è formata dai vettori originali nelle colonne che contengono un pivot.

\begin{theorem}[Nullità più rango]
  Se $A$ è una matrice $n\times m$, allora:
  $$\nul A+\rk A=m$$
  Il teorema è un caso speciale del teorema della dimensione.
\end{theorem}

\begin{example}
  $$
    A=
    \begin{pmatrix}
      1 & 1  & 2 & 2 \\
      2 & 0  & 4 & 2 \\
      1 & -1 & 2 & 0 \\
      0 & 2  & 0 & 2 \\
    \end{pmatrix}
    \sim
    \begin{pmatrix}
      1 & 1  & 2 & 2  \\
      0 & -2 & 0 & -2 \\
      0 & -2 & 0 & -2 \\
      0 & 2  & 0 & 2  \\
    \end{pmatrix}
    \sim
    \begin{pmatrix}
      1 & 1 & 2 & 2 \\
      0 & 1 & 0 & 1 \\
      0 & 0 & 0 & 0 \\
      0 & 0 & 0 & 0 \\
    \end{pmatrix}
  $$
  $$\rk A = 2$$
  $$\nul A=m-\rk A=4-2=2$$
\end{example}
\begin{example}
  $$
    A=
    \begin{pmatrix}
      1 & 1  & 2 & 2 \\
      2 & 0  & 4 & 2 \\
      1 & -1 & 2 & 0 \\
      0 & 2  & 0 & 2 \\
    \end{pmatrix}
  $$
  $$
    \begin{pmatrix}
      A & 0 \\
    \end{pmatrix}
    =
    \begin{pmatrix}
      1 & 1  & 2 & 2 & 0 \\
      2 & 0  & 4 & 2 & 0 \\
      1 & -1 & 2 & 0 & 0 \\
      0 & 2  & 0 & 2 & 0 \\
    \end{pmatrix}
    \sim
    \begin{pmatrix}
      1 & 1 & 2 & 2 & 0 \\
      0 & 1 & 0 & 1 & 0 \\
      0 & 0 & 0 & 0 & 0 \\
      0 & 0 & 0 & 0 & 0 \\
    \end{pmatrix}
  $$
  $$\rk A=2\impl \nul A = 4-2=2$$
  $$
    \begin{cases}
      x_1+x_2+2x_3+2x_4=0 \\
      x_2+x_4=0           \\
    \end{cases}
  $$
  Dato che i pivot sono $x_1$ e $x_2$, allora essi sono anche le variabili vincolate. Viceversa $x_3$ e $x_4$ sono libere.
  $$
    \begin{cases}
      x_1 +x_2=-2x_3-2x_4 \\
      x_2=-x_4            \\
    \end{cases}
    \impl
    \begin{cases}
      x_1=-2x_3-x_4 \\
      x_2=-x_4      \\
    \end{cases}
  $$
  $$
    \begin{pmatrix}
      x_1 \\
      x_2 \\
      x_3 \\
      x_4 \\
    \end{pmatrix}
    =
    \begin{pmatrix}
      -2x_3-x_4 \\
      -x_4      \\
      x_3       \\
      x_4       \\
    \end{pmatrix}
    =
    x_3
    \begin{pmatrix}
      -2 \\
      0  \\
      1  \\
      0  \\
    \end{pmatrix}
    +x_4
    \begin{pmatrix}
      -1 \\
      -1 \\
      0  \\
      1  \\
    \end{pmatrix}
  $$
  $\left\{ \(-2,0,1,0\),\(-1,-1,0,1\) \right\}$ genera le soluzioni del sistema.
  Dato che $\nul A=2$, l'insieme è una base.
  $N(A)$ è, quindi, $\sp{\(-2,0,1,0\),\(-1,-1,0,1\)}$.
\end{example}

\section{Trasformazioni lineari}

\subsection{Funzioni}

\begin{definition}[Funzione]
  Dati due insiemi $A$ e $B$, una \textbf{funzione} $f: A\to B$ è una legge che associa a ogni elemento di $A$ uno ed un solo elemento di $B$ che si indica con $f(a)$.
\end{definition}
\begin{definition}[Funzione iniettiva, suriettiva e biiettiva]
  Una funzione si dice \textbf{iniettiva} se $f(a_1)=f(a_2)\impl a_1=a_2$. Una funzione si dice \textbf{suriettiva} se $\forall b\in B\exists a\in A:f(a)=b$. Una funzione si dice \textbf{biiettiva} se è iniettiva e suriettiva.
\end{definition}
\begin{definition}[Funzione identità]
  La \textbf{funzione identità} $I_A:A\to A$ è definita come $I_A\(a\)\walrus a$.
\end{definition}
\begin{definition}[Funzione composta]
  Dati gli insiemi $A$, $B$, $C$ e le funzioni $f:A\to B$ e $g:B\to C$, la \textbf{funzione composta} $g\circ f$ è definita come:
  $$g\circ f:A\to C\walrus g(f(a))$$
\end{definition}
\begin{definition}[Funzione invertibile]
  Una funzione $f:A\to B$ si dice \textbf{invertibile} se esiste una funzione $g:B\to A$, tale che
  $$g\circ f = I_A$$
  $$f\circ g = I_B$$
  Se $f$ è una funzione invertibile, allora la funzione $g$ è unica e la si indica con $f^{-1}$.
\end{definition}
\begin{lemma}
  Una funzione è invertibile se e solo se è biiettiva.
\end{lemma}

\subsection{Trasformazioni lineari}

\begin{definition}[Trasformazione lineare]
  Dati due spazi vettoriali $V$ e $W$, una \textbf{trasformazione lineare} da $V$ in $W$ è una funzione $f:V\to W$ tale che, dati $v_1,v_2\in V$:
  $$f(v_1+v_2)=f(v_1)+f(v_2)$$
  e dato $v\in V$ e $t\in\reals$:
  $$f(tv)=tf(v)$$
\end{definition}

\begin{observation}
  Se $T:V\to W$ è una trasformazione lineare, allora $T(0)=0$.
\end{observation}
\begin{proof}
  $$T(0)=T(0+0)=T(0)+T(0)\impl T(0)=0$$
\end{proof}

Se $A$ è una matrice $n\times m$, la funzione $T_A:R^m\to R^n$, definita come $T_A(X)\walrus AX$, è una trasformazione lineare.
\begin{proof}
  $$T_A(X_1+X_2)=A\(X_1+X_2\)=AX_1+AX_2=T_A(X_1)+T_A(X_2)$$
  $$T_A(tX)=A\(tX\)=tAX=tT_A(X)$$
\end{proof}

\begin{example}
  $$
    A\walrus
    \begin{pmatrix}
      0 & 1 \\
      1 & 0 \\
    \end{pmatrix}
  $$
  $$
    T_A
    \begin{pmatrix}
      x \\
      y \\  
    \end{pmatrix}
    =
    \begin{pmatrix}
      y \\
      x \\
    \end{pmatrix}
  $$
  $$
    A\walrus
    \begin{pmatrix}
      1 & 0 & 1 \\
      0 & 1 & 1 \\
    \end{pmatrix}
  $$
  $$
    T_A
    \begin{pmatrix}
      x \\
      y \\
      z \\
    \end{pmatrix}
    =
    \begin{pmatrix}
      x+z \\
      y+z \\
    \end{pmatrix}
  $$
\end{example}

\begin{theorem}
  Se $T:R^m\to R^n$ è una trasformazione lineare, sia $v_i=T(e_i)$ e sia $A\(n\times m\)$, tale che $A\walrus\(v_1,v_2,\dots,v_m\)$.
  Allora, $T=T_A$.
\end{theorem}
\begin{proof}
  $$
    T\(X\)=T
    \begin{pmatrix}
      x_1    \\
      x_2    \\
      \vdots \\
      v_m    \\
    \end{pmatrix}
    =\sum_{i=1}^mx_iT\(e_i\)=\sum_{i=1}^mx_iv_i=AX
  $$
\end{proof}

La matrice $A$ è chiamata matrice associata a $T$, o rappresentativa di $T$, nelle basi canoniche.
\begin{definition}[Nucleo]
  Se $T:V\to W$ è una trasformazione lineare, l'insieme 
  $$\ker T\walrus\left\{ v\in V:T(v)=\vec{0} \right\}$$
  è detto \textbf{nucleo} di $T$.
\end{definition}

\begin{definition}[Immagine]
  Se $T:V\to W$ è una trasformazione lineare, l'insieme
  $$\im T\walrus\left\{ w\in W:\exists v\in V:w=T(v) \right\}$$
  è detto \textbf{immagine} di $T$.
\end{definition}

\begin{lemma}
  Il nucleo e l'immagine di una trasformazione lineare $T:V\to W$ sono sottospazi vettoriali rispettivamente di $V$ e $W$.
\end{lemma}
\begin{proof}
  $$T(0)=0\impl 0 \in \ker T$$
  $$v,w\in \ker T\impl T(v+w)=T(v)+T(w)=0+0=0\in \ker T$$
  $$v\in \ker T, t\in \reals\impl T(tv)=tT(v)=t0=0\in \ker T$$
\end{proof}

\begin{observation}
  $$\ker T_A=N(A)$$
  $$\dim \ker T_A=\nul A$$
\end{observation}
\begin{proof}
  $$\ker T_A=\left\{ v:T_A(v)=0 \right\}=\left\{ v:Av=0 \right\}=N(A)$$
\end{proof}

\begin{observation}
  $$\im T_A=R(A)$$
  $$\dim \im T_A=\rk A$$
\end{observation}
\begin{proof}
  $$\im T_A=\left\{ T_A(v):v\in\reals^n \right\}=\left\{ Av:v\in\reals^n \right\}=\left\{ \sum_{i=0}^mx_iA^i \right\}=R\(A\)$$
\end{proof}

\begin{lemma}
  $T:V\to W$ è iniettiva se e solo se $\ker T=\left\{ 0 \right\}$.
\end{lemma}
\begin{proof}
  Sia $T$ iniettiva. Sia $v\in \ker T$.
  Allora $T(v)=0$. Sapendo che $T(0)=0$, allora $v=0$. Pertanto, $\ker T = \left\{ 0 \right\}$.
  
  Sia $\ker T=\left\{ 0 \right\}$. Allora:
  $$T(v)=T(w)\impl T(v)-T(w)=0$$
  Per la linearità delle trasformazioni lineari:
  $$T(v-w)=0\impl v-w\in \ker T\impl v-w=0\impl v=w$$
  Pertanto, $T$ è iniettiva.
\end{proof}

\begin{lemma}
  $T:V\to W$ è suriettiva se e solo se $\im T=W$.
\end{lemma}
\begin{proof}
  Sia $T$ suriettiva. Sia $w\in W$. Allora:
  $$\exists v\in V:T\(v\)=w\impl w\in \im T\impl W=\im T$$
  
  Sia $\im T=W$. Sia $w\in W$. Allora:
  $$w\in \im T\impl \exists v\in V:T\(v\)=w$$
  Pertanto, $T$ è suriettiva.
\end{proof}

\begin{theorem}[Teorema della dimensione]
  Se $T:V\to W$ è una trasformazione lineare e $V$ è di dimensione finita, allora $\im T$ è di dimensione finita e, in particolare:
  $$\dim V=\dim \ker T+\dim \im T$$
\end{theorem}
\begin{proof}
  Sia $v\walrus\(v_1,\dots,v_k\)$ una base di $\ker T$.
  La si completa ad una base di $V$ $\left\{ v_1,\dots,v_k,v_{k+1},\dots,v_n \right\}$.
  $\left\{ T(v_{k+1}),\dots,T(v_n) \right\}$ è linearmente indipendente se:
  $$x_1T(v_{k+1})+x_2T(v_{k+2})+\cdots+x_{n-k}T(v_n)=0$$
  $$T(x_1v_{k+1})+T(x_2v_{k+2})+\cdots+T(x_{n-k}v_n)=0$$
  $$T(x_1v_{k+1}+x_2v_{k+2}\cdots+x_{n-k}v_n)=0$$
  $$x_1v_{k+1}+x_2v_{k+2}\cdots+x_{n-k}v_n\in \ker T$$
  Ne segue che:
  $$x_1v_{k+1}+x_2v_{k+2}\cdots+x_{n-k}v_n=y_1v_1+y_2v_2+\cdots+y_kv_k$$
  $$x_1v_{k+1}+x_2v_{k+2}\cdots+x_{n-k}v_n-y_1v_1-y_2v_2-\cdots-y_kv_k=0$$
  Dato che $\left\{ v_1,\dots,v_k,v_{k+1},\dots,v_n \right\}$ è linearmente indipendente, allora $x_1=x_2=x_{n-k}=0$.
  Inoltre, $\left\{ T(v_{k+1}),\dots,T(v_n) \right\}$ genera $\im T$, infatti, se $T\(v\)\in \im T$ e:
  $$v=y_1v_1+y_2v_2+\cdots+y_kv_k+x_1v_{k+1}+x_2v_{k+2}\cdots+x_{n-k}v_n$$
  si ha che:
  \begin{align*}
    T(v) & = y_1T(v_1)+y_2T(v_2)+\cdots+y_kT(v_k)+x_1T(v_{k+1})+x_2T(v_{k+2})\cdots+x_{n-k}T(v_n) \\
         & =x_1T(v_{k+1})+x_2T(v_{k+2})\cdots+x_{n-k}T(v_n)                                       
  \end{align*}
  Pertanto, $\left\{ T(v_{k+1}),\dots,T(v_n) \right\}$ è una base di $\im T$.
  Infine, essendo una base, si ha che:
  $$\dim \im T=n-k=\dim V-\dim \ker T$$
  $$\dim V=\dim \ker T+\dim \im T$$
\end{proof}

\begin{corollary}
  Se $A$ è una matrice $n\times m$, si ha che:
  $$m=\nul A + \rk A$$
\end{corollary}
\begin{proof}
  Per il teorema della dimesione
  $$\dim \im T_A+\dim \ker T_A = \dim \reals^m$$
  $$
    \begin{cases}
      \dim \im T_A = \rk A \\
      \dim \ker T_A=\nul A \\
      \dim \reals^m=m      \\
    \end{cases}
    \impl m=\nul A + \rk A
  $$
\end{proof}

\begin{corollary}
  Siano $V$ e $W$ spazi vettoriali di dimesione finita e $T:V\to W$ una trasformazione lineare.
  Se $T$ è iniettiva, allora $\dim V\le \dim W$.
  Se $T$ è suriettiva, allora $\dim V\ge \dim W$.
\end{corollary}
\begin{proof}
  \hfill\break
  \noindent
  Sia $T$ iniettiva:
  $$\dim V=\dim \ker T+\dim \im T=0+\dim \im T\le \dim W$$
  Sia $T$ suriettiva:
  $$\dim V=\dim \ker T+\dim \im T=\dim \ker T + \dim W \ge \dim W$$
\end{proof}

\begin{corollary}
  Siano $V$ e $W$ spazi vettoriali di dimesione finita e $T:V\to W$ una trasformazione lineare.
  Se $\dim V=\dim W$, allora $T$ è iniettiva se e solo se è biiettiva.
\end{corollary}
\begin{proof}
  Sia $T$ iniettiva:
  $$\dim V = \dim \ker T+\dim \im T=0+\dim \im T=\dim W$$
  $$\im T\subseteq W\impl \im T = W$$
  Pertanto, $T$ è anche suriettiva, e di conseguenza biiettiva.
  Se $T$ biiettiva, allora è per definizione iniettiva.
\end{proof}

\begin{corollary}
  Siano $V$ e $W$ spazi vettoriali di dimesione finita e $T:V\to W$ una trasformazione lineare.
  Se $\dim V=\dim W$, allora $T$ è suriettiva se e solo se è biiettiva.
\end{corollary}
\begin{proof}
  Sia $T$ suriettiva:
  $$\dim V = \dim \ker T+\dim \im T=\dim \ker T+\dim W$$
  $$\dim V=\dim \ker T + \dim V\impl \dim \ker T=0\impl \ker T=\left\{ 0 \right\}$$
  Pertanto, $T$ è anche iniettiva, e di conseguenza biiettiva.
  Se $T$ biiettiva, allora è per definizione suriettiva.
\end{proof}

\begin{corollary}
  Siano $V$ e $W$ spazi vettoriali di dimesione finita e $T:V\to W$ una trasformazione lineare.
  Se $T$ è biiettiva, allora $\dim V=\dim W$.
\end{corollary}
\begin{proof}
  Dato che $T$ è iniettiva, $\dim V\le \dim W$. Dato che $T$ è suriettiva, $\dim V\ge \dim W$.
  Pertanto, $\dim V=\dim W$.
\end{proof}

\begin{lemma}
  Se $T:V\to W$, $S:W\to U$ sono trasformazioni lineari, allora $S\circ T:T\to U$ è una trasformazione lineare.
\end{lemma}
\begin{proof}
  \begin{align*}
    S\circ T\(v_1+v_2\) & =S(T(v_1+v_2))               \\
                        & =S(T(v_1)+T(v_2))            \\
                        & =S(T(v_1))+S(T(v_2))         \\
                        & =S\circ T(v_1)+S\circ T(v_2) 
  \end{align*}
  $$S\circ T(tv)=S(T(tv))=S(tT(v))=tS(T(v))=t\(S\circ T(v)\)$$
\end{proof}

\begin{theorem}
  Sia A una matrice $n\times m$ e B una matrice $m\times k$, allora $T_A\circ T_B=T_{AB}$
\end{theorem}
\begin{proof}
  $$T_A:\reals^m\to\reals^n$$
  $$T_B:\reals^k\to\reals^m$$
  $$T_A\circ T_B:\reals^k\to\reals^n$$
  $$T_A\circ T_B\(X\)=T_A(T_B(X))=ABX=T_{AB}(X)\impl T_A\circ T_B = T_{AB}$$
\end{proof}

\begin{lemma}
  Se $T:V\to W$ è una trasformazione lineare biiettiva, allora $T^{-1}:W\to V$ è lineare.
\end{lemma}
\begin{proof}
  $$v_1\walrus T^{-1}(w_1)\impl T(v_1)=w_1$$
  $$v_2\walrus T^{-1}(w_2)\impl T(v_2)=w_2$$
  $$T^{-1}\(w_1+w_2\)=T^{-1}(T(v_1)+T(v_2))=T^{-1}(T(v_1+v_2))=v_1+v_2=T^{-1}(w_1)+T^{-1}(w_2)$$
  $$v\walrus T^{-1}(w)\impl T(v)=w$$
  $$T^{-1}(tw)=T^{-1}(tT(v))=T^{-1}(T(tv))=tv=tT^{-1}(w)$$
\end{proof}

\begin{lemma}
  Sia $A$ una matrice $n\times m$, allora $T_A$ è invertibile se e solo se $n=m$ e $A$ è invertibile. Inoltre:
  $$T_A^{-1}=T_{A^{-1}}$$
\end{lemma}
\begin{proof}
  $$T_A:\reals^m\to\reals^n$$
  $$T_A^{-1}:\reals^n\to\reals^m$$
  Per il teorema della dimensione, $n=m$.
  $$T_A:\reals^n\to\reals^n$$
  $$T_A^{-1}:\reals^n\to\reals^n$$
  $$T_A^{-1}=T_B$$
  $$I_{\reals^n}:\reals^n\to\reals^n=T_C$$
  $$T_C(X)=CX=X\iff C=I$$
  $$T_A\circ T_B=I_{\reals^n}=T_{I_n}\impl T_{AB}=T_{I_n}\impl AB=I_n\impl B=A^{-1}$$
  $$T_B\circ T_A=I_{\reals^n}=T_{I_n}\impl T_{BA}=T_{I_n}\impl BA=I_n\impl B=A^{-1}$$
\end{proof}

\begin{theorem}
  Se $A$ è una matrice quadrata di ordine $n$, allora:
  $$\rk A=n\iff \det A\neq 0$$
  In particolare, $v_1,v_2,\dots,v_n$ in $\reals^n$ formano una base di $\reals^n$ se e solo se il determinante della matrice che ha per righe (o per colonne) i vettori dati è diverso da 0.
\end{theorem}
\begin{proof}
  $\det A\neq 0$ se e solo se $A$ è invertibile se e solo se $T_A$ è invertibile se e solo se $T_A$ è suriettiva.
  Sia $T_A$ suriettiva:
  $$\im T_A=\reals^n\impl R\(A\)=\reals^n\impl \dim R\(A\)=\rk A = n$$
  Sia $\rk A = n$:
  $$\rk A=n=\dim R(A)=\dim \reals^n\impl \dim \im T_A=\dim \reals^n$$
  $$\im T_A\subseteq\reals^n\impl \im T_A=\reals^n$$
  Pertanto, $T_A$ è suriettiva.
\end{proof}

Se $A$ è una matrice quadrata di ordine $n$, allora sono equivalenti:
\begin{itemize}
  \item $\det A\neq 0$;
  \item $\exists A^{-1}$;
  \item $\rk A=n$;
  \item le righe di $A$ formano una base di $\reals^n$;
  \item le colonne di $A$ formano una base di $\reals^n$.
\end{itemize}

\begin{example}
  $$A\walrus\left\{ \(1,2\),\(2,1\) \right\}$$
  $$\det A = 1-4=-3$$
  $\sp{A}$ è una base di $\reals^2$.
\end{example}

\paragraph*{Metodo dei determinanti minori}

Sia $A$ una matrice $n\times m$. Il rango di $A$ è uguale al massimo $r$ per cui esiste un minore $M$ di ordine $r$ il cui determinante sia diverso da 0. In alternativa, si può dire che il rango di $A$ è il massimo ordine di un determinante minore non nullo.

\begin{example}
  $$
    A\walrus
    \begin{pmatrix}
      1 & 0 & 2  & 1 \\
      1 & 0 & -1 & 2 \\
      2 & 0 & 1  & 3 \\
    \end{pmatrix}
  $$
  $$
    M_1=
    \begin{pmatrix}
      1 \\
    \end{pmatrix}
  $$
  $$\det M_1=1\impl \rk A \ge 1$$
  $$
    M_2=
    \begin{pmatrix}
      2  & 1 \\
      -1 & 2 \\
    \end{pmatrix}
  $$
  $$\det M_2=4+1=5\impl \rk A \ge 2$$
  $$
    M_3=
    \begin{pmatrix}
      1 & 2  & 1 \\
      1 & -1 & 2 \\
      2 & 1  & 3 \\
    \end{pmatrix}
  $$
  $$
    \det M_3=
    \begin{vmatrix}
      -1 & 2 \\
      1  & 3 \\
    \end{vmatrix}
    -
    \begin{vmatrix}
      2 & 1 \\
      1 & 3 \\
    \end{vmatrix}
    +
    2\begin{vmatrix}
      2  & 1 \\
      -1 & 2 \\
    \end{vmatrix}
    =0
    \impl \rk A < 3
  $$
  $$\rk A=2$$
\end{example}

\paragraph*{Metodo degli orlati o di Kronecker}
Se $A$ è una matrice $n\times m$ e $M$ è un minore di $A$ di ordine $r$, un minore orlato di $M$ è un minore di $A$ di ordine $n+1$ che si ottiene aggiungendo a $M$ una riga ed una colonna.
Sia $A$ una matrice $n\times m$. Se $M$ è un minore di $A$ di ordine $r$ tale che $\det M\neq 0$ e ogni minore orlato di $M$ ha determinante nullo, allora $\rk A=r$.

\begin{example}
  $$
    A\walrus
    \begin{pmatrix}
      1 & 0 & 2  & -1 \\
      1 & 1 & -1 & 0  \\
      0 & 1 & 0  & 1  \\
    \end{pmatrix}
  $$
  $$
    M_1=
    \begin{pmatrix}
      1 \\
    \end{pmatrix}
  $$
  $$\det M_1 = 1\impl \rk A\ge 1$$
  $$
    M_2=
    \begin{pmatrix}
      1 & 0 \\
      1 & 1 \\
    \end{pmatrix}
  $$
  $$
    \det M_2 = 1\impl \rk A\ge 2
  $$
  $$
    M_3=
    \begin{pmatrix}
      1 & 0 & 2  \\
      1 & 1 & -1 \\
      0 & 1 & 0  \\
    \end{pmatrix}
  $$
  $$
    \det M_3=
    \begin{vmatrix}
      1 & 2  \\
      1 & -1 \\
    \end{vmatrix}
    =-3\impl \rk A\ge 3
  $$
  $$\rk A = 3$$
\end{example}


\begin{example}
  $$
    A\walrus
    \begin{pmatrix}
      1  & 0 & 1  & 0 \\
      2  & 0 & 2  & 0 \\
      -1 & 0 & -1 & 0 \\
    \end{pmatrix}
  $$
  $$
    M_1=
    \begin{pmatrix}
      1 \\
    \end{pmatrix}
  $$
  $$\det M_1=1\impl \rk A\ge 1$$
  $$
    M_2\in\left\{ 
    \begin{pmatrix}
      1 & 0 \\
      2 & 0 \\
    \end{pmatrix}
    ,
    \begin{pmatrix}
      1  & 0 \\
      -1 & 0 \\
    \end{pmatrix}
    ,
    \begin{pmatrix}
      1 & 1 \\
      2 & 2 \\
    \end{pmatrix}
    ,
    \begin{pmatrix}
      1  & 1  \\
      -1 & -1 \\
    \end{pmatrix}
    ,
    \begin{pmatrix}
      1 & 0 \\
      2 & 0 \\
    \end{pmatrix}
    ,
    \begin{pmatrix}
      1  & 0 \\
      -1 & 0 \\
    \end{pmatrix}
    \right\}
  $$
  $$
    \det M_2=0\impl \rk A<2
  $$
  $$\rk A = 1$$
\end{example}

\begin{example}
  Determinare per quali valori del parametro $k$ la seguente matrice ha rango 2.
  $$
    A\walrus
    \begin{pmatrix}
      k   & k+1  & -1 \\
      k-2 & -k-1 & k  \\
    \end{pmatrix}
  $$
  
  $$M_1=-1\impl \rk A\ge 1$$
  $$
    M_2\in\left\{ 
    \begin{pmatrix}
      k   & -1 \\
      k-2 & k  \\
    \end{pmatrix}
    ,
    \begin{pmatrix}
      k+1  & -1 \\
      -k-1 & k  \\
    \end{pmatrix}
    \right\}
  $$
  $$
    \det M_2\neq 0\impl k^2+k-2\neq0\vee k(k+1)-k-1\neq0\impl \(k\neq1\wedge k\neq -2\) \vee k\neq\pm1
  $$
  $$k\neq 1\wedge \(k\neq -2\vee k\neq -1\)$$
\end{example}

\subsection{Applicazioni ai sistemi lineari}

\begin{theorem}
  Un sistema lineare $AX=B$ ha $\infty^k$ soluzioni se ha almeno una soluzione e $\nul A=k$.
  Se un sistema $AX=B$ ha soluzione, allora le soluzioni sono $X_P+N\(A\)$.
  Se il sistema ha $\infty^k$ soluzioni, allora tutte le soluzioni si possono scrivere come $X_P+t_1X_1+\cdots+t_kX_k$, dove $\left\{ X_1,\dots,X_k \right\}$ è una base di $N(A)$.
\end{theorem}

Se il sistema $AX=B$ ha $\infty^k$ soluzioni, allora l'espressione $X=X_P+t_1X_1+\cdots+t_kX_k$ che descrive le soluzioni in termini di $k$ parametri viene chiamata \textbf{soluzione generale}.

\begin{observation}
  In generale, $v$ è combinazione lineare di $v_1,\dots,v_k$ se e solo se
  $\sp{v_1,\dots,v_k}=\sp{v_1,\dots,v_k,v}$
\end{observation}
\begin{proof}
  Se $v$ è combinazione lineare di $v_1,\dots,v_k$, allora:
  $$\sp{v_1,\dots,v_k}\subseteq\sp{v_1,\dots,v_k,v}\wedge\sp{v_1,\dots,v_k,v}\subseteq\sp{v_1,\dots,v_k}\impl \sp{v_1,\dots,v_k}=\sp{v_1,\dots,v_k,v}$$
  Se $\sp{v_1,\dots,v_k}=\sp{v_1,\dots,v_k,v}$, allora:
  $$v\in\sp{v_1,\dots,v_k,v}\impl v\in\sp{v_1,\dots,v_k}$$
  Pertanto, $v$ è combinazione lineare di $v_1,\dots,v_k$.
\end{proof}

\begin{theorem}[Teorema di Rouché--Capelli]
  Sia $A$ una matrice $n\times m$ e si consideri il sistema lineare di $n$ equazioni e $m$ incognite $AX=B$. Il sistema ha soluzione se e solo se $r\walrus\rk A=\rk \(A,B\)$. In tal caso il sistema ha $\infty^{m-r}$.
\end{theorem}
\begin{proof}
  Il sistema ha soluzione se e solo se $\exists \bar{X}:A\bar{X}=B$. Si ha:
  $$\bar{x_1}A^1+\cdots+\bar{x_m}A^m=B$$
  Pertanto, $\sp{A^1,\dots,A^m}=\sp{A^1,\dots,A^m,B}$, da cui
  $$R\(A\)=R\(A\vert B\)$$
  Dire che i due spazi colonna sono uguali, equivale a dire che i ranghi sono uguali:
  $$\rk A=\rk \(A\vert B\)$$
  Viceversa, se $\rk A=\rk \(A\vert B\)$, allora:
  $$\dim R\(A\)=\dim R\(A\vert B\)$$
  e, poiché $R\(A\)\subseteq R\(A\vert B\)$, allora:
  $$R\(A\)=R\(A\vert B\)$$
  
  Se $r\walrus \rk A=\rk \(A\vert B\)$, allora il sistema ha $\infty^{m-r}$ soluzioni, poiché, per il teorema del rango più nullità:
  $$\nul A = m-\rk A=m-r$$
\end{proof}

\begin{example}
  Determinare per quali valori $k$ il sistema ha soluzione.
  $$
    \begin{cases}
      kx+(k+1)y-z=2       \\
      (k-2)x-(k+1)y+kz=-2 \\
    \end{cases}
  $$
  $$
    A=
    \begin{pmatrix}
      k   & k+1  & -1 \\
      k-2 & -k-1 & k  \\
    \end{pmatrix}
  $$
  $$
    \(A\vert B\)=
    \begin{pmatrix}
      k   & k+1  & -1 & 2  \\
      k-2 & -k-1 & k  & -2 \\
    \end{pmatrix}
  $$
  $$
    M=
    \begin{pmatrix}
      k+1  & -1 \\
      -k-1 & k  \\
    \end{pmatrix}
  $$
  $$\rk A=2\iff\det M\neq 0\iff \(k+1\)\(k-1\)\neq 0\iff k\neq \pm1\impl \rk \(A\vert B\)=2$$
  $$
    k=1\impl \rk A=1\impl \rk \(A\vert B\)=\rk
    \begin{pmatrix}
      1  & 2  & -1 & 2  \\
      -1 & -2 & 1  & -2 \\
    \end{pmatrix}
    =1
  $$
  $$
    k=-1\impl \rk A = 2\impl \rk \(A\vert B\)=\rk
    \begin{pmatrix}
      -1 & 0 & -1 & 2 \\
      -3 & 0 & -1 & 2 \\
    \end{pmatrix}
    =2
  $$
  In conclusione, il sistema ha sempre soluzione.
\end{example}

\subsection{Sistemi crameriani}

\begin{definition}[Sistema crameriano]
  Un sistema di $n$ equazioni e $n$ incognite è detto \textbf{crameriano}.
\end{definition}

\begin{theorem}[Teorema di Cramer]
  Sia $A$ una matrice quadrata di ordine $n$. Il sistema crameriano $AX=B$ ha un'unica soluzione $S=A^{-1}B$ se e solo se $\det A\neq 0$.
\end{theorem}
\begin{proof}
  $$\det A\neq0\iff \exists A^{-1}$$
  $$AX=B$$
  $$A^{-1}AX=A^{-1}B\iff IX=A^{-1}B\iff X=A^{-1}B$$
  
  Se il sistema ha un'unica soluzione, allora $\rk A=\rk\(A,B\)$ e $n-\rk A=0\iff n=\rk A$. Pertanto, $\det A\neq0$.
\end{proof}

\begin{observation}
  Il teorema di Cramer non dice che se $\det A=0$ allora il sistema non ha soluzione, bensì che, se $\det A=0$, il sistema o non ha soluzione o ha più soluzioni diverse.
\end{observation}

\begin{theorem}
  Sia $AX=B$ un sistema crameriano e $\det A\neq 0$. Sia $S=A^{-1}B$ la soluzione del sistema. Allora:
  $$S_i=\frac{\det f(A,B,i)}{\det A}$$
  dove $f(A,B,i)$ è la matrice che si ottiene sostituendo alla $i$--esima colonna di $A$ il vettore $B$.
\end{theorem}
\begin{proof}
  Sia $S$ soluzione del sistema crameriano $AX=B$.
  $$AS=B$$
  $$S_1A^1+S_2A^2+\cdots+S_nA^n=B$$
  \begin{align*}
    f(A,B,i) & =\(A^1,\dots,A^{i-1},B,A^{i+1},\dots,A^n\)                           \\
             & =\(A^1,\dots,A^{i-1},S_1A^1+S_2A^2+\cdots+S_nA^n,A^{i+1},\dots,A^n\) 
  \end{align*}
  \begin{align*}
    \det f(A,B,i) & =\det \(A^1,\dots,A^{i-1},S_1A^1+S_2A^2+\cdots+S_nA^n,A^{i+1},\dots,A^n\) \\
                  & =S_1\det \(A^1,\dots,A^{i-1},A^1,A^{i+1},\dots,A^n\)+                     \\
                  & +S_2\det \(A^1,\dots,A^{i-1},A^2,A^{i+1},\dots,A^n\)+                     \\
                  & +\cdots+S_i\det \(A^1,\dots,A^{i-1},A^i,A^{i+1},\dots,A^n\)+\cdots+       \\
                  & +S_n\det \(A^1,\dots,A^{i-1},A^n,A^{i+1},\dots,A^n\)                      \\
                  & =0+0+\dots+S_i\det A+\cdots+0                                             \\
                  & =S_i\det A                                                                
  \end{align*}
  $$S_i=\frac{\det f(A,B,i)}{\det A}$$
\end{proof}

\begin{example}
  $$
    \begin{cases}
      x+y-z=1  \\
      x-y+z=1  \\
      -x+y+z=1 \\
    \end{cases}
  $$
  $$
    A=
    \begin{pmatrix}
      1  & 1  & -1 \\
      1  & -1 & 1  \\
      -1 & 1  & 1  \\
    \end{pmatrix}
  $$
  $$\det A = -2$$
  $$
    S_1=\frac{ 
      \begin{vmatrix}
        1 & 1  & -1 \\
        1 & -1 & 1  \\
        1 & 1  & 1  \\
      \end{vmatrix}
    }{-4}=\frac{-4}{-4}=1
  $$
  $$
    S_2=\frac{ 
      \begin{vmatrix}
        1  & 1 & -1 \\
        1  & 1 & 1  \\
        -1 & 1 & 1  \\
      \end{vmatrix}
    }{-4}=\frac{-4}{-4}=1
  $$
  $$
    S_3=\frac{ 
      \begin{vmatrix}
        1  & 1  & 1 \\
        1  & -1 & 1 \\
        -1 & 1  & 1 \\
      \end{vmatrix}
    }{-4}=\frac{-4}{-4}=1
  $$
  $$S=\(1,1,1\)$$
\end{example}

\begin{example}
  Determinare per quali valori di $k$ il sistema ha un'unica soluzione, e calcolarla.
  $$
    \begin{cases}
      (k+1)x+z=1     \\
      x+(1-k)y+z=-1  \\
      2x+(1+k)y+2z=k \\
    \end{cases}
  $$
  $$
    A=
    \begin{pmatrix}
      k+1 & 0   & 1 \\
      1   & 1-k & 1 \\
      2   & 1+k & 2 \\
    \end{pmatrix}
  $$
  $$\exists!\ \mathrm{soluzione}\iff\det A\neq0$$
  $$
    \det A=
    \(k+1\)
    \begin{vmatrix}
      1-k & 1 \\
      1+k & 2 \\  
    \end{vmatrix}
    +
    \begin{vmatrix}
      1 & 1-k \\
      2 & 1+k \\
    \end{vmatrix}
    =\(k+1\)\(1-3k\)+\(3k-1\)=-k\(3k-1\)=k-3k^2
  $$
  $$\det A\neq0\iff k\neq0\vee k\neq\nicefrac{1}{3}$$
  $$
    S_1=\frac{
      \begin{vmatrix}
        1  & 0   & 1 \\
        -1 & 1-k & 1 \\
        k  & 1+k & 2 \\
      \end{vmatrix}
    }{k-3k^2}
    =\frac{
      \begin{vmatrix}
        1-k & 1 \\
        1+k & 2 \\
      \end{vmatrix}
      +
      \begin{vmatrix}
        -1 & 1-k \\
        k  & 1+k \\ 
      \end{vmatrix}
    }{k-3k^2}
    =\frac{k^2-5k}{k-3k^2}=\frac{k-5}{1-3k}
  $$
  $$
    S_2=\frac{
      \begin{vmatrix}
        k+1 & 1  & 1 \\
        1   & -1 & 1 \\
        2   & k  & 2 \\
      \end{vmatrix}
    }{k-3k^2}
    =\frac{
      -
      \begin{vmatrix}
        1 & 1 \\
        k & 2 \\
      \end{vmatrix}
      -
      \begin{vmatrix}
        k+1 & 1 \\
        2   & 2 \\
      \end{vmatrix}
      -
      \begin{vmatrix}
        k+1 & 1 \\
        2   & k \\
      \end{vmatrix}
    }{k-3k^2}
    =\frac{-2k-k^2}{k-3k^2}=\frac{k+2}{3k-1}
  $$
  $$
    S_3=\frac{
      \begin{vmatrix}
        k+1 & 0   & 1  \\
        1   & 1-k & -1 \\
        2   & 1+k & k  \\
      \end{vmatrix}
    }{k-3k^2}
    =\frac{
      \(k+1\)
      \begin{vmatrix}
        1-k & -1 \\
        1+k & k  \\
      \end{vmatrix}
      +
      \begin{vmatrix}
        1 & 1-k \\
        2 & 1+k \\
      \end{vmatrix}
    }{k-3k^2}
    =\frac{k^2-k^3+6k}{k-3k^2}=\frac{k^2-k-6}{3k-1}
  $$
\end{example}

\documentclass[a4paper,12pt]{article}

\usepackage[italian]{babel}
\usepackage[a4paper, left=18mm, right=18mm, top=20mm, bottom=20mm]{geometry}
\usepackage{amssymb}
\usepackage{mathtools}
\usepackage{interval}
\usepackage{amsthm}
\usepackage{thmtools}
\usepackage{cancel}
\usepackage{hyperref}
\usepackage{tikz}
\usepackage{pgfplots}
\usepackage{nicefrac}
\usepackage{enumitem}
\usepackage{verbatim}
\usepackage{tabularray}
\usepackage{bold-extra}
\usepackage{tabularray}

\hypersetup{
  colorlinks=true,
  linkcolor=black,
    filecolor=magenta,      
    urlcolor=cyan,
    pdftitle={Esercizi geometria e algebra lineare},
    % bookmarks=true,
    bookmarksopen=true,
    pdfpagemode=UseOutlines,
    pdfauthor={Amato Michele Pasquale},
}

\title{\huge Geometria e algebra lineare\\\Large Esercitazione}
\author{Amato Michele Pasquale}
\date{\today}

\pgfplotsset{compat = newest}
\makeatletter
\renewcommand\l@subsection{\@dottedtocline{2}{1.5em}{3em}}
\makeatother
\setitemize{noitemsep,topsep=3pt,parsep=0pt,partopsep=0pt}
\setenumerate{noitemsep,topsep=3pt,parsep=0pt,partopsep=0pt}

\newcommand{\abs}[1]{\left\lvert #1 \right\rvert}
\newcommand{\ceil}[1]{\left\lceil #1 \right\rceil}
\newtheorem{theorem}{Teorema}
\newtheorem{definition}{Definizione}
\newtheorem{lemma}{Lemma}
\newtheorem{axiom}{Assioma}
\newtheorem{corollary}{Corollario}
\renewcommand\qedsymbol{$\blacksquare$}
\newcommand{\asin}{\arcsin}
\newcommand{\acos}{\arccos}
\newcommand{\atan}{\arctan}
\newcommand{\impl}{\Rightarrow}
\setitemize{noitemsep,topsep=3pt,parsep=0pt,partopsep=0pt}
\setenumerate{noitemsep,topsep=3pt,parsep=0pt,partopsep=0pt}
\newcommand{\triang}[1]{\overset{\triangle}{#1}}
\renewcommand{\a}{\alpha}
\renewcommand{\b}{\beta}
\renewcommand{\c}{\gamma}
\renewcommand{\(}{\left(}
\renewcommand{\)}{\right)}
\renewcommand{\mod}[1]{\ \( \mathrm{mod}\ #1 \)}
\DeclareMathOperator{\agg}{agg}
\DeclareMathOperator{\sol}{Sol}
\renewcommand{\emptyset}{\varnothing}
\newcommand{\walrus}{\coloneqq}
\newcommand{\reals}{\mathbb{R}}
\newcommand{\dabs}[1]{\left\lvert\left\lvert #1 \right\rvert\right\rvert}
\renewcommand{\sp}[1]{\left< #1 \right>}
\DeclareMathOperator{\nul}{null}
\DeclareMathOperator{\rk}{rk}
\DeclareMathOperator{\im}{im}
\DeclareMathOperator{\diag}{diag}
\renewcommand{\l}{\lambda}

\newcounter{nexercise}
\newcommand{\exercisecnt}{\stepcounter{nexercise}\thenexercise}

\newenvironment{exercise}
{\newpage\par\noindent{\bf\Large Esercizio \exercisecnt}\vspace{1em}\hfill\break}
{\hfill\vspace{1cm}}
\newenvironment{solution}
{\par\noindent{\bf\large Soluzione}\vspace{1em}\hfill\break}
{\hfill}


\begin{document}
\pagenumbering{gobble}
\maketitle
\newpage
\pagenumbering{arabic}

\begin{exercise}
  Data la matrice:
  $$
    A=
    \begin{pmatrix}
      1 & 3 & 4 \\
      1 & 2 & 1 \\
      1 & 1 & 2 \\
    \end{pmatrix}
  $$
  Verificare che $\bar{\l}=1$ è un suo autovalore e trovare un autovettore relativo a $\bar{\l}$.
\end{exercise}
\begin{solution}
  \begin{align*}
    p_A\left( t \right) & = 
    \begin{vmatrix}
      1-t & 3   & 4   \\
      1   & 2-t & 1   \\
      1   & 1   & 2-t \\
    \end{vmatrix}
    \\
                        & = 
    \left( 1-t \right)
    \begin{vmatrix}
      2-t & 1   \\
      1   & 2-t \\
    \end{vmatrix}
    -3
    \begin{vmatrix}
      1 & 1   \\
      1 & 2-t \\
    \end{vmatrix}
    +4
    \begin{vmatrix}
      1 & 2-t \\
      1 & 1   \\
    \end{vmatrix}
    \\
                        & = 
    \left( 1-t \right)\left( 4+t^2-4t-1 \right)-3\left( 2-t-1 \right)+4\left( 1-2+t \right)
    \\
                        & = 
    \left( 1-t \right)\left( t^2-4t+3 \right)-3\left( 1-t \right)+4\left( t-1 \right)
    \\
                        & = 
    \left( 1-t \right)\left( t^2-4t+3-3-4 \right)
    \\
                        & = 
    \left( 1-t \right)\left( t^2-4t-4 \right)
  \end{align*}
  $$p_A\left( t \right)=0\iff t\in\left\{ 1,2\pm\sqrt{2} \right\}$$
  $\bar{\l}=1$ è un autovalore di $A$.
  $$
    A-\bar{\l}I=
    \begin{pmatrix}
      0 & 3 & 4 \\
      1 & 1 & 1 \\
      1 & 1 & 1 \\
    \end{pmatrix}
    \sim
    \begin{pmatrix}
      1 & 1 & 1 \\
      0 & 3 & 4 \\
      0 & 0 & 0 \\
    \end{pmatrix}
    \sim
    \begin{pmatrix}
      1 & 1 & 1               \\
      0 & 1 & \nicefrac{4}{3} \\
      0 & 0 & 0               \\
    \end{pmatrix}
    \sim
    \begin{pmatrix}
      1 & 0 & -\nicefrac{1}{3} \\
      0 & 1 & \nicefrac{4}{3}  \\
      0 & 0 & 0                \\
    \end{pmatrix}
  $$
  $$
    N\left( A-\bar{\l}I \right)=\sp{
      \begin{pmatrix}
        \nicefrac{1}{3}  \\
        -\nicefrac{4}{3} \\
        1                \\
      \end{pmatrix}
    }=\sp{
      \begin{pmatrix}
        1  \\
        -4 \\
        3  \\
      \end{pmatrix}
    }
  $$
  $\left( -1,4,3 \right)$ è un autovettore di $A$ relativo all'autovalore $\bar{\l}$.
\end{solution}

\begin{exercise}
  Data la matrice:
  $$
    A=
    \begin{pmatrix}
      1 & 3 & 1  \\
      3 & 1 & -1 \\
      0 & 0 & k  \\
    \end{pmatrix}
  $$
  Stabilire per quali valori $k$ la matrice $A$ è diagonalizzabile.
\end{exercise}
\begin{solution}
  $$
    p_A\left( t \right)=
    \begin{vmatrix}
      1-t & 3   & 1   \\
      3   & 1-t & -1  \\
      0   & 0   & k-t \\
    \end{vmatrix}
    =
    \left( k-t \right)
    \begin{vmatrix}
      1-t & 3   \\
      3   & 1-t \\
    \end{vmatrix}
    =
    \left( k-t \right)\left( t^2-2t-8 \right)
  $$
  $$p_A\left( t \right)=0\iff t\in \left\{ k,-2,4 \right\}$$
  Per il primo criterio di diagonalizzabilità, $A$ è diagonalizzabile se $k\notin\left\{ -2,4 \right\}$.
\end{solution}

\begin{exercise}
  Data la matrice:
  $$
    A=
    \begin{pmatrix}
      1 & k^2 \\
      3 & h   
    \end{pmatrix}
  $$
  Stabilire per quali valori $k$ e $h$ la matrice $A$ è diagonalizzabile.
\end{exercise}
\begin{solution}
  $$
    p_A\left( t \right)=
    \begin{vmatrix}
      1-t & k^2 \\
      3   & h-t \\
    \end{vmatrix}
    =
    \left( 1-t \right)\left( h-t \right)-3k^2
    =
    t^2+\left( -1-h \right)t+\left( h-3k^2 \right)
  $$
  $$\Delta=\left( -1-h \right)^2-4\left( h-3k^2 \right)=h^2-2h+12k^2+1=\left( h-1 \right)^2+12k^2$$
  $$\Delta\ge0\ \forall\left( h,k \right)\in\reals^2$$
  $$\Delta=0\iff h-1=0\wedge k=0\iff h=1\wedge k=0$$
  $A$ è diagonalizzabile per $h\neq 1\vee k\neq0$ per il primo criterio di diagonalizzabilità, poiché si hanno due radici del polinomio caratteristico.
  $$h=1\wedge k=0\iff p_A\left( t \right)=t^2-2t+1=\left( t-1 \right)^2$$
  $$p_A\left( t \right)=0\iff t=1$$
  $$
    A-I=
    \begin{pmatrix}
      0 & 0 \\
      3 & 0 \\  
    \end{pmatrix}
  $$ 
  $$m_g\left( 1 \right)=\nul\left( A-I \right)=1$$
  $$m_a\left( 1 \right)=2$$
  Pertanto, se $h=1\wedge k=0$, la matrice non è diagonalizzabile.
\end{solution}

\begin{exercise}
  Data la matrice:
  $$
    A=
    \begin{pmatrix}
      1 & 2 & 0 \\
      6 & 0 & 0 \\
      3 & 2 & 1 \\
    \end{pmatrix}
  $$
  Stabilire se è diagonalizzabile e, in caso affermativo, trovare la relativa matrice diagonale e modale.
\end{exercise}
\begin{solution}
  $$
    p_A\left( t \right)=
    \begin{vmatrix}
      1-t & 2  & 0   \\
      6   & -t & 0   \\
      3   & 2  & 1-t \\
    \end{vmatrix}
    =
    \left( 1-t \right)
    \begin{vmatrix}
      1-t & 2  \\
      6   & -t \\
    \end{vmatrix}
    =
    \left( 1-t \right)\left( t^2-t-12 \right)
  $$
  $$p_A\left( t \right)=0\iff t\in\left\{ 1,-3,4 \right\}$$
  $A$ è diagonalizzabile per il primo criterio di diagonalizzabilità.
  $$
    A-1\cdot I=
    \begin{pmatrix}
      0 & 2  & 0 \\
      6 & -1 & 0 \\
      3 & 2  & 0 \\
    \end{pmatrix}
    \sim
    \begin{pmatrix}
      0 & 1 & 0 \\
      6 & 0 & 0 \\
      3 & 0 & 0 \\
    \end{pmatrix}
    \sim
    \begin{pmatrix}
      1 & 0 & 0 \\
      0 & 1 & 0 \\
      0 & 0 & 0 \\
    \end{pmatrix}
  $$
  $$
    N\left( A-1\cdot I \right)=\sp{
      \begin{pmatrix}
        0 \\
        0 \\
        1 \\
      \end{pmatrix}
    }
  $$
  $$
    A+3 I=
    \begin{pmatrix}
      4 & 2 & 0 \\
      6 & 3 & 0 \\
      3 & 2 & 4 \\
    \end{pmatrix}
    \sim
    \begin{pmatrix}
      2 & 1 & 0 \\
      3 & 2 & 4 \\
      0 & 0 & 0 \\
    \end{pmatrix}
    \sim
    \begin{pmatrix}
      2 & 1 & 0 \\
      1 & 1 & 4 \\
      0 & 0 & 0 \\
    \end{pmatrix}
    \sim
  $$
  $$
    \sim
    \begin{pmatrix}
      1 & 1 & 4 \\
      2 & 1 & 0 \\
      0 & 0 & 0 \\
    \end{pmatrix}
    \sim
    \begin{pmatrix}
      1 & 1  & 4  \\
      0 & -1 & -8 \\
      0 & 0  & 0  \\
    \end{pmatrix}
    \sim
    \begin{pmatrix}
      1 & 0 & -4 \\
      0 & 1 & 8  \\
      0 & 0 & 0  \\
    \end{pmatrix}
  $$
  $$
    N\left( A+3I \right)=\sp{
      \begin{pmatrix}
        4  \\
        -8 \\
        1  \\
      \end{pmatrix}
    }
  $$
  $$
    A-4I=
    \begin{pmatrix}
      -3 & 2  & 0  \\
      6  & -4 & 0  \\
      3  & 2  & -3 \\
    \end{pmatrix}
    \sim
    \begin{pmatrix}
      3 & -2 & 0  \\
      3 & 2  & -3 \\
      0 & 0  & 0  \\
    \end{pmatrix}
    \sim
    \begin{pmatrix}
      3 & -2 & 0  \\
      0 & 4  & -3 \\
      0 & 0  & 0  \\
    \end{pmatrix}
    \sim
  $$
  $$
    \sim
    \begin{pmatrix}
      3 & -2 & 0                \\
      0 & 1  & -\nicefrac{3}{4} \\
      0 & 0  & 0                \\
    \end{pmatrix}
    \sim
    \begin{pmatrix}
      3 & 0 & -\nicefrac{6}{4} \\
      0 & 1 & -\nicefrac{3}{4} \\
      0 & 0 & 0                \\
    \end{pmatrix}
    \sim
    \begin{pmatrix}
      1 & 0 & -\nicefrac{1}{2} \\
      0 & 1 & -\nicefrac{3}{4} \\
      0 & 0 & 0                \\
    \end{pmatrix}
  $$
  $$
    N\left( A-4I \right)=\sp{
      \begin{pmatrix}
        2 \\
        3 \\
        4 \\
      \end{pmatrix}
    }
  $$
  La matrice modale è:
  $$
    \begin{pmatrix}
      0 & 4  & 2 \\
      0 & -8 & 3 \\
      1 & 1  & 4 \\
    \end{pmatrix}
  $$
  La matrice diagonale è:
  $$
    \begin{pmatrix}
      1 & 0  & 0 \\
      0 & -3 & 0 \\
      0 & 0  & 4 \\
    \end{pmatrix} 
  $$
\end{solution}

\begin{exercise}
  Data la matrice:
  $$
    A=
    \begin{pmatrix}
      0  & 1 & 0 \\
      1  & 3 & 3 \\
      -1 & 4 & 4 \\
    \end{pmatrix}
  $$
  Stabilire se è diagonalizzabile e, in caso affermativo, trovare la relativa matrice diagonale e modale.
\end{exercise}
\begin{solution}
  \begin{align*}
    p_A\left( t \right) & = 
    \begin{vmatrix}
      -t & 1   & 0   \\
      1  & 3-t & 3   \\
      -1 & 4   & 4-t \\
    \end{vmatrix}
    \\
                        & = 
    -t
    \begin{vmatrix}
      3-t & 3   \\
      4   & 4-t \\
    \end{vmatrix}
    -
    \begin{vmatrix}
      1  & 3   \\
      -1 & 4-t \\
    \end{vmatrix}
    \\
                        & = 
    -t\left( t^2-7t \right)-\left( 7-t \right) 
    \\
                        & = 
    -t^2\left( t-7 \right)+\left( t-7 \right)
    \\
                        & = 
    \left( t-7 \right)\left( -t^2+1 \right)
  \end{align*}
  $$p_A\left( t \right)=0\iff t\in \left\{ 7,\pm1 \right\}$$
  $A$ è diagonalizzabile per il primo criterio di diagonalizzabilità.
  $$
    A-7I=
    \begin{pmatrix}
      -7 & 1  & 0  \\
      1  & -4 & 3  \\
      -1 & 4  & -3 \\
    \end{pmatrix}
    \sim
    \begin{pmatrix}
      1  & -4 & 3 \\
      -7 & 1  & 0 \\
      0  & 0  & 0 \\
    \end{pmatrix}
    \sim
    \begin{pmatrix}
      1 & -4  & 3  \\
      0 & -27 & 21 \\
      0 & 0   & 0  \\
    \end{pmatrix}
    \sim
  $$
  $$
    \sim
    \begin{pmatrix}
      1 & -4 & 3  \\
      0 & 9  & -7 \\
      0 & 0  & 0  \\
    \end{pmatrix}
    \sim
    \begin{pmatrix}
      1 & -4 & 3                \\
      0 & 1  & -\nicefrac{7}{9} \\
      0 & 0  & 0                \\
    \end{pmatrix}
    \sim
    \begin{pmatrix}
      1 & 0 & -\nicefrac{1}{9} \\
      0 & 1 & -\nicefrac{7}{9} \\
      0 & 0 & 0                \\
    \end{pmatrix}
  $$
  $$
    N\left( A-7I \right)=\sp{
      \begin{pmatrix}
        1 \\
        7 \\
        9 \\
      \end{pmatrix}
    }
  $$
  $$
    A+1\cdot I=
    \begin{pmatrix}
      1  & 1 & 0 \\
      1  & 4 & 3 \\
      -1 & 4 & 5 \\
    \end{pmatrix}
    \sim
    \begin{pmatrix}
      1 & 1 & 0 \\
      1 & 4 & 3 \\
      0 & 8 & 8 \\
    \end{pmatrix}
    \sim
    \begin{pmatrix}
      1 & 1 & 0 \\
      0 & 3 & 3 \\
      0 & 1 & 1 \\
    \end{pmatrix}
    \sim
    \begin{pmatrix}
      1 & 1 & 0 \\
      0 & 1 & 1 \\
      0 & 0 & 0 \\
    \end{pmatrix}
    \sim
    \begin{pmatrix}
      1 & 0 & -1 \\
      0 & 1 & 1  \\
      0 & 0 & 0  \\
    \end{pmatrix}
  $$
  $$
    N\left( A+1\cdot I \right)=\sp{
      \begin{pmatrix}
        1  \\
        -1 \\
        1  \\
      \end{pmatrix}
    }
  $$
  $$
    A-1\cdot I=
    \begin{pmatrix}
      -1 & 1 & 0 \\
      1  & 2 & 3 \\
      -1 & 4 & 3 \\
    \end{pmatrix}
    \sim
    \begin{pmatrix}
      1 & -1 & 0 \\
      0 & 3  & 3 \\
      0 & 6  & 6 \\
    \end{pmatrix}
    \sim
    \begin{pmatrix}
      1 & -1 & 0 \\
      0 & 1  & 1 \\
      0 & 0  & 0 \\
    \end{pmatrix}
    \sim
    \begin{pmatrix}
      1 & 0 & 1 \\
      0 & 1 & 1 \\
      0 & 0 & 0 \\
    \end{pmatrix}
  $$
  $$
    N\left( A-1\cdot I \right)=\sp{
      \begin{pmatrix}
        -1 \\
        -1 \\
        1  \\
      \end{pmatrix}
    }
  $$
  La matrice modale è:
  $$
    \begin{pmatrix}
      1 & 1  & -1 \\
      7 & -1 & -1 \\
      9 & 1  & 1  \\
    \end{pmatrix}
  $$
  La matrice diagonale è:
  $$
    \begin{pmatrix}
      7 & 0  & 0 \\
      0 & -1 & 0 \\
      0 & 0  & 1 \\
    \end{pmatrix} 
  $$
\end{solution}

\begin{exercise}
  Data la matrice:
  $$
    A=
    \begin{pmatrix}
      3  & 0 & 0  \\
      3  & 3 & 6  \\
      -4 & 0 & -5 \\
    \end{pmatrix}
  $$
  Verificare che $A$ sia diagonalizzabile. Calcolare una base $B$ formata da autovettori di $A$. Calcolare la matrice rappresentativa di $T_A$ nella base $B$.
\end{exercise}
\begin{solution}
  $$
    p_A\left( t \right)=
    \begin{vmatrix}
      3-t & 0   & 0    \\
      3   & 3-t & 6    \\
      -4  & 0   & -5-t \\
    \end{vmatrix}
    =
    \left( 3-t \right)
    \begin{vmatrix}
      3-t & 6    \\
      0   & -5-t \\
    \end{vmatrix}
    =
    \left( 3-t \right)\left( 3-t \right)\left( -5-t \right)
  $$
  $$p_A\left( t \right)=0\iff t\in\left\{ 3,-5 \right\}$$
  $$m_a\left( 3 \right)=2$$
  $$m_a\left( -5 \right)=1$$
  $$
    m_g\left( 3 \right)=3-\rk
    \begin{pmatrix}
      0  & 0 & 0  \\
      3  & 0 & 6  \\
      -4 & 0 & -8 \\
    \end{pmatrix}
    =3-1=2
  $$
  $$
    m_g\left( -5 \right)=3-\rk
    \begin{pmatrix}
      8  & 0 & 0 \\
      3  & 8 & 6 \\
      -4 & 0 & 0 \\
    \end{pmatrix}
    =3-2=1
  $$
  $A$ è diagonalizzabile per il secondo criterio di diagonalizzabilità.
  $$
    A-3I=
    \begin{pmatrix}
      0  & 0 & 0  \\
      3  & 0 & 6  \\
      -4 & 0 & -8 \\
    \end{pmatrix}
    \sim
    \begin{pmatrix}
      1 & 0 & 2 \\
      0 & 0 & 0 \\
      0 & 0 & 0 \\
    \end{pmatrix}
  $$
  $$
    N\left( A-3I \right)=\sp{
      \begin{pmatrix}
        0 \\
        1 \\
        0 \\
      \end{pmatrix}
      ,
      \begin{pmatrix}
        -2 \\
        0  \\
        1  \\
      \end{pmatrix}
    }
  $$
  $$
    A+5I=
    \begin{pmatrix}
      8  & 0 & 0 \\
      3  & 8 & 6 \\
      -4 & 0 & 0 \\
    \end{pmatrix}
    \sim
    \begin{pmatrix}
      1 & 0 & 0 \\
      3 & 8 & 6 \\
      0 & 0 & 0 \\
    \end{pmatrix}
    \sim
    \begin{pmatrix}
      1 & 0 & 0 \\
      0 & 8 & 6 \\
      0 & 0 & 0 \\
    \end{pmatrix}
    \sim
    \begin{pmatrix}
      1 & 0 & 0               \\
      0 & 1 & \nicefrac{3}{4} \\
      0 & 0 & 0               \\
    \end{pmatrix}
  $$
  $$
    N\left( A+5I \right)=\sp{
      \begin{pmatrix}
        0  \\
        -3 \\
        4  \\  
      \end{pmatrix}
    }
  $$
  $$
    B=\left\{ 
    \begin{pmatrix}
      0 \\
      1 \\
      0 \\
    \end{pmatrix}
    ,
    \begin{pmatrix}
      -2 \\
      0  \\
      1  \\
    \end{pmatrix}
    ,
    \begin{pmatrix}
      0  \\
      -3 \\
      4  \\  
    \end{pmatrix}
    \right\}
  $$
  La matrice rappresentativa di $T_A$ nella base $B$ non è altro che la matrice diagonale:
  $$
    \begin{pmatrix}
      3 & 0 & 0  \\
      0 & 3 & 0  \\
      0 & 0 & -5 \\
    \end{pmatrix}
  $$
\end{solution}

\begin{exercise}
  Dato $k\in\reals$, sia:
  $$
    A_k=
    \begin{pmatrix}
      1 & k & 0 \\
      k & 1 & k \\
      0 & 1 & 1 \\
    \end{pmatrix}
  $$
  \begin{enumerate}
    \item verificare che $\left( 1,0,-1 \right)$ è una autovettore di $A_k\ \forall k$ e calcolarne il relativo autovalore;
    \item determinare se esiste una base $B$ di $\reals^3$ formata da autovettori di $A_1$ e in tal caso calcolarla;
    \item determinare per quali valori di $k$ la matrice $A_k$ non è diagonalizzabile.
  \end{enumerate}
\end{exercise}
\begin{solution}
  $$
    \begin{pmatrix}
      1 & k & 0 \\
      k & 1 & k \\
      0 & 1 & 1 \\
    \end{pmatrix}
    \begin{pmatrix}
      1  \\
      0  \\
      -1 \\
    \end{pmatrix}
    =
    \begin{pmatrix}
      1  \\
      0  \\
      -1 \\
    \end{pmatrix}
    =
    \l
    \begin{pmatrix}
      1  \\
      0  \\
      -1 \\
    \end{pmatrix}
    =
    \begin{pmatrix}
      \l  \\
      0   \\
      -\l \\
    \end{pmatrix}
  $$
  $$
    \begin{pmatrix}
      1  \\
      0  \\
      -1 \\
    \end{pmatrix}
    =
    \begin{pmatrix}
      \l  \\
      0   \\
      -\l \\
    \end{pmatrix}
    \iff \l=1
  $$
  $\left( 1,0,-1 \right)$ è un autovettore di $A_k\ \forall k$, con relativo autovalore $1$.
  $$
    A_1=
    \begin{pmatrix}
      1 & 1 & 0 \\
      1 & 1 & 1 \\
      0 & 1 & 1 \\
    \end{pmatrix}
  $$
  \begin{align*}
    p_{A_1}\left( t \right)
     & = 
    \begin{vmatrix}
      1-t & 1   & 0   \\
      1   & 1-t & 1   \\
      0   & 1   & 1-t \\
    \end{vmatrix}
    \\
     & = 
    \left( 1-t \right)
    \begin{vmatrix}
      1-t & 1   \\
      1   & 1-t \\
    \end{vmatrix}
    -
    \begin{vmatrix}
      1 & 1   \\
      0 & 1-t \\
    \end{vmatrix}
    \\
     & = 
    \left( 1-t \right)\left( t^2+2t \right)-\left( 1-t \right)
    \\
     & = 
    \left( 1-t \right)\left( t^2+2t-1 \right)
  \end{align*}
  $$p_{A_1}\left( t \right)=0\iff t\in\left\{ 1,1\pm\sqrt{2} \right\}$$
  $$
    A_1-I=
    \begin{pmatrix}
      0 & 1 & 0 \\
      1 & 0 & 1 \\
      0 & 1 & 0 \\
    \end{pmatrix}
    \sim
    \begin{pmatrix}
      1 & 0 & 1 \\
      0 & 1 & 0 \\
      0 & 0 & 0 \\
    \end{pmatrix}
  $$
  $$
    N\left( A_1-I \right)=\sp{
      \begin{pmatrix}
        -1 \\
        0  \\
        1  \\
      \end{pmatrix}
    }
  $$
  $$
    A_1-\left( 1-\sqrt{2} \right)I=
    \begin{pmatrix}
      \sqrt{2} & 1        & 0        \\
      1        & \sqrt{2} & 1        \\
      0        & 1        & \sqrt{2} \\
    \end{pmatrix}
    \sim
    \begin{pmatrix}
      2 & \sqrt{2} & 0        \\
      1 & \sqrt{2} & 1        \\
      0 & 1        & \sqrt{2} \\
    \end{pmatrix}
    \sim
    \begin{pmatrix}
      1 & 0        & -1       \\
      1 & \sqrt{2} & 1        \\
      0 & 1        & \sqrt{2} \\
    \end{pmatrix}
    \sim
  $$
  $$
    \sim
    \begin{pmatrix}
      1 & 0        & -1       \\
      0 & \sqrt{2} & 2        \\
      0 & 1        & \sqrt{2} \\
    \end{pmatrix}
    \sim
    \begin{pmatrix}
      1 & 0 & -1        \\
      0 & 2 & 2\sqrt{2} \\
      0 & 1 & \sqrt{2}  \\
    \end{pmatrix}
    \sim
    \begin{pmatrix}
      1 & 0 & -1       \\
      0 & 1 & \sqrt{2} \\
      0 & 0 & 0        \\
    \end{pmatrix}
  $$
  $$
    N\left( A_1-\left( 1-\sqrt{2} \right)I \right)=\sp{
      \begin{pmatrix}
        1         \\
        -\sqrt{2} \\
        1         \\
      \end{pmatrix}
    }
  $$
  $$
    A-\left( 1+\sqrt{2} \right)I=
    \begin{pmatrix}
      -\sqrt{2} & 1         & 0         \\
      1         & -\sqrt{2} & 1         \\
      0         & 1         & -\sqrt{2} \\
    \end{pmatrix}
    \sim
    \begin{pmatrix}
      1  & -\sqrt{2} & 1         \\
      -2 & \sqrt{2}  & 0         \\
      0  & 1         & -\sqrt{2} \\
    \end{pmatrix}
    \sim
    \begin{pmatrix}
      1 & -\sqrt{2} & 1         \\
      0 & -\sqrt{2} & 2         \\
      0 & 1         & -\sqrt{2} \\
    \end{pmatrix}
    \sim
  $$
  $$
    \sim
    \begin{pmatrix}
      1 & -\sqrt{2} & 1         \\
      0 & -2        & 2\sqrt{2} \\
      0 & 1         & -\sqrt{2} \\
    \end{pmatrix}
    \sim
    \begin{pmatrix}
      1 & -\sqrt{2} & 1         \\
      0 & 1         & -\sqrt{2} \\
      0 & 0         & 0         \\
    \end{pmatrix}
    \sim
    \begin{pmatrix}
      1 & -\sqrt{2} & 1  \\
      0 & \sqrt{2}  & -2 \\
      0 & 0         & 0  \\
    \end{pmatrix}
    \sim
    \begin{pmatrix}
      1 & 0 & -1        \\
      0 & 1 & -\sqrt{2} \\
      0 & 0 & 0         \\
    \end{pmatrix}
  $$
  $$
    N\left( A-\left( 1+\sqrt{2} \right)I \right)=\sp{
      \begin{pmatrix}
        1        \\
        \sqrt{2} \\
        1        \\
      \end{pmatrix}
    }
  $$
  $$
    B=\left\{ 
    \begin{pmatrix}
      -1 \\
      0  \\
      1  \\
    \end{pmatrix}
    ,
    \begin{pmatrix}
      1         \\
      -\sqrt{2} \\
      1         \\
    \end{pmatrix}
    ,
    \begin{pmatrix}
      1        \\
      \sqrt{2} \\
      1        \\
    \end{pmatrix}
    \right\}
  $$
  $$
    \reals^3=\sp{
      \begin{pmatrix}
        -1 \\
        0  \\
        1  \\
      \end{pmatrix}
      ,
      \begin{pmatrix}
        1         \\
        -\sqrt{2} \\
        1         \\
      \end{pmatrix}
      ,
      \begin{pmatrix}
        1        \\
        \sqrt{2} \\
        1        \\
      \end{pmatrix}
    }
  $$
  \begin{align*}
    p_{A_k}\left( t \right)
     & = 
    \begin{vmatrix}
      1-t & k   & 0   \\
      k   & 1-t & k   \\
      0   & 1   & 1-t \\
    \end{vmatrix}
    \\
     & = 
    \left( 1-t \right)
    \begin{vmatrix}
      1-t & k   \\
      1   & 1-t \\
    \end{vmatrix}
    -k
    \begin{vmatrix}
      k & k   \\
      0 & 1-t \\
    \end{vmatrix}
    \\
     & = 
    \left( 1-t \right)\left( 1+t^2-2t-k \right)-k\left( k-kt \right)
    \\
     & = 
    \left( 1-t \right)\left( t^2-2t+1-k \right)-k^2\left( 1-t \right)
    \\
     & = 
    \left( 1-t \right)\left( t^2-2t+1-k -k^2\right)
  \end{align*}
  $t^2-2t+1-k-k^2=0\iff \left( t-1 \right)^2-k\left( 1+k \right)=0\iff \left( t-1 \right)^2=k\left( 1+k \right)\iff \left( t-1 \right)^2= k+k^2 \iff t-1=\pm\sqrt{k+k^2}\iff t=1\pm\sqrt{k+k^2}$
  $$p_{A_k}\left( t \right)=0\iff t\in\left\{ 1,1\pm\sqrt{k+k^2} \right\}$$
  $$k+k^2=k(1+k)=0\iff k\in\left\{ -1,0 \right\}$$
  $$k+k^2<0\iff -1<k<0$$
  La matrice $A_k$ non è diagonalizzabile per $-1<k<0$.
  $$k=0\impl m_a\left( 1 \right)=3 \wedge m_g\left( 1 \right)=2$$
  $$k=-1\impl m_a\left( 1 \right)=3 \wedge m_g\left( 1 \right)=1$$
  La matrice $A_k$ non è diagonalizzabile per $k=-1\vee k=0$.
\end{solution}

\begin{exercise}
  Dato $h\in\reals$, sia:
  $$
    A_h=
    \begin{pmatrix}
      1 & 1 & h \\
      1 & 1 & 1 \\
      h & 1 & 1 \\
    \end{pmatrix}
  $$
  \begin{enumerate}
    \item verificare che $\left( -1,0,1 \right)$ è un autovettore di $A_h$, e in tal caso calcolarne il relativo autovalore;
    \item calcolare una matrice $M$ tale che $M^{-1}A_1M$ è diagonale;
    \item verificare che l'insieme $B$ formato dalle colonne di $M$ è una base di $\reals^3$ e calcolare la matrice $T_{A_1}$ in $B$.
  \end{enumerate}
\end{exercise}
\begin{solution}
  $$
    \begin{pmatrix}
      1 & 1 & h \\
      1 & 1 & 1 \\
      h & 1 & 1 \\
    \end{pmatrix}
    \begin{pmatrix}
      -1 \\
      0  \\
      1  \\
    \end{pmatrix}
    =
    \begin{pmatrix}
      h-1 \\
      0   \\
      1-h \\
    \end{pmatrix}
    =
    \l
    \begin{pmatrix}
      -1 \\
      0  \\
      1  \\
    \end{pmatrix}
    \iff
    \l=1-h
  $$
  $\l=1-h$ è l'autovalore di $A_h$ relativo all'autovettore $\left( -1,0,1 \right)$.
  $$
    A_1=
    \begin{pmatrix}
      1 & 1 & 1 \\
      1 & 1 & 1 \\
      1 & 1 & 1 \\
    \end{pmatrix}
  $$
  \begin{align*}
    p_{A_1}\left( t \right)
     & = 
    \begin{vmatrix}
      1-t & 1   & 1   \\
      1   & 1-t & 1   \\
      1   & 1   & 1-t \\
    \end{vmatrix}
    \\
     & = 
    \left( 1-t \right)
    \begin{vmatrix}
      1-t & 1   \\
      1   & 1-t \\
    \end{vmatrix}
    -
    \begin{vmatrix}
      1 & 1   \\
      1 & 1-t \\
    \end{vmatrix}
    +
    \begin{vmatrix}
      1 & 1-t \\
      1 & 1   \\
    \end{vmatrix}
    \\
     & = 
    \left( 1-t \right)\left( t^2-2t \right)-\left( -t \right)+ t 
    \\
     & = 
    t\left( 1-t \right)\left( t-2 \right)+2t
    \\
     & = 
    t\left( \left( 1-t \right)\left( t-2 \right)+2 \right)
    \\
     & = 
    t\left( 3t-t^2 \right)
    \\
     & = 
    t^2\left( 3-t \right)
  \end{align*}
  $$p_{A_1}\left( t \right)=0\iff t\in\left\{ 0,3 \right\}$$
  $$
    A_1-0\cdot I=
    \begin{pmatrix}
      1 & 1 & 1 \\
      1 & 1 & 1 \\
      1 & 1 & 1 \\
    \end{pmatrix}
    \sim
    \begin{pmatrix}
      1 & 1 & 1 \\
      0 & 0 & 0 \\
      0 & 0 & 0 \\
    \end{pmatrix}
  $$
  $$
    N\left( A_1-0\cdot I \right)=\sp{
      \begin{pmatrix}
        -1 \\
        1  \\
        0  \\
      \end{pmatrix}
      ,
      \begin{pmatrix}
        -1 \\
        0  \\
        1  \\
      \end{pmatrix}
    }
  $$
  $$
    A_1-3I=
    \begin{pmatrix}
      -2 & 1  & 1  \\
      1  & -2 & 1  \\
      1  & 1  & -2 \\
    \end{pmatrix}
    \sim
    \begin{pmatrix}
      1 & 1  & -2 \\
      0 & -3 & 3  \\
      0 & 3  & -3 \\
    \end{pmatrix}
    \sim
    \begin{pmatrix}
      1 & 1 & -2 \\
      0 & 1 & -1 \\
      0 & 0 & 0  \\
    \end{pmatrix}
    \sim
    \begin{pmatrix}
      1 & 0 & -1 \\
      0 & 1 & -1 \\
      0 & 0 & 0  \\
    \end{pmatrix}
  $$
  $$
    N\left( A_1-3I \right)=\sp{
      \begin{pmatrix}
        1 \\
        1 \\
        1 \\
      \end{pmatrix}
    }
  $$
  $$
    M=
    \begin{pmatrix}
      -1 & -1 & 1 \\
      1  & 0  & 1 \\
      0  & 1  & 1 \\
    \end{pmatrix}
  $$
  $$
    M^{-1}A_1M=
    \begin{pmatrix}
      0 & 0 & 0 \\
      0 & 0 & 0 \\
      0 & 0 & 3 \\
    \end{pmatrix}
  $$
  $$
    \det M=
    \begin{vmatrix}
      -1 & -1 & 1 \\
      1  & 0  & 1 \\
      0  & 1  & 1 \\
    \end{vmatrix}
    =
    -
    \begin{vmatrix}
      -1 & 1 \\
      1  & 1 \\
    \end{vmatrix}
    +
    \begin{vmatrix}
      -1 & -1 \\
      1  & 0  \\
    \end{vmatrix}
    =
    1+1+1=3
    \impl
    \rk M=3
    \impl
    R\left( M \right)=\reals^3
  $$
  $$
    B=\left\{ 
    \begin{pmatrix}
      -1 \\
      1  \\
      0  \\
    \end{pmatrix}
    ,
    \begin{pmatrix}
      -1 \\
      0  \\
      1  \\
    \end{pmatrix}
    ,
    \begin{pmatrix}
      1 \\
      1 \\
      1 \\
    \end{pmatrix}
    \right\}
  $$
  $$
    \begin{pmatrix}
      1 & 1 & 1 \\
      1 & 1 & 1 \\
      1 & 1 & 1 \\
    \end{pmatrix}
    \begin{pmatrix}
      -1 \\
      1  \\
      0  \\
    \end{pmatrix}
    =
    \begin{pmatrix}
      0 \\
      0 \\
      0 \\
    \end{pmatrix}
  $$
  $$
    \begin{pmatrix}
      -1 & -1 & 1 & 0 \\
      1  & 0  & 1 & 0 \\
      0  & 1  & 1 & 0 \\
    \end{pmatrix}
    \sim
    \begin{pmatrix}
      1 & 0  & 1 & 0 \\
      0 & -1 & 2 & 0 \\
      0 & 1  & 1 & 0 \\
    \end{pmatrix}
    \sim
    \begin{pmatrix}
      1 & 0 & 1 & 0 \\
      0 & 1 & 1 & 0 \\
      0 & 0 & 3 & 0 \\
    \end{pmatrix}
    \sim
    \begin{pmatrix}
      1 & 0 & 0 & 0 \\
      0 & 1 & 0 & 0 \\
      0 & 0 & 1 & 0 \\
    \end{pmatrix}
  $$
  $$
    \begin{pmatrix}
      1 & 1 & 1 \\
      1 & 1 & 1 \\
      1 & 1 & 1 \\
    \end{pmatrix}
    \begin{pmatrix}
      -1 \\
      0  \\
      1  \\
    \end{pmatrix}
    =
    \begin{pmatrix}
      0 \\
      0 \\
      0 \\
    \end{pmatrix}
  $$
  $$
    \begin{pmatrix}
      -1 & -1 & 1 & 0 \\
      1  & 0  & 1 & 0 \\
      0  & 1  & 1 & 0 \\
    \end{pmatrix}
    \sim
    \begin{pmatrix}
      1 & 0  & 1 & 0 \\
      0 & -1 & 2 & 0 \\
      0 & 1  & 1 & 0 \\
    \end{pmatrix}
    \sim
    \begin{pmatrix}
      1 & 0 & 1 & 0 \\
      0 & 1 & 1 & 0 \\
      0 & 0 & 3 & 0 \\
    \end{pmatrix}
    \sim
    \begin{pmatrix}
      1 & 0 & 0 & 0 \\
      0 & 1 & 0 & 0 \\
      0 & 0 & 1 & 0 \\
    \end{pmatrix}
  $$
  $$
    \begin{pmatrix}
      1 & 1 & 1 \\
      1 & 1 & 1 \\
      1 & 1 & 1 \\
    \end{pmatrix}
    \begin{pmatrix}
      1 \\
      1 \\
      1 \\
    \end{pmatrix}
    =
    \begin{pmatrix}
      3 \\
      3 \\
      3 \\
    \end{pmatrix}
  $$
  $$
    \begin{pmatrix}
      -1 & -1 & 1 & 3 \\
      1  & 0  & 1 & 3 \\
      0  & 1  & 1 & 3 \\
    \end{pmatrix}
    \sim
    \begin{pmatrix}
      1 & 0  & 1 & 3 \\
      0 & -1 & 2 & 6 \\
      0 & 1  & 1 & 3 \\
    \end{pmatrix}
    \sim
    \begin{pmatrix}
      1 & 0 & 1 & 3 \\
      0 & 1 & 1 & 3 \\
      0 & 0 & 3 & 9 \\
    \end{pmatrix}
    \sim
    \begin{pmatrix}
      1 & 0 & 1 & 3 \\
      0 & 1 & 1 & 3 \\
      0 & 0 & 1 & 3 \\
    \end{pmatrix}
    \sim
    \begin{pmatrix}
      1 & 0 & 0 & 0 \\
      0 & 1 & 0 & 0 \\
      0 & 0 & 1 & 3 \\
    \end{pmatrix}
  $$
  La matrice rappresentativa di $T_{A_1}$ nella base $B$ è:
  $$
    \begin{pmatrix}
      0 & 0 & 0 \\
      0 & 0 & 0 \\
      0 & 0 & 3 \\
    \end{pmatrix}
  $$
\end{solution}

\end{document}
\section{Serie numeriche}

% \paragraph*{Motivazione}
% $$I=\intv{0}{1}=\bigcup_{n=0}^\infty I_n$$
% $$I_0=\intv{0}{\nicefrac{1}{2}}$$
% $$I_1=\intv{\nicefrac{1}{2}}{\nicefrac{1}{2}+\nicefrac{1}{4}}$$
% $$I_2=\intv{\nicefrac{1}{2}+\nicefrac{1}{4}}{\nicefrac{1}{2}+\nicefrac{1}{4}+\nicefrac{1}{8}}$$
% % TODO: disegnare i pezzetti
% $$\abs{I}=1$$
% $$\abs{I_n}=2^{-\left( n+1 \right)}$$
% $$1=\abs{I_0}+\abs{I_1}+\cdots+\abs{I_n}+\cdots=\sum_{n=0}^\infty\abs{I_n}$$
\begin{definition}[Serie numerica]
  Sia $\left\{ a_n \right\}_{n=0}^\infty\subset\reals$ una successione. Si definisce un'altra successione $\left\{ s_N \right\}_{N=0}^\infty$, detta delle somme parziali:
  $$s_N\walrus \sum_{n=0}^Na_n=a_0+a_1+\cdots+a_N$$
  
  Se la successione $\left\{ s_N \right\}$ ammette limite, si dice che la serie definita dai coefficienti $\left\{ a_n \right\}$ converge. 
  Se $\left\{ s_N \right\}$ converge a $\pm\infty$, si dice che la successione diverge.  
  Se $\left\{ s_N \right\}$ non ammette limite si dice che la serie è indeterminata.
  
  $$\sum_{n=0}^\infty a_n\walrus\lim_{N\to+\infty}s_N=\lim_{N\to+\infty}\sum_{n=0}^Na_n$$
\end{definition}

Nella pratica, tuttavia, si usa il simbolo:
$$\sum_{n=0}^\infty a_n$$
per denotare tutto ciò, anche quando la serie non converge.

\begin{example}
  $$\sum_{n=1}^\infty\frac{1}{2^n}=1$$
  $$s_N=\sum_{n=1}^N\frac{1}{2^n}=\frac{1}{2}+\frac{1}{4}+\cdots+\frac{1}{2^N}=\frac{2^{N-1}+2^{N-2}+\cdots+1}{2^N}=\frac{2^N-1}{2^N}$$
  $$\lim_{N\to\infty}s_N=\lim_{N\to\infty}\frac{2^N-1}{2^N}=\lim_{N\to\infty}\left( \frac{2^N}{2^N}+\frac{1}{2^N} \right)=1+0=1$$
\end{example}

\begin{example}
  $$a_n\ge0\quad \forall n\in\rintv{0}{\infty}$$
  $$f:\rintv{0}{+\infty}\to\reals\qquad f\left( x \right)=a_n\quad x\in\rintv{n}{n+1}$$
  $$\sum_{n=0}^\infty a_n=\int_0^\infty f\left( x \right)\,dx$$
  % TODO: grafico; funzione curva, ridotta a scalini
\end{example}

\begin{lemma}[Serie di Mengoli]
  $$\sum_{n=1}^\infty\frac{1}{n\left( n+1 \right) }=1$$
\end{lemma}
\begin{proof}
  $$a_n=\frac{1}{n\left( n+1 \right)}\qquad n\ge1$$
  $$s_N=\sum_{n=1}^N\frac{1}{n\left( n+1 \right)}=\sum_{n=1}^N\left( \frac{1}{n}-\frac{1}{n+1} \right)=1-\frac{1}{N+1}$$
  $$\sum_{n=0}^\infty\frac{1}{n\left( n+1 \right)}=\lim_{N\to+\infty}s_N=\lim_{N\to+\infty}\left( 1-\frac{1}{N+1} \right)=1$$
\end{proof}

\begin{example}
  $$\sum_{n=1}^\infty\ln\left( \frac{n+1}{n} \right)=\sum_{n=1}^\infty\left( \ln\left( n+1 \right)-\ln n \right)$$
  $$s_N=\sum_{n=1}^N\left( \ln\left( n+1 \right)-\ln n \right)=\ln\left( N+1 \right)-\ln 1=\ln\left( N+1 \right)$$
  $$\sum_{n=1}^\infty\ln\left( \frac{n+1}{n} \right)=\lim_{N\to+\infty}\ln\left( N+1 \right)=+\infty$$
\end{example}

\begin{example}
  $$\sum_{n=0}^\infty\left( -1 \right)^n$$
  $$
    s_N=\sum_{n=0}^N\left( -1 \right)^n=\left( -1 \right)^0+\left( -1 \right)^1+\left( -1 \right)^2+\cdots+\left( -1 \right)^N=
    \begin{cases}
      1 & N\equiv 0 \mod{2} \\
      0 & N\equiv 1 \mod{2} \\
    \end{cases}
  $$
  $$\nexists \lim_{N\to+\infty}s_N$$
\end{example}

\begin{theorem}[Serie geometrica]
  Sia $\reals\ni q\ge0$.
  $$
    \sum_{n=0}^\infty q^n=
    \begin{cases}
      +\infty       & q\ge1    \\
      \frac{1}{1-q} & 0\le q<1 \\
    \end{cases}
  $$
\end{theorem}
\begin{proof}
  $$q=1\impl \sum_{n=0}^\infty q^n=1+1+\cdots=+\infty$$
  $$
    q\neq1
    \impl s_N=\sum_{n=0}^N q^n=\frac{1-q^{N+1}}{1-q}
    \impl \sum_{n=0}^\infty q^n=
    \lim_{N\to+\infty}s_N=
    \begin{cases}
      +\infty       & q>1      \\
      \frac{1}{1-q} & 0\le q<1 \\
    \end{cases}
  $$
\end{proof}

\begin{observation}
  Il carattere di una serie (la sua convergenza, la sua divergenza o la sua non determinatezza) non è alterato se si trascurano un numero finito di termini:
  $$\sum_{n=n_0}^\infty a_n$$ ha lo stesso carattere di $$\sum_{n=n_0+k}^\infty a_n$$
  Infatti, si nota che:
  $$\sum_{n=n_0}^Na_n=\sum_{n=n_0+k}^Na_n+\sum_{n=n_0}^{n_0+k-1}a_n$$
\end{observation}

\begin{observation}
  Raramente si riesce a determinare esattamente la somma di una serie convergente.
\end{observation}

\subsection{Proprietà}

\begin{theorem}[Linearità]
  Siano $\left\{ a_n \right\},\left\{ b_n \right\}$ successioni e $a,b\in\reals$. Vale la seguente relazione:
  $$\sum_n\left( \alpha a_n+\beta b_n \right)=\alpha\sum_na_n+\beta\sum_nb_n$$
  In particolare, se due delle serie cui sopra convergono anche la terza converge. Di contro, se $\sum_na_n$ converge e $\sum_nb_n$ non converge, allora $\sum_n\left( \alpha a_n+\beta b_n \right)$ non converge.
\end{theorem}

\begin{theorem}[Confronto]
  Se $0\le a_n\le b_n\ \forall n$, allora:
  \begin{itemize}[label=--]
    \item se $\sum_nb_n$ converge, anche $\sum_na_n$ converge e, in particolare $$\sum_na_n\le\sum_nb_n$$
    \item se $\sum_na_n$ diverge, anche $\sum_nb_n$ diverge, poiché le somme parziali di $a_n$ sono più piccole di quelle di $b_n$.
  \end{itemize}
\end{theorem}

\begin{theorem}[Confronto asintotico]
  Siano $\left\{ a_n \right\},\left\{ b_n \right\}$ successioni, tali che $0\le a_n,0\le b_n$ e $a_n\sim b_n$ per $n\to+\infty$. Allora:
  $$\reals\ni\sum_na_n\iff \sum_nb_n\in \reals$$
  oppure:
  $$\infty=\sum_na_n\iff\sum_nb_n=\infty$$
\end{theorem}
\begin{proof}
  $$a_n\sim b_n\iff \lim_{n}\frac{a_n}{b_n}=1$$
  $$\forall\epsilon>0\ \exists\ N:n\ge N\impl\abs{\frac{a_n}{b_n}-1}<\epsilon$$
  $$\Updownarrow$$
  $$-\epsilon<\frac{a_n}{b_n}-1<+\epsilon$$
  $$\Updownarrow$$
  $$1-\epsilon<\frac{a_n}{b_n}<1+\epsilon$$
  $$\Updownarrow$$
  $$\left( 1-\epsilon \right)b_n<a_n<\left( 1+\epsilon \right)b_n$$
  Pertanto, per il teorema del confronto, $\sum_na_n$ ha lo stesso carattere di $\sum_nb_n$
\end{proof}

\begin{theorem}
  Condizione necessaria affinché $\sum_na_n$ converga è che $\lim_na_n=0$.
\end{theorem}
\begin{proof}
  $$a_N\walrus s_N-s_{N-1}$$
  Se $\sum_na_n$ converge, allora $\exists\ \lim_Ns_N=\lim_Ns_{N-1}$ e quindi:
  $$\lim_Na_N=\lim_N\left( s_N-s_{N-1} \right)=\left( \lim_Ns_N \right)-\left( \lim_Ns_{N-1} \right)=0$$
\end{proof}

\begin{example}
  $$\sum_{n=1}^\infty\frac{1}{n^2}=1+\frac{1}{4}+\frac{1}{9}+\cdots$$
  $$a_n=\frac{1}{n^2}\stackrel{n\to\infty}{\sim}\frac{1}{n\left( n+1 \right)}=b_n$$
  Per il criterio del confronto asintotico $\sum\frac{1}{n}$ converge.
\end{example}

\begin{observation}
  La condizione di cui sopra è \emph{necessaria}, ma non sufficiente.
\end{observation}

\begin{example}
  \emph{Serie armonica}
  $$\sum_{n=1}^\infty\frac{1}{n}=1+\frac{1}{2}+\frac{1}{3}+\cdots$$
  $$a_n=\frac{1}{n}\xrightarrow{n\to\infty}0$$
  La condizione necessaria è verificata.
  Considerata la funzione $f$, definita nel seguente modo:
  $$f\left( x \right)\walrus\ln\left( 1+x \right)$$
  Si ha che:
  $$f'\left( x \right)=\left( 1+x \right)^{-1}$$
  $$f''\left( x \right)=-\left( 1+x \right)^{-2}$$
  Poiché $f''\left( x \right)<0$, $f$ è concava, ossia il suo grafico è al di sotto delle sue tangenti. In particolare, $f'\left( 0 \right)=1$:
  % TODO: grafico di ln(x+1) con tangente in x=0
  Pertanto:
  $$\ln\left( 1+x \right)\le x\iff\ln\left( 1+\frac{1}{n} \right)\le \frac{1}{n}\iff\ln\left( \frac{n+1}{n} \right)\le \frac{1}{n}$$
  $$\sum_{n=1}^N\ln\left( \frac{n+1}{n} \right)\le\sum_{n=1}^N\frac{1}{n}$$
  Da cui:
  $$+\infty\le\sum_{n=1}^N\frac{1}{n}\impl \sum_{n=1}^N\frac{1}{n}=+\infty$$
  Pertanto, la serie armonica diverge per il criterio del confronto.
\end{example}

\begin{example}
  $$\sum_{n=1}^\infty\frac{1}{n^\alpha}\qquad\alpha\ge2$$
  $$\frac{1}{n^\alpha}\le\frac{1}{n^2}$$
  $$\sum_{n=1}^\infty \frac{1}{n^\alpha}\le\sum_{n=1}^\infty\frac{1}{n^2}<+\infty$$
  In particolare, $\sum n^{-\alpha}$ converge.
\end{example}

\begin{example}
  $$\sum_{n=0}^\infty\left( \sqrt{n+1}-\sqrt{n} \right)=\sum_{n=0}^\infty\frac{\left( \sqrt{n+1}-\sqrt{n} \right)\left( \sqrt{n+1}+\sqrt{n} \right)}{\left( \sqrt{n+1}+\sqrt{n} \right)}=\sum_{n=0}^\infty\frac{1}{\sqrt{n+1}+\sqrt{n}}$$
  $$a_n=\frac{1}{\sqrt{n+1}+\sqrt{n}}\sim \frac{1}{2\sqrt{n}}$$
  $$\frac{1}{n}\le \frac{1}{2\sqrt{n}}$$
  $$+\infty=\sum_n\frac{1}{n}\le\sum_n\frac{1}{2\sqrt{n}}$$
  Per il teorema del confronto asintotico, la serie diverge.
\end{example}

\begin{example}
  $$\sum_{n\ge1}\frac{\sqrt[n]{e}-1}{n}$$
  $$a_n=\frac{\sqrt[n]{e}-1}{n}=\frac{1}{n}\left( e^{\nicefrac{1}{n}}-1 \right)\sim \frac{1}{n}\cdot \frac{1}{n}=\frac{1}{n^2}$$
  Pertanto, per il teorema del confronto asintotico, la serie converge.
\end{example}

\begin{theorem}[Confronto integrale]
  Siano $\left\{ a_n \right\}$ e una funzione $f:\rintv{0}{\infty}\to\rintv{0}{\infty}\in\mathcal{R}\left( \rintv{0}{\infty} \right)$, tali che $a_n\ge0$ e $f$ decrescente.
  Se $a_n\sim f\left( n \right)$ per $n\to+\infty$, allora $\sum a_n$ converge.
\end{theorem}
\begin{proof}
  $$x\in\rintv{n}{n+1}\impl f\left( x \right)\ge f\left( n+1 \right)$$
  $$+\infty>\int_0^\infty f\left( x \right)\,dx\ge\sum_{n=0}^\infty f\left( n+1 \right)$$
  % TODO: grafico di 1/x scalinato a intervalli regolari; colorare area sotto gli scalini
  Quindi $\sum f\left( n+1 \right)$ è convergente e, dato che $a_n\sim f\left( n+1 \right)$, anche $\sum a_n$ è convergente.
\end{proof}

\begin{example}
  $$\sum_{n\ge1}\frac{1}{n^\alpha}\qquad\alpha\in\ointv{1}{2}$$
  $$f\left( x \right)=\frac{1}{x^\alpha}\iff f\left( n \right)=\frac{1}{n^\alpha}$$
  Poiché $f$ è integrabile su $\ointv{1}{\infty}$ per $\alpha>1$, per il teorema del confronto integrale, si ha che:
  $$\sum_{n\ge1}\frac{1}{n^\alpha}\in\reals$$
\end{example}

\begin{theorem}[Criterio della radice]
  Sia $\left\{ a_n \right\}$, tale che $0\le a_n$ ed
  $$\exists\ \l\walrus\lim_n\sqrt[n]{a_n}\in\intv{0}{\infty}$$
  Allora:
  \begin{itemize}
    \item se $0\le\l<1$, la serie $\sum a_n$ converge;
    \item se $\l>1$, la serie $\sum a_n$ diverge;
    \item se $\l=1$, nulla si può dire sula carattere di $\sum a_n$.
  \end{itemize}
\end{theorem}
\begin{proof}
  Dalle ipotesi segue che:
  $$\forall \epsilon>0\ \exists\ N:n\ge N\impl\abs{\sqrt[n]{a_n}-\l}<\epsilon$$
  $$\Updownarrow$$
  $$-\epsilon<\sqrt[n]{a_n}-\l<+\epsilon\iff\l-\epsilon<\sqrt[n]{a_n}<\l+\epsilon$$
  In virtù del fatto che $\epsilon$ è arbitrariamente piccolo, si sceglie $0<\epsilon<\l$, per cui si ha:
  $$\left( \l-\epsilon \right)^n<a_n<\left( \l+\epsilon \right)^n$$
  da cui:
  $$\sum_n\left( \l-\epsilon \right)^n<\sum_na_n<\sum_n\left( \l+\epsilon \right)^n$$
  Se $\l<1$, scegliendo $\epsilon:\l+\epsilon<1$ la serie $\sum\left( \l+\epsilon \right)^n$ converge; quindi, per il teorema del confronto, anche $\sum a_n$ converge.
  
  \noindent Se $\l>1$, scegliendo $\epsilon:\l-\epsilon>1$ la serie $\sum\left( \l-\epsilon \right)^n$ diverge; quindi, per il teorema del confronto, anche $\sum a_n$ diverge.
\end{proof}

\begin{example}
  $$\sum_{n\ge1}n^{-\alpha}\qquad a_n=n^{-\alpha}$$
  $$\sqrt{a_n}=a_n^{\nicefrac{1}{n}}=n^{-\nicefrac{\alpha}{n}}=e^{-\nicefrac{\alpha}{n}\ln n}\xrightarrow{n\to\infty}1\quad \forall\a>0$$
  Pertanto $\sum n^{-\alpha}$ converge se $\alpha>1$ e, viceversa, diverge se $0<\alpha\le 1$.
\end{example}

\begin{example}
  $$\sum_{n\ge1}\frac{n}{2^n}\qquad a_n=\frac{n}{2^n}$$
  $$a_n^{\nicefrac{1}{n}}=\frac{n^{\nicefrac{1}{n}}}{2}=\frac{e^{\nicefrac{\ln n}{n}}}{2}\to \frac{1}{2}<1$$
  Pertanto la serie converge.
\end{example}

\begin{theorem}[Criterio del rapporto]
  Sia $\left\{ a_n \right\}$, tale che $0\le a_n$ ed
  $$\exists\ \l\walrus \lim_n\frac{a_{n+1}}{a_n}\in\intv{0}{\infty}$$
  Allora:
  \begin{itemize}
    \item se $0\le\l<1$, la serie $\sum a_n$ converge;
    \item se $\l>1$, la serie $\sum a_n$ diverge;
    \item se $\l=1$, nulla si può dire sula carattere di $\sum a_n$.
  \end{itemize}
\end{theorem}
\begin{proof}
  Dalle ipotesi segue che:
  $$\forall \epsilon\ \exists\ N:n\ge N\impl\abs{\frac{a_{n+1}}{a_n}-\l}<\epsilon$$
  $$\Updownarrow$$
  $$-\epsilon<\frac{a_{n+1}}{a_n}-\l<+\epsilon\iff\l-\epsilon<\frac{a_{n+1}}{a_n}<\l+\epsilon$$
  In virtù del fatto che $\epsilon$ è arbitrariamente piccolo, si sceglie $0<\epsilon<\l$.
  Poiché $a_n\ge0$, si ha:
  $$\left( \l-\epsilon \right)a_n<a_{n+1}<\left( \l+\epsilon \right)a_n$$
  Se $\l<1$, scegliendo $\epsilon:\l+\epsilon<1$, si ottiene: $$a_{n+1}<\left( \l+\epsilon \right)a_n<\left( \l+\epsilon \right)\left( \l+\epsilon \right)a_{n-1}<\cdots<\left( \l+\epsilon \right)^{n+1}a_0$$ quindi si sta confrontando la serie con una serie $\sum\left( \l+\epsilon \right)^n$ geometrica convergente e, per il criterio del confronto, $\sum a_n$ è convergente.
  
  \noindent Se $\l>1$, scegliendo $\epsilon:\l-\epsilon>1$, si ottiene: $$a_{n+1}>\left( \l-\epsilon \right)a_n>\left( \l-\epsilon \right)^2a_{n-1}>\cdots>\left( \l-\epsilon \right)^{n+1}a_0$$ quindi si sta confrontando la serie con una serie $\sum\left( \l-\epsilon \right)^n$ geometrica divergente e, per il criterio del confronto, $\sum a_n$ è divergente.
\end{proof}

\begin{example}
  \emph{Serie esponenziale}
  
  Sia $x\ge0$ fissato.
  $$\sum_{n=0}^\infty\frac{x^n}{n!}$$
  $$a_n=\frac{x^n}{n!}\ge0$$
  $$\frac{a_{n+1}}{a_n}=\frac{x^{n+1}}{\left( n+1 \right)!}\cdot \frac{n!}{x^n}=x\cdot\frac{1}{n+1}\xrightarrow{n\to\infty}0=\l$$
  Per il criterio del rapporto, la serie converge.
  
  In più, applicando il teorema di Taylor con resto di Lagrange, si ottiene:
  $$e^x=\lim_{N\to+\infty}\sum_{n=0}^N\frac{x^n}{n!}=\sum_{n=0}^\infty\frac{x^n}{n!}\quad \forall x\ge0$$
\end{example}

\begin{example}
  $$\sum_{n\ge1}n!\cdot\left( \frac{b}{n} \right)^n\qquad b>0$$
  $$a_n=n!\cdot\left( \frac{b}{n} \right)^n$$
  $$\frac{a_{n+1}}{a_n}=\frac{\left( n+1 \right)!}{n!}\cdot\left( \frac{b}{n+1} \right)^{n+1}\cdot\left( \frac{n}{b} \right)^n=b\cdot\left( n+1 \right)\cdot \frac{1}{n+1}\cdot\left( \frac{n}{n+1} \right)^n=\frac{b}{\left( 1+\frac{1}{n} \right)^n}\xrightarrow{\infty}\frac{b}{e}$$
  La serie converge se $b<e$, in quanto $\nicefrac{b}{e}<1$. La serie diverge se $b>e$, in quanto $\nicefrac{b}{e}>1$.
  Se $b=e$, usare la formula di Stirling. % TODO: da continuare
\end{example}

\begin{observation}
  Per le serie a termini non negativi $a_n\ge0$ le somme parziali $s_N$ costituiscono una successione $\left\{ s_N \right\}\subset\reals$ monotona crescente, che ha limite finito o $+\infty$: tali serie, quindi, non possono essere indeterminate.
\end{observation}

\begin{definition}[Convergenza assoluta]
  La serie $\sum a_n$ è detta assolutamente convergente se la serie $\sum\abs{a_n}$ converge.
\end{definition}

\begin{theorem}
  Se $\sum a_n$ converge assolutamente, allora converge semplicemente e, in particolare:
  $$\abs{\sum_na_n}\le\sum_n\abs{a_n}$$
\end{theorem}
\begin{proof}
  Si considerano le somme parziali dei termini positivi e di quelli negativi separatamente:
  $$s_N^+\walrus \sum_{n=0}^Na_n:a_n\ge0\qquad s_N^-\walrus \sum_{n=0}^N\left( -a_n \right):a_n\le0$$
  $$s_N^+\ge0\qquad s_N^-\ge0$$
  $$s_N=s_N^+-s_N^-$$
  Per le somme parziali dei termini positivi si osserva che:
  $$s_N^+=\sum_{\substack{n=0\\a_n\ge0}}^N\abs{a_n}\le\sum_{n=0}^N\abs{a_n}<+\infty$$
  Essendo $\left\{ s_N^+ \right\}$ monotona crescente e superiormente limitata, per il criterio di convergenza delle successioni monotone limitate, si ha:
  $$\exists\ s^+\walrus\lim_{N\to+\infty}s_N^+\le\sum_{n=0}^\infty\abs{a_n}<+\infty$$
  Per le somme parziali dei termini negativi si osserva che:
  $$s_N^-=\sum_{\substack{n=0\\a_n\le0}}^N\abs{a_n}\le\sum_{n=0}^N\abs{a_n}<+\infty$$
  Essendo $\left\{ s_N^- \right\}$ monotona crescente e superiormente limitata, per il criterio di convergenza delle successioni monotone limitate, si ha:
  $$\exists\ s^-\walrus\lim_{N\to+\infty}s_N^-\le\sum_{n=0}^\infty\abs{a_n}<+\infty$$
  Per le proprietà dei limiti, si ha:
  $$\exists\ \lim_{N\to+\infty}s_N=\lim_{N\to+\infty}\left( s_N^+-s_N^- \right)=\lim_{N\to+\infty}s_N^+-\lim_{N\to+\infty}s_N^-=s^+-s^-\in\reals$$
\end{proof}

\begin{example}
  $$\sum_{n\ge1}\frac{\sin n}{n^2}$$
  $$a_n=\frac{\sin n}{n^2}$$
  $$\abs{a_n}=\abs{\frac{\sin n}{n^2}}\le\frac{1}{n^2}$$
  Per il teorema del confronto, si ottiene:
  $$\sum_{n=1}^\infty\abs{a_n}\le\sum_{n=1}^\infty\frac{1}{n^2}<+\infty$$
  Pertanto, $\sum a_n$ converge assolutamente e, per il teorema di cui sopra, anche semplicemente.
\end{example}

\begin{theorem}[Criterio di Leibnitz]
  Sia data una serie nella seguente forma:
  $$\sum_{n=0}^\infty\left( -1 \right)^na_n$$
  Se $a_n\ge0$ e $\lim a_n=0$ e $\left\{ a_n \right\}$ è definitivamente decrescente\footnote{$\exists\ n_0\ge0:n>n_0\impl a_n\ge a_{n+1}$}, allora la serie converge:
  $$\exists\ s\walrus \lim_{N\to+\infty}s_N=\sum_{n=0}^\infty\left( -1 \right)^na_n\in\reals$$
  Inoltre $s_{2N}\ge s\ge s_{2N+1}\ \forall N\ge0$, con $\left\{ s_{2N} \right\}$ decrescente e $\left\{ s_{2N+1} \right\}$ crescente.
  In più:
  $$R_N\walrus \sum_{n=N}^\infty\left( -1 \right)^na_n$$
  è una serie convergente e $\abs{R_N}\le a_N\ \forall N\ge0$.
\end{theorem}
\begin{proof}
  Poiché $\lim \abs{\left( -1 \right)^na_n}=\lim \abs{a_n}=0$ la condizione necessaria della convergenza è verificata.
  
  \noindent Considerata la successione $\left\{ s_{2N} \right\}$, si nota che essa è decrescente:
  \begin{align*}
    s_0 & =a_0\ge0                                                           \\
    s_2 & =\underbrace{a_0}_{s_0}+\underbrace{a_2-a_1}_{\le0}\le s_0         \\
    s_4 & =\underbrace{a_0-a_1+a_2}_{s_2}+\underbrace{a_4-a_3}_{\le0}\le s_2 \\
        & \vdots                                                             
  \end{align*}
  
  \noindent Considerata la successione $\left\{ s_{2N+1} \right\}$, si nota che essa è crescente:
  \begin{align*}
    s_1 & =a_0-a_1\ge0                                                           \\
    s_3 & =\underbrace{a_0-a_1}_{s_1}+\underbrace{a_2-a_3}_{\ge0}\ge s_1         \\
    s_5 & =\underbrace{a_0-a_1+a_2-a_3}_{s_3}+\underbrace{a_4-a_5}_{\ge0}\ge s_3 \\
        & \vdots                                                                 
  \end{align*}
  Inoltre:
  $$s_{2N+1}=s_{2N}-a_{2N+1}\quad\forall N\ge0$$
  $$s_1\le s_3\le \cdots\le s_{2N+1}\le s_{2N}\le \cdots \le s_2\le s_0$$
  Pertanto, le successioni $\left\{ s_{2N} \right\}$ e $\left\{ s_{2N+1} \right\}$ sono monotone e limitate e, per il teorema di convergenza delle successioni monotone limitate:
  $$\exists\ \lim_{N}s_{2N}\qquad \exists\ \lim_{N}s_{2N+1}$$
  Inoltre:
  $$\lim_N\left( s_{2N+1}-s_{2N} \right)=\lim_N\left( -a_{2N+1} \right)=0$$
  $$\Updownarrow$$
  $$\reals\ni s\walrus\lim_Ns_{2N}=\lim_Ns_{2N+1}$$
  Vale a dire:
  $$\forall \epsilon>0\ \exists\ M_p\left( \epsilon \right)\ge0:2N\ge M_p\left( \epsilon \right)\impl\abs{s_{2N}-s}<\epsilon$$
  $$\forall \epsilon>0\ \exists\ M_d\left( \epsilon \right)\ge0:2N+1\ge M_d\left( \epsilon \right)\impl\abs{s_{2N+1}-s}<\epsilon$$
  Ponendo $M\left( \epsilon \right)\walrus \max\left\{ M_p\left( \epsilon \right),M_d\left( \epsilon \right) \right\}$, si ha:
  $$\forall \epsilon>0\ \exists\ M\left( \epsilon \right)\ge0:N\ge M\left( \epsilon \right)\impl\abs{s_N-s}<\epsilon$$
  ossia:
  $$\exists\ \lim_{N}s_N=s$$
  In particolare:
  $$s_{2N+1}\le s\le s_{2N}\iff s_{2N+1}\le \sum_{n=0}^\infty\left( -1 \right)^na_n\le s_{2N}$$
  
  Per quanto riguarda il resto:
  $$R_N=\sum_{n=N}^\infty\left( -1 \right)^na_n=\sum_{n=0}^\infty\left( -1 \right)^na_n-\sum_{n=0}^{N-1}\left( -1 \right)^na_n=s-s_{N-1}$$
  $$\lim_NR_N=\lim_N\left( s-s_{N-1} \right)=s-\lim_Ns_{N-1}=s-s=0$$
  Usando quanto ottenuto prima:
  $$s_{2N-1}\le s\le s_{2N}\qquad s_{2N+1}\le s\le s_{2N}$$
  Da cui:
  $$0\le s-s_{2N-1}\le s_{2N}-s_{2N-1}=a_{2N}\qquad 0\le s_{2N}-s\le s_{2N}-s_{2N+1}=a_{2N+1}$$
  In ogni caso $\abs{s_N-s}\le a_N$, sia per $N$ pari sia per $N$ dispari, per cui:
  $$\abs{R_N}\le a_{N-1}\quad \forall N\ge0$$
\end{proof}

\begin{example}
  $$\sum_{n=1}^\infty\frac{\left( -1 \right)^n}{n}$$
  $a_n=\nicefrac{1}{n}\ge0$ è decrescente e $\lim_na_n=0$. Allora, per il criterio di Leibnitz, la serie converge.
  Tuttavia la serie non converge assolutamente:
  $$\sum_{n=1}^\infty\abs{\frac{\left( -1 \right)^n}{n}}=\sum_{n=1}^\infty\frac{1}{n}=+\infty$$
\end{example}

\begin{example}
  $$\sum_{n=2}^\infty\left( -1 \right)^n\frac{\ln n}{n}$$
  $$a_n=\frac{\ln n}{n}$$
  $$a_n\ge0\quad \forall n\ge2$$
  $$\lim_na_n=\lim_n\frac{\ln n}{n}=0$$
  Per determinare se $\left\{ a_n \right\}$ è definitivamente descrescente, si usa la funzione ausiliaria:
  $$f\left( x \right)\walrus \frac{\ln x}{x}\qquad f\left( n \right)=\frac{\ln n}{n}=a_n\qquad x\in\reals,n\in\mathbb{N}\setminus\left\{ 0,1 \right\}$$
  $f$ è derivabile poiché combinazione di funzioni derivabili:
  $$f'\left( x \right)=\frac{\nicefrac{1}{x}\cdot x-\ln x\cdot 1}{x^2}=\frac{1-\ln x}{x^2}$$
  $$f'\left( x \right)\le0\iff \ln x\ge 1\iff x\ge e$$
  $f$ è decrescente su $\rintv{e}{+\infty}$, per cui $\left\{ a_n \right\}$ è decrescente per $n\ge e$, ossia per $n\ge3$.
  Per il criterio di Leibnitz, la serie converge semplicemente.
\end{example}

\begin{example}
  $$\sum_{n=1}^\infty\left( -1 \right)^n\frac{\sqrt{n}+\left( -1 \right)^n}{n}=\sum_{n=1}^\infty\left( \frac{\left( -1 \right)^n}{\sqrt{n}}+\frac{1}{n} \right)$$
  La serie $\sum\frac{\left( -1 \right)^n}{\sqrt{n}}$ converge semplicemente per il criterio di Leibnitz (vedi esempio), ma $\sum \frac{1}{n}$ non converge.
  Quindi, la serie originale non converge. 
  
  Infatti, se fosse convergente, si avrebbe:
  $$\sum_{n=1}^\infty\frac{1}{n}=\sum_{n=1}^\infty\left( -1 \right)^n\frac{\sqrt{n}+\left( -1 \right)^n}{n}-\sum_{n=1}^\infty\frac{\left( -1 \right)^n}{\sqrt{n}}$$
  e quindi la serie armonica sarebbe convergente, per linearità.
  Inoltre:
  $$\frac{\sqrt{n}+\left( -1 \right)^n}{n}\ge0$$
  $$\lim_n\frac{\sqrt{n}+\left( -1 \right)^n}{n}=0$$
  $$\frac{\sqrt{n+1}+\left( -1 \right)^n}{n+1}\nleq \frac{\sqrt{n}+\left( -1 \right)^n}{n}\quad \forall n\ge1$$
  Pertanto non tutte le ipotesi del criterio di Leibnitz sono soddisfatte.
\end{example}

\subsection{Serie di Taylor}

\begin{definition}[Serie di Taylor]
  Sia $f\in \mathcal{C}^\infty\left( \ointv{a}{b} \right)$, ossia $f$ è derivabile un numero arbitrario di volte, e $x_0\in\ointv{a}{b}$.
  La serie di Taylor di $f$ in $x_0$ è:
  $$\sum_{n=0}^\infty\frac{f^{\left( n \right)}\left( x_0 \right)}{n!}\left( x-x_0 \right)^n=\lim_{N\to\infty}\sum_{n=0}^N\frac{f^{\left( n \right)}\left( x_0 \right)}{n!}\left( x-x_0 \right)^n=\lim_{N\to\infty}T_{x_0,N}^f\left( x \right)$$
\end{definition}
\begin{observation}
  La serie di Taylor di $f$ converge e il suo valore è $f\left( x \right)$.
\end{observation}
\begin{proof}
  Per il teorema di Taylor con il resto di Lagrange:
  $$f\left( x \right)=T_N\left( x \right)+R_N\left( x \right)\qquad R_N\left( x \right)=\frac{f^{\left( N+1 \right)}\left( c \right)}{\left( N+1 \right)!}\left( x-x_0 \right)^{N+1}$$
  dove $\abs{c-x_0}<\abs{x-x_0}$.
  
  Sia $\reals \ni M_N\walrus \sup\limits_{x\in\ointv{a}{b}} \abs{f^{\left( N \right)}\left( x \right)}$. Si osserva che:
  $$\abs{f\left( x \right)-T_N\left( x \right)}=\abs{R_N\left( x \right)}\le\frac{M_{N+1}}{\left( N+1 \right)!}\left( b-a \right)^{N+1}\quad \forall x\in\ointv{a}{b}$$
  Quindi:
  $$\lim_{N\to\infty}\frac{M_{N+1}}{\left( N+1 \right)!}\left( b-a \right)^{N+1}=0\impl f\left( x \right)=T_N\left( x \right)$$
  vale a dire che la serie di Taylor converge e la sua somma è proprio $f\left( x \right)$:
  $$\sum_{n=0}^\infty\frac{f^{\left( n \right)}\left( x_0 \right)}{n!}\left( x-x_0 \right)^n=f\left( x \right)\quad \forall x\in\ointv{a}{b}$$
  La condizione sufficiente è verificata, per la formula di Stirling, se $M_N\le k^N\ k\in\reals^+$.
\end{proof}

\begin{example}
  $$f\left( x \right)=e^x\qquad x_0=0$$
  $$f^{\left( n \right)}\left( x \right)=e^x\quad \forall n\ge0$$
  $$f^{\left( n \right)}\left( x_0 \right)=e^0=1\quad \forall n\ge0$$
  $$M_N\walrus \sup_{x\in\ointv{a}{b}}\abs{f^{\left( n \right)}\left( x \right)}=1\quad \forall N\ge0,\forall\ \ointv{a}{b}\subset\reals$$
  $$e^x=\sum_{n=0}^\infty\frac{x^n}{n!}$$
  $$e^1=\sum_{n=0}^\infty\frac{1}{n!}\qquad e^{-1}=\frac{1}{e}=\sum_{n=0}^\infty\frac{\left( -1 \right)^n}{n!}$$
\end{example}

\begin{example}
  $$f\left( x \right)=\sin x\qquad x_0=0$$
  $$
    f^{\left( n \right)}=
    \begin{cases}
      \sin x  & n\equiv 0\mod{4} \\
      \cos x  & n\equiv 1\mod{4} \\
      -\sin x & n\equiv 2\mod{4} \\
      -\cos x & n\equiv 3\mod{4} \\
    \end{cases}
  $$
  $$M_N=1\quad \forall N\ge0$$
  $$\sin x=\sum_{n=0}^\infty\left( -1 \right)^n\frac{x^{2n+1}}{\left( 2n+1 \right)!}$$
\end{example}

\begin{example}
  $$f\left( x \right)=\cos x\qquad x_0=0$$
  $$
    f^{\left( n \right)}=
    \begin{cases}
      \cos x  & n\equiv 0\mod{4} \\
      -\sin x & n\equiv 1\mod{4} \\
      -\cos x & n\equiv 2\mod{4} \\
      \sin x  & n\equiv 3\mod{4} \\
    \end{cases}
  $$
  $$M_N=1\quad \forall N\ge0$$
  $$\cos x=\sum_{n=0}^\infty\left( -1 \right)^n\frac{x^{2n}}{\left( 2n \right)!}$$
\end{example}

\begin{example}
  $$f\left( x \right)=\ln\left( 1+x \right)\qquad x_0=0$$
  $$f^{\left( n \right)}\left( x \right)=\left( -1 \right)^{n-1}\left( n-1 \right)!\left( 1+x \right)^{-n}$$
  $$M_N\le \left( N-1 \right)!\left( 1+a \right)^{-N}\quad \forall x\in\ointv{a}{b}\subset\ointv{-1}{+\infty}$$
  $$\ln\left( 1+x \right)=\sum_{n=1}^\infty\left( -1 \right)^{n-1}\frac{x^n}{n}=-\sum_{n=1}^\infty\left( -1 \right)^n\frac{x^n}{n}$$
\end{example}

\begin{example}
  $$\frac{1}{1-x}=\sum_{n=0}^\infty x^n\qquad\abs{x}<1$$
\end{example}

\begin{theorem}[Formula di Eulero]
  Sia $\t\in\reals$. Vale la seguente identità:
  $$e^{i\t}=\cos\t+i \sin\t$$
\end{theorem}
\begin{proof}
  $$e^{z}=\sum_{n=0}^\infty\frac{z^n}{n!}\qquad z\in\complex$$
  $$\lim_n\abs{e^z-\sum_{n=0}^N\frac{z^n}{n!}}=0$$
  Sia $z=i\t$. Allora:
  \begin{align*}
    e^{z} & =e^{i\t}                                                                                                                                            \\
          & =\sum_{n=0}^\infty\frac{\left( i\t \right)^n}{n!}                                                                                                   \\
          & =\sum_{n=0}^\infty\frac{\left( i\t \right)^{2n}}{\left( 2n \right)!}+\sum_{n=0}^\infty\frac{\left( i\t \right)^{2n+1}}{\left( 2n+1 \right)!}        \\
          & =\sum_{n=0}^\infty\left( -1 \right)^n\frac{\t^{2n}}{\left( 2n \right)!}+i\sum_{n=0}^\infty\left( -1 \right)^n\frac{\t^{2n+1}}{\left( 2n+1 \right)!} \\
          & =\cos\t+i\sin\t                                                                                                                                     
  \end{align*}
\end{proof}


\end{document}