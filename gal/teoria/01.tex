\section{Vettori}

\begin{definition}[Vettore]
  Si definisce \textbf{vettore} un insieme ordinato di $n$ elementi.
\end{definition}

Un vettore si indica con la notazione:
$$\(x_1,x_2,x_3,\dots,x_n\)$$
dove $n$ è il numero di elementi. Un vettore avente 2 elementi prende il nome di ``2--vettore'', quello che ne ha 3 ``3--vettore'' e così via.

\begin{definition}[Vettore nullo]
  Si definisce \textbf{vettore nullo} e si indica con $\vec{0}$ un vettore che ha come elementi solo 0.
  $$\vec{0}\walrus\(0,0,0,\dots,0\)$$
\end{definition}

\subsection{Operazioni}

\begin{definition}[Somma di vettori]
  La somma di due vettori aventi lo stesso numero di elementi è un vettore avente la loro stessa dimensione e, per ogni elemento, la somma dei rispettivi elementi.
  $$\(x_1,x_2,\dots,x_n\)+\(y_1,y_2,\dots,y_n\)\walrus\(x_1+y_1,x_2+y_2,\dots,x_n+y_n\)$$
\end{definition}
\begin{example}
  $$\(1,2\)+\(3,4\)=\(4,6\)$$
\end{example}

\begin{definition}[Prodotto per uno scalare]
  Il prodotto di uno scalare per un vettore è un vettore avente la stessa dimensione e, per ogni elemento, il prodotto dell'elemento per lo scalare.
  $$t\(x_1,x_2,\dots,x_n\)\walrus\(tx_1,tx_2,\dots,tx_n\)$$
\end{definition}
\begin{example}
  $$2\(2,5,-1\)=\(4,10,-2\)$$
\end{example}

\begin{definition}[Vettore opposto]
  Si definisce \textbf{vettore opposto} e si indica con $-v$ il prodotto scalare del vettore $v$ per $-1$.
  $$-v\walrus\(-1\)v$$
\end{definition}
\begin{example}
  $$-\(1,-5\)=\(-1,5\)$$
\end{example}

\begin{definition}[Differenza tra vettori]
  La differenza fra due vettori della stessa dimensione è la somma del primo più l'opposto del secondo.
  $$v_1-v_2\walrus v_1+\(-v_2\)$$
\end{definition}
\begin{example}
  $$\(10,7,8\)-\(2,9,1\)=\(8,-2,7\)$$
\end{example}

\subsection{Proprietà}

I vettori godono di alcune proprietà:
\paragraph*{Proprietà associativa}
$$\(v_1+v_2\)+v_3=v_1+\(v_2+v_3\)$$
\begin{proof}
  \begin{align*}
    \(v_1+v_2\)+v_3 & = \(\(a_1,a_2,\dots,a_n\)+\(b_1,b_2,\dots,b_n\)\) + \(c_1,c_2,\dots,c_n\) \\
                    & =\(a_1+b_1,a_2+b_2,\dots,a_n+b_n\)+\(c_1,c_2,\dots,c_n\)                  \\
                    & =\(a_1+b_1+c_1,a_2+b_2+c_2,\dots,a_n+b_n+c_n\)                            
  \end{align*}
  \begin{align*}
    v_1+\(v_2+v_3\) & = \(a_1,a_2,\dots,a_n\)+\(\(b_1,b_2,\dots,b_n\) + \(c_1,c_2,\dots,c_n\)\) \\
                    & =\(a_1,a_2,\dots,a_n\)+\(b_1+c_1,b_2+c_2,\dots,b_n+c_n\)                  \\
                    & =\(a_1+b_1+c_1,a_2+b_2+c_2,\dots,a_n+b_n+c_n\)                            
  \end{align*}
\end{proof}
\paragraph*{Proprietà commutativa}
$$v_1+v_2=v_2+v_1$$
\begin{proof}
  \begin{align*}
    v_1+v_2 & =\(a_1,a_2,\dots,a_n\)+\(b_1,b_2,\dots,b_n\) \\
            & =\(a_1+b_1,a_2+b_2,\dots,a_n+b_n\)           \\
            & =\(b_1+a_1,b_2+a_2,\dots,b_n+a_n\)           \\
            & =v_2+v_1                                     
  \end{align*}
\end{proof}
\paragraph*{Esistenza dell'elemento neutro}
$$v+\vec{0}=\vec{0}+v=v$$
\begin{proof}
  \begin{align*}
    v+\vec{0} & =\(a_1,a_2,\dots,a_n\)+\(0,0,\dots,0\) \\
              & =\(a_1+0,a_2+0,\dots,a_n+0\)           \\
              & =\(a_1,a_2,\dots,a_n\)                 \\
              & =v                                     
  \end{align*}
  Per la proprietà commutativa, anche $\vec{0}+v = v$.
\end{proof}
\paragraph*{Esistenza dell'opposto}
$$v+\(-v\)=\vec{0}$$
\begin{proof}
  \begin{align*}
    v+(-v) & =\(a_1,a_2,\dots,a_n\)+\(-a_1,-a_2,\dots,-a_n\) \\
           & =\(a_1-a_1,a_2-a_2,\dots,a_n-a_n\)              \\
           & =\(0,0,\dots,0\)                                \\
           & =\vec{0}                                        
  \end{align*}
\end{proof}
\paragraph*{Proprietà distributiva 1}
$$t\(v_1+v_2\)=tv_1+tv_2$$
\begin{proof}
  \begin{align*}
    t\(v_1+v_2\) & =t\(\(a_1,a_2,\dots,a_n\)+\(b_1,b_2,\dots,b_n\)\)  \\
                 & =t\(a_1+b_1,a_2+b_2,\dots,a_n+b_n\)                \\
                 & =\(t\(a_1+b_1\),t\(a_2+b_2\),\dots,t\(a_n+b_n\)\)  \\
                 & =\(ta_1+tb_1,ta_2+tb_2,\dots,ta_n+tb_n\)           \\
                 & =\(ta_1,ta_2,\dots,ta_n\)+\(tb_1,tb_2,\dots,tb_n\) \\
                 & =t\(a_1,a_2,\dots,a_n\)+t\(b_1,b_2,\dots,b_n\)     \\
                 & =tv_1+tv_2                                         
  \end{align*}
\end{proof}
\paragraph*{Proprietà distributiva 2}
$$\(t_1+t_2\)v=t_1v+t_2v$$
\begin{proof}
  \begin{align*}
    \(t_1+t_2\)v & =\(t_1+t_2\)\(a_1,a_2,\dots,a_n\)                              \\
                 & =\(\(t_1+t_2\)a_1,\(t_1+t_2\)a_2,\dots,\(t_1+t_2\)a_n\)        \\
                 & =\(t_1a_1+t_2a_1,t_1a_2+t_2a_2,\dots,t_1a_n+t_2a_n\)           \\
                 & =\(t_1a_1,t_1a_2,\dots,t_1a_n\)+\(t_2a_1,t_2a_2,\dots,t_2a_n\) \\
                 & =t_1\(a_1,a_2,\dots,a_n\)+t_2\(a_1,a_2,\dots,a_n\)             \\
                 & =t_1v+t_2v                                                     
  \end{align*}
\end{proof}
\paragraph*{Proprietà associativa mista}
$$\(ts\)v=t\(sv\)$$
\begin{proof}
  \begin{align*}
    \(ts\)v & =\(ts\)\(a_1,a_2,\dots,a_n\) \\
            & =\(tsa_1,tsa_2,\dots,tsa_n\) 
  \end{align*}
  \begin{align*}
    t\(sv\) & =t\(s\(a_1,a_2,\dots,a_n\)\) \\
            & =t\(sa_1,sa_2,\dots,sa_n\)   \\
            & =\(tsa_1,tsa_2,\dots,tsa_n\) 
  \end{align*}
\end{proof}
\paragraph*{Legge di unità}
$$1v=v$$
\begin{proof}
  $$1v=1\(a_1,a_2,\dots,a_n\)=\(1\cdot a_1,1\cdot a_2,\dots,1\cdot a_n\)=\(a_1,a_2,\dots,a_n\)=v$$
\end{proof}
