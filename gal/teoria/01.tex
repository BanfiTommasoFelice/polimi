% lezione del 12/09/2022

\section{Vettori}

\begin{definition}[Vettore]
  Si definisce \textbf{vettore} un insieme ordinato di $n$ elementi.
\end{definition}

Un vettore si indica con la notazione:
$$\(x_1,x_2,x_3,...,x_n\)$$
dove $n$ è il numero di elementi. Un vettore avente 2 elementi prende il nome di ``2--vettore'', quello che ne ha 3 ``3--vettore'' e così via.

\begin{definition}[Somma di vettori]
  La somma di due vettori aventi lo stesso numero di elementi è un vettore avente la loro stessa dimensione e, per ogni elemento, la somma dei rispettivi elementi.
  $$\(x_1,x_2,...,x_n\)+\(y_1,y_2,...,y_n\)=\(x_1+y_1,x_2+y_2,...,x_n+y_n\)$$
\end{definition}

\begin{example}
  $$\(1,2\)+\(3,4\)=\(4,6\)$$
\end{example}

\begin{definition}[Prodotto per uno scalare]
  Il prodotto di uno scalare per un vettore è un vettore avente la stessa dimensione e, per ogni elemento, il prodotto dell'elemento per lo scalare.
  $$t\(x_1,x_2,...,x_n\)=\(tx_1,tx_2,...,tx_n\)$$
\end{definition}

\begin{example}
  $$2\(2,5,-1\)=\(4,10,-2\)$$
\end{example}

\begin{definition}[Vettore nullo]
  Si definisce \textbf{vettore nullo} e si indica con $\vec{0}$ un vettore che ha come elementi solo 0.
  $$\vec{0}=\(0,0,0,...,0\)$$
\end{definition}

\begin{definition}[Vettore opposto]
  Si definisce \textbf{vettore opposto} e si indica con $-v$ il prodotto scalare del vettore $v$ per $-1$.
  $$-v=\(-1\)v$$
\end{definition}

\begin{example}
  $$-\(1,-5\)=\(-1,5\)$$
\end{example}

I vettori godono di alcune proprietà:
\paragraph*{Proprietà associativa}
$$\(v_1+v_2\)+v_3=v_1+\(v_2+v_3\)$$
\paragraph*{Proprietà commutativa}
$$v_1+v_2=v_2+v_1$$
\paragraph*{Esistenza dell'elemento neutro}
$$v+\vec{0}=\vec{0}+v=v$$
\paragraph*{Esistenza dell'opposto}
$$v+\(-v\)=\vec{0}$$
\paragraph*{Proprietà distributiva 1}
$$t\(v_1+v_2\)=tv_1+tv_2$$
\paragraph*{Proprietà distributiva 2}
$$\(t_1+t_2\)v=t_1v+t_2v$$
\paragraph*{Proprietà associativa mista}
$$\(ts\)v=t\(sv\)$$
\paragraph*{Legge di unità}
$$1v=v$$

\begin{definition}[Differenza tra vettori]
  La differenza fra due vettori della stessa dimensione è la somma del primo più l'opposto del secondo.
  $$v_1-v_2=v_1+\(-v_2\)$$
\end{definition}
\begin{example}
  $$\(10,7,8\)-\(2,9,1\)=\(8,-2,7\)$$
\end{example}

\section{Matrici}

\begin{definition}[Matrice]
  Si definisce \textbf{matrice} $n\times m$ una tabella avente $n$ righe e $m$ colonne.
\end{definition}
\begin{example}
  $$
    \begin{pmatrix}
      1 & 2 & 3 \\
      4 & 5 & 6 
    \end{pmatrix}
  $$
\end{example}
\begin{definition}[Matrice quadrata]
  Si definisce \textbf{matrice quadrata} una matrice avente lo stesso numero di righe e colonne.
\end{definition}
\begin{example}
  $$
    \begin{pmatrix}
      1 & 2 \\
      3 & 4 
    \end{pmatrix}
  $$
\end{example}

Nel caso particolare di una matrice $0\times 0$, la matrice prende il nome di \textbf{matrice vuota}. Le matrici $1\times m$ sono gli $m$--vettori. Le matrici $n\times 1$ prendono il nome di $n$--vettori colonna e sono equiparati agli $n$--vettori.

Una matrice generica $n\times m$ si indica nel seguente modo:
$$\(a_{ij}\)$$

La matrice avente tutti gli elementi pari a 0 prende il nome di \textbf{matrice nulla}:
$$
  \begin{pmatrix}
    0 & 0 \\
    0 & 0 
  \end{pmatrix}
$$

\begin{definition}[Matrice identità]
  Si definisce \textbf{matrice identità} una matrice quadrata che ha tutti gli elementi nulli, eccetto quelli sulla diagonale, che sono 1.
\end{definition}
\begin{example}
  $$
    \begin{pmatrix}
      1 & 0 & 0 \\
      0 & 1 & 0 \\
      0 & 0 & 1 
    \end{pmatrix}
  $$
\end{example}

La matrice identità di lunghezza $n$ prende il nome di \textbf{matrice identità di ordine $n$} e si indica con $I_n$.

\begin{definition}[Somma di matrici]
  La somma di due matrici della stessa dimensione è una matrice della loro dimensione dove ogni elemento è la somma dei rispettivi altri due.
  $$\(a_{ij}\)+\(b_{ij}\)=\(a_{ij}+b_{ij}\)$$
\end{definition}

\begin{example}
  $$
    \begin{pmatrix}
      1  & -2 & 3 \\
      -1 & 0  & 2 
    \end{pmatrix}
    +
    \begin{pmatrix}
      5  & 1  & 2 \\
      -2 & -1 & 4 
    \end{pmatrix}
    =
    \begin{pmatrix}
      6  & -1 & 5 \\
      -3 & -1 & 6 
    \end{pmatrix}
  $$
\end{example}

\begin{definition}[Prodotto per uno scalare]
  Il prodotto di una matrice per uno scalare è una matrice avente la stessa dimensione e dove ogni elemento è il prodotto dell'elemento per lo scalare.
  $$t\(a_{ij}\)=\(ta_{ij}\)$$
\end{definition}
\begin{example}
  $$
    3
    \begin{pmatrix}
      1  & -2 & 3 \\
      -1 & 0  & 2 
    \end{pmatrix}
    =
    \begin{pmatrix}
      3  & -6 & 9 \\
      -3 & 0  & 6 
    \end{pmatrix}
  $$
\end{example}

Le proprietà di cui godono i vettori valgono anche per le matrici.

\begin{definition}[Matrici conformabili]
  Una matrice $A$ si dice conformabile ad una matrice $B$ se il numero di colonne di $A$ è uguale al numero di righe di $B$.
\end{definition}
\begin{example}
  $$
    \begin{pmatrix}
      1 & 2 & 3 \\
      4 & 5 & 6 
    \end{pmatrix}
    \text{conformabile a}
    \begin{pmatrix}
      1 & 2 \\
      3 & 4 \\
      5 & 6 
    \end{pmatrix}
  $$
\end{example}

\begin{definition}[Prodotto tra matrici]
  Il prodotto di $A\left( n\times m \right)$ e $B\left( m\times k \right)$, con $A$ conformabile a $B$, è una nuova matrice $C\left( n\times k \right)$ i cui elementi sono dati da:
  $$c_{ij}=\sum_{t=1}^ma_{it}b_{tj}$$
\end{definition}
\begin{example}
  $$
    \begin{pmatrix}
      0 & 2 & -2 \\
      1 & 0 & 1  
    \end{pmatrix}
    \cdot
    \begin{pmatrix}
      1 & 1 & 1  & 1  \\
      0 & 1 & 1  & 0  \\
      3 & 3 & -1 & -3 
    \end{pmatrix}
    =
    \begin{pmatrix}
      -6 & -4 & 4 & 6  \\
      4  & 4  & 0 & -2 
    \end{pmatrix}
  $$
  $$c_{11}=\begin{pmatrix}0&2&-2\end{pmatrix}\cdot\begin{pmatrix}1&0&3\end{pmatrix}=0+0-6=-6$$
  $$c_{12}=\begin{pmatrix}0&2&-2\end{pmatrix}\cdot\begin{pmatrix}1&1&3\end{pmatrix}=0+2-6=-4$$
  $$c_{13}=\begin{pmatrix}0&2&-2\end{pmatrix}\cdot\begin{pmatrix}1&1&-1\end{pmatrix}=0+2+2=4$$
  $$c_{14}=\begin{pmatrix}0&2&-2\end{pmatrix}\cdot\begin{pmatrix}1&0&-3\end{pmatrix}=0+0+6=6$$
  $$c_{21}=\begin{pmatrix}1&0&1\end{pmatrix}\cdot\begin{pmatrix}1&0&3\end{pmatrix}=1+0+3=4$$
  $$c_{22}=\begin{pmatrix}1&0&1\end{pmatrix}\cdot\begin{pmatrix}1&1&3\end{pmatrix}=1+0+3=4$$
  $$c_{23}=\begin{pmatrix}1&0&1\end{pmatrix}\cdot\begin{pmatrix}1&1&-1\end{pmatrix}=1+0-1=0$$
  $$c_{24}=\begin{pmatrix}1&0&1\end{pmatrix}\cdot\begin{pmatrix}1&0&-3\end{pmatrix}=1+0-3=-2$$
\end{example}

% lezione del 14/09/2022

Il prodotto ``riga per colonna'' gode di alcune proprietà:

\paragraph*{Proprietà associativa}
$$\(AB\)C=A\(BC\)$$
\paragraph*{Proprietà distributiva 1}
$$A\(B+C\)=AB+AC$$
\paragraph*{Proprietà distributiva 2}
$$\(A+B\)C=AC+BC$$
\paragraph*{Proprietà associativa mista}
$$t\(AB\)=\(tA\)B=A\(tB\)$$
\paragraph*{Proprietà speciale della matrice identità}
Sia $A$ una matrice $n\times m$
$$A=I_nA=AI_m$$

\begin{definition}[Matrice trasposta]
  La \textbf{trasposta} di una matrice $A$ è la matrice che si ottiene scambiando le righe con le colonne e si indica con $A^T$ o $A'$.
\end{definition}
\begin{example}
  $$
    A=\begin{pmatrix}
      1 & 2 & 3 \\
      4 & 5 & 6 
    \end{pmatrix}
  $$
  $$A'=
    \begin{pmatrix}
      1 & 4 \\
      2 & 5 \\
      3 & 6 
    \end{pmatrix}
  $$
\end{example}

Una matrice $A$ si dice \textbf{simmetrica} se $A=A'$. Banalmente, se una matrice è simmetrica, allora sarà anche quadrata:
$$\(a_{ij}\)=\(a_{ji}\)$$

Una matrice $A$ si dice \textbf{antisimmetrica} o emisimmetrica se $A=-A'$. Se una matrice è antisimmetrica, allora sarà quadrata e avrà la diagonale nulla:
$$\(a_{ij}\)=-\(a_{ji}\)$$
$$a_{ii}=-a_{ii}\impl a_{ii}=0$$

La trasposta gode di alcune proprietà:

$$\(A+B\)'=A'+B'$$
$$\(sA\)'=sA'$$
$$\(A'\)'=A$$
$$\(AB\)'=B'A'$$

Se una matrice $A$ è quadrata, allora ha senso definire $A^2=AA$ o, più in generale:
$$A^n=A\cdot A\cdot A\cdots A$$
Analogamente a quanto avviene per le potenze reali
$$A^0=I$$
Inoltre, valgono le proprietà:
$$A^nA^m=A^{n+m}$$
$$\(A^n\)^m=A^{nm}$$

\begin{definition}[Matrice invertibile]
  Una matrice quadrata $A$ si dice \textbf{invertibile} se esiste una matrice quadrata $B$ tale che
  $$AB=BA=I$$
  e tale matrice si indica con $A^{-1}$:
  $$AA^{-1}=A^{-1}A=I$$
\end{definition}

\begin{theorem}
  Se la matrice $A$ è invertibile, allora la matrice $A^{-1}$ è unica.
\end{theorem}
\begin{proof}
  Siano $A,B$ due matrici quadrate per cui $AB=BA=I$. Per assurdo, si suppone che anche un'altra matrice $C$ soddisfa la medesima equazione: $AC=CA=I$. 
  Per le proprietà di sopra, si ha:
  $$B=BI=B(AC)=(BA)C=IC=C$$
  Per la catena di uguaglianze, si ha che $B=C$, pertanto $B$ è unica.
\end{proof}

Se $A$ è una matrice invertibile, allora si pone
$$A^{-n}=\(A^{-1}\)^n$$

\begin{example}
  $$A=
    \begin{pmatrix}
      0 & 1 \\
      1 & 0 
    \end{pmatrix}
  $$
  $$A^2=
    \begin{pmatrix}
      0 & 1 \\
      1 & 0 
    \end{pmatrix}
    \begin{pmatrix}
      0 & 1 \\
      1 & 0 
    \end{pmatrix}=
    \begin{pmatrix}
      1 & 0 \\
      0 & 1 
    \end{pmatrix}
  $$
  $$A^2=I\impl A^{-1}$$
  $$A^3=A^2A=IA=A$$
  $$A^{123}=A$$
  $$A^{-72}=I$$
\end{example}

\begin{example}
  $$A=\begin{pmatrix}
      0 & 1 & 0 \\
      0 & 0 & 1 \\
      0 & 0 & 0 
    \end{pmatrix}$$
  $$A^2=\begin{pmatrix}
      0 & 1 & 0 \\
      0 & 0 & 1 \\
      0 & 0 & 0 
    \end{pmatrix}\begin{pmatrix}
      0 & 1 & 0 \\
      0 & 0 & 1 \\
      0 & 0 & 0 
    \end{pmatrix}=\begin{pmatrix}
      0 & 0 & 1 \\
      0 & 0 & 0 \\
      0 & 0 & 0 
    \end{pmatrix}$$
  $$A^3=AA^2=\begin{pmatrix}
      0 & 1 & 0 \\
      0 & 0 & 1 \\
      0 & 0 & 0 
    \end{pmatrix}\begin{pmatrix}
      0 & 0 & 1 \\
      0 & 0 & 0 \\
      0 & 0 & 0 
    \end{pmatrix}=\begin{pmatrix}
      0 & 0 & 0 \\
      0 & 0 & 0 \\
      0 & 0 & 0 
    \end{pmatrix}=\vec{0}$$
  $$A^{123}=A^3A^{120}=\vec{0}A^{120}=\vec{0}$$
  $$A^3A^{-1}=A^2$$
  $$A^3A^{-1}=\vec{0}$$
  $$A^2\neq \vec{0}\impl \nexists A^{-1}$$
\end{example}

\begin{definition}[Determinante]
  Il \textbf{determinante} è un numero che viene assegnato ad ogni matrice quadrata.
\end{definition}

Il determinante di una matrice quadrata di ordine 1 è l'unico elemento della matrice stessa:
$$\det\begin{pmatrix}a\end{pmatrix}=a$$
Il determinate di una matrice quadrata di ordine 2 è il prodotto di una diagonale meno il prodotto dell'altra diagonale:
$$\det\begin{pmatrix}a & b \\ c & d\end{pmatrix}=ad-bc$$
Il determinante di un matrice quadrata di ordine 3 è dato dalla formula:
$$\det\begin{pmatrix}a_{11}&a_{12}&a_{13}\\a_{21}&a_{22}&a_{23}\\a_{31}&a_{32}&a_{33}\end{pmatrix}=a_{11}a_{22}a_{33}+a_{12}a_{23}a_{31}+a_{13}a_{21}a_{32}-a_{31}a_{22}a_{13}-a_{32}a_{23}a_{11}-a_{33}a_{21}a_{12}$$

\paragraph*{Regola di Sarrus}
Poiché è difficile ricordare la formula per il calcolo del determinante di una matrice quadrata di ordine 3, si ricorre solitamente alla regola di Sarrus, che consta di un ampliamento della matrice originale, raddoppiandola a destra:
$$
  \begin{pmatrix}
    a_{11} & a_{12} & a_{13} \\
    a_{21} & a_{22} & a_{23} \\
    a_{31} & a_{32} & a_{33} \\
  \end{pmatrix}
  \mapsto 
  \begin{pmatrix}
    a_{11} & a_{12} & a_{13} & a_{11} & a_{12} & a_{13} \\
    a_{21} & a_{22} & a_{23} & a_{21} & a_{22} & a_{23} \\
    a_{31} & a_{32} & a_{33} & a_{31} & a_{32} & a_{33} \\
  \end{pmatrix}
$$
Il determinante è la somma dei prodotti sulle diagonali discendenti meno la somma dei prodotti sulle diagonali ascendenti.

\begin{example}
  $$
    \det
    \begin{pmatrix}
      1 & 2 & -1 \\
      0 & 2 & -1 \\
      1 & 1 & 0  
    \end{pmatrix}=
    0-2+0+2+1-0=1
  $$
\end{example}

\subsection*{Sviluppo di Laplace}
Lo sviluppo di Laplace è una formula ricorsiva che permette di calcolare il determinante di una matrice di ordine $n$ ricorrendo al calcolo del determinante di una matrice di ordine $n-1$.

\begin{definition}[Sottomatrice]
  Si definisce sottomatrice di $A$ una matrice ottenuta togliendo ad $A$ un certo numero di righe e/o un certo numero di colonne. Se la sottomatrice è quadrata di ordine $m$ allora prende il nome di minore di ordine $m$.
\end{definition}

\begin{definition}[Minore complementare]
  Il minore complementare $M_{ij}$ dell'elemento $a_{ij}$ è il minore che si ottiene togliendo ad $A$ la $i$--esima riga e la $j$--esima colonna.
\end{definition}

\begin{definition}[Complemento algebrico]
  Il complemento algebrico di $a_{ij}$ è :
  $$A_{ij}=\(-1\)^{i+j}\det M_{ij}$$
\end{definition}

\begin{example}
  $$A=\begin{pmatrix}
      1 & 0  & 1 \\
      2 & 1  & 0 \\
      1 & -1 & 1 
    \end{pmatrix}$$
  $$M_{22}=\begin{pmatrix}
      1 & 1 \\
      1 & 1 
    \end{pmatrix}$$
  $$A_{22}=\(-1\)^4
    \det \begin{pmatrix}
      1 & 1 \\
      1 & 1 
    \end{pmatrix}=1-1=0$$
\end{example}

\begin{theorem}
  Sia $A$ una matrice $n\times n$. Fissato $i$, il determinante di $A$ è:
  $$\det A=\sum_{j=1}^na_{ij}A_{ij}$$
\end{theorem}

\begin{example}
  \begin{align*}
    \det
    \begin{pmatrix}
      1 & 2 & -1 & 1 \\
      2 & 0 & 1  & 0 \\
      1 & 1 & 1  & 1 \\
      0 & 2 & 1  & 0 
    \end{pmatrix} & =     
    -2\det
    \begin{pmatrix}
      2 & -1 & 1 \\
      1 & 1  & 1 \\
      2 & 1  & 0 
    \end{pmatrix}
    -
    \det
    \begin{pmatrix}
      1 & 2 & 1 \\
      1 & 1 & 1 \\
      0 & 2 & 0 
    \end{pmatrix}        \\
                      & = 
    -2\(2\det
    \begin{pmatrix}
      -1 & 1 \\
      1  & 1 
    \end{pmatrix}
    -
    \det
    \begin{pmatrix}
      2 & 1 \\
      1 & 1 
    \end{pmatrix}
    \)-\(-2\det
    \begin{pmatrix}
      1&1\\
      1&1
    \end{pmatrix}
    \)\\
    &=-2\(2\cdot\(-2\)-1\)-\(-2\cdot0\)\\
    &=10+0=10
  \end{align*}
\end{example}