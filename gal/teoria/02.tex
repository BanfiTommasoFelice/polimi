\section{Matrici}

\begin{definition}[Matrice]
  Si definisce \textbf{matrice} $n\times m$ una tabella avente $n$ righe e $m$ colonne.
\end{definition}
\begin{example}
  $$
    \begin{pmatrix}
      1 & 2 & 3 \\
      4 & 5 & 6 
    \end{pmatrix}
  $$
\end{example}
\begin{definition}[Matrice quadrata]
  Si definisce \textbf{matrice quadrata} una matrice avente lo stesso numero di righe e colonne.
  Il caso particolare di una matrice $0\times 0$ prende il nome di \textbf{matrice vuota}.
\end{definition}
\begin{example}
  $$
    \begin{pmatrix}
      1 & 2 \\
      3 & 4 \\
    \end{pmatrix}
  $$
\end{example}


Le matrici $1\times m$ sono gli $m$--vettori. Le matrici $n\times 1$ prendono il nome di $n$--vettori colonna e sono equiparati agli $n$--vettori.
Una matrice generica $n\times m$ si indica nel seguente modo:
$$\(a_{ij}\)_{\substack{1\le i\le n\\1\le j\le m}}$$
La matrice avente tutti gli elementi pari a 0 prende il nome di \textbf{matrice nulla}:
$$
  \begin{pmatrix}
    0 & 0 \\
    0 & 0 
  \end{pmatrix}
$$

\begin{definition}[Matrice identità]
  Si definisce \textbf{matrice identità} una matrice quadrata che ha tutti gli elementi nulli, eccetto quelli sulla diagonale, che sono 1.
  La matrice identità di lunghezza $n$ prende il nome di \textbf{matrice identità di ordine $n$} e si indica con $I_n$.
\end{definition}
\begin{example}
  $$
    I_3\walrus
    \begin{pmatrix}
      1 & 0 & 0 \\
      0 & 1 & 0 \\
      0 & 0 & 1 
    \end{pmatrix}
  $$
\end{example}

\subsection{Operazioni elementari}

\begin{definition}[Somma di matrici]
  La somma di due matrici della stessa dimensione è una matrice della loro dimensione dove ogni elemento è la somma dei rispettivi altri due.
  $$\(a_{ij}\)+\(b_{ij}\)\walrus\(a_{ij}+b_{ij}\)$$
\end{definition}

\begin{example}
  $$
    \begin{pmatrix}
      1  & -2 & 3 \\
      -1 & 0  & 2 
    \end{pmatrix}
    +
    \begin{pmatrix}
      5  & 1  & 2 \\
      -2 & -1 & 4 
    \end{pmatrix}
    =
    \begin{pmatrix}
      6  & -1 & 5 \\
      -3 & -1 & 6 
    \end{pmatrix}
  $$
\end{example}

\begin{definition}[Prodotto per uno scalare]
  Il prodotto di una matrice per uno scalare è una matrice avente la stessa dimensione e dove ogni elemento è il prodotto dell'elemento per lo scalare.
  $$t\(a_{ij}\)\walrus\(ta_{ij}\)$$
\end{definition}
\begin{example}
  $$
    3
    \begin{pmatrix}
      1  & -2 & 3 \\
      -1 & 0  & 2 
    \end{pmatrix}
    =
    \begin{pmatrix}
      3  & -6 & 9 \\
      -3 & 0  & 6 
    \end{pmatrix}
  $$
\end{example}

\subsubsection*{Proprietà}
Le matrici godono delle stesse proprietà dei vettori. Pertanto:
\begin{itemize}
  \item $\(A+B\)+c=A+\(B+c\)$
  \item $A+B=B+A$
  \item $A+\vec{0}=\vec{0}+A=A$
  \item $A+\(-A\)=\vec{0}$
  \item $t\(A+B\)=tA+tB$
  \item $\(t_1+t_2\)A=t_1A+t_2A$
  \item $\(ts\)A=t\(sA\)$
  \item $1A=A$
\end{itemize}

\subsection{Prodotto}

\begin{definition}[Matrici conformabili]
  Una matrice $A$ si dice \textbf{conformabile} ad una matrice $B$ se il numero di colonne di $A$ è uguale al numero di righe di $B$.
\end{definition}
\begin{example}
  $$
    \begin{pmatrix}
      1 & 2 & 3 \\
      4 & 5 & 6 \\
    \end{pmatrix}
    \text{ conformabile a }
    \begin{pmatrix}
      1 & 2 \\
      3 & 4 \\
      5 & 6 \\
    \end{pmatrix}
  $$
\end{example}

\begin{definition}[Prodotto tra matrici]
  Il prodotto di $A\left( n\times m \right)$ e $B\left( m\times k \right)$, con $A$ conformabile a $B$, è una nuova matrice $C\left( n\times k \right)$ i cui elementi sono dati da:
  $$c_{ij}=\sum_{t=1}^ma_{it}b_{tj}$$
\end{definition}
\begin{example}
  $$
    \begin{pmatrix}
      0 & 2 & -2 \\
      1 & 0 & 1  
    \end{pmatrix}
    \cdot
    \begin{pmatrix}
      1 & 1 & 1  & 1  \\
      0 & 1 & 1  & 0  \\
      3 & 3 & -1 & -3 
    \end{pmatrix}
    =
    \begin{pmatrix}
      -6 & -4 & 4 & 6  \\
      4  & 4  & 0 & -2 
    \end{pmatrix}
  $$
  \begin{center}
    \begin{tblr}{llllllll}
      $c_{11}$ & $=$ & $\(0,2,-2\)$ & $\(1,0,3\)$  & $\mapsto$ & $\(0,0,-6\)$ & $\mapsto$ & $-6$ \\
      $c_{12}$ & $=$ & $\(0,2,-2\)$ & $\(1,1,3\)$  & $\mapsto$ & $\(0,2,-6\)$ & $\mapsto$ & $-4$ \\
      $c_{13}$ & $=$ & $\(0,2,-2\)$ & $\(1,1,-1\)$ & $\mapsto$ & $\(0,2,2\)$  & $\mapsto$ & $4$  \\
      $c_{14}$ & $=$ & $\(0,2,-2\)$ & $\(1,0,-3\)$ & $\mapsto$ & $\(0,0,6\)$  & $\mapsto$ & $6$  \\
      $c_{21}$ & $=$ & $\(1,0,1\)$  & $\(1,0,3\)$  & $\mapsto$ & $\(1,0,3\)$  & $\mapsto$ & $4$  \\
      $c_{22}$ & $=$ & $\(1,0,1\)$  & $\(1,1,3\)$  & $\mapsto$ & $\(1,0,3\)$  & $\mapsto$ & $4$  \\
      $c_{23}$ & $=$ & $\(1,0,1\)$  & $\(1,1,-1\)$ & $\mapsto$ & $\(1,0,-1\)$ & $\mapsto$ & $0$  \\
      $c_{24}$ & $=$ & $\(1,0,1\)$  & $\(1,0,-3\)$ & $\mapsto$ & $\(1,0,-3\)$ & $\mapsto$ & $-2$ 
    \end{tblr}
  \end{center}
\end{example}

\subsubsection*{Proprietà}

Il prodotto ``riga per colonna'' gode delle seguenti proprietà:

\paragraph*{Proprietà associativa}
$$\(AB\)C=A\(BC\)$$
\paragraph*{Proprietà distributiva 1}
$$A\(B+C\)=AB+AC$$
\paragraph*{Proprietà distributiva 2}
$$\(A+B\)C=AC+BC$$
\paragraph*{Proprietà associativa mista}
$$t\(AB\)=\(tA\)B=A\(tB\)$$
\paragraph*{Proprietà speciale della matrice identità}
Sia $A$ una matrice $n\times m$.
$$A=I_nA=AI_m$$

\subsection{Trasposta}

\begin{definition}[Matrice trasposta]
  La \textbf{trasposta} di una matrice $A$ è la matrice che si ottiene scambiando le righe con le colonne e si indica con $A^T$ o $A'$.
\end{definition}
\begin{example}
  $$
    A=
    \begin{pmatrix}
      1 & 2 & 3 \\
      4 & 5 & 6 \\
    \end{pmatrix}
  $$
  $$A'=
    \begin{pmatrix}
      1 & 4 \\
      2 & 5 \\
      3 & 6 \\
    \end{pmatrix}
  $$
\end{example}

\begin{definition}[Matrice simmetrica]
  Una matrice $A$ si dice \textbf{simmetrica} se $A=A'$.
\end{definition}
\begin{lemma}
  Se una matrice è simmetrica, allora è anche quadrata.
\end{lemma}
\begin{proof}
  $$A=A'\impl\(a_{ij}\)_{\substack{i\le n\\j\le m}}=\(a_{ji}\)_{\substack{j\le m\\i\le n}}\impl n=m$$
\end{proof}
\begin{definition}[Matrice antisimmetrica]
  Una matrice $A$ si dice \textbf{antisimmetrica} o emisimmetrica se $A=-A'$.
\end{definition}
\begin{lemma}
  Se una matrice è antisimmetrica, allora è quadrata e ha la diagonale nulla.
\end{lemma}
\begin{proof}
  $$A=A'\impl\(a_{ij}\)_{\substack{i\le n\\j\le m}}=\(-a_{ji}\)_{\substack{j\le m\\i\le n}}\impl n=m$$
  $$a_{ii}=-a_{ii}\impl a_{ii}=0$$
\end{proof}

\subsubsection*{Proprietà}

La trasposta gode delle seguenti proprietà:
$$\(A+B\)'=A'+B'$$
$$\(sA\)'=sA'$$
$$\(A'\)'=A$$
$$\(AB\)'=B'A'$$

\subsection{Potenza}

Se una matrice $A$ è quadrata, allora ha senso definire $A^2=AA$ o, più in generale:
$$A^n=\underbrace{A\cdot A\cdot A\cdots A}_{n\text{ volte}}$$
Analogamente a quanto avviene per le potenze reali
$$A^0=I$$

\subsubsection*{Proprietà}

Valgono le seguenti proprietà:
$$A^nA^m=A^{n+m}$$
$$\(A^n\)^m=A^{nm}$$

\subsection{Inversa}

\begin{definition}[Matrice inversa]
  Una matrice quadrata $A$ si dice \textbf{invertibile} se esiste una matrice quadrata $B$ tale che
  $$AB=BA=I$$
  e tale matrice si indica con $A^{-1}$ e viene detta \textbf{matrice inversa}:
  $$AA^{-1}=A^{-1}A=I$$
\end{definition}

\begin{theorem}
  Se la matrice $A$ è invertibile, allora la matrice $A^{-1}$ è unica.
\end{theorem}
\begin{proof}
  Siano $A$ e $B$ due matrici quadrate tali che $AB=BA=I$. Per assurdo, si suppone che anche un'altra matrice $C$ soddisfi la medesima equazione: $AC=CA=I$. 
  Per le proprietà cui sopra, si ha:
  $$B=BI=B(AC)=(BA)C=IC=C$$
  La catena di uguaglianze dimostra che che $B=C$, pertanto $A^{-1}$ è unica.
\end{proof}

Se $A$ è una matrice invertibile, allora, per le proprietà delle potenze, vale:
$$A^{-n}=\(A^{-1}\)^n=\underbrace{A^{-1}\cdot A^{-1}\cdot A^{-1}\cdots A^{-1}}_{n\text{ volte}}$$

\begin{example}
  $$
    A\walrus
    \begin{pmatrix}
      0 & 1 \\
      1 & 0 \\
    \end{pmatrix}
  $$
  $$
    A^2=
    \begin{pmatrix}
      0 & 1 \\
      1 & 0 \\
    \end{pmatrix}
    \begin{pmatrix}
      0 & 1 \\
      1 & 0 \\
    \end{pmatrix}=
    \begin{pmatrix}
      1 & 0 \\
      0 & 1 \\
    \end{pmatrix}=I
  $$
  $$AA^{-1}=I=A^2\impl A^{-1}=A$$
  $$A^3=A^2A=IA=A$$
  $$
    A^n=
    \begin{cases}
      I & n\equiv 1\mod{2} \\
      A & n\equiv 0\mod{2} \\
    \end{cases}
  $$
\end{example}

\begin{example}
  $$
    A=
    \begin{pmatrix}
      0 & 1 & 0 \\
      0 & 0 & 1 \\
      0 & 0 & 0 \\
    \end{pmatrix}
  $$
  $$
    A^2=
    \begin{pmatrix}
      0 & 1 & 0 \\
      0 & 0 & 1 \\
      0 & 0 & 0 \\
    \end{pmatrix}
    \begin{pmatrix}
      0 & 1 & 0 \\
      0 & 0 & 1 \\
      0 & 0 & 0 \\
    \end{pmatrix}
    =
    \begin{pmatrix}
      0 & 0 & 1 \\
      0 & 0 & 0 \\
      0 & 0 & 0 \\
    \end{pmatrix}
  $$
  $$
    A^3=AA^2=
    \begin{pmatrix}
      0 & 1 & 0 \\
      0 & 0 & 1 \\
      0 & 0 & 0 \\
    \end{pmatrix}
    \begin{pmatrix}
      0 & 0 & 1 \\
      0 & 0 & 0 \\
      0 & 0 & 0 \\
    \end{pmatrix}
    =
    \begin{pmatrix}
      0 & 0 & 0 \\
      0 & 0 & 0 \\
      0 & 0 & 0 \\
    \end{pmatrix}
    =\vec{0}$$
  $$A^{123}=A^3A^{120}=\vec{0}A^{120}=\vec{0}$$
  $$
    \begin{cases}
      A^3A^{-1}=\vec{0}A^{-1}=\vec{0} \\
      A^3A^{-1}=A^2\neq\vec{0}        \\
    \end{cases}
    \impl
    \nexists A^{-1}
  $$
\end{example}

\subsection{Determinante}

\begin{definition}[Determinante]
  Il \textbf{determinante} è un numero che viene assegnato ad ogni matrice quadrata.
\end{definition}

\paragraph*{Matrici del primo ordine}
Il determinante di una matrice quadrata di ordine 1 è l'unico elemento della matrice stessa:
$$\det\begin{pmatrix}a\end{pmatrix}=a$$

\paragraph*{Matrici del secondo ordine}
Il determinate di una matrice quadrata di ordine 2 è il prodotto di una diagonale meno il prodotto dell'altra diagonale:
$$
  \det
  \begin{pmatrix}
    a & b \\ 
    c & d \\
  \end{pmatrix}
  =ad-bc
$$

\paragraph*{Matrici del terzo ordine}
Il determinante di un matrice quadrata di ordine 3 è dato dalla formula:
$$
  \det
  \begin{pmatrix}
    a_{11} & a_{12} & a_{13} \\
    a_{21} & a_{22} & a_{23} \\
    a_{31} & a_{32} & a_{33} \\
  \end{pmatrix}
  =
  \begin{matrix}
    a_{11}a_{22}a_{33}+a_{12}a_{23}a_{31}+a_{13}a_{21}a_{32}- \\
    -a_{31}a_{22}a_{13}-a_{32}a_{23}a_{11}-a_{33}a_{21}a_{12}
  \end{matrix}
$$

\subsubsection*{Regola di Sarrus}
Poiché è difficile ricordare la formula per il calcolo del determinante di una matrice quadrata di ordine 3, si ricorre solitamente alla regola di Sarrus, che consta di un ampliamento della matrice originale, raddoppiandola a destra:
$$
  \begin{pmatrix}
    a_{11} & a_{12} & a_{13} \\
    a_{21} & a_{22} & a_{23} \\
    a_{31} & a_{32} & a_{33} \\
  \end{pmatrix}
  \mapsto 
  \begin{pmatrix}
    a_{11} & a_{12} & a_{13} & a_{11} & a_{12} & a_{13} \\
    a_{21} & a_{22} & a_{23} & a_{21} & a_{22} & a_{23} \\
    a_{31} & a_{32} & a_{33} & a_{31} & a_{32} & a_{33} \\
  \end{pmatrix}
$$
Il determinante è la somma dei prodotti sulle diagonali discendenti meno la somma dei prodotti sulle diagonali ascendenti.

\begin{example}
  $$
    \det
    \begin{pmatrix}
      1 & 2 & -1 \\
      0 & 2 & -1 \\
      1 & 1 & 0  
    \end{pmatrix}
    =0-2+0+2+1-0=1
  $$
\end{example}

\subsubsection*{Sviluppo di Laplace}
Lo sviluppo di Laplace è una formula ricorsiva che permette di calcolare il determinante di una matrice di ordine $n$ ricorrendo al calcolo del determinante di una matrice di ordine $n-1$.

\begin{definition}[Sottomatrice]
  Si definisce \textbf{sottomatrice} di $A$ una matrice ottenuta togliendo ad $A$ un certo numero di righe e/o un certo numero di colonne. Se la sottomatrice è quadrata di ordine $m$ allora prende il nome di \textbf{minore di ordine $m$}.
\end{definition}

\begin{definition}[Minore complementare]
  Il \textbf{minore complementare} $M_{ij}$ dell'elemento $a_{ij}$ è il minore che si ottiene togliendo da $A$ la $i$--esima riga e la $j$--esima colonna.
\end{definition}

\begin{definition}[Complemento algebrico]
  Il \textbf{complemento algebrico} di $\begin{pmatrix}a_{ij}\end{pmatrix}$ è:
  $$A_{ij}=\(-1\)^{i+j}\det M_{ij}$$
\end{definition}

\begin{example}
  $$
    A\walrus
    \begin{pmatrix}
      1 & 0  & 1 \\
      2 & 1  & 0 \\
      1 & -1 & 1 \\
    \end{pmatrix}
  $$
  $$
    M_{22}=
    \begin{pmatrix}
      1 & 1 \\
      1 & 1 \\
    \end{pmatrix}
  $$
  $$
    A_{22}=\(-1\)^4\det
    \begin{pmatrix}
      1 & 1 \\
      1 & 1 \\
    \end{pmatrix}
    =1-1=0$$
\end{example}

\begin{theorem}[Sviluppo di Laplace]
  Sia $A$ una matrice $n\times n$. Fissato $i$, il determinante di $A$ è:
  $$\det A=\sum_{j=1}^na_{ij}A_{ij}$$
\end{theorem}

\begin{example}
  \begin{align*}
    \begin{vmatrix}
      1 & 2 & -1 & 1 \\
      2 & 0 & 1  & 0 \\
      1 & 1 & 1  & 1 \\
      0 & 2 & 1  & 0 
    \end{vmatrix} 
     & = -2                               
    \begin{vmatrix}
      2 & -1 & 1 \\
      1 & 1  & 1 \\
      2 & 1  & 0 
    \end{vmatrix}
    -
    \begin{vmatrix}
      1 & 2 & 1 \\
      1 & 1 & 1 \\
      0 & 2 & 0 
    \end{vmatrix}                        \\
     & =                                  
    -2\(2
    \begin{vmatrix}
      -1 & 1 \\
      1  & 1 
    \end{vmatrix}
    -
    \begin{vmatrix}
      2 & 1 \\
      1 & 1 
    \end{vmatrix}
    \)-\(-2
    \begin{vmatrix}
      1 & 1 \\
      1 & 1 
    \end{vmatrix}
    \)                                    \\
     & =-2\(2\cdot\(-2\)-1\)-\(-2\cdot0\) \\
     & =10+0=10                           
  \end{align*}
\end{example}

\subsubsection*{Proprietà}

Il determinante gode delle seguenti proprietà:
\paragraph*{Normalizzazione}
$$\det I=1$$
\begin{proof}
  $$\det I_1=\det \begin{pmatrix}
      1
    \end{pmatrix}=1$$
  Sia $\det I_n=1$.
  \begin{align*}
    \det I_{n+1}=\det 
    \begin{pmatrix}
      1      & 0      & \cdots & 0      \\
      0      & 1      & \cdots & 0      \\
      \vdots & \vdots & \ddots & \vdots \\
      0      & 0      & \cdots & 1      \\
    \end{pmatrix}=
    \det
    \begin{pmatrix}
      1      & \cdots & 0      \\
      \vdots & \ddots & \vdots \\
      0      & \cdots & 1      \\
    \end{pmatrix}=\det I_{n}=1
  \end{align*}
  Per induzione, $\det I_n=1\;\forall n\ge 1$.
\end{proof}
\paragraph*{Simmetria}
$$\det A=\det A'$$
\paragraph*{Teorema di Binet}
$$\det \(AB\)=\det A\det B$$
\paragraph*{Conseguenze}
$$\det A \neq 0\iff \det A^{-1}=\frac{1}{\det A}$$
$$\det A\neq 0\iff \exists A^{-1}$$
\begin{proof}
  $$AA^{-1}=I$$
  $$\det\(AA^{-1}\)=\det I$$
  $$\det\(A\)\det\(A^{-1}\)=\det I$$
  $$\det\(A^{-1}\)=\frac{\det I}{\det A}=\frac{1}{\det A}$$
\end{proof}

\paragraph*{Alternanza}
Se $A'$ si ottiene di $A$ scambiando due righe, allora $\det A'=-\det A$
\begin{example}
  $$
    A=
    \begin{pmatrix}
      1 & 0 & 1 \\
      0 & 1 & 0 \\
      1 & 1 & 0 \\
    \end{pmatrix}
  $$
  $$
    \det A=  
    \begin{vmatrix}
      1 & 1 \\
      1 & 0 \\
    \end{vmatrix}
    =-1
  $$
  $$
    A'=
    \begin{pmatrix}
      1 & 0 & 1 \\
      0 & 1 & 1 \\
      1 & 0 & 0 \\
    \end{pmatrix}
  $$
  $$\det A'=-\det A=1$$
\end{example}
\paragraph*{Multilinearità}
Se $A'$ si ottiene da $A$ moltiplicando una riga di $A$ per uno scalare $t$, allora $\det A'=t\det A$
\begin{example}
  $$
    A=
    \begin{pmatrix}
      1 & 1 & 1 \\
      0 & 1 & 0 \\
      1 & 1 & 0 \\
    \end{pmatrix}
  $$
  $$
    \det A=
    \begin{vmatrix}
      1 & 1 & 1 \\
      0 & 1 & 0 \\
      1 & 1 & 0 \\
    \end{vmatrix}
    =
    \begin{vmatrix}
      1 & 1 \\
      1 & 0 \\
    \end{vmatrix}
    =-1
  $$
  $$
    A'=
    \begin{pmatrix}
      1 & 0 & 1 \\
      1 & 1 & 1 \\
      1 & 0 & 0 \\
    \end{pmatrix}
  $$
  $$\det A'=1\det A=-1$$
\end{example}

\begin{theorem}
  Se una martice ha due righe uguali allora il suo determinante è 0.
\end{theorem}

\subsection{Calcolo dell'inversa}

\begin{definition}[Aggiunto classico]
  Sia $A$ una matrice quadrata.
  Si definisce l'\textbf{aggiunto classico} di $A$ e si indica con $\agg A$:
  $$\agg A=\(A_{ij}\)'$$
\end{definition}
\begin{theorem}
  $$A\agg A=\agg A \cdot A=\det A\cdot I$$
  da cui
  $$A^{-1}=\frac{1}{\det A}\agg A$$
\end{theorem}

\begin{example}
  $$
    A\walrus
    \begin{pmatrix}
      a & b \\
      c & d \\
    \end{pmatrix}
  $$
  $$\det A=ad-bc\neq 0$$
  $$
    C=
    \begin{pmatrix}
      d  & -c \\
      -b & a  \\
    \end{pmatrix}
  $$
  $$
    \agg A=C'=
    \begin{pmatrix}
      d  & -b \\
      -c & a  \\
    \end{pmatrix}
  $$
  $$
    A^{-1}=\frac{1}{ad-bc}
    \begin{pmatrix}
      d  & -b \\
      -c & a  \\
    \end{pmatrix}
  $$
  $$
    A\walrus
    \begin{pmatrix}
      2 & 1 \\
      0 & 1 \\
    \end{pmatrix}
  $$
  $$\det A=2$$
  $$
    A^{-1}=\frac{1}{2}
    \begin{pmatrix}
      1 & -1 \\
      0 & 2  \\
    \end{pmatrix}
    =
    \begin{pmatrix}
      \frac{1}{2} & -\frac{1}{2} \\
      0           & 1            \\            
    \end{pmatrix}
  $$
\end{example}

\begin{example}
  $$
    A=
    \begin{pmatrix}
      1 & 0 & 1 \\
      1 & 1 & 1 \\
      1 & 1 & 0 \\
    \end{pmatrix}
  $$
  $$
    \det A=
    \begin{vmatrix}
      1 & 1 \\
      1 & 0 \\
    \end{vmatrix}
    +
    \begin{vmatrix}
      1 & 1 \\
      1 & 1 \\
    \end{vmatrix}
    =-1
  $$
  $$
    C=
    \begin{pmatrix}
      -1 & 1  & 0  \\
      1  & -1 & -1 \\
      -1 & 0  & 1  \\
    \end{pmatrix}
  $$
  $$
    C'=
    \begin{pmatrix}
      -1 & 1  & -1 \\
      1  & -1 & 0  \\
      0  & -1 & 1  \\
    \end{pmatrix}
  $$
  $$
    A^{-1}=-1C'=
    \begin{pmatrix}
      1  & -1 & 1  \\
      -1 & 1  & 0  \\
      0  & 1  & -1 \\
    \end{pmatrix}
  $$
\end{example}

\begin{definition}[Matrice a blocchi]
  Una \textbf{matrice a blocchi} è una matrice della forma
  $$A=\begin{pmatrix}
      A_1 & 0   \\
      0   & A_2 \\
    \end{pmatrix}$$
  dove $A_1$ e $A_2$ sono matrici quadrate.
\end{definition}
\begin{example}
  $$
    A\walrus
    \begin{pmatrix}
      1 & 2 & 0 & 0 & 0 \\
      2 & 1 & 0 & 0 & 0 \\
      0 & 0 & 1 & 0 & 1 \\
      0 & 0 & 1 & 1 & 2 \\
      0 & 0 & 2 & 1 & 2 \\
    \end{pmatrix}
  $$
\end{example}
\begin{theorem}
  Se $A$ è una matrice a blocchi, il suo determinante è il prodotto dei determinanti dei blocchi e l'inversa è la matrice a blocchi aventi come blocchi le inverse dei blocchi di $A$:
  $$\det A=\det A_1\cdot \det A_2$$
  $$
    A^{-1}=
    \begin{pmatrix}
      A_1^{-1} & 0        \\
      0        & A_2^{-1} \\  
    \end{pmatrix}
  $$
\end{theorem}
\begin{example}
  $$
    A=
    \begin{pmatrix}
      1 & 1 & 0 & 0 \\
      1 & 0 & 0 & 0 \\
      0 & 0 & 0 & 1 \\
      0 & 0 & 2 & 1 \\
    \end{pmatrix}
  $$
  $$
    \det A=
    \begin{vmatrix}
      1 & 1 \\
      1 & 0 \\
    \end{vmatrix}
    \cdot 
    \begin{vmatrix}
      0 & 1 \\
      2 & 1 \\
    \end{vmatrix}
    =-1\cdot \(-2\)=2$$
  $$
    A^{-1}=
    \begin{pmatrix}
      0 & 1  & 0            & 0           \\
      1 & -1 & 0            & 0           \\
      0 & 0  & -\frac{1}{2} & \frac{1}{2} \\
      0 & 0  & 1            & 0           \\
    \end{pmatrix}
  $$
\end{example}
