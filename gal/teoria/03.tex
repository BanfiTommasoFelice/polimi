\section{Sistemi lineari}

Si considera il sistema di $m$ equazioni in $n$ incognite:
$$
  \begin{cases}
    a_{11}x_1+a_{12}x_2+\cdots +a_{1n}x_n=b_1 \\
    a_{21}x_1+a_{22}x_2+\cdots +a_{2n}x_n=b_2 \\
    \vdots                                    \\
    a_{m1}x_1+a_{m2}x_2+\cdots +a_{mn}x_n=b_m \\
  \end{cases}
$$

La matrice $A\walrus\(a_{ij}\)$ è detta \textbf{matrice dei coefficienti}, il vettore colonna $B\walrus\(b_{i}\)$ è detto \textbf{vettore dei termini noti} e il vettore $X\walrus\(x_i\)$ è detto \textbf{vettore delle incognite}.

Si osserva che il sistema è equivalente a:
$$AX=B$$

\subsection{Matrice completa}

\begin{definition}[Matrice completa]
  La \textbf{matrice completa} del sistema $AX=B$ è la matrice $\(A,B\)$.
\end{definition}

\begin{example}
  $$
    \begin{cases}
      2x+y=1 \\
      x-y=-1
    \end{cases}
    \mapsto
    \begin{pmatrix}
      2 & 1  & 1  \\
      1 & -1 & -1 
    \end{pmatrix}
  $$
\end{example}

\begin{definition}[Soluzione]
  Una \textbf{soluzione} di un sistema $AX=B$ è un $m$--vettore colonna $\bar{X}$ che soddisfa l'equazione, cioè $A\bar{X}=B$.
  L'insieme delle soluzioni del sistema si indica con $\sol\(A,B\)$.
\end{definition}

\begin{definition}[Sistema omogeneo]
  Un sistema $AX=B$ si dice \textbf{omogeneo} se $B=\vec{0}$.
\end{definition}

\begin{theorem}[Legge di sovrapposizione]
  Se $X_1,X_2$ sono soluzioni di un sistema omogeneo, anche $X_1+X_2$ lo è.  
  Se $X$ è soluzione di un sistema omogeneo, allora anche $tX$ lo è.
\end{theorem}
\begin{proof}
  $$A\(X_1+X_2\)=AX_1+AX_2=\vec{0}+\vec{0}=\vec{0}$$
  $$A\(tX\)=tAX=t\vec{0}=\vec{0}$$
\end{proof}

Si suppone che $AX=B$ abbia soluzione $X_P$. Allora tutte le soluzioni del sistema sono date da $X_P+Y\ \forall Y\in\sol\(A,0\)$. In simboli:
$$\sol\(A,B\)=X_P+\sol\(A,0\)$$

\begin{proof}
  Se $Y\in\sol\(A,0\)$ allora
  $$A\(X_P+Y\)=AX_P+AY=B+\vec{0}=B$$
  Se $X'\in\sol\(A,B\)$ sia $Y=X'-X_P$ cosicché $X'=X_P+Y$ si ha che
  $$AY=A\(X'-X_P\)=AX'-AX_P=B-B=\vec{0}$$
  quindi
  $$X'=X_P+Y,\ Y\in\sol\(A,0\)$$.
\end{proof}

\begin{example}
  $$
    \begin{cases}
      x+y=0 \\
      x-y=2 \\
    \end{cases}
    \mapsto
    \begin{pmatrix}
      1 & 1  & 0 \\
      1 & -1 & 2 \\
    \end{pmatrix}
  $$
  $$
    X_P=
    \begin{pmatrix}
      1  \\
      -1 \\
    \end{pmatrix}
  $$
  Ogni soluzione si scrive come $X_P+Y$ dove $Y$ è soluzione di 
  $$
    \begin{cases}
      x+y=0 \\
      x-y=0 \\
    \end{cases}
    \impl
    Y=
    \begin{pmatrix}
      0 \\
      0 \\
    \end{pmatrix}
  $$
  Pertanto $X_P$ è l'unica soluzione.
\end{example}

\begin{example}
  $$
    \begin{cases}
      x+y+z=1  \\
      2x+3z=2  \\
      x-y+2z=1 \\
    \end{cases}
    \impl
    \begin{cases}
      x+y+z=1 \\
      2x+3z=2 \\
      2x+3z=2 \\
    \end{cases}
    \impl
    \begin{cases}
      2x+2y+2z=2 \\
      2x+3z=2    \\
    \end{cases}
    \impl
  $$
  $$
    \impl
    \begin{cases}
      x+y+z=1 \\
      -2y+z=0 \\
    \end{cases}
    \impl
    \begin{cases}
      z=2y     \\
      x+y+2y=1 \\
    \end{cases}
    \impl
    \begin{cases}
      z=2y   \\
      x+3y=1 \\
    \end{cases}
    \impl
    \begin{cases}
      y=\frac{1}{2}z   \\
      x=1-\frac{3}{2}z \\
    \end{cases}
  $$
  Le soluzioni sono date da:
  $$
    \begin{pmatrix}
      1 \\
      0 \\
      0
    \end{pmatrix}
    +
    z
    \begin{pmatrix}
      -\nicefrac{3}{2} \\
      \nicefrac{1}{2}  \\
      1                \\
    \end{pmatrix}
  $$
\end{example}

\subsection{Metodo di Gauss--Jordan}

\begin{definition}[Pivot]
  In una matrice qualsiasi il primo elemento non nullo di una riga viene chiamato \textbf{pivot} della riga.
\end{definition}
  
\begin{definition}[Matrice a gradini]
  Una matrice si dice \textbf{a gradini} se il pivot di ogni riga è in una colonna successiva a quella del pivot della riga precedente.
  In particolare, in una matrice a gradini, se una riga è nulla allora tutte le righe successive devono essere nulle.
\end{definition}
  

Si definiscono operazioni elementari di riga, operanti su una matrice:
\begin{itemize}
  \item scambiare due righe;
  \item moltiplicare una riga per $n\in\mathbb{R}\setminus\left\{ 0 \right\}$;
  \item sommare ad una riga un multiplo di un'altra riga.
\end{itemize}
Si scrive $A\sim B$ se $B$ si ottiene a partire da $A$ mediante una sequenza di operazioni elementari di riga.

Il metodo di Gauss--Jordan consiste nel trasformare una matrice completa in una matrice a gradini.

\begin{example}
  $$
    \begin{pmatrix}
      1 & 1  & 1 & 1 \\
      2 & 0  & 3 & 2 \\
      1 & -1 & 2 & 1 \\
    \end{pmatrix}
    \sim
    \begin{pmatrix}
      1 & 1  & 1 & 1 \\
      1 & -1 & 2 & 1 \\
      2 & 0  & 3 & 2 \\
    \end{pmatrix}
    \sim
    \begin{pmatrix}
      1 & 1  & 1 & 1 \\
      0 & -2 & 1 & 0 \\
      2 & 0  & 3 & 2 \\
    \end{pmatrix}
    \sim
    \begin{pmatrix}
      1 & 1  & 1 & 1 \\
      0 & -2 & 1 & 0 \\
      0 & -2 & 1 & 0 \\
    \end{pmatrix}
    \sim
    \begin{pmatrix}
      1 & 1  & 1 & 1 \\
      0 & -2 & 1 & 0 \\
      0 & 0  & 0 & 0 \\
    \end{pmatrix}
  $$
  $$
    \begin{pmatrix}
      1 & 1  & 2 & 2 \\
      2 & 0  & 4 & 2 \\
      1 & -1 & 2 & 0 \\
      0 & 2  & 0 & 2 \\
    \end{pmatrix}
    \sim
    \begin{pmatrix}
      1 & 1  & 2 & 2  \\
      0 & -2 & 0 & -2 \\
      0 & -2 & 0 & -2 \\
      0 & 2  & 0 & 2  \\
    \end{pmatrix}
    \sim
    \begin{pmatrix}
      1 & 1  & 2 & 2  \\
      0 & -2 & 0 & -2 \\
      0 & 0  & 0 & 0  \\
      0 & 0  & 0 & 0  \\
    \end{pmatrix}
  $$
  $$
    \begin{pmatrix}
      0 & 2  & 1  & 0 & 1  \\
      0 & 3  & -1 & 1 & 1  \\
      1 & 1  & -2 & 1 & 0  \\
      1 & -1 & -3 & 1 & -1 \\
    \end{pmatrix}
    \sim
    \begin{pmatrix}
      1 & -1 & -3 & 1 & -1 \\
      1 & 1  & -2 & 1 & 0  \\
      0 & 3  & -1 & 1 & 1  \\
      0 & 2  & 1  & 0 & 1  \\
    \end{pmatrix}
    \sim
    \begin{pmatrix}
      1 & -1 & -3 & 1 & -1 \\
      0 & 2  & 1  & 0 & 1  \\
      0 & 3  & -1 & 1 & 1  \\
      0 & 2  & 1  & 0 & 1  \\
    \end{pmatrix}
    \sim
  $$
  $$
    \sim
    \begin{pmatrix}
      1 & -1 & -3           & 1 & -1           \\
      0 & 2  & 1            & 0 & 1            \\
      0 & 0  & -\frac{5}{2} & 1 & -\frac{1}{2} \\
      0 & 0  & 0            & 0 & 0            \\
    \end{pmatrix}
    \sim
    \begin{pmatrix}
      1 & -1 & -3 & 1 & -1 \\
      0 & 2  & 1  & 0 & 1  \\
      0 & 0  & -5 & 2 & -1 \\
      0 & 0  & 0  & 0 & 0  \\
    \end{pmatrix}
  $$  
\end{example}


Un sistema si dice a gradini se la sua matrice completa lo è. Dato che ogni matrice può essere trasformata mediante una sequenza di operazioni elementari di riga in una matrice a gradini, ogni sistema è equivalente ad una sistema a gradini: basta ridurre la sua matrice a gradini.

In un sistema a gradini le variabili che corrispondono ai pivot sono dette \textbf{variabili vincolate}, mentre le altre variabili sono dette \textbf{variabili libere}.
La soluzione del sistema si trova sostituendo i vari coefficienti con la matrice completa. In particolare, se l'ultimo pivot è nell'ultima colonna, il sistema non ha soluzione.

