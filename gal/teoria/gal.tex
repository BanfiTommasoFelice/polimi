\documentclass[a4paper,12pt]{article}

\usepackage[italian]{babel}
\usepackage[a4paper, left=18mm, right=18mm, top=20mm, bottom=20mm]{geometry}
\usepackage{amssymb}
\usepackage{mathtools}
\usepackage{interval}
\usepackage{amsthm}
\usepackage{thmtools}
\usepackage{cancel}
\usepackage{hyperref}
\usepackage{tikz}
\usepackage{pgfplots}
\usepackage{nicefrac}
\usepackage{enumitem}
\usepackage{verbatim}
\usepackage{tabularray}
\usepackage{fontspec}
\usepackage{bold-extra}
\usepackage{tabularray}

\hypersetup{
  colorlinks=true,
  linkcolor=black,
    filecolor=magenta,      
    urlcolor=cyan,
    pdftitle={Geometria e algebra lineare},
    % bookmarks=true,
    bookmarksopen=true,
    pdfpagemode=UseOutlines,
    pdfauthor={Amato Michele Pasquale},
}

\title{\huge Geometria e algebra lineare}
\author{Amato Michele Pasquale}
\date{\today}

\pgfplotsset{compat = newest}
\makeatletter
\renewcommand\l@subsection{\@dottedtocline{2}{1.5em}{3em}}
\makeatother
\setitemize{noitemsep,topsep=3pt,parsep=0pt,partopsep=0pt}
\setenumerate{noitemsep,topsep=3pt,parsep=0pt,partopsep=0pt}

\newcommand{\abs}[1]{\left\lvert #1 \right\rvert}
\newcommand{\ceil}[1]{\left\lceil #1 \right\rceil}
\newtheorem{theorem}{Teorema}
\newtheorem{definition}{Definizione}
\newtheorem{lemma}{Lemma}
\newtheorem{axiom}{Assioma}
\newtheorem{corollary}{Corollario}
\renewcommand\qedsymbol{$\blacksquare$}
\newcommand{\asin}{\arcsin}
\newcommand{\acos}{\arccos}
\newcommand{\atan}{\arctan}
\newcommand{\impl}{\Rightarrow}
\setitemize{noitemsep,topsep=3pt,parsep=0pt,partopsep=0pt}
\setenumerate{noitemsep,topsep=3pt,parsep=0pt,partopsep=0pt}
\newcommand{\triang}[1]{\overset{\triangle}{#1}}
\renewcommand{\a}{\alpha}
\renewcommand{\b}{\beta}
\renewcommand{\c}{\gamma}
\renewcommand{\(}{\left(}
\renewcommand{\)}{\right)}
\renewcommand{\mod}[1]{\ \( \mathrm{mod}\ #1 \)}
\DeclareMathOperator{\agg}{agg}
\DeclareMathOperator{\sol}{Sol}
\renewcommand{\emptyset}{\varnothing}
\newcommand{\walrus}{\coloneqq}
\newcommand{\reals}{\mathbb{R}}
\newcommand{\dabs}[1]{\left\lvert\left\lvert #1 \right\rvert\right\rvert}
\renewcommand{\sp}[1]{\left< #1 \right>}
\DeclareMathOperator{\nul}{null}
\DeclareMathOperator{\rk}{rk}

\newenvironment{example}
{\par\noindent{\small\bf Esempio}\hspace{0.5em}}
{\hfill$\nsim$}
\newenvironment{observation}
{\par\noindent{\small\bf Osservazione}\hspace{0.5em}}
{\hfill}

\begin{document}
\pagenumbering{gobble}
\maketitle
\vfill
\begin{center}
  \emph{}
\end{center}
\newpage
\pagenumbering{roman}
\tableofcontents
\newpage
\pagenumbering{arabic}
% lezione del 12/09/2022

Il problema che ci si pone è trovare un modo opportuno per rappresentare all'interno di un sistema di calcolo le informazioni in modo efficiente, rispetto alla realtà fisica del sistema e alla loro manipolazione.

\begin{definition}[Alfabeto]
  Si definisce \textbf{alfabeto} un insieme di simboli utilizzabili e, pertanto, distinguibili tra loro.
\end{definition}

\begin{definition}[Codice]
  Si definisce \textbf{codice} l'insieme delle sequenze di simboli o delle regole per definire le combinazioni ammissibili.
\end{definition}

Dati l'insieme degli elementi da rappresentare e l'insieme delle configurazioni ammissibili, il codice ne definisce la relazione biunivoca.
Le configurazioni ammissibili hanno tutte egual dimensione. Tale dimensione dipende sia dall'alfabeto, sia dalla quantità di elementi da rappresentare: siano $S$ l'alfabeto di riferimento e $\abs{S}$ la sua cardinalità (ossia il numero di simboli che lo compone), se si vogliono rappresentare $n$ elementi, ogni elemento avrà dimensione:
$$k=\ceil{\log_{\abs{S}} n}$$
Al contrario, se gli elementi di un codice hanno lunghezza $k$, le combinazioni ammissibili sono:
$$n=\abs{S}^k$$

\section{Rappresentazione binaria}

I componenti elettronici che costituiscono il sistema di calcolo sono caratterizzati da una realtà costituita da due stati (condensatore carico/scarico, tensione alta/bassa, etc...). Si effettua, quindi, una mappatura diretta con un sistema costituito da \textbf{due simboli} che, pertanto, si chiama \textbf{binario}. L'alfabeto di riferimento diventa $\left\{ 0,1 \right\}$.

La cifra della codifica (0 o 1) prende il nome di \textbf{bit}, dall'inglese \emph{\textbf{bi}nary digi\textbf{t}}. L'insieme ordinato di 8 bit prende il nome di \textbf{byte}. Come per le cifre decimali, si ha una nomenclatura per le potenze della base:
\begin{center}
  \begin{tblr}{colspec={c|c}, cells={c,m}, columns={20mm}}
    \textbf{Nome} & \textbf{Quantità} \\
    \hline
    KB            & $2^{10}$          \\
    MB            & $2^{20}$          \\
    GB            & $2^{30}$          \\
    TB            & $2^{40}$          \\
  \end{tblr}
\end{center}

\begin{example}
  Se si vogliono rappresentare i giorni della settimana usando l'alfabeto binario, si calcola la dimensione del singolo elemento:
  $$k=\ceil{\log_27}=3$$
  e si assegna ad ogni combinazione di 3 bit un giorno della settimana distinto:
  \begin{center}
    \begin{tblr}{colspec={c|c|c|c|c|c|c}, cells={c,m}, columns={18mm}}
      Lunedì & Martedì & Mercoledì & Giovedì & Venerdì & Sabato & Domenica \\
      \hline
      000    & 001     & 010       & 011     & 100     & 101    & 110      
    \end{tblr}
  \end{center}
  Da notare che non viene utilizzata la combinazione 111, in quanto le combinazioni ammissibili sono $2^3=8$ ma per le necessità del caso ne servono solo 7.
\end{example}

Nella scelta della codifica da adottare, bisogna tenere a mente alcuni aspetti:
\begin{itemize}
  \item l'insieme degli elementi da rappresentare;
  \item il grado di semplificazione delle operazioni più eseguite;
  \item il grado di conservazione delle proprietà dell'insieme originale.
\end{itemize}

L'informazione può essere, per comodità, suddivisa in aree:

\begin{center}
  \begin{tblr}{colspec={c|c|c|c|c|c}, cells={c,m}, columns={20mm}}
    \SetCell[c=6]{c} Informazione                                                                       \\ \hline
    \SetCell[c=3]{c} Numerica &          &           & \SetCell[c=3]{c} Non numerica                    \\ \hline
    Naturali                  & Relativi & Razionali & Testi                         & Audio & Immagini 
  \end{tblr}
\end{center}

% lezione del 13/09/2022

La notazione che si usa in base 10 è una \textbf{notazione posizionale pesata}, vale a dire che ogni cifra vale in base alla posizione che essa occupa all'interno del numero. Il valore del numero, infatti, è dato da
$$\text{valore}=\sum_{0}^nc_ib^i$$
dove $c_i$ è la cifra in posizione $i$ e $b$ è la base di riferimento.

\begin{example}
  $$315_{10}=3\cdot10^2+1\cdot10^1+5\cdot10^0$$
\end{example}

Nella notazione in base 2, si definisce un codice che associa al valore numerico una configurazione, in cui la cifra più a destra è la cifra meno significativa (\textbf{LSB}, Least Significant Bit), mentre quella più a sinistra è quella più significativa (\textbf{MSB}, Most Significant Bit).

\section{Conversione di base}
Idealmente, ogni qualvolta si vuole rappresentare una quantità in una certa base, si vogliono implicitamente rappresentare in un'altra base tutte le quantità minori uguali a quella di partenza, non avrebbe senso altrimenti\footnote{se me ne serve solo uno, uso il primo valore disponibile}.

Se $n$ è il valore da rappresentare, significa che in base 2 si avrà bisogno di $k$ bits:
$$k=\ceil{\log_2\(n+1\)}$$
\paragraph*{Osservazione} L'argomento del logaritmo è $n+1$ in quanto bisogna anche considerare lo 0.

Per passare da una base $a$ ad una base $b$, con $a<b$, si procede nel seguente modo: si moltiplica ogni coefficiente per la base elevata alla sua posizione e poi si sommano tutti i prodotti ottenuti.

\begin{example}
  $$1010_2=1\cdot2^3+0\cdot2^2+1\cdot2^1+0\cdot2^0=8+0+2+0=10_{10}$$
\end{example}

Per passare da una base $a$ ad una base $b$, con $a>b$, si procede nel seguente modo:

\begin{enumerate}
  \item si divide il numero per la base;
  \item si prende il resto;
  \item si ripete il processo con il quoziente ottenuto.
\end{enumerate}

Il processo si conclude quando il quoziente diventa 0. Il risultato non è altro che la sequenza ordinata dei resti ottenuti, letta al contrario. 

\begin{example}
  \begin{center}
    \begin{tblr}{c|c}
      17 & 1 \\
      8  & 0 \\
      4  & 0 \\
      2  & 0 \\
      1  & 1 
    \end{tblr}
    $$\impl 17_{10}=10001_2$$
  \end{center}
\end{example}

Le modalità con cui si trasformano le quantità fra le diverse basi sono interscambiabili, pertanto si dicono metodi.

\section{Modulo e segno}

Nella notazione decimale si utilizza la forma ``modulo e segno'' per esprimere i numeri relativi:
$$-11\;\;+4$$

Tale ragionamento non si può estendere alla rappresentazione binaria, perché richiederebbe l'introduzione un nuovo simbolo per `$+$' o `$-$'. Pertanto, per convenzione, il primo bit indica il segno, che può essere 0 se è positivo, 1 se è negativo:
$$+17_{10\text{MS}}=010001_{2\text{MS}}$$
$$-17_{10\text{MS}}=110001_{2\text{MS}}$$
$$-23_{10\text{MS}}=110111_{2\text{MS}}$$

Si devono eseguire le operazioni aritmetiche usando la notazione modulo e segno.
\begin{center}
  \begin{tblr}{c|c}
    $
      \begin{aligned}
        x & =+11_{10\text{MS}} \\
        y & =+8_{10\text{MS}}  \\
        z & =-7_{10\text{MS}}  \\
      \end{aligned}
    $
     & 
    $
      \begin{aligned}
        x+y & =+19_{10\text{MS}} \\
        x+z & =+4_{10\text{MS}}  \\
      \end{aligned}
    $
  \end{tblr}
\end{center}

Nella rappresentazione decimale si eseguono i passaggi descritti nel seguente algoritmo:

\begin{algorithm}[H]
  \caption{Somma di due valori in notazione modulo e segno}\label{algo:sommams}
  \begin{algorithmic}[1]
    \If{$\sgn x= \sgn y$}
    \State $\abs{n}=\abs{x}+\abs{y}$
    \State $\sgn n=\sgn x$
    \Else
    \State $a = \max \(\abs{x},\abs{y}\)$
    \State $b = \min \(\abs{x},\abs{y}\)$
    \State $\abs{n}=a-b$
    \If{$\abs{x}=a$}
    \State $\sgn n=\sgn a$
    \Else
    \State $\sgn n=\sgn b$
    \EndIf
    \EndIf
  \end{algorithmic}
\end{algorithm}

Tuttavia, per un calcolatore un tal numero di operazioni renderebbe il calcolo proibitivamente lento. Per questo, si introduce una nuova notazione, più efficiente di quella di modulo e segno, incentrata sull'agilità dell'elaborazione delle informazioni.

Ovviamente, la finestra dei valori che si vuole rappresentare si estende anche al negativo, per cui se si vuole rappresentare il $17$, si vuole rappresentare tutti i valori da $-17$ a $17$.

Si descrivono le possibili combinazioni avendo a disposizione 4 bit:
\begin{center}
  \begin{tblr}{c|c||c|c}
    $0000_2$ & $0_{10}$ & $1000_2$ & $-0_{10}$ \\
    $0001_2$ & $1_{10}$ & $1001_2$ & $-1_{10}$ \\
    $0010_2$ & $2_{10}$ & $1010_2$ & $-2_{10}$ \\
    $0011_2$ & $3_{10}$ & $1011_2$ & $-3_{10}$ \\ \hline
    $0100_2$ & $4_{10}$ & $1100_2$ & $-4_{10}$ \\
    $0101_2$ & $5_{10}$ & $1101_2$ & $-5_{10}$ \\
    $0110_2$ & $6_{10}$ & $1110_2$ & $-6_{10}$ \\
    $0111_2$ & $7_{10}$ & $1111_2$ & $-7_{10}$ \\
  \end{tblr}
\end{center}

Si nota subito che lo 0 ha una codifica ridondante:
$$0000_{2\text{MS}}=1000_{2\text{MS}}=0_{10}$$

Si pone l'attenzione sull'identità:
$$x+\(-x\)=0$$

Volendo rispettare l'identità con un valore $x=3_{10}$ si ha:

\begin{center}
  \begin{tblr}{ccccc}
    $^10$ & $^10$ & $^11$ & $1$ & $+$ \\
    $1$   & $1$   & $0$   & $1$ & $=$ \\
    \hline
    $0$   & $0$   & $0$   & $0$       
  \end{tblr}
\end{center}

quindi risulta che $1101_2$ è la codifica di $0011_2$ al negativo. Da questo ragionamento, esteso agli altri numeri risulta che:
\begin{center}
  \begin{tblr}{c|c||c|c}
    $0000_2$ & $0_{10}$ & $1000_2$ & $\pm8_{10}$ \\
    $0001_2$ & $1_{10}$ & $1001_2$ & $-7_{10}$   \\
    $0010_2$ & $2_{10}$ & $1010_2$ & $-6_{10}$   \\
    $0011_2$ & $3_{10}$ & $1011_2$ & $-5_{10}$   \\ \hline
    $0100_2$ & $4_{10}$ & $1100_2$ & $-4_{10}$   \\
    $0101_2$ & $5_{10}$ & $1101_2$ & $-3_{10}$   \\
    $0110_2$ & $6_{10}$ & $1110_2$ & $-2_{10}$   \\
    $0111_2$ & $7_{10}$ & $1111_2$ & $-1_{10}$   \\
  \end{tblr}
\end{center}

Esiste anche un metodo algebrico per trovare l'opposto:
$$-x=\sim x+1$$
oppure un metodo grafico:
\emph{tutti i bit che dall'LSB al primo bit 1 rimangono invariati, mentre tutti gli altri si cambiano}.

L'efficacia di questa notazione sta nella facilità con cui si eseguono le operazioni:
$$2_{10}+4_{10}=0010_2+0100_2=0110_2=6_{10}$$
$$-3_{10}+\(-1_{10}\)=1101_2+1111_2=\cancel{1}1100_2=-4_{10}$$
$$5_{10}+\(-3_{10}\)=0101_2+1101_2=\cancel{1}0010_2=2_{10}$$

Questa notazione prende il nome di complemento in base 2.
Si chiama complemento in quanto i valori negativi si completano a quelli positivi.

Per passare dalla base 10 alla base 2 in complemento a 2, si effettuano i seguenti passaggi:
\begin{algorithm}[H]
  \caption{Conversione da 10MS a 2C2}\label{algo:10to2c2}
  \begin{algorithmic}[1]
    \State $n=\abs{x}$
    \If{$x<0$}
    \State $n=-n$
    \EndIf
  \end{algorithmic}
\end{algorithm}
\begin{example}
  $$x=-13_{10}$$
  $$\abs{x}=13_{10}=1101_2=01101_{2\text{MS}}\equiv 01101_{2\text{C}2}$$
  $$x=-\abs{x}=10011_{2\text{C}2}$$
\end{example}

Si presentano casi ambigui in cui apparentemente sembra che l'aritmetica non funzioni:
$$5_{10}+3_{10}=0101_2+0011_2=1000_2=-8_{10}$$
$$-2_{10}+\(-6_{10}\)=1110_2+1010_2=\cancel{1}1000_2=-8_{10}$$
Si nota che tutti i numeri positivi cominciano con 0 mentre quelli negativi cominciano con 1.
Convenzionalmente, in virtù di quanto appena detto, la combinazione $1000_2$ assume il valore $-8_{10}$.

Usando la notazione modulo e segno il range dei valori ammissibili è $\interval{-7}{7}$, mentre in complemento a 2 il range è $\interval{-8}{7}$.

Considerando che la dimensione dell'informazione è sempre fissa e determinata, ci si ritrova in casi particolari detti di \textbf{overflow} (\emph{traboccamento}):
$$6_{10}+4_{10}=0110_2+0100_2=1010_2=-6_{10}$$
L'overflow è facilmente risolvibile con un aumento dello spazio a disposizione:
$$6_{10}+4_{10}=00110_2+00100_2=01010_2=10_{10}$$
Si ha overflow quando due valori concordi generano un valore discorde dai primi due.
Quando gli operandi sono discordi è impossibile generare overflow.

\paragraph*{Osservazione} L'estensione di un valore in complemento a 2 si esegue ripetendo il MSB quante volte se ne ha bisogno:

\begin{example}
  $$5_{10}=0101_{2\text{C}2}=000000101_{2\text{C}2}$$
  $$-5_{10}=1011_{2\text{C}2}=111111011_{2\text{C}2}$$
\end{example}

\begin{example}
  $$x=+12_{10\text{MS}}=01100_{2\text{C}2}$$
  $$y=-3_{10\text{MS}}=101_{2\text{C}2}=11101_{2\text{C}2}$$
  $$x+y=01100_2+11101_2=\cancel{1}01001_2$$
  $$x-y=x+\(-y\)=01100_2+00011_2=01111_2$$
\end{example}

% lezione del 15/09/2022

Per passare dalla base 2 in complemento a 2 alla base 10, si possono verificare due scenari:
\begin{enumerate}
  \item il numero è positivo, pertanto lo si converte usando la formula dei pesi;
  \item il numero è negativo: in questo caso si calcola l'opposto, lo si converte e gli si cambia il segno.
\end{enumerate}
\begin{example}
  $$010110_{2\text{C}2}=22_{10\text{MS}}$$
  $$1011101_{2\text{C}2}=-35_{10\text{MS}}$$
\end{example}

\section{Rappresentazione esadecimale}
Il problema principale con la rappresentazione binaria è l'enorme spazio richiesto. Inoltre, in virtù dell'uso sui calcolatori, la base da cercare deve essere una potenza di 2.
Per anni è stata utilizzata la base 8, ma ben presto è stata resa obsoleta, in favore della base 16.

Il codice esadecimale si crea associando le cifre esadecimali e combinazioni di 4 bit del codice binario.
\begin{center}
  \begin{tblr}{c|c|c}
    \textbf{esadecimale} & \textbf{binario} & \textbf{decimale} \\\hline
    0                    & 0000             & 0                 \\
    1                    & 0001             & 1                 \\
    2                    & 0010             & 2                 \\
    3                    & 0011             & 3                 \\
    4                    & 0100             & 4                 \\
    5                    & 0101             & 5                 \\
    6                    & 0110             & 6                 \\
    7                    & 0111             & 7                 \\
    8                    & 1000             & 8                 \\
    9                    & 1001             & 9                 \\
    A                    & 1010             & 10                \\
    B                    & 1011             & 11                \\
    C                    & 1100             & 12                \\
    D                    & 1101             & 13                \\
    E                    & 1110             & 14                \\
    F                    & 1111             & 15                \\
  \end{tblr}
\end{center}

Per effettuare la conversione dalla base 2 alla base 16 basta considerare ogni quadrupla e convertirla in loco:
\begin{center}
  \begin{tblr}{cccc}
    0101 & 0101 & 1101 & 1010 \\ \hline
    5    & 5    & D    & A    
  \end{tblr}
\end{center}
Banalmente, la stessa cosa avviene per il processo inverso.

Al fine di rappresentare attraverso la notazione modulo e segno un numero esadecimale, gli si prepone il segno, facendo la conversione dal binario.

\section{Numeri razionali}

Si vuole rappresentare i valori espressi da:
$$\frac{m}{n},\; m\in\mathbb{Z},n\in\mathbb{N}/\left\{ 0 \right\}$$

Nella notazione decimale si usa scrivere prima la parte intera, un separatore decimale e poi la parte frazionaria.
In base 2 si può fare la stessa cosa:
$$101.01_2=2^2+2^0+2^{-2}=5.25_{10}$$
Avendo a disposizione una certa quantità bit, si sceglie la parte di essi che contiene la parte frazionaria e quella che contiene la parte intera. 

\begin{example}
  Si considerano 3 bit per la parte intera e 2 per quella frazionaria. Si ha che il range massimo esprimibile è $\interval{0}{7.75}$, con salti di $0.25$.
  
  Tuttavia, se si vuole rappresentare il valore $6.3$, non si può. Al limite, si può esprimere una sua approssimazione, che in questo caso è $6.25$.
\end{example}

Il fatto che non si possa rappresentare un valore con \textbf{precisione} non è dovuto ad una cattiva gestione dei bit, ma bensì al loro stato di finitezza.

La notazione utilizza si chiama a virgola fissa: viene stabilita la dimensione a priori sia per la parte intera sia per la parte frazionaria.
In una notazione del genere l'\textbf{errore assoluto} $\epsilon_A$ è costante: si sbaglia sempre della stessa quantità.

Esiste, tuttavia, una notazione alternativa che consente di variare l'errore assoluto e mantenere costante l'errore relativo: la notazione in \textbf{virgola mobile} (\emph{floating point} in inglese).
L'errore relativo $$\epsilon_R=\frac{\epsilon_A}{\text{valore}}$$

Per rappresentare un numero razionale in base 2, si procede come al solito per la parte intera, ovvero si divide per 2 segnando il resto, ma al contrario per la parte frazionaria, ovvero si moltiplica per 2 segnando l'unità. Un'altra differenza fondamentale è il verso di lettura delle cifre: mentre per la parte intera si procede dal basso verso l'alto, per la parte frazionaria si procede dall'alto verso il basso.

\begin{example}
  Si vuole rappresentare il numero $13.75$.
  \begin{center}
    \begin{minipage}{0.2\linewidth}
      \begin{center}
        \begin{tblr}{c|c}
          13 & 1 \\
          6  & 0 \\
          3  & 1 \\
          1  & 1 \\
          0  &   
        \end{tblr}
      \end{center}
    \end{minipage}
    \begin{minipage}{0.2\linewidth}
      \begin{center}
        \begin{tblr}{c|c}
          0.75 &   \\
          1.5  & 1 \\
          1.0  & 1 \\
          0    &   \\
        \end{tblr}
      \end{center}
    \end{minipage}
  \end{center}
  $$13.75_{10}=1101.11_2$$
\end{example}

\begin{example}
  Si vuole rappresentare il numero $7.32$.
  \begin{center}
    \begin{minipage}{0.2\linewidth}
      \begin{center}
        \begin{tblr}{c|c}
          7 & 1 \\
          3 & 1 \\
          1 & 1 \\
          0 &   
        \end{tblr}
      \end{center}
    \end{minipage}
    \begin{minipage}{0.2\linewidth}
      \begin{center}
        \begin{tblr}{c|c}
          0.32     &          \\
          0.64     & 0        \\
          1.28     & 1        \\
          0.56     & 0        \\
          1.12     & 1        \\
          0.24     & 0        \\
          0.48     & 0        \\
          0.96     & 0        \\
          1.92     & 1        \\
          1.84     & 1        \\
          $\cdots$ & $\cdots$ 
        \end{tblr}
      \end{center}
    \end{minipage}
  \end{center}
  $$7.32_{10}=111.010100011...$$
\end{example}

Se si considerano i razionali da una diversa prospettiva, si nota che:
$$13.75_{10}=1.375\cdot 10^1$$
$$7.32_{10}=7.32\cdot 10^0$$
Analogamente:
$$1101.11_2=1.10111\cdot2^3$$
$$111.010100011_2=1.11010100011\cdot2^2$$

La notazione in virgola mobile, quindi, ha:
\begin{itemize}
  \item un bit per il \textbf{segno};
  \item una parte per l'\textbf{esponente}, a cui viene aggiunto un numero tale che il più piccolo esponente possibile sia 0;
  \item una parte per la \textbf{mantissa}, ossia la parte frazionaria della notazione scientifica\footnote{si prende la parte frazionaria in quanto la parte intera sarà sempre 1, quindi è inutile sprecare un bit}.
\end{itemize}
Lo standard che regola la notazione in virgola mobile è lo \textbf{IEEE 754}. Lo standard prevede 3 versioni:
\begin{itemize}
  \item 32 bit: 1 per il segno, 8 per l'esponente, 23 per la mantissa;
  \item 64 bit: 1 per il segno, 11 per l'esponente, 52 per la mantissa;
  \item 128 bit: 1 per il segno, 15 per l'esponente, 112 per la mantissa.
\end{itemize}

Il valore di un numero espresso attraverso lo standard IEEE 754 è dato da:
$$\text{valore}=\(-1\)^S\(1+M\)\cdot2^E$$
dove $S$ è il segno, $M$ è la mantissa e $E$ è l'esponente

\paragraph*{Nota bene}
La costante che bisogna aggiungere all'esponente si chiama ``eccesso'' ed è:
\begin{itemize}
  \item 127 per la singola precisione;
  \item 1023 per la doppia precisione;
  \item 16383 per la quadrupla precisione.
\end{itemize}

\begin{example}
  $$+5.65_{10\text{MS}}$$
  $$+101.10{1001}_{2\text{MS}}=1.01101001\cdot2^2=01000000101101001100110011001100$$
  $$-0.028_{10\text{MS}}$$
  $$-0.00000111001010_2=-1.1100101\cdot2^{-6}=101001111110101...$$
\end{example}

Lo standard, tuttavia, presenta delle falle che vengono compensate dall'introduzione di combinazioni di bit speciali che, proprio per questo, vengono dette \textbf{denormarlizzate}.

\paragraph*{Forma denormarlizzata generale}
Quando tutti i bit dell'esponente sono 0, la mantissa non è sommata ad 1 ma bensì a 0.

\paragraph*{Zero}
Quando tutti i bit (indifferentemente dal primo) sono 0, il valore è 0.

\paragraph*{Infinito}
Quando tutti i bit dell'esponente sono 1 e quelli della mantissa sono tutti 0, il valore è infinito, che può essere sia negativo che positivo.

\paragraph*{NaN}
Quando tutti i bit dell'esponente sono 1 e almeno uno di quelli della mantissa è 1, il valore non è un numero (\textbf{NaN}, \emph{Not a Number}).

% lezione del 16/09/2022

\section{L'informazione non numerica}

Per ovvi motivi, non si può rappresentare solo l'informazione numerica. L'obiettivo è, quindi, cercare di definire una codifica per ogni possibile carattere rappresentabile. Inizialmente si hanno:
\begin{itemize}
  \item alfabeto base (a...z, A...Z);
  \item caratteri numerici (0...9);
  \item caratteri di interpunzione (.,;:);
  \item caratteri speciali.
\end{itemize}

Tutti questi elementi sono circa 120, per cui sono necessari 7 bit.
Per la rappresentazione di tali simboli si utilizza il \textbf{codice ASCII} (\emph{American Standard Code for Information Interchange}): il codice dispone di 7 bit, per cui ha 128 combinazioni.
Per far fronte alle necessità sorte nel tempo, si è aggiunto un bit al codice ASCII, creando il codice \textbf{ASCII esteso}: esso dispone di 8 bit, e comprende caratteri nazionali, simboli e cornici.


\section{Successioni}

\begin{definition}[Successione]
  Una successione in un dato insieme $X$ è una funzione $f:D\to X$ con dominio $D$ numerabile (il più delle volte $D\subseteq \mathbb{N}$). La si indica con $\left\{ x_n \right\}_{n\in D}\subseteq X$.
\end{definition}

\subsection{Limite}

\begin{definition}[Limite di successione]
  Una successione $\left\{ x_n \right\}_{n\in\mathbb{N}}\subseteq\mathbb{Q}$ converge al numero $\l\in\mathbb{Q}$, in simboli:
  $$\lim_{n\to+\infty} x_n=\l\ \mathrm{oppure}\ x_n\xrightarrow{n\to+\infty}\l$$
  se:
  $$\forall\epsilon>0\ \exists\ N\(\epsilon\)\in\mathbb{N}:n\ge N\(\epsilon\)\impl \abs{x_n-\l}<\epsilon$$
  cioè:
  $$\forall\epsilon>0\ \exists\ N\(\epsilon\)\in\mathbb{N}:n\ge N\(\epsilon\)\impl \l-\epsilon<x_n<\l+\epsilon$$
  cioè:
  $$\forall\epsilon>0\ \exists\ N\(\epsilon\)\in\mathbb{N}:n\ge N\(\epsilon\)\impl x_n\in\ointv{\l-\epsilon}{\l+\epsilon}$$
  \begin{center}
    \begin{tikzpicture}[scale=1.5]
      \draw[-stealth] (0,0) -- (6,0) node [right] {$\mathbb{Q}$};
      \draw (3,0) node [label=below:$\l$] {};
      \draw (3,0) circle (1.5pt);
      \draw (2.5,0) node [label=$x_n$] {};
      \fill (2.5,0) circle (1.5pt);
      \draw (2,0) node [label=below:$\l-\epsilon$] {$\vert$};
      \draw (4,0) node [label=below:$\l+\epsilon$] {$\vert$};
    \end{tikzpicture}
  \end{center}
\end{definition}

\begin{example}
  $$\lim_{n\to+\infty}\frac{1}{n}=0$$
  \begin{center}
    \begin{tikzpicture}[scale=5]
      \draw[-stealth] (-0.3,0) -- (1.3,0) node [right] {$\mathbb{Q}$};
      \draw (0,0) node [label=below:$0$] {$\vert$};
      \foreach \n in {1,...,5} {
          \draw (1/\n,0) node {$\vert$};
          \node[below=8pt] at (1/\n,0) {$\frac{1}{\n}$};
        }
    \end{tikzpicture}
  \end{center}
  Fissato $\epsilon>0$, si ha:
  $$0-\epsilon<\frac{1}{n}<0+\epsilon$$
  $$-\epsilon<\frac{1}{n}<\epsilon\iff \frac{1}{n}<\epsilon$$
  $$n>\frac{1}{\epsilon}$$
  Si sceglie $N\(\epsilon\)$ come il più piccolo numero naturale maggiore di $\nicefrac{1}{\epsilon}$, in modo tale da avere:
  $$n\ge N\(\epsilon\)>\frac{1}{\epsilon}$$
  e, quindi, verificare la definizione di limite.
\end{example}

\begin{example}
  $$\lim_{n\to+\infty}n=+\infty$$
  \begin{center}
    \begin{tikzpicture}[scale=1.5]
      \draw[-stealth] (0,0) -- (5,0) node [right] {$\mathbb{Q}$};
      \draw (1,0) node [label=below:$a_{n}$] {};
      \fill (1,0) circle (1.5pt);
      \draw (4,0) node [label=below:$a_{n+1}$] {};
      \fill (4,0) circle (1.5pt);
      \draw (2,0) node [label=below:$\l$] {};
      \draw (2,0) circle (1.5pt);
    \end{tikzpicture}
  \end{center}
  Si dice che la successione tende a $+\infty$, ossia il limite esiste ma non è un numero $\in\mathbb{Q}$.
\end{example}

\subsection{Proprietà}

\begin{lemma}
  Se $\exists \lim_{n\to+\infty}a_n\in\mathbb{Q}$, allora tale limite è unico.
\end{lemma}
\begin{proof}
  Siano $a_n\xrightarrow{n\to+\infty}\l_1$ e $a_n\xrightarrow{n\to+\infty}\l_2$. Si ha:
  $$\forall\epsilon>0\ \exists\ N_1\(\epsilon\)\in\mathbb{N}:n\ge N_1\(\epsilon\)\impl \abs{a_n-\l_1}<\epsilon$$
  $$\forall\epsilon>0\ \exists\ N_2\(\epsilon\)\in\mathbb{N}:n\ge N_2\(\epsilon\)\impl \abs{a_n-\l_2}<\epsilon$$
  Fissato $\epsilon>0$, se $n\ge N\(\epsilon\)\walrus\max\left\{ N_1\(\epsilon\),N_2\(\epsilon\) \right\}$, allora $n\ge N_1\(\epsilon\)\wedge n\ge N_2\(\epsilon\)$ e:
  $$0\le \abs{\l_1-\l_2}<\abs{\l_1-a_n+a_n-\l_2}\le\abs{\l_1-a_n}+\abs{a_n-\l_2}\le\epsilon+\epsilon=2\epsilon$$
  $$0\le\abs{\l_1-\l_2}\le2\epsilon\impl \abs{\l_1-\l_2}=0\iff \l_1=\l_2$$
\end{proof}
\begin{lemma}
  Se $\exists \lim_{n\to+\infty}a_n\in\mathbb{Q}$, allora $\left\{ a_n \right\}$ è limitata.
\end{lemma}
\begin{proof}
  $$\l\walrus\lim_{n\to+\infty}a_n$$
  $$\forall\epsilon>0\ \exists\ N\(\epsilon\)\in\mathbb{N}:n\ge N\(\epsilon\)\impl a_n\in\ointv{\l-\epsilon}{\l+\epsilon}$$
  Allora $\left\{ a_n \right\}_{n\ge N\(\epsilon\)}\subset \ointv{\l-\epsilon}{\l+\epsilon}$, ed è quindi limitata, e $\left\{ a_n \right\}_{n< N\(\epsilon\)}$ è limitata in quanto insieme finito. Quindi:
  $$\left\{ a_n \right\}_{n\ge0}=\left\{ a_n \right\}_{n< N\(\epsilon\)}\cup \left\{ a_n \right\}_{n\ge N\(\epsilon\)}$$
  è limitata, in poiché unione di insiemi limitati.
\end{proof}
\begin{lemma}
  $$
    \begin{cases}
      \exists\ \l_1\walrus\lim_{n\to+\infty}a_n \\
      \exists\ \l_2\walrus\lim_{n\to+\infty}b_n \\
    \end{cases}
    \impl
    \lim_{n\to+\infty}\(a_n+b_n\)=\lim_{n\to+\infty}a_n+\lim_{n\to+\infty}b_n
  $$
\end{lemma}
\begin{proof}
  Dalle ipotesi, si ha:
  $$0\le\abs{\(a_n+b_n\)-\(\l_1+\l_2\)}=\abs{\(a_n-\l_1\)-\(b_n-\l_2\)}\le\abs{a_n-\l_1}+\abs{b_n-\l_2}<\epsilon+\epsilon=2\epsilon$$
  Dato che si può scegliere $\epsilon$ arbitrariamente piccolo, si ha:
  $$\abs{\(a_n+b_n\)-\(\l_1+\l_2\)}=0\iff \(a_n+b_n\)-\(\l_1+\l_2\)=0\iff a_n+b_n=\l_1+\l_2$$
\end{proof}
\begin{lemma}
  $$
    \begin{cases}
      \exists\ \l_1\walrus\lim_{n\to+\infty}a_n \\
      \exists\ \l_2\walrus\lim_{n\to+\infty}b_n \\
    \end{cases}
    \impl
    \lim_{n\to+\infty}\(a_n\cdot b_n\)=\(\lim_{n\to+\infty}a_n\)\cdot\(\lim_{n\to+\infty}b_n\)
  $$
\end{lemma}
\begin{proof}
  Dalle ipotesi, si ha:
  \begin{align*}
    0 & \le \abs{a_nb_n-\l_1\l_2}                         \\
      & =\abs{a_nb_n-\l_1b_n+\l_1b_n-\l_1\l_2}            \\
      & =\abs{b_n\(a_n-\l_1\)+\l_1\(b_n-\l_2\)}           \\
      & \le\abs{b_n\(a_n-\l_1\)}+\abs{\l_1\(b_n-\l_2\)}   \\
      & =\abs{b_n}\abs{a_n-\l_1}+\abs{\l_1}\abs{b_n-\l_2} 
  \end{align*}
  Dato che $\left\{ b_n \right\}$ converge, allora è anche limitata, per cui:
  $$\exists\ M\ge0:\abs{b_n}<M\ \forall n\ge0$$
  Da cui:
  $$0\le \abs{a_nb_n-\l_1\l_2} \le M\abs{a_n-\l_1}+\abs{\l_1}\abs{b_n-\l_2}<M\epsilon+\abs{\l_1}\epsilon=\epsilon\(M+\abs{\l_1}\)$$
  Dato che si può scegliere $\epsilon$ arbitrariamente piccolo, si ha:
  $$\abs{a_nb_n-\l_1\l_2}=0\iff a_nb_n-\l_1\l_2=0\iff a_nb_n=\l_1\l_2$$
\end{proof}
\begin{lemma}
  $$\lim_{n\to+\infty}\frac{a_n}{b_n}=\frac{\lim_{n\to+\infty}a_n}{\lim_{n\to+\infty}b_n}$$
\end{lemma}

\begin{lemma}[Convergenza delle successioni costanti]
  Sia $\left\{ c \right\}$, con $c\in\mathbb{Q}$ fissato.
  $$\lim_{n\to+\infty} c=c$$
\end{lemma}
\begin{proof}
  $$\forall \epsilon>0\ \abs{a_n-\l}=\abs{c-c}=0<\epsilon$$
\end{proof}
\begin{observation}
  Se $\exists\ \l\walrus\lim_{n\to+\infty}a_n$, si verifica il seguente fenomeno:
  \begin{center}
    \begin{tikzpicture}[scale=1.5]
      \draw[-stealth] (0,0) -- (6,0) node [right] {$\mathbb{Q}$};
      \draw[dashdotted] (3,-.3) node [below] {$N\(\epsilon\)$} -- (3,3.3);
      \draw[] (0,1.5) node [left] {$\l$} -- (6,1.5);
      \draw[dashed] (0,0.9) node [left] {$\l-\epsilon$} -- (6,0.9);
      \draw[dashed] (0,2.1) node [left] {$\l+\epsilon$} -- (6,2.1);
      \foreach \i in {1,...,8} {
          \pgfmathrandominteger{\x}{1}{15}
          \fill (\i / 3,\x / 5) circle (1.5pt);
        }
      \foreach \i in {6,...,11} {
          \pgfmathrandominteger{\x}{1}{5}
          \fill (\i / 2,\x / 5 + 1) circle (1.5pt);
        }
    \end{tikzpicture}
  \end{center}
\end{observation}

\begin{example}
  \begin{align*}
    \lim_{n\to+\infty} \frac{n}{n+1} & =\lim_{n\to+\infty}\frac{n+1-1}{n+1}                               \\
                                     & =\lim_{n\to+\infty}\frac{n+1}{n+1}-\lim_{n\to+\infty}\frac{1}{n+1} \\
                                     & =\lim_{n\to+\infty}1-\lim_{n\to+\infty}\frac{1}{n+1}               \\
                                     & =1-0=1                                                             
  \end{align*}
\end{example}

\begin{theorem}[Permanenza del segno]
  $$a_n\ge0\wedge\exists\ \l\walrus\lim_{n\to+\infty}a_n\impl\l\ge0$$
\end{theorem}
\begin{proof}
  Per assurdo, sia $\l<0$. Sia $\epsilon\walrus-\nicefrac{\l}{2}>0$. Poiché esiste il limite, si ha:
  $$\exists\ N\(\epsilon\)\ge0:n\ge N\(\epsilon\)\impl \l-\epsilon<a_n<\l+\epsilon$$
  da cui:
  $$\l+\nicefrac{\l}{2}<a_n<\l-\nicefrac{\l}{2}$$
  $$\nicefrac{3}{2}\l<a_n<\nicefrac{1}{2}\l<0$$
  per cui si ha l'assurdo.
\end{proof}

\begin{theorem}[Monotonia]
  $$
    \begin{cases}
      a_n\le b_n                                \\
      \exists\ \l_1\walrus\lim_{n\to+\infty}a_n \\
      \exists\ \l_2\walrus\lim_{n\to+\infty}b_n \\
    \end{cases}
    \impl
    \l_1\le\l_2
  $$
\end{theorem}

\begin{theorem}
  $$\lim_{n\to+\infty}a_n=0\wedge \left\{ b_n \right\}\subseteq\ointv{a}{b}\impl\exists\ \lim_{n\to+\infty}a_nb_n=0$$  
\end{theorem}

\begin{definition}
  $$\lim_{n\to-\infty}a_n=\l\iff \forall\epsilon>0\ \exists\ N\(\epsilon\)\le0:n\le N\(\epsilon\)\impl\abs{a_n-\l}<\epsilon$$
\end{definition}

\begin{definition}
  $$\lim_{n\to+\infty}a_n=+\infty\iff \forall M\in\mathbb{Q}\ \exists\ N\(M\)\ge0:n\ge N\(M\)\impl a_n\ge M$$
\end{definition}

Tutte le proprietà dei limiti continuano a valere anche nel caso in cui $\left\{ a_n \right\}$ e $\left\{ b_n \right\}$ convergano a $\pm\infty$, tranne nei casi seguenti, detti di indecisione:
\begin{itemize}
  \item $+\infty-\infty$
  \item $0\cdot \infty$
  \item $\nicefrac{\infty}{\infty}$
  \item $\nicefrac{0}{0}$
\end{itemize}
Per convenzione $\infty\cdot\infty=\infty$.

\begin{example}
  $$a_n\walrus\(1+\frac{1}{n}\)^n$$
  \begin{align*}
    a_n & =\(1+\frac{1}{n}\)^n                                                               \\
        & =\sum_{k=0}^n\binom{n}{k}1^{n-k}\(\frac{1}{n}\)^k                                  \\
        & =\sum_{k=0}^n\frac{n!}{k!\(n-k\)!}\frac{1}{n^k}                                    \\
        & =\frac{n!}{0!n!n^0}+\frac{n!}{1!\(n-1\)!n^1}+\sum_{k=2}^n\frac{n!}{n^kk!\(n-k\)!}  \\
        & =2+\sum_{k=2}^n\frac{n\(n-1\)\cdots\(n-k+1\)\(n-k\)!}{n^kk!\(n-k\)!}               \\
        & =2+\sum_{k=2}^n\frac{n\(n-1\)\cdots\(n-k+1\)}{n^kk!}                               \\
        & =2+\sum_{k=2}^n\frac{1}{k!}\cdot\frac{n}{n}\cdot\frac{n-1}{n}\cdots\frac{n-k+1}{n} \\
        & =2+\sum_{k=2}^n\frac{1}{k!}\(1-\frac{1}{n}\)\cdots\(1-\frac{k-1}{n}\)              
  \end{align*}
  \begin{align*}
    a_{n+1} & =2+\sum_{k=2}^{n+1}\frac{1}{k!}\(1-\frac{1}{n+1}\)\cdots\(1-\frac{k-1}{n+1}\) \\
            & \ge2+\sum_{k=2}^{n}\frac{1}{k!}\(1-\frac{1}{n+1}\)\cdots\(1-\frac{k-1}{n+1}\) \\
            & \ge2+\sum_{k=2}^{n}\frac{1}{k!}\(1-\frac{1}{n}\)\cdots\(1-\frac{k-1}{n}\)     \\
            & =a_n                                                                          
  \end{align*}
  $$a_{n+1}\ge a_n$$
  Essendo crescente, $\left\{ a_n \right\}$ è inferiormente limitata:
  $$a_n\ge a_1=\(1+\frac{1}{1}\)^1=2$$
  Inoltre:
  \begin{align*}
    a_n & <2+\sum_{k=2}^n\frac{1}{k!}                           \\
        & \le 2+\sum_{k=2}^n\frac{1}{2^{k-1}}                   \\
        & =2+\sum_{k=0}^n\frac{1}{2^{k-1}}-2-1                  \\
        & =\sum_{k=0}^n\frac{1}{2^{k-1}}-1                      \\
        & =\sum_{k=0}^n\frac{2}{2^{k}}                          \\
        & =2\sum_{k=0}^n\frac{1}{2^k}-1                         \\
        & =2\(\frac{1-\(\frac{1}{2}\)^{n+1}}{1-\frac{1}{2}}\)-1 \\
        & =4\(1-\frac{1}{2^{n+1}}\)-1                           
  \end{align*}
  Passando al limite:
  $$
    \lim_na_n<\lim_n 4\(1-\frac{1}{2^{n+1}}\)-1 = 4-1=3
  $$
  In conclusione, $\forall n\ge1\ a_n\in\rintv{2}{3}$.
\end{example}

\subsection{Costruzione di $\reals$}

\begin{theorem}[Proprietà di Cauchy]
  Se $\exists\ \lim_{n}a_n\in\mathbb{Q}$, allora:
  $$\forall\epsilon>0\ \exists\ N\(\epsilon\)\ge0:n,m\ge N\(\epsilon\)\impl \abs{a_n-a_m}<\epsilon$$
\end{theorem}
\begin{proof}
  Sia $\l\walrus\lim_na_n$. Allora:
  $$\forall\epsilon>0\ \exists\ N\(\epsilon\)\ge0:n\ge N\(\epsilon\)\impl \abs{x_n-\l}<\epsilon$$
  Da cui, se $n,m\ge N\(\epsilon\)$:
  $$0\le\abs{a_n-a_m}=\abs{\(a_n-\l\)+\(\l-a_m\)}\le\abs{a_n-\l}+\abs{a_m-\l}\le 2\epsilon$$
  Perciò:
  $$\abs{a_n-a_m}<2\epsilon$$
\end{proof}


\begin{definition}[Relazione di equivalenza]
  Sia $X$ un insieme. Una \textbf{relazione di equivalenza} su $X$ è un insieme $R\subset X\times X$, avente le seguenti proprietà:
  \begin{itemize}
    \item riflessività: $\(x,x\)\in R\ \forall x\in X$ oppure $x\stackrel{R}{\sim}x$
    \item simmetria: $\(x,y\)\in R\impl \(y,x\)\in R$ oppure $x\stackrel{R}{\sim}y\impl y\stackrel{R}{\sim}x$
    \item transitività: $\(x,y\),\(y,z\)\in R\impl \(x,z\)\in R$ oppure $x\stackrel{R}{\sim}y\wedge y\stackrel{R}{\sim}z\impl x\stackrel{R}{\sim}z$
  \end{itemize}
\end{definition}

\begin{definition}[Insieme quoziente]
  Sia $\left[ x \right]_R\walrus\left\{ y\in X:x\stackrel{R}{\sim}y \right\}\subset X$. L'\textbf{insieme quozionte} è definito come:
  $$\nicefrac{X}{R}\walrus\left\{ \left[ x \right]_R:x\in X \right\}$$
\end{definition}

Sia $X\walrus\left\{ \text{successioni di Cauchy in }\mathbb{Q} \right\}$. Le successioni $\left\{ a_n \right\},\left\{ b_n \right\}$ sono in relazione $R$ se:
$$\lim_{n\to+\infty}\(a_n-b_n\)=0$$
Si dimostra che che $R$ è una relazione di equivalenza.

\begin{definition}[Numeri reali]
  L'insieme dei \textbf{numeri reali} è definito come:
  $$\reals\walrus \nicefrac{X}{R}$$
\end{definition}

Si definiscono le seguenti operazioni su $\reals$.

\begin{definition}[Somma in $\reals$]
  Siano $\left[ \left\{ a_n \right\} \right]_R,\left[ \left\{ b_n \right\} \right]_R\in\reals$. La loro somma è:
  $$\left[ \left\{ a_n \right\} \right]_R+\left[ \left\{ b_n \right\} \right]_R\walrus\left[ \left\{ a_n+b_n \right\} \right]_R$$
\end{definition}

\begin{definition}[Prodotto in $\reals$]
  Siano $\left[ \left\{ a_n \right\} \right]_R,\left[ \left\{ b_n \right\} \right]_R\in\reals$. Il loro prodotto è:
  $$\left[ \left\{ a_n \right\} \right]_R\cdot\left[ \left\{ b_n \right\} \right]_R\walrus\left[ \left\{ a_n\cdot b_n \right\} \right]_R$$
\end{definition}

\begin{observation}
  Queste operazioni godono delle ``solite'' proprietà: associatività, distributività, elementi neutri, etc...
\end{observation}

\begin{definition}[Positività]
  Si dirà che $\left[ \left\{ a_n \right\} \right]_R\in\reals$ è positivo se:
  $$\exists\ q>0\in\mathbb{Q},\exists\ N\ge0:n\ge N\impl a_n\ge q$$
\end{definition}

\begin{definition}[Ordine in $\reals$]
  Siano $\left[ \left\{ a_n \right\} \right]_R,\left[ \left\{ b_n \right\} \right]_R\in\reals$. Si stabilisce l'ordine:
  $$\left[ \left\{ a_n \right\} \right]_R<\left[ \left\{ b_n \right\} \right]_R\iff \left[ \left\{ b_n \right\} \right]_R-\left[ \left\{ a_n \right\} \right]_R>0$$
\end{definition}

\begin{observation}
  Si può ``immergere'' $\mathbb{Q}$ in $\reals$, poiché esiste una funzione iniettiva $j:\mathbb{Q}\to\reals$:
  $$j\(x\)\walrus\left[ \left\{ x \right\} \right]_R\in\reals$$
  $$j\(x\)+j\(y\)=j\(x+y\)$$
  $$j\(x\)\cdot j\(y\)=j\(x\cdot y\)$$
\end{observation}

\begin{definition}[Intervalli in $\reals$]
  Siano $a,b\in\reals:a<b$:
  \begin{itemize}
    \item l'insieme $\ointv{a}{b}\walrus\left\{ x\in\reals:a<x<b \right\}$ è detto \textbf{intervallo aperto};
    \item l'insieme $\intv{a}{b}\walrus\left\{ x\in\reals:a\le x\le b \right\}$ è detto \textbf{intervallo chiuso};
    \item l'insieme $\lintv{a}{b}\walrus\left\{ x\in\reals:a<x\le b \right\}$ è detto \textbf{intervallo semiaperto a sinistra};
    \item l'insieme $\rintv{a}{b}\walrus\left\{ x\in\reals:a\le x<b \right\}$ è detto \textbf{intervallo semiaperto a destra}.
  \end{itemize}
\end{definition}

\begin{definition}[Limite di successione]
  Una successione $\left\{ x_n \right\}\subseteq\reals$ converge al numero $\l\in\reals$, in simboli:
  $$\lim_{n\to+\infty} x_n=\l\ \mathrm{oppure}\ x_n\xrightarrow{n\to+\infty}\l$$
  se:
  $$\forall\epsilon>0\ \exists\ N\(\epsilon\)\in\mathbb{N}:n\ge N\(\epsilon\)\impl \abs{x_n-\l}<\epsilon$$
\end{definition}

\begin{observation}
  Tutte le proprietà dei limiti di successioni razionali restano valide anche per limiti di successioni di numeri reali.
\end{observation}

\begin{theorem}[Teorema fondamentale di completezza di $\reals$]
  La condizione di Cauchy per successioni di numeri reali è necessaria e sufficiente per la convergenza in $\reals$.
\end{theorem}

\begin{theorem}[Convergenza delle successioni monotone limitate]
  Sia $\left\{ x_n \right\}\subset\reals$ una successione monotona crescente e superiormente limitata:
  $$x_n\le x_{n+1}$$
  $$\exists\ M\in\reals:x_n\le M$$
  Allora $\left\{ x_n \right\}\subset\reals$ è di Cauchy e quindi convergente:
  $$\exists\ \lim_{n\to+\infty}x_n\in\reals$$
\end{theorem}
\begin{proof}
  Per la completezza di $\reals$ è sufficiente dimostrare che $\left\{ x_n \right\}$ è di Cauchy, ossia:
  $$\forall\epsilon>0\ \exists\ N\(\epsilon\)\ge1:n,m\ge N\(\epsilon\)\impl \abs{x_n-x_m}<\epsilon$$
  Per assurdo, se $\left\{ x_n \right\}$ non è di Cauchy, allora:
  $$\exists\epsilon>0\ \forall N\(\epsilon\)\ge1:\exists\ n,m\ge N\(\epsilon\)\impl \abs{x_n-x_m}\ge\epsilon$$
  Senza ledere la generalità della tesi, si assume che $x_n>x_m$. In questo modo:
  $$x_n-x_m=\abs{x_n-x_m}\ge\epsilon\iff x_n\ge\epsilon+x_m$$
  $$N\(\epsilon\)=1\impl\exists\ n_1>m_1:x_{n_1}\ge\epsilon+x_{m_1}$$
  $$N\(\epsilon\)=n_1\impl \exists\ n_2>m_2\ge N\(\epsilon\) \ge n_1 >m_1:x_{n_2}\ge\epsilon+x_{m_2}\ge\epsilon+x_{n_1}\ge 2\epsilon+x_{m_1}$$
  $$N\(\epsilon\)=n_k>m_k\impl\exists\ n_{k+1}>m_{k+1}\ge N\(\epsilon\)\ge n_k>m_k$$
  Per la monotonia del limite, si ha:
  $$\lim_{k\to+\infty}x_{n_{k+1}}\ge \lim_{k\to+\infty}k\epsilon+x_{m_1}=\epsilon\(+\infty\)+x_{m_1}=+\infty$$
  Ma ciò contraddice l'ipotesi di limitatezza della successione, per cui $\left\{ x_n \right\}$ è di Cauchy e pertanto converge in $\reals$.
\end{proof}

\begin{definition}[Maggiorante e minorante]
  Sia $A\subseteq\reals$.
  $x_0\in\reals$ è detto \textbf{maggiorante} di $A$ se $x\in A\impl x<x_0$.
  Analogamente, $x_0$ è detto \textbf{minorante} di $A$ se $x\in A\impl x>x_0$.
\end{definition}

\begin{definition}[Estremi superiore ed inferiore]
  Sia $A\subseteq\reals$.
  $x_0$ è detto \textbf{estremo superiore} di $A$, indicato con $\sup A$, se $x_0$ è maggiorante di $A$ ed è il suo più piccolo maggiorante, vale a dire che $\forall\epsilon>0\ x_0-\epsilon$ non è maggiorante di $A$.
  Analogamente, $x_0$ è detto \textbf{estremo inferiore} di $A$, indicato con $\inf A$, se $x_0$ è minorante di $A$ ed è il suo più grande minorante, vale a dire che $\forall\epsilon>0\ x_0+\epsilon$ non è minorante di $A$.
\end{definition}

\begin{definition}[Massimo e minimo]
  Sia $A\subseteq\reals$.
  Se $\sup A\in A$, allora $\sup A$ è detto \textbf{massimo} di $A$.
  Se $\inf A\in A$, allora $\inf A$ è detto \textbf{minimo} di $A$.
\end{definition}

\begin{theorem}[Esistenza di $\sup A$ e/o $\inf A$]
  Se $A\subseteq\reals$ è superiormente limitato, ossia ha almeno un maggiorante, allora $\exists\ \sup A\in\reals$.
\end{theorem}
\begin{proof}
  Poiché $A$ è superiormente limitato, $\exists\ b_0\in\reals:x\in A\impl x\le b_0$. Sia $a_0\in A:\intv{a_0}{b_0}\cap A\neq\emptyset$. Sia $c$ il punto medio dell'intervallo: $$c\walrus\frac{a_0+b_0}{2}$$
  In almeno uno degli intervalli $\intv{a_0}{c}$ e $\intv{c}{b_0}$ vi sono punti di $A$. Senza ledere la generalità della dimostrazione, si considera l'ultimo intervallo e lo si indica con $\intv{a_1}{b_1}$:
  $$\intv{a_1}{b_1}\subset\intv{a_0}{b_0}$$
  $$\intv{a_1}{b_1}\cap A\neq \emptyset$$
  $$x\in A\impl x\le b_1$$
  Iterando questo processo (che prende il nome di dicotomia), si ottiene una successione decrescente di intervalli:
  $$\intv{a_{n+1}}{b_{n+1}}\subset\intv{a_n}{b_n}\subset\cdots\subset\intv{a_0}{b_0}$$
  per cui si verifica che $\forall n\ge0\ b_n$ è un maggiorante di $A$.
  Si ha che $\left\{ a_n \right\}$ è crescente e superiormente limitata da $b_0$ e $\left\{ b_n \right\}$ è decrescente e inferiormente limitata da $a_0$.
  Per il teorema di convergenza delle successioni monotone limitate:
  $$\exists\ \lim_{n\to+\infty}a_n\quad\wedge\quad\exists\ \lim_{n\to+\infty}b_n$$
  Inoltre:
  $$\lim_{n\to+\infty}\(b_n-a_n\)=\lim_{n\to+\infty}\frac{b_0-a_0}{2^n}=0$$
  da cui:
  $$\reals\ni\bar{x}\walrus \lim_{n\to+\infty}a_n=\lim_{n\to+\infty}b_n$$
  $$\forall x\in A\impl x\le b_n$$
  $$\forall x\in A\impl \lim_{n\to+\infty}x\le \lim_{n\to+\infty}b_n$$
  $$\forall x\in A\impl x\le \bar{x}$$
  ossia $\bar{x}$ è un maggiorante di $A$.
  Sia $\epsilon>0$.
  $$\bar{x}=\lim_{n\to+\infty}a_n\impl\exists\ n\ge1:\bar{x}-\epsilon<a_n\le\bar{x}$$
  $$\intv{a_n}{b_n}\cap A\neq\emptyset\impl \exists\ x\in\intv{a_n}{b_n}\cap A\impl x\ge a_n$$
  $$\bar{x}-\epsilon<a_n\le x\le b_n$$
  pertanto $\bar{x}-\epsilon$ non è un maggiorante.
  Quindi $\bar{x}\in R$ è il più piccolo maggiorante, ossia $\bar{x}=\sup A$.
\end{proof}

\begin{definition}[Punti di accumulazione]
  Sia $E\subseteq\reals$. Un punto $\bar{x}\in\reals$ è detto \textbf{punto di accumulazione} di $E$ se:
  $$\forall\epsilon>0\ \(E\setminus\left\{ \bar{x} \right\}\)\cap\ointv{\bar{x}-\epsilon}{\bar{x}+\epsilon}\neq\emptyset$$
  ovvero:
  $$\forall\epsilon>0\ \exists\ x\in E:x\neq \bar{x}\wedge \abs{x-\bar{x}}<\epsilon$$
  ovvero:
  $$\forall\epsilon>0\ \#\(E\cap\ointv{\bar{x}-\epsilon}{\bar{x}+\epsilon}\)=+\infty$$
  Se $\bar{x}\in E$ non è di accumulazione, è detto \textbf{isolato}.
\end{definition}
\begin{observation}
  Non è detto che $\bar{x}\in E$.
\end{observation}

\begin{definition}[Insieme derivato]
  In generale, si denota con $E'$ l'insieme dei punti di accumulazione di $E$, detto \textbf{insieme derivato} di $E$.
\end{definition}

\begin{theorem}[Proprietà di Bolzano--Weierstrass]
  Sia $E\subseteq \reals$ limitato e infinito. Allora $\exists\ \bar{x}\in E'$.
\end{theorem}
\begin{proof}
  Poiché limitato, $E\subseteq\intv{a_0}{b_0}$. Sia $c$ il punto medio dell'intervallo:
  $$c\walrus\frac{a_0+b_0}{2}$$
  In almeno uno fra gli intervalli $\intv{a_0}{c}$ e $\intv{c}{b_0}$ vi sono infiniti punti di $E$, dato che $E\subseteq\intv{a_0}{b_0}=\intv{a_0}{c}\cup\intv{c}{b_0}$ e $\#\(E\)=+\infty$. Considerato tale intervallo, e indicandolo con $\intv{a_1}{b_1}$, si ha:
  $$\#\(E\cap\intv{a_1}{b_1}\)=+\infty$$
  Iterando la dicotomia, si ottiene una successione decrescente di intervalli:
  $$\intv{a_{n+1}}{b_{n+1}}\subset\intv{a_n}{b_n}\subset\cdots\subset\intv{a_0}{b_0}$$
  da cui:
  $$\#\(E\cap\intv{a_n}{b_n}\)=+\infty$$
  Si ha che $\left\{ a_n \right\}$ è crescente e superiormente limitata da $b_0$ e $\left\{ b_n \right\}$ è decrescente e inferiormente limitata da $a_0$.
  Per il teorema di convergenza delle successioni monotone limitate:
  $$\exists\ \lim_{n\to+\infty}a_n\quad\wedge\quad\exists\ \lim_{n\to+\infty}b_n$$
  Inoltre:
  $$\lim_{n\to+\infty}\(b_n-a_n\)=\lim_{n\to+\infty}\frac{b_0-a_0}{2^n}=0$$
  da cui:
  $$\reals\ni\bar{x}\walrus \lim_{n\to+\infty}a_n=\lim_{n\to+\infty}b_n$$
  Sia $x_n\in E\cap\intv{a_n}{b_n}:x_n\neq\bar{x}$. Si ha:
  $$a_n\le x_n\le b_n$$
  $$\bar{x}=\lim_{n\to+\infty}a_n\le \lim_{n\to+\infty}x_n\le \lim_{n\to+\infty}b_n=\bar{x}$$
  $$\bar{x}=\lim_{n\to+\infty}x_n$$
  Pertanto, $\bar{x}$ è di accumulazione per $E$: $$\bar{x}\in E'$$
\end{proof}

\section{Numeri complessi}

% TODO: aggiungere grafici dove vi è un TODO vuoto

\subsection{Costruzione}

\begin{definition}[Campo complesso]
  $$\complex\walrus\reals\times\reals=\reals^2$$
  $$\complex\ni z\walrus\(a,b\)\in\reals^2$$
\end{definition}

% TODO: inserire piano di Gauss

\begin{definition}[Somma in $\complex$]
  $$z,w\in\complex$$
  $$z\walrus\(a,b\)$$
  $$w\walrus\(c,d\)$$
  $$z+w\walrus\(a+c,b+d\)$$
\end{definition}

% TODO

\begin{definition}[Prodotto in $\complex$]
  $$z,w\in\complex$$
  $$z\walrus\(a,b\)$$
  $$w\walrus\(c,d\)$$
  $$z\cdot w\walrus\(ac-bd,ad+bc\)$$
\end{definition}

% TODO

\subsubsection*{Proprietà}

\begin{itemize}
  \item $z+w=w+z$
  \item $zw=wz$
  \item $z+\(w+v\)=\(z+w\)+v$
  \item $z\(wv\)=\(zw\)v$
  \item $z\(w+t\)=zw+zt$
\end{itemize}

L'elemento neutro della somma è $\(0,0\)$, mentre quello del prodotto è $\(1,0\)$. % TODO: dimostrare

% TODO: da rivedere la forma

\paragraph*{Immersione di $\reals$ in $\complex$}
$$f:\reals\to\complex\quad f\(x\)\walrus\(x,0\)$$
$f$ è iniettiva:
$$f(x)=f(y)\iff \(x,0\)=\(y,0\)\iff x=y$$

$$f(x)+f(y)=\(x,0\)+\(y,0\)=\(x+y,0\)=f(x+y)$$
$$f(x)f(y)=\(x,0\)\(y,0\)=\(xy-0,0+0\)=f(xy)$$

\subsection{Forma algebrica}

$$z=\(a,b\)\in\complex$$
$$\complex \ni i\walrus\(0,1\)$$
\begin{align*}
  z & =\(a,b\)             \\
    & =\(a,0\)+\(0,b\)     \\
    & =f(a)+\(0,1\)\(b,0\) \\
    & =f(a)+\(0,1\)f(b)    \\
    & =a+\(0,1\)b          \\
    & =a+ib                
\end{align*}

\begin{observation}
  $$i^2=\(0,1\)\(0,1\)=\(0-1,0+0\)=f(-1)=-1\iff i^2+1=0$$
\end{observation}

\subsection{Operazioni}

$$\(a+ib\)+\(c+id\)=\(a+c\)+i\(b+d\)$$
$$\(a+ib\)+\(c+id\)=\(ac-bd\)+i\(ad+bc\)$$
$$-\(a+ib\)=-a-ib$$ % TODO
$$z^{-1}=\frac{1}{z}=\frac{1}{a+ib}=\frac{a}{a^2+b^2}-i\frac{b}{a^2+b^2}$$

\begin{definition}[Parte reale e immaginaria]
  Sia $\complex\ni z\walrus a+ib$. $a$ è detta \textbf{parte reale} di $z$, mentre $b$ è detta \textbf{parte immaginaria} di $z$. Si indicano rispettivamente con $\re\(z\)$ e $\im\(z\)$.
\end{definition}

\begin{definition}[Coniugato]
  Il \textbf{coniugato} di $\complex\ni z\walrus a+ib$ è:
  $$\bar{z}\walrus a-ib$$
\end{definition}

% TODO

\begin{definition}[Modulo in $\complex$]
  Sia $\complex\ni z\walrus a+ib$.
  $$\abs{z}=\sqrt{a^2+b^2}$$
\end{definition}

% TODO

\begin{observation}
  Mentre $\abs{z}$ è la distanza di $z$ dall'origine, $\abs{z-w}$ è la distanza fra $z$ e $w$.
  % TODO
\end{observation}

\begin{lemma}
  $$\abs{z}=\abs{-z}$$
  % TODO
\end{lemma}

\begin{lemma}
  $$\abs{z}=\abs{\bar{z}}$$
\end{lemma}

\begin{lemma}
  $$\bar{z+w}=\bar{z}+\bar{w}$$ % FIXME: la barra sopra non è lunga abbastanza
\end{lemma}

\begin{lemma}
  $$\bar{zw}=\bar{z}\bar{w}$$ % FIXME: la barra sopra non è lunga abbastanza
\end{lemma}

\begin{lemma}
  $$\abs{z}=\sqrt{z\bar{z}}\iff \abs{z}^2=z\bar{z}$$
\end{lemma}
\begin{proof}
  $$z\bar{z}=\(a+ib\)\(a-ib\)=a^2-iab+iab-i^2b=a^2+b^2=\abs{z}^2$$
\end{proof}

\begin{lemma}
  $$\frac{1}{z}=\frac{\bar{z}}{\abs{z}^2}$$
\end{lemma}
\begin{proof}
  $$\frac{1}{z}=\frac{1}{z}\cdot\frac{\bar{z}}{\bar{z}}=\frac{\bar{z}}{z\bar{z}}=\frac{\bar{z}}{\abs{z}^2}$$
\end{proof}

\begin{lemma}[Disuguaglianza triangolare]
  $$\abs{z+w}\le \abs{z}+\abs{w}$$
  $$\abs{z_1-z_2}\le \abs{z_1-z_3}+\abs{z_3-z_2}$$ 
  % TODO
\end{lemma}

\subsection{Forma trigonometrica}

Sia $\complex\ni z\neq0$. Allora:
$$z=z\frac{\abs{z}}{\abs{z}}=\abs{z}\frac{z}{\abs{z}}=\r w$$
$$\r=\abs{z}\quad w=\frac{z}{\abs{z}}$$
$$\abs{w}=\abs{\frac{z}{\abs{z}}}=\frac{\abs{z}}{\abs{\abs{z}}}=\frac{\abs{z}}{\abs{z}}=1$$
$$w=\(\cos\t,\sin\t\)$$
$$z=\r\(\cos\t+i\sin\t\)$$
$$\re\(z\)=\r\cos\t$$
$$\im\(z\)=\r\sin\t$$
$$\t\walrus \arg\(z\)$$

\begin{theorem}[I formula di De Moivre]
  $$z_1=\r_1\(\cos\t_1+i\sin\t_1\)$$
  $$z_2=\r_2\(\cos\t_2+i\sin\t_2\)$$
  $$z_1z_2=\r_1\r_2\(\cos\(\t_1+\t_2\)+i\sin\(\t_1+\t_2\)\)$$
\end{theorem}
\begin{proof}
  \begin{align*}
    z_1z_2 & =\r_1\(\cos\t_1+i\sin\t_1\)\r_2\(\cos\t_2+i\sin\t_2\)                              \\
           & =\r_1\r_2\(\cos\t_1\cos\t_2+i\cos\t_1\sin\t_2+i\sin\t_1\cos\t_2-\sin\t_1\sin\t_2\) \\
           & =\r_1\r_2\(\cos\(\t_1+\t_2\)+i\sin\(\t_1+\t_2\)\)                                  
  \end{align*}
\end{proof}

% TODO

\begin{corollary}
  $$z=\r\(\cos\t+i\sin\t\)$$
  $$z^n=\r^n\(\cos n\t+i\sin n\t\)$$
\end{corollary}

\begin{theorem}[II formula di De Moivre]
  Sia $\complex\setminus\left\{ 0 \right\}\ni w\walrus\r\(\cos\t+i\sin\t\)$ e $n\in\mathbb{N}\setminus\left\{ 0 \right\}$. Allora l'equazione $z^n=w$ ha esattamente $n$ soluzioni, ossia:
  $$z_k^n=w\quad \forall k\in\rintv{0}{n}$$
  In particolare:
  $$z_k=\r^{\nicefrac{1}{n}}\(\cos\t_k+i\sin\t_k\)\quad \t_k=\frac{\t}{n}+\frac{2\pi}{n}k$$
\end{theorem}
\begin{proof}
  \begin{align*}
    z_k^n & =\(\r^{\nicefrac{1}{n}}\(\cos\t_k+i\sin\t_k\)\)^n \\
          & =\r^{\nicefrac{n}{n}}\(\cos n\t_k+i\sin n\t_k\)   \\
          & =\r\(\cos\(\t+2k\pi\)+i\sin\(\t+2k\pi\)\)         \\
          & =\r\(\cos\t+i\sin\t\)                             \\
          & =w                                                
  \end{align*}
\end{proof}

% TODO

\begin{corollary}
  Le $n$ radici $n$--esime di $w\in\complex\setminus\left\{ 0 \right\}$ sono i vertici di un poligono regolare di $n$ lati, inscritto nella circonferenza centrata in $0\in\complex$ e di raggio $\r^{\nicefrac{1}{n}}$.
\end{corollary}

\begin{definition}[Polinomio]
  Un \textbf{polinomio} è una funzione:
  $$P_n:\complex\to\complex\quad P_n\(z\)\walrus\sum_{i=0}^nc_iz^i$$
  $n$ è detto \textbf{grado} del polinomio e $\left\{ c_i:0\le i\le n \right\}\subset\complex$ è l'insieme dei \textbf{coefficienti} del polinomio.
\end{definition}

\begin{definition}[Equazione algebrica]
  Sia $P_n$ un polinomio di grado $n$. L'equazione $P(z)=0$ prende il nome di \textbf{equazione algebrica}. 
\end{definition}

\begin{theorem}[Teorema fondamentale dell'algebra]
  Un'equazione algebrica di grado $n$ ha $N$ soluzioni $w\walrus\left\{ w_i:1\le i\le N \right\}$, cui corrispondono gli indici $m\walrus\left\{ m_i:1\le i\le N \right\}$ tali per cui $\sum m_i=n$. Tali indici si dicono \textbf{molteplicità} di $w$.
  $$P_n(z)=c_n\(z-w_1\)^{m_1}\cdots\(z-w_N\)^{m_N}$$
\end{theorem}

\subsection{Trasformazioni del piano di Gauss}

\subsubsection*{Traslazione}

Sia $z_0\in\complex$.
$$T:\complex\to\complex\quad T\(z\)\walrus z+z_0$$
% TODO

\subsubsection*{Rotazione}

Sia $w\in\complex:\abs{w}=1$.
$$R:\complex\to\complex\quad R\(z\)\walrus wz$$
% TODO

\subsubsection*{Dilatazione}

Sia $\reals\ni\r>0$.
$$D:\complex\to\complex\quad D\(z\)\walrus \r z$$
% TODO

\subsubsection*{Inversione}

$$I:\complex\to\complex\quad I\(z\)\walrus -z$$
% TODO

\subsubsection*{Simmetria con l'asse reale}

$$S:\complex\to\complex\quad S\(z\)\walrus \bar{z}$$
% TODO

\subsubsection*{Funzione di Zukeowski}

$$f:\complex\setminus\left\{ 0 \right\}\to\complex\quad f\(z\)\walrus \frac{1}{2}\(z+\frac{1}{z}\)$$

\begin{example}
  $$\complex\ni z:\abs{z}=1$$
  $$f\(z\)=\frac{1}{2}\(z+\frac{1}{z}\)=\frac{1}{2}\(z+\frac{\bar{z}}{\abs{z}^2}\)=\frac{1}{2}\(z+\bar{z}\)=\frac{1}{2}\re\(z\)$$
\end{example}

% TODO
% il disegno è più complicato: si deve disegnare la trasformazione come due grafici, il primo mostra una circonferenza e delle rette, parallele all'asse delle ascisse, che vengono curvate dalla circonferenza (similmente a quanto accade con la gravità), il secondo mostra gli effetti della trasformazione, ossia un segmento dove prima c'era la circonferenza e rette parallele all'asse reale dove prima c'erano curve. il tutto è completato da frecce circolari che indicano il passaggio diretto e inverso della trasformazione (quindi, da sinistra a destra una freccia f, da destra a sinistra una freccia f^{-1})

\begin{observation}
  La funzione non è un'isometria, perciò non conserva le distanze, ma bensì gli angoli.
\end{observation}

\subsection{Forma esponenziale}

$$z=\r\(\cos\t+i\sin\t\)=\r e^{i\t}$$
$$zw=\r_1\r_2\(\cos\(\t_1+\t_2\)+i\sin\(\t_1+\t_2\)\)=\r_1\r_2 e^{i\(\t_1+\t_2\)}$$
\section{Spazi vettoriali}

\subsection{Vettori liberi}

I vettori fisici sono quantità caratterizzate da un verso, una direzione e un modulo.
Un modo di rappresentare i vettori fisici è dato dall'uso dei segmenti orientati.

\begin{center}
  \begin{tikzpicture}[scale=1.5]
    \draw[-stealth] (0,0) node[label={$P$}] {} -- (3,1) node[label={$Q$}] {};
  \end{tikzpicture}
\end{center}
In un segmento orientato $PQ$:
\begin{itemize}
  \item la lunghezza (ossia la distanza fra $P$ e $Q$) è il modulo del vettore;
  \item la retta passante per i due estremi è la direzione;
  \item l'ordine dei due estremi indica il verso.
\end{itemize}
Il segmento $PP$ rappresenta il vettore nullo.

\begin{definition}[Equipollenza]
  Due segmenti si dicono \textbf{equipollenti} se hanno in comune lunghezza, direzione e verso.
\end{definition}

\begin{definition}[Vettore libero]
  Un \textbf{vettore libero} è la classe di equipollenza di un segmento orientato, ossia l'insieme dei segmenti orientati che sono equipollenti fra loro. Il vettore libero individuato dal segmento orientato $PQ$ si indica con $\vec{PQ}$. In particolare, il vettore nullo è $\vec{0}$.
\end{definition}

\begin{definition}[Legge di Galileo]
  La somma di due vettori liberi è congruente al segmento orientato che congiunge l'origine del primo vettore con la destinazione del secondo vettore:
  $$\vec{PQ}+\vec{QR}\walrus\vec{PR}$$
  \begin{center}
    \begin{tikzpicture}[scale=1.2]
      \draw[-stealth] (0,0) node[label={$P$}] {} -- (2,2) node[label={$Q$}] {};
      \draw[-stealth] (2,2) node[label={$Q$}] {} -- (5,0) node[label={$R$}] {};
      \draw[-stealth] (0,0) -- (5,0);
    \end{tikzpicture}
  \end{center}
\end{definition}

\begin{definition}[Prodotto scalare]
  Il prodotto di un vettore libero $\vec{v}$ per uno scalare $t\in\reals$ è un vettore libero che ha:
  \begin{itemize}
    \item modulo $t\dabs{\vec{v}}$;
    \item stessa direzione;
    \item stesso verso se $t>0$, verso opposto se $t<0$.
  \end{itemize}
  \begin{center}
    \begin{tikzpicture}[scale=1.2]
      \draw[-stealth] (0,0) -- (2,0) node[midway,above] {$\vec{v}$};
      \draw[-stealth] (0,-1) -- (4,-1) node[midway,above] {$2\vec{v}$};
      \draw[-stealth] (0,-2) -- (-1,-2) node[midway,above] {$-\frac{1}{2}\vec{v}$};
    \end{tikzpicture}
  \end{center}
\end{definition}

\begin{theorem}[Regola del parallelogramma]
  La somma di due vettori liberi $\vec{v}$ e $\vec{w}$ è la diagonale del parallelogramma che ha per lati i due vettori.
  \begin{center}
    \begin{tikzpicture}[scale=1.2]
      \draw[-stealth] (0,0) node[label={$P$}] {} -- (1,2) node[label={$Q$}] {};
      \draw[-stealth] (0,0) -- (4,0) node[label={$S$}] {};
      \draw[-stealth] (0,0) -- (5,2) node[label={$R$}] {};
      \draw[dashed] (4,0) -- (5,2);
      \draw[dashed] (1,2) -- (5,2);
    \end{tikzpicture}
  \end{center}
  $$\vec{PQ}+\vec{PS}=\vec{PR}$$
\end{theorem}
\begin{proof}
  Per la legge di Galileo, $\vec{PQ}+\vec{QR}=\vec{PR}$ e $\vec{PS}+\vec{SR}=\vec{PR}$. Poiché i lati opposti di un parallelogramma sono congruenti ($\vec{PQ}\cong\vec{SR}$ e $\vec{PS}\cong\vec{QR}$), si ha:
  $$\vec{PQ}+\vec{PS}=\vec{PR}$$
\end{proof}

\subsubsection*{Proprietà}

Le proprietà di cui godono i vettori liberi sono analoghe a quelle dei vettori, per cui:
\begin{itemize}
  \item $(\vec{v}+\vec{w})+\vec{u}=\vec{v}+(\vec{w}+\vec{u})$
  \item $\vec{v}+\vec{w}=\vec{w}+\vec{v}$
  \item $\vec{v}+\vec{0}=\vec{v}$
  \item $\vec{v}+(-1)\vec{v}=\vec{0}$
  \item $t(\vec{v}+\vec{w})=t\vec{v}+t\vec{w}$
  \item $(t+s)\vec{v}=t\vec{v}+s\vec{v}$
  \item $(ts)\vec{v}=t(s\vec{v})$
  \item $1\vec{v}=\vec{v}$
\end{itemize}

\begin{lemma}
  Sia $\vec{PQ}$ un qualsiasi vettore libero.
  $$\vec{QP}=-\vec{PQ}$$
\end{lemma}
\begin{proof}
  $$\vec{PQ}+\vec{QP}=\vec{PP}=\vec{0}\impl\vec{QP}=-\vec{PQ}$$
\end{proof}

\subsection{Sistema di assi cartesiano}

\subsubsection*{Nel piano}
Dare un sistema di assi cartesiani nel piano equivale a fissare un punto $O$ e i vettori $\vec{OU_1}$ e $\vec{OU_2}$, ortogonali fra loro, detti \textbf{versori degli assi cartesiani} e indicati rispettivamente con $\vec{i}$ e $\vec{j}$.

Il sistema di assi cartesiani individuato da $O$ e $\vec{i}$ e $\vec{j}$ si indica con $$S\walrus\(O,\left\{ \vec{i},\vec{j} \right\}\)$$

Se $\vec{v}$ è un vettore, allora si può scrivere $\vec{v}=\vec{OP}$. La coppia $\(x,y\)$ delle \textbf{coordinate} di $P$ è detta 2--vettore delle coordinate di $\vec{v}$ e si nota che $\vec{v}=x\vec{i}+y\vec{j}$.

Questo ha delle conseguenze non trascurabili.
Se $\(x_1,y_1\)$ è il 2--vettore delle coordinate di $\vec{v_1}$ e $\(x_2,y_2\)$ è il 2--vettore delle coordinate di $\vec{v_2}$, allora il 2--vettore delle coordinate di $\vec{v_1}+\vec{v_2}$ è $\(x_1+x_2,y_1+y_2\)$.
Se $\(x,y\)$ è il 2--vettore delle coordinate di $\vec{v}$, allora $\(tx,ty\)$ è il 2--vettore delle coordinate di $t\vec{v}$.

Siano $a$ il 2--vettore delle coordinate del vettore libero $\vec{v}$ e $b$ il 2--vettore delle coordinate del vettore libero $\vec{w}$. Allora, $\vec{v}+\vec{w}=a+b$ e $t\vec{v}=ta$.

\subsubsection*{Nello spazio}
Quanto visto per i vettori nel piano vale anche per i vettori nello spazio:
\begin{itemize}
  \item si hanno tre versori $\vec{i}$, $\vec{j}$ e $\vec{k}$;
  \item il 3--vettore delle coordinate di $\vec{v}=\vec{OP}$ è la tripla $\(x,y,z\)$ delle coordinate di $P$;
  \item si ha che $\vec{v}=x\vec{i}+y\vec{j}+z\vec{k}$.
\end{itemize}

\subsection{Spazi vettoriali}

\begin{definition}[Spazio vettoriale]
  Uno \textbf{spazio vettoriale} è un insieme $V$ su cui è definita:
  \begin{itemize}
    \item un'operazione di \textbf{somma} che associa a due elementi $v,w$ un elemento $v+w\in V$;
    \item un'operazione di \textbf{prodotto scalare} che associa ad un elemento $v$ un elemento $tv\in V$.
  \end{itemize}
\end{definition}
Inoltre, devono necessariamente valere le seguenti proprietà:
\begin{itemize}
  \item $(v+w)+u=v+(w+u)$
  \item $v+w=w+v$
  \item $v+0=v$
  \item $v+(-1)v=0$
  \item $t(v+w)=tv+tw$
  \item $(t+s)v=tv+sv$
  \item $(ts)v=t(sv)$
  \item $1v=v$
\end{itemize}

L'insieme di tutti gli $n$--vettori si indica con $\reals^n$.
L'insieme di tutte le matrici $n\times m$ si indica con $\reals^{n\times m}$.
L'insieme dei polinomi si indica con $\reals\left[ x \right]$.
L'insieme dei polinomi di grado minore o uguale a $n$ si indica con $\reals_n\left[ x \right]$.

\begin{definition}[Sottospazio]
  Un \textbf{sottospazio} di uno spazio vettoriale $V$ è un sottoinsieme non vuoto $W$ tale che $v,w\in W\impl v+w\in W\wedge v\in W,t\in \reals\impl tv\in W$.
\end{definition}

\begin{example}
  $R_n\left[ x \right]$ è un sottospazio di $R\left[ x \right]$.
\end{example}

\begin{example}
  Se $V$ è uno spazio vettoriale allora $V$ è un sottospazio di se stesso.
\end{example}

\begin{example}
  Se $AX=0$ è un sistema omogeneo di $n$ equazioni e $m$ incognite, allora l'insieme delle soluzioni $\sol \(A,0\)$ è un sottospazio di $\reals^m$, per la legge di sovrapposizione.
  \begin{proof}
    L'insieme delle soluzioni non è mai vuoto, perché $0$ è una soluzione.
    Per la legge di sovrapposizione:
    $$x_1,x_2 \text{ soluzioni}\impl x_1+x_2 \text{ soluzione}$$
    $$x\text{ soluzione}\impl tx \text{ soluzione}$$
  \end{proof}
\end{example}

\begin{theorem}
  $\vec{0}$ appartiene ad ogni sottospazio.  
\end{theorem}
\begin{proof}
  Se $W$ è un sottospazio, allora non è vuoto. Sia $w\in W$, allora $(-1)w\in W$ e quindi $w+(-1)w=\vec{0}\in W$.
\end{proof}
\begin{corollary}
  Se $\vec{0}$ non appartiene ad un insieme $W$, allora $W$ non può essere un sottospazio.
\end{corollary}

Se $W$ è un sottospazio di $V$, allora con la somma e il prodotto per uno scalare indotte da $V$, $W$ è uno spazio vettoriale. Ne segue che ogni proprietà che vale per gli spazi vettoriali, vale anche per tutti i sottospazi.

\begin{example}
  \begin{itemize}
    \item $\left\{ \(x,y\):2x-y=1 \right\}$ non è un sottospazio in quanto 0 non è una soluzione;
    \item $\left\{ \(x,y\):x^2+y^2=1 \right\}$ non è un sottospazio in quanto 0 non è una soluzione;
    \item $\left\{ \(x,y\):x^2-y^2=0 \right\}$ non è un sottospazio in quanto non è chiuso rispetto alla somma;
    \item $\left\{ \(x,y\):x^2+y^2=0 \right\}$ non è un sottospazio in quanto non è chiuso rispetto al prodotto scalare.
  \end{itemize}
\end{example}

\subsection{Spazi generati}

\begin{definition}[Combinazione lineare]
  In uno spazio vettoriale $V$, si dice che il vettore $v$ è una \textbf{combinazione lineare} dei vettori $v_1,v_2,...,v_k$ se
  $$v=a_1v_1+a_2v_2+\cdots+a_kv_k$$
  con $a_1,a_2,...,a_k\in\reals$ detti \textbf{coefficienti della combinazione lineare}.
\end{definition}

\begin{example}
  $\(1,3,-1,4\)$ è combinazione lineare di $\(1,1,0,1\)$, $\(1,0,1,1\)$, $\(0,0,0,1\)$, infatti
  $$\(1,3,-1,4\)=2\(1,1,0,1\)-\(1,0,1,1\)+3\(0,0,0,1\)$$
\end{example}

\begin{definition}[Vettori linearmente dipendenti]
  I vettori $v_1,v_2,...,v_n$ si dicono \textbf{linearmente dipendenti} se esiste una combinazione lineare di essi con coefficienti non tutti nulli.
\end{definition}
\begin{example}
  I vettori $\(1,0,1\), \(1,1,1\), \(-1,1,-1\)$ sono linearmente dipendenti:
  $$\(1,0,1\) -\frac{1}{2}\(1,1,1\) +\frac{1}{2}\(-1,1,-1\)=\(0,0,0\)$$
\end{example}
\begin{lemma}
  Due vettori sono linearmente dipendenti se e solo se uno è un multiplo dell'altro.
\end{lemma}
\begin{proof}
  Siano $v$ e $w$ due vettori. Perché essi siano linearmente dipendenti, si deve verificare:
  $$a,b\in\reals\setminus\left\{ 0 \right\}$$
  $$av+bw=\vec{0}$$
  $$av=-bw\impl v=-\frac{b}{a}w$$
\end{proof}

\begin{definition}[Vettori linearmente indipendenti]
  I vettori $v_1,v_2,...,v_n$ si dicono \textbf{linearmente indipendenti} se non sono linearmente dipendenti, vale a dire esiste una sola combinazione lineare ed è quella nulla.
  Per convenzione $\emptyset$ è linearmente indipendente.
\end{definition}
\begin{example}
  I vettori $\(1,1,1\), \(1,1,0\), \(1,0,0\)$ sono linearmente indipendenti:
  $$a\(1,1,1\) +b\(1,1,0\) +c\(1,0,0\)=\(0,0,0\)$$
  $$\(a,a,a\) +\(b,b,0\) +\(c,0,0\)=\(0,0,0\)$$
  $$
    \(a+b+c,a+b,a\)=\(0,0,0\)
    \iff
    \begin{cases}
      a+b+c=0 \\
      a+b=0   \\
      a=0     \\
    \end{cases}
    \iff
    \begin{cases}
      a=0 \\
      b=0 \\
      c=0 \\
    \end{cases}
  $$
\end{example}

\begin{definition}[Spazio generato]
  È detto \textbf{spazio generato} l'insieme di tutte le combinazioni lineari dei vettori $v_1,v_2,\dots,v_n$ è  e lo si indica con $L\(v_1,v_2,...,v_n\)$, oppure $\sp{v_1,v_2,...,v_n}$ oppure $\mathrm{span}\(v_1,v_2,...,v_n\)$. Per convenzione $\sp{\emptyset} = \left\{ 0 \right\}$.
\end{definition}
Lo spazio generato da $v_1,v_2,...,v_n$ è un sottospazio. Infatti:
\begin{itemize}
  \item non è vuoto: $\vec{0}\in\sp{v_1,v_2,...,v_n}$;
  \item è chiuso rispetto alla somma: $\(a_1v_1+a_2v_2+\cdots+a_nv_n\)+\(b_1v_1+b_2v_2+\cdots+b_nv_n\)=\(a_1+b_1\)v_1+\(a_2+b_2\)v_2+\cdots+\(a_n+b_n\)v_n\in \sp{v_1,v_2,...,v_n}$;
  \item è chiuso rispetto al prodotto per uno scalare: $t\(a_1v_1+a_2v_2+\cdots+a_nv_n\)=ta_1v_1+ta_2v_2+\cdots+ta_nv_n\in \sp{v_1,v_2,...,v_n}$.
\end{itemize}

Lo spazio generato da $v_1,v_2,...,v_n$ è il più piccolo sottospazio che contiene $v_1,v_2,...,v_n$, nel senso che $\sp{v_1,v_2,...,v_n}$ contiene $v_1,v_2,...,v_n$ e se $W$ è un sottospazio che contiene $v_1,v_2,...,v_n$, allora $\sp{v_1,v_2,...,v_n}$ è contenuto in $W$.

\begin{definition}[Insieme di generatori]
  Si dice che $\left\{ v_1,v_2,...,v_n \right\}$ è un \textbf{insieme di generatori} per lo spazio $V$ se $\sp{v_1,v_2,...,v_n}=V$: si dice che $V$ è generato da $v_1,v_2,...,v_n$ oppure che $v_1,v_2,...,v_n$ generano $V$.
\end{definition}

\begin{example}
  $\(1,0,0\),\(0,1,0\),\(0,0,1\)$ generano $\reals^3$, infatti se $\(x,y,z\)$ è un generico 3--vettore, allora:
  $$\(x,y,z\)=x\(1,0,0\)+y\(0,1,0\)+z\(0,0,1\)$$
\end{example}

\begin{definition}[Spazio finitamente generato]
  Uno spazio vettoriale $V$ si dice \textbf{finitamente generato} se esiste un insieme finito $\left\{ v_1,v_2,...,v_n \right\}$ di vettori che generano $V$.
\end{definition}

\begin{definition}[Base]
  Una \textbf{base} di uno spazio vettoriale $V$ finitamente generato è un insieme $\left\{ v_1,v_2,...,v_n \right\}$ di generatori di $V$ linearmente indipendente.
\end{definition}

\begin{theorem}[Teorema della base]
  Sia $V$ uno spazio finitamente generato. Allora:
  \begin{itemize}
    \item $V$ ha una base e tutte le basi di $V$ hanno lo stesso numero di elementi;
    \item ogni insieme linearmente indipendente in $V$ è contenuto in una base di $V$;
    \item ogni insieme di generatori di $V$ contiene una base.
  \end{itemize}
\end{theorem}

\begin{definition}[Dimensione]
  Il numero elementi di una qualsiasi base di uno spazio $V$ è detto \textbf{dimensione} di $V$ e lo si indica con $\dim V$.
\end{definition}

\begin{example}
  $$\dim \reals^3=3$$
  $$\dim \left\{ 0 \right\}=0$$
\end{example}
\paragraph*{Base canonica}
Sia $e_i\walrus\(0,0,\dots,0,0,1,0,0,\dots,0\)\in \reals^n$. È detta \textbf{base canonica} l'insieme $C^n\walrus\left\{ e_1,e_2,\dots,e_n \right\}$.

\begin{example}
  $$V=\sp{\(1,1,0\),\(1,2,-1\)}$$
  $\left\{ \(1,1,0\),\(1,2,-1\) \right\}$ è un insieme di generatori di $\left<\(1,1,0\),\(1,2,-1\)\right>$. 
  $$a\(1,1,0\)+b\(1,2,-1\)=\vec{0}\impl\(a+b,a+2b,-b\)=\vec{0}$$
  $$
    \begin{cases}
      a+b=0  \\
      a+2b=0 \\
      -b=0   \\
    \end{cases}
    \impl
    a=b=0
  $$
  Quindi l'insieme dei generatori $\left\{ \(1,1,0\),\(1,2,-1\) \right\}$ è linearmente indipendente e, pertanto, una base di $\sp{\(1,1,0\),\(1,2,-1\)}$.
\end{example}

\paragraph*{Assenza di una base per uno spazio infinito}
Si considera lo spazio $\reals\left[ n \right]$.
Per il principio di identità dei polinomi (che dice che due polinomi sono uguali se e solo se hanno coefficienti uguali) l'insieme $\left\{ x^0,x^1,x^2,\dots,x^n \right\}$ è linearmente indipendente $\forall n\in\mathbb{N}$.
$\vec{0}$ di $\reals\left[ x \right]$ è il polinomio nullo $P(x)=0$. Poiché $P(n)=\vec{0}\impl a_0=a_1=\cdots=a_n=0$ e poiché si può sempre aggiungere un elemento all'insieme del polinomio, l'insieme è sempre indipendente.
Per assurdo, se $\reals\left[ n \right]$ fosse di dimensione finita e $\dim \reals\left[ n \right]=k$, allora, per il teorema della base, $\left\{ x^0,x^1,x^2,\dots,x^n \right\}$ sarebbe contenuto in una base di e quindi $k\ge n\;\forall n$.


\begin{example}
  È $\left\{\(1,2\),\(2,1\)\right\}$ una base di $\reals^2$?
  
  \noindent Si verifica che l'insieme sia indipendente:
  $$a\(1,2\)+b\(2,1\)=0\impl\(a,2a\)+\(2b,b\)=0$$
  $$
    \begin{cases}
      a+2b=0 \\
      2a+b=0 \\
    \end{cases}
    \impl
    \begin{cases}
      2a=-4b  \\
      -4b+b=0 \\
    \end{cases}
    \impl
    \begin{cases}
      3b=0 \\
      2a=0
    \end{cases}
    \impl
    a=b=0
  $$
  Si verifica che l'insieme generi $\reals^2$:
  $$
    \begin{cases}
      a+2b=x \\
      2a+b=y \\
    \end{cases}
    \impl
    \begin{cases}
      2x-4b=2a  \\
      2x-4b+b=y \\
    \end{cases}
    \impl
    \begin{cases}
      x=a+2b  \\
      y=2x-3b \\
    \end{cases}
  $$
  Quindi l'insieme è un generatore ed è indipendente, il che vuol dire che è una base.
\end{example}

\begin{theorem}
  Sia $V$ uno spazio di dimensione finita. Se $W$ è un sottospazio di $V$ allora anche $W$ è di dimensione finita e $\dim W\le \dim V$.
\end{theorem}

\begin{theorem}
  Sia $V$ uno spazio di dimensione finita. Se $W$ è un sottospazio di $V$ e $\dim W=\dim V$ allora $V=W$.
\end{theorem}
\begin{proof}
  Sia $n=\dim V=\dim W$. Sia $w=\left\{ w_1,w_2,\dots,w_n \right\}$ una base di $W$. Siccome $w$ è linearmente indipendente allora, per il teorema della base, è contenuto in una base $b$ di $V$. Dato che $b$ ha $n$ elementi, si ha che $b=w$. In particolare, $W=\sp{w}=\sp{b}=V$.
\end{proof}

\begin{theorem}
  Sia $V$ uno spazio di dimensione finita. Se $v_1,v_2,\dots,v_k$ sono linearmente indipendenti in $V$, allora $k\le \dim V$.
\end{theorem}
\begin{proof}
  Siccome $v=\left\{ v_1,v_2,\dots,v_k \right\}$ è linearmente indipendente allora, per il teorema della base, è contenuto in una base $b$ di $V$. Si ha che $k\le \dim V$.
\end{proof}

\begin{theorem}
  Sia $V$ uno spazio di dimensione finita. Se $v_1,v_2,\dots,v_k$ generano $V$ allora $\dim V\le k$.
\end{theorem}
\begin{proof}
  Siccome $v=\left\{ v_1,v_2,\dots,v_k \right\}$ genera $V$ allora, per il teorema della base, contiene una base $b$ di $V$. Si ha che $k\ge \dim V$.
\end{proof}

\begin{theorem}
  Sia $V$ uno spazio di dimensione finita. Se $v_1,v_2,\dots,v_k$ sono linearmente indipendenti e $\dim V=k$ allora $v_1,v_2,\dots,v_k$ è una base di $V$.
\end{theorem}
\begin{proof}
  Siccome $v=\left\{ v_1,v_2,\dots,v_k \right\}$ è linearmente indipendente allora, per il teorema della base, è contenuto in una base $b$ di $V$. Dato che $b$ ha $k$ elementi, ne segue che $b=v$.
\end{proof}

\begin{theorem}
  Sia $V$ uno spazio di dimensione finita. Se $v_1,v_2,\dots,v_k$ generano $V$ e $\dim V=k$, allora $\left\{ v_1,v_2,\dots,v_k \right\}$ è una base di V.
\end{theorem}
\begin{proof}
  Siccome $v=\left\{ v_1,v_2,\dots,v_k \right\}$ è generatore di $V$ allora, per il teorema della base, è contiene una base $b$ di $V$. Dato che $b$ ha $k$ elementi, ne segue che $b=v$.
\end{proof}

\paragraph*{Notazione} Sia $A$ una matrice $n\times m$. Si indicano $A_i$ l'$i$--esima riga e $A^i$ l'$i$--esima colonna.

\begin{definition}[Spazio riga e colonna]
  Sia $A$ una matrice $n\times m$.
  Si definisce spazio riga $\sp{A_i}=\sp{A_1,A_2,\dots,A_n}$ di $A$ lo spazio generato dalle righe di $A$ e spazio colonna $\sp{A^i}=\sp{A^1,A^2,\dots,A^m}$ lo spazio generato dalle colonne di $A$.
\end{definition}

\begin{observation}
  Se $A$ è una matrice $n \times m$ e $X\walrus\(x_1,x_2,\dots,x_m\)'$, allora:
  $$AX=\sum_{i=1}^mx_iA^i$$
\end{observation}
\begin{proof}
  $$e_i\walrus\(0,0,\dots,0,0,1,0,0,\dots,0\)'\in \reals^n$$
  $$Ae_i=A^i$$
  $$AX=A\(\sum_{i=1}^mx_ie_i\)=\sum_{i=1}^mx_iAe_i=\sum_{i=1}^mx_iA^i$$
\end{proof}

\begin{observation}
  Se ad una matrice $A$ si applicano le operazioni elementari di riga, allora lo spazio riga non cambia (questo non accade però per lo spazio colonna, che invece cambia).
\end{observation}
\begin{proof}
  Le operazioni elementari di riga sono 3:
  \begin{enumerate}
    \item scambio di righe;
    \item prodotto di una riga per uno scalare;
    \item somma di una riga ad un'altra.
  \end{enumerate}
  Sia $A$ una matrice $n\times m$.
  Lo scambio di righe non produce alcun effetto sullo spazio riga:
  $$\sp{A_1,\dots,A_i,\dots,A_j,\dots,A_n}=\sp{A_1,\dots,A_j,\dots,A_i,\dots,A_n}$$
  Il prodotto di una riga per uno scalare non produce alcun effetto sullo spazio riga:
  $$\sp{A_1,\dots,A_i,\dots,A_n}=\sp{A_1,\dots,tA_i,\dots,A_n}$$
  La somma di una riga ad un'altra non produce alcun effetto sullo spazio riga:
  $$a_1A_1+\cdots+a_iA_i+\cdots+a_j\(tA_i+A_j\)+\cdots+a_nA_n=$$
  $$=a_1A_1+\cdots+\(a_i+a_jt\)A_i+\cdots+a_jA_j+\cdots+a_nA_n$$
  $$\sp{A_1,\dots,A_i,\dots,A_j,\dots,A_n}=\sp{A_1,\dots, A_i,\dots,tA_i+A_j,\dots,A_n}$$
\end{proof}

Quindi, ogni matrice $A$ può essere trasformata in una matrice a gradini $B$ attraverso una serie di operazioni di riga. Se una matrice $B$ è a gradini, allora le sue righe non nulle sono linearmente indipendenti e quindi sono una base del suo spazio riga.

Quindi, se $B$ è una riduzione a gradini di $A$, allora $\sp{A}=\sp{B}$. In particolare le righe non nulle di $B$ sono una base dello spazio riga di $A$ e la sua dimensione è il numero dei pivot di $B$.

\begin{example}
  $$
    \begin{pmatrix}
      1 & 1  & 2 & 2 \\
      2 & 0  & 4 & 2 \\
      1 & -1 & 2 & 0 \\
      0 & 2  & 0 & 2 \\
    \end{pmatrix}
    \sim
    \begin{pmatrix}
      1 & 1  & 2 & 2  \\
      0 & -2 & 0 & -2 \\
      0 & -2 & 0 & -2 \\
      0 & 2  & 0 & 2  \\
    \end{pmatrix}
    \sim
    \begin{pmatrix}
      1 & 1 & 2 & 2 \\
      0 & 1 & 0 & 1 \\
      0 & 0 & 0 & 0 \\
      0 & 0 & 0 & 0 \\
    \end{pmatrix}
  $$
  $$
    \mathfrak{B} =\left\{ \(1,1,2,2\),\(0,1,0,1\) \right\}
  $$
  $$
    \dim\mathfrak{B}=2
  $$
\end{example}

\begin{example}
  $$
    \begin{pmatrix}
      0 & 0 & 2 & 2 & -1 & 1 \\
      1 & 1 & 2 & 1 & 1  & 1 \\
      1 & 1 & 4 & 3 & 0  & 2 \\
    \end{pmatrix}
    \sim
    \begin{pmatrix}
      1 & 1 & 2 & 1 & 1  & 1 \\
      1 & 1 & 4 & 3 & 0  & 2 \\
      0 & 0 & 2 & 2 & -1 & 1 \\
    \end{pmatrix}
    \sim
  $$
  $$
    \sim
    \begin{pmatrix}
      1 & 1 & 2 & 1 & 1  & 1 \\
      0 & 0 & 2 & 2 & -1 & 1 \\
      0 & 0 & 2 & 2 & -1 & 1 \\
    \end{pmatrix}
    \sim
    \begin{pmatrix}
      1 & 1 & 2 & 1 & 1  & 1 \\
      0 & 0 & 2 & 2 & -1 & 1 \\
      0 & 0 & 0 & 0 & 0  & 0 \\
    \end{pmatrix}
  $$
  Siccome la dimesione dello spazio generato è 2, le righe della matrice non sono linearmente indipendenti.
\end{example}
\begin{example}
  $$
    \begin{pmatrix}
      1  & 2 & 2 & 2 \\
      1  & 0 & 1 & 1 \\
      1  & 2 & 3 & 0 \\
      -1 & 1 & 1 & 1 \\
    \end{pmatrix}
    \sim
    \begin{pmatrix}
      1  & 0 & 1 & 1 \\
      1  & 2 & 2 & 2 \\
      1  & 2 & 3 & 0 \\
      -1 & 1 & 1 & 1 \\
    \end{pmatrix}
    \sim
    \begin{pmatrix}
      1 & 0 & 1 & 1  \\
      0 & 2 & 1 & 1  \\
      0 & 2 & 2 & -1 \\
      0 & 1 & 2 & 2  \\
    \end{pmatrix}
    \sim
    \begin{pmatrix}
      1 & 0 & 1 & 1  \\
      0 & 2 & 1 & 1  \\
      0 & 0 & 1 & -2 \\
      0 & 0 & 3 & 3  \\
    \end{pmatrix}
    \sim
    \begin{pmatrix}
      1 & 0 & 1 & 1  \\
      0 & 2 & 1 & 1  \\
      0 & 0 & 1 & -2 \\
      0 & 0 & 0 & 3  \\
    \end{pmatrix}
  $$
  Siccome la dimesione dello spazio generato è 4, le righe della matrice sono linearmente indipendenti.
\end{example}

\subsection{Operazioni}

\begin{definition}[Spazio somma]
  Se $U,W$ sono sottospazi di uno stesso spazio $V$, allora lo \textbf{spazio somma} è:
  $$U+W\walrus \left\{ u+w:u\in U,w\in W \right\}$$
\end{definition}

\begin{definition}[Spazio intersezione]
  Se $U,W$ sono sottospazi di uno stesso spazio $V$, allora lo \textbf{spazio intersezione} è:
  $$U\cap W\walrus \left\{ v:v\in U\wedge v\in W \right\}$$
\end{definition}

\begin{observation}
  Se $U,W$ sono sottospazi di uno stesso spazio $V$, allora $U+W$ e $U\cap W$ sono sottospazi di $V$.
\end{observation}
\begin{proof}
  \hfill\break
  \noindent Spazio somma $U+W$:
  \begin{itemize}
    \item $0\in U+W$
    \item $v,z\in U+W\impl v+z\in U+W \because v+z=(u_1+w_1)+(u_2+w_2)=(u_1+u_2)+(w_1+w_2)$
    \item $v\in U+W\impl kv\in U+W\because kv=k(u+w)=ku+kw$
  \end{itemize}
  
  \noindent Spazio intersezione $U\cap W$:
  \begin{itemize}
    \item $0\in U\cap W$
    \item $v,z\in U\cap W\impl v+z\in U\cap W \because v+z=(u_1+w_1)+(u_2+w_2)=(u_1+u_2)+(w_1+w_2)$
    \item $v\in U\cap W\impl kv\in U\cap W\because kv=k(u+w)=ku+kw$
  \end{itemize}
\end{proof}

\begin{theorem}[Formula di Grassmann]
  $$\dim\(U+W\)+\dim \(U\cap W\)=\dim U+\dim W$$
\end{theorem}

\begin{definition}[Somma diretta]
  Si dice che uno spazio $V$ è \textbf{somma diretta} di due sottospazi $U,W$ se $V=U+W$ e $U\cap W=\left\{ 0 \right\}$. Si indica come:
  $$V=U \oplus W$$
\end{definition}

\begin{definition}[Complemento]
  Se $U$ è un sottospazio di $V$, un \textbf{complemento} di $U$ è un sottospazio $W$ tale che $V=U\oplus W$.
\end{definition}

\begin{example}
  $$U\walrus \left\{ (x,y):x=y \right\}$$
  $$W\walrus \left\{ (x,y):x=-y \right\}$$
  $W$ è complemento di $U$ in $\reals^2$.
\end{example}

\paragraph*{Osservazione} 
$V=U\oplus W$ se e solo se ogni vettore $v\in V$ può essere scritto univocamente come $v=u+w, u\in U, w\in W$.
\begin{proof}
  Sia $V=U\oplus W$, allora $v\in V$ può essere scritto come:
  $$v\in V=u+w, u\in U, w\in W$$
  Se, per assurdo, si suppone che tale scrittura non sia univoca, allora:
  $$v=u'+w'\impl u'+w'=u+w\impl u'-u=w'-w$$
  $$u'-u\in U,\ w'-w\in W$$
  $$u'-u\in U\cap W\impl u'-u=0\impl u=u'$$
  $$w'-w\in U\cap W\impl w'-w=0\impl w=w'$$
  
  Se $v\in V$ si scrive in modo univoco come $u+w,\ u\in U,\ w\in W$, allora
  $$V=U+W$$
  Inoltre, se $v\in U\cap W$:
  $$v\in U\wedge v\in W\impl v=0\impl V=U\oplus W$$
\end{proof}

\begin{definition}[Proiezione]
  Sia $V=U\oplus W$, dato $v\in V$, possiamo scrivere in modo univoco $v=u+w,\ u\in U,\ w\in W$.
  Il vettore $u$ è detto \textbf{proiezione} di $v$ su $U$, relativa alla somma diretta $V=U\oplus W$.
\end{definition}

\subsection{Rango e nullità}

\begin{definition}[Spazio nullo]
  Lo \textbf{spazio nullo} di una matrice $A$ è $\sol \(A,0\)$ e si indica con $N\(A\)$.
\end{definition}

\begin{definition}[Immagine]
  L'\textbf{immagine} di una matrice $A$ è lo spazio colonna della matrice e lo si indica con $R\(A\)$.
\end{definition}

\begin{definition}[Nullità]
  La dimensione di $N(A)$ è detta \textbf{nullità} di A e la si indica con $null(A)$:
  $$\nul A=\dim N(A)=\dim \sol \(A,0\)$$
\end{definition}

\begin{definition}[Rango]
  La dimensione di $R(A)$ è detta \textbf{rango} di $A$ e la si indica con $\rk A$, quindi se $A$ è una matrice $n\times m$:
  $$\rk A=\dim R(A)=\dim\sp{A^1,A^2,\dots,A^m}$$
  Si nota che $\rk A\le \min \left\{ n,m \right\}$.
\end{definition}

\begin{theorem}[Teorema del rango]
  Il rango di una matrice $A$ è uguale al rango di $A'$:
  $$\rk A=\rk A'$$
\end{theorem}
\begin{proof}
  Sia $A$ una matrice $n\times m$ e sia $r\walrus rk(A)$. Sia $B$ la base di $\sp{A^1,A^2,\dots,A^m}$ e $C$ la matrice avente per colonne gli elementi di $B$. Allora si ha che ogni colonna di $A$ è una combinazione lineare delle colonne di $C$:
  $$A^j=\sum_{i=1}^n k_{ij}C^i$$ % FIXME: forse l'indice deve arrivare ad m
  
  Sia $K=\(k_{ij}\)$. Siccome $A^j=CK^j$, si ha che $A=CK$ e quindi $A'=K'C'$. In particolare, le colonne di $A'$ sono combinazioni lineari delle colonne di $K'$ e quindi $\sp{A'^i}\subseteq \sp{K'^i}$.
  Pertanto $\rk A'\le \rk K'$ e dato che $K$ è $r\times m$, $\rk K'\le r$ e quindi $\rk A'\le r = \rk A$.
  Analogamente, scambiando i ruoli, si ha che $\rk A \le \rk A'$. Si conclude che $\rk A = \rk A'$.
\end{proof}

\paragraph*{Conseguenze}
Sia $r\walrus \rk A$:
\begin{itemize}
  \item la dimensione dello spazio riga è $r$;
  \item la dimesione dello spazio colonna è $r$;
  \item il massimo numero di colonne linearmente indipendenti è $r$;
  \item il massimo numero di righe linearmente indipendenti è $r$.
\end{itemize}

\begin{example}
  $$
    A=
    \begin{pmatrix}
      1 & 1  & 2 & 2 \\
      2 & 0  & 4 & 2 \\
      1 & -1 & 2 & 0 \\
      0 & 2  & 0 & 2 \\
    \end{pmatrix}
    \sim
    \begin{pmatrix}
      1 & 1  & 2 & 2  \\
      0 & -2 & 0 & -2 \\
      0 & -2 & 0 & -2 \\
      0 & 2  & 0 & 2  \\
    \end{pmatrix}
    \sim
    \begin{pmatrix}
      1 & 1 & 2 & 2 \\
      0 & 1 & 0 & 1 \\
      0 & 0 & 0 & 0 \\
      0 & 0 & 0 & 0 \\
    \end{pmatrix}
  $$
  $$\rk A = 2$$
\end{example}

\paragraph*{Metodo dei perni}
Il metodo dei perni consente di calcolare una base di uno spazio, partendo dalla matrice ridotta dei vettori colonna che generano tale spazio. Esso consiste nel trasformare la matrice dei vettori colonna che generano uno spazio in una matrice a gradini. La base è formata dai vettori originali nelle colonne che contengono un pivot.

\begin{theorem}[Nullità più rango]
  Se $A$ è una matrice $n\times m$, allora:
  $$\nul A+\rk A=m$$
  Il teorema è un caso speciale del teorema della dimensione.
\end{theorem}

\begin{example}
  $$
    A=
    \begin{pmatrix}
      1 & 1  & 2 & 2 \\
      2 & 0  & 4 & 2 \\
      1 & -1 & 2 & 0 \\
      0 & 2  & 0 & 2 \\
    \end{pmatrix}
    \sim
    \begin{pmatrix}
      1 & 1  & 2 & 2  \\
      0 & -2 & 0 & -2 \\
      0 & -2 & 0 & -2 \\
      0 & 2  & 0 & 2  \\
    \end{pmatrix}
    \sim
    \begin{pmatrix}
      1 & 1 & 2 & 2 \\
      0 & 1 & 0 & 1 \\
      0 & 0 & 0 & 0 \\
      0 & 0 & 0 & 0 \\
    \end{pmatrix}
  $$
  $$\rk A = 2$$
  $$\nul A=m-\rk A=4-2=2$$
\end{example}
\begin{example}
  $$
    A=
    \begin{pmatrix}
      1 & 1  & 2 & 2 \\
      2 & 0  & 4 & 2 \\
      1 & -1 & 2 & 0 \\
      0 & 2  & 0 & 2 \\
    \end{pmatrix}
  $$
  $$
    \begin{pmatrix}
      A & 0 \\
    \end{pmatrix}
    =
    \begin{pmatrix}
      1 & 1  & 2 & 2 & 0 \\
      2 & 0  & 4 & 2 & 0 \\
      1 & -1 & 2 & 0 & 0 \\
      0 & 2  & 0 & 2 & 0 \\
    \end{pmatrix}
    \sim
    \begin{pmatrix}
      1 & 1 & 2 & 2 & 0 \\
      0 & 1 & 0 & 1 & 0 \\
      0 & 0 & 0 & 0 & 0 \\
      0 & 0 & 0 & 0 & 0 \\
    \end{pmatrix}
  $$
  $$\rk A=2\impl \nul A = 4-2=2$$
  $$
    \begin{cases}
      x_1+x_2+2x_3+2x_4=0 \\
      x_2+x_4=0           \\
    \end{cases}
  $$
  Dato che i pivot sono $x_1$ e $x_2$, allora essi sono anche le variabili vincolate. Viceversa $x_3$ e $x_4$ sono libere.
  $$
    \begin{cases}
      x_1 +x_2=-2x_3-2x_4 \\
      x_2=-x_4            \\
    \end{cases}
    \impl
    \begin{cases}
      x_1=-2x_3-x_4 \\
      x_2=-x_4      \\
    \end{cases}
  $$
  $$
    \begin{pmatrix}
      x_1 \\
      x_2 \\
      x_3 \\
      x_4 \\
    \end{pmatrix}
    =
    \begin{pmatrix}
      -2x_3-x_4 \\
      -x_4      \\
      x_3       \\
      x_4       \\
    \end{pmatrix}
    =
    x_3
    \begin{pmatrix}
      -2 \\
      0  \\
      1  \\
      0  \\
    \end{pmatrix}
    +x_4
    \begin{pmatrix}
      -1 \\
      -1 \\
      0  \\
      1  \\
    \end{pmatrix}
  $$
  $\left\{ \(-2,0,1,0\),\(-1,-1,0,1\) \right\}$ genera le soluzioni del sistema.
  Dato che $\nul A=2$, l'insieme è una base.
  $N(A)$ è, quindi, $\sp{\(-2,0,1,0\),\(-1,-1,0,1\)}$.
\end{example}

\end{document}