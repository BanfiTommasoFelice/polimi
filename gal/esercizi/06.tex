\documentclass[a4paper,12pt]{article}

\usepackage[italian]{babel}
\usepackage[a4paper, left=18mm, right=18mm, top=20mm, bottom=20mm]{geometry}
\usepackage{amssymb}
\usepackage{mathtools}
\usepackage{interval}
\usepackage{amsthm}
\usepackage{thmtools}
\usepackage{cancel}
\usepackage{hyperref}
\usepackage{tikz}
\usepackage{pgfplots}
\usepackage{nicefrac}
\usepackage{enumitem}
\usepackage{verbatim}
\usepackage{tabularray}
\usepackage{bold-extra}
\usepackage{tabularray}

\hypersetup{
  colorlinks=true,
  linkcolor=black,
    filecolor=magenta,      
    urlcolor=cyan,
    pdftitle={Esercizi geometria e algebra lineare},
    % bookmarks=true,
    bookmarksopen=true,
    pdfpagemode=UseOutlines,
    pdfauthor={Amato Michele Pasquale},
}

\title{\huge Geometria e algebra lineare\\\Large Esercitazione}
\author{Amato Michele Pasquale}
\date{\today}

\pgfplotsset{compat = newest}
\makeatletter
\renewcommand\l@subsection{\@dottedtocline{2}{1.5em}{3em}}
\makeatother
\setitemize{noitemsep,topsep=3pt,parsep=0pt,partopsep=0pt}
\setenumerate{noitemsep,topsep=3pt,parsep=0pt,partopsep=0pt}

\newcommand{\abs}[1]{\left\lvert #1 \right\rvert}
\newcommand{\ceil}[1]{\left\lceil #1 \right\rceil}
\newtheorem{theorem}{Teorema}
\newtheorem{definition}{Definizione}
\newtheorem{lemma}{Lemma}
\newtheorem{axiom}{Assioma}
\newtheorem{corollary}{Corollario}
\renewcommand\qedsymbol{$\blacksquare$}
\newcommand{\asin}{\arcsin}
\newcommand{\acos}{\arccos}
\newcommand{\atan}{\arctan}
\newcommand{\impl}{\Rightarrow}
\setitemize{noitemsep,topsep=3pt,parsep=0pt,partopsep=0pt}
\setenumerate{noitemsep,topsep=3pt,parsep=0pt,partopsep=0pt}
\newcommand{\triang}[1]{\overset{\triangle}{#1}}
\renewcommand{\a}{\alpha}
\renewcommand{\b}{\beta}
\renewcommand{\c}{\gamma}
\renewcommand{\(}{\left(}
\renewcommand{\)}{\right)}
\renewcommand{\mod}[1]{\ \( \mathrm{mod}\ #1 \)}
\DeclareMathOperator{\agg}{agg}
\DeclareMathOperator{\sol}{Sol}
\renewcommand{\emptyset}{\varnothing}
\newcommand{\walrus}{\coloneqq}
\newcommand{\reals}{\mathbb{R}}
\newcommand{\dabs}[1]{\left\lvert\left\lvert #1 \right\rvert\right\rvert}
\renewcommand{\sp}[1]{\left< #1 \right>}
\DeclareMathOperator{\nul}{null}
\DeclareMathOperator{\rk}{rk}
\DeclareMathOperator{\im}{im}
\DeclareMathOperator{\diag}{diag}
\renewcommand{\l}{\lambda}

\newcounter{nexercise}
\newcommand{\exercisecnt}{\stepcounter{nexercise}\thenexercise}

\newenvironment{exercise}
{\newpage\par\noindent{\bf\Large Esercizio \exercisecnt}\vspace{1em}\hfill\break}
{\hfill\vspace{1cm}}
\newenvironment{solution}
{\par\noindent{\bf\large Soluzione}\vspace{1em}\hfill\break}
{\hfill}


\begin{document}
\pagenumbering{gobble}
\maketitle
\newpage
\pagenumbering{arabic}

\begin{exercise}
  Data la matrice:
  $$
    A=
    \begin{pmatrix}
      1 & 3 & 4 \\
      1 & 2 & 1 \\
      1 & 1 & 2 \\
    \end{pmatrix}
  $$
  Verificare che $\bar{\l}=1$ è un suo autovalore e trovare un autovettore relativo a $\bar{\l}$.
\end{exercise}
\begin{solution}
  \begin{align*}
    p_A\left( t \right) & = 
    \begin{vmatrix}
      1-t & 3   & 4   \\
      1   & 2-t & 1   \\
      1   & 1   & 2-t \\
    \end{vmatrix}
    \\
                        & = 
    \left( 1-t \right)
    \begin{vmatrix}
      2-t & 1   \\
      1   & 2-t \\
    \end{vmatrix}
    -3
    \begin{vmatrix}
      1 & 1   \\
      1 & 2-t \\
    \end{vmatrix}
    +4
    \begin{vmatrix}
      1 & 2-t \\
      1 & 1   \\
    \end{vmatrix}
    \\
                        & = 
    \left( 1-t \right)\left( 4+t^2-4t-1 \right)-3\left( 2-t-1 \right)+4\left( 1-2+t \right)
    \\
                        & = 
    \left( 1-t \right)\left( t^2-4t+3 \right)-3\left( 1-t \right)+4\left( t-1 \right)
    \\
                        & = 
    \left( 1-t \right)\left( t^2-4t+3-3-4 \right)
    \\
                        & = 
    \left( 1-t \right)\left( t^2-4t-4 \right)
  \end{align*}
  $$p_A\left( t \right)=0\iff t\in\left\{ 1,2\pm\sqrt{2} \right\}$$
  $\bar{\l}=1$ è un autovalore di $A$.
  $$
    A-\bar{\l}I=
    \begin{pmatrix}
      0 & 3 & 4 \\
      1 & 1 & 1 \\
      1 & 1 & 1 \\
    \end{pmatrix}
    \sim
    \begin{pmatrix}
      1 & 1 & 1 \\
      0 & 3 & 4 \\
      0 & 0 & 0 \\
    \end{pmatrix}
    \sim
    \begin{pmatrix}
      1 & 1 & 1               \\
      0 & 1 & \nicefrac{4}{3} \\
      0 & 0 & 0               \\
    \end{pmatrix}
    \sim
    \begin{pmatrix}
      1 & 0 & -\nicefrac{1}{3} \\
      0 & 1 & \nicefrac{4}{3}  \\
      0 & 0 & 0                \\
    \end{pmatrix}
  $$
  $$
    N\left( A-\bar{\l}I \right)=\sp{
      \begin{pmatrix}
        \nicefrac{1}{3}  \\
        -\nicefrac{4}{3} \\
        1                \\
      \end{pmatrix}
    }=\sp{
      \begin{pmatrix}
        1  \\
        -4 \\
        3  \\
      \end{pmatrix}
    }
  $$
  $\left( -1,4,3 \right)$ è un autovettore di $A$ relativo all'autovalore $\bar{\l}$.
\end{solution}

\begin{exercise}
  Data la matrice:
  $$
    A=
    \begin{pmatrix}
      1 & 3 & 1  \\
      3 & 1 & -1 \\
      0 & 0 & k  \\
    \end{pmatrix}
  $$
  Stabilire per quali valori $k$ la matrice $A$ è diagonalizzabile.
\end{exercise}
\begin{solution}
  $$
    p_A\left( t \right)=
    \begin{vmatrix}
      1-t & 3   & 1   \\
      3   & 1-t & -1  \\
      0   & 0   & k-t \\
    \end{vmatrix}
    =
    \left( k-t \right)
    \begin{vmatrix}
      1-t & 3   \\
      3   & 1-t \\
    \end{vmatrix}
    =
    \left( k-t \right)\left( t^2-2t-8 \right)
  $$
  $$p_A\left( t \right)=0\iff t\in \left\{ k,-2,4 \right\}$$
  Per il primo criterio di diagonalizzabilità, $A$ è diagonalizzabile se $k\notin\left\{ -2,4 \right\}$.
\end{solution}

\begin{exercise}
  Data la matrice:
  $$
    A=
    \begin{pmatrix}
      1 & k^2 \\
      3 & h   
    \end{pmatrix}
  $$
  Stabilire per quali valori $k$ e $h$ la matrice $A$ è diagonalizzabile.
\end{exercise}
\begin{solution}
  $$
    p_A\left( t \right)=
    \begin{vmatrix}
      1-t & k^2 \\
      3   & h-t \\
    \end{vmatrix}
    =
    \left( 1-t \right)\left( h-t \right)-3k^2
    =
    t^2+\left( -1-h \right)t+\left( h-3k^2 \right)
  $$
  $$\Delta=\left( -1-h \right)^2-4\left( h-3k^2 \right)=h^2-2h+12k^2+1=\left( h-1 \right)^2+12k^2$$
  $$\Delta\ge0\ \forall\left( h,k \right)\in\reals^2$$
  $$\Delta=0\iff h-1=0\wedge k=0\iff h=1\wedge k=0$$
  $A$ è diagonalizzabile per $h\neq 1\vee k\neq0$ per il primo criterio di diagonalizzabilità, poiché si hanno due radici del polinomio caratteristico.
  $$h=1\wedge k=0\iff p_A\left( t \right)=t^2-2t+1=\left( t-1 \right)^2$$
  $$p_A\left( t \right)=0\iff t=1$$
  $$
    A-I=
    \begin{pmatrix}
      0 & 0 \\
      3 & 0 \\  
    \end{pmatrix}
  $$ 
  $$m_g\left( 1 \right)=\nul\left( A-I \right)=1$$
  $$m_a\left( 1 \right)=2$$
  Pertanto, se $h=1\wedge k=0$, la matrice non è diagonalizzabile.
\end{solution}

\begin{exercise}
  Data la matrice:
  $$
    A=
    \begin{pmatrix}
      1 & 2 & 0 \\
      6 & 0 & 0 \\
      3 & 2 & 1 \\
    \end{pmatrix}
  $$
  Stabilire se è diagonalizzabile e, in caso affermativo, trovare la relativa matrice diagonale e modale.
\end{exercise}
\begin{solution}
  $$
    p_A\left( t \right)=
    \begin{vmatrix}
      1-t & 2  & 0   \\
      6   & -t & 0   \\
      3   & 2  & 1-t \\
    \end{vmatrix}
    =
    \left( 1-t \right)
    \begin{vmatrix}
      1-t & 2  \\
      6   & -t \\
    \end{vmatrix}
    =
    \left( 1-t \right)\left( t^2-t-12 \right)
  $$
  $$p_A\left( t \right)=0\iff t\in\left\{ 1,-3,4 \right\}$$
  $A$ è diagonalizzabile per il primo criterio di diagonalizzabilità.
  $$
    A-1\cdot I=
    \begin{pmatrix}
      0 & 2  & 0 \\
      6 & -1 & 0 \\
      3 & 2  & 0 \\
    \end{pmatrix}
    \sim
    \begin{pmatrix}
      0 & 1 & 0 \\
      6 & 0 & 0 \\
      3 & 0 & 0 \\
    \end{pmatrix}
    \sim
    \begin{pmatrix}
      1 & 0 & 0 \\
      0 & 1 & 0 \\
      0 & 0 & 0 \\
    \end{pmatrix}
  $$
  $$
    N\left( A-1\cdot I \right)=\sp{
      \begin{pmatrix}
        0 \\
        0 \\
        1 \\
      \end{pmatrix}
    }
  $$
  $$
    A+3 I=
    \begin{pmatrix}
      4 & 2 & 0 \\
      6 & 3 & 0 \\
      3 & 2 & 4 \\
    \end{pmatrix}
    \sim
    \begin{pmatrix}
      2 & 1 & 0 \\
      3 & 2 & 4 \\
      0 & 0 & 0 \\
    \end{pmatrix}
    \sim
    \begin{pmatrix}
      2 & 1 & 0 \\
      1 & 1 & 4 \\
      0 & 0 & 0 \\
    \end{pmatrix}
    \sim
  $$
  $$
    \sim
    \begin{pmatrix}
      1 & 1 & 4 \\
      2 & 1 & 0 \\
      0 & 0 & 0 \\
    \end{pmatrix}
    \sim
    \begin{pmatrix}
      1 & 1  & 4  \\
      0 & -1 & -8 \\
      0 & 0  & 0  \\
    \end{pmatrix}
    \sim
    \begin{pmatrix}
      1 & 0 & -4 \\
      0 & 1 & 8  \\
      0 & 0 & 0  \\
    \end{pmatrix}
  $$
  $$
    N\left( A+3I \right)=\sp{
      \begin{pmatrix}
        4  \\
        -8 \\
        1  \\
      \end{pmatrix}
    }
  $$
  $$
    A-4I=
    \begin{pmatrix}
      -3 & 2  & 0  \\
      6  & -4 & 0  \\
      3  & 2  & -3 \\
    \end{pmatrix}
    \sim
    \begin{pmatrix}
      3 & -2 & 0  \\
      3 & 2  & -3 \\
      0 & 0  & 0  \\
    \end{pmatrix}
    \sim
    \begin{pmatrix}
      3 & -2 & 0  \\
      0 & 4  & -3 \\
      0 & 0  & 0  \\
    \end{pmatrix}
    \sim
  $$
  $$
    \sim
    \begin{pmatrix}
      3 & -2 & 0                \\
      0 & 1  & -\nicefrac{3}{4} \\
      0 & 0  & 0                \\
    \end{pmatrix}
    \sim
    \begin{pmatrix}
      3 & 0 & -\nicefrac{6}{4} \\
      0 & 1 & -\nicefrac{3}{4} \\
      0 & 0 & 0                \\
    \end{pmatrix}
    \sim
    \begin{pmatrix}
      1 & 0 & -\nicefrac{1}{2} \\
      0 & 1 & -\nicefrac{3}{4} \\
      0 & 0 & 0                \\
    \end{pmatrix}
  $$
  $$
    N\left( A-4I \right)=\sp{
      \begin{pmatrix}
        2 \\
        3 \\
        4 \\
      \end{pmatrix}
    }
  $$
  La matrice modale è:
  $$
    \begin{pmatrix}
      0 & 4  & 2 \\
      0 & -8 & 3 \\
      1 & 1  & 4 \\
    \end{pmatrix}
  $$
  La matrice diagonale è:
  $$
    \begin{pmatrix}
      1 & 0  & 0 \\
      0 & -3 & 0 \\
      0 & 0  & 4 \\
    \end{pmatrix} 
  $$
\end{solution}

\begin{exercise}
  Data la matrice:
  $$
    A=
    \begin{pmatrix}
      0  & 1 & 0 \\
      1  & 3 & 3 \\
      -1 & 4 & 4 \\
    \end{pmatrix}
  $$
  Stabilire se è diagonalizzabile e, in caso affermativo, trovare la relativa matrice diagonale e modale.
\end{exercise}
\begin{solution}
  \begin{align*}
    p_A\left( t \right) & = 
    \begin{vmatrix}
      -t & 1   & 0   \\
      1  & 3-t & 3   \\
      -1 & 4   & 4-t \\
    \end{vmatrix}
    \\
                        & = 
    -t
    \begin{vmatrix}
      3-t & 3   \\
      4   & 4-t \\
    \end{vmatrix}
    -
    \begin{vmatrix}
      1  & 3   \\
      -1 & 4-t \\
    \end{vmatrix}
    \\
                        & = 
    -t\left( t^2-7t \right)-\left( 7-t \right) 
    \\
                        & = 
    -t^2\left( t-7 \right)+\left( t-7 \right)
    \\
                        & = 
    \left( t-7 \right)\left( -t^2+1 \right)
  \end{align*}
  $$p_A\left( t \right)=0\iff t\in \left\{ 7,\pm1 \right\}$$
  $A$ è diagonalizzabile per il primo criterio di diagonalizzabilità.
  $$
    A-7I=
    \begin{pmatrix}
      -7 & 1  & 0  \\
      1  & -4 & 3  \\
      -1 & 4  & -3 \\
    \end{pmatrix}
    \sim
    \begin{pmatrix}
      1  & -4 & 3 \\
      -7 & 1  & 0 \\
      0  & 0  & 0 \\
    \end{pmatrix}
    \sim
    \begin{pmatrix}
      1 & -4  & 3  \\
      0 & -27 & 21 \\
      0 & 0   & 0  \\
    \end{pmatrix}
    \sim
  $$
  $$
    \sim
    \begin{pmatrix}
      1 & -4 & 3  \\
      0 & 9  & -7 \\
      0 & 0  & 0  \\
    \end{pmatrix}
    \sim
    \begin{pmatrix}
      1 & -4 & 3                \\
      0 & 1  & -\nicefrac{7}{9} \\
      0 & 0  & 0                \\
    \end{pmatrix}
    \sim
    \begin{pmatrix}
      1 & 0 & -\nicefrac{1}{9} \\
      0 & 1 & -\nicefrac{7}{9} \\
      0 & 0 & 0                \\
    \end{pmatrix}
  $$
  $$
    N\left( A-7I \right)=\sp{
      \begin{pmatrix}
        1 \\
        7 \\
        9 \\
      \end{pmatrix}
    }
  $$
  $$
    A+1\cdot I=
    \begin{pmatrix}
      1  & 1 & 0 \\
      1  & 4 & 3 \\
      -1 & 4 & 5 \\
    \end{pmatrix}
    \sim
    \begin{pmatrix}
      1 & 1 & 0 \\
      1 & 4 & 3 \\
      0 & 8 & 8 \\
    \end{pmatrix}
    \sim
    \begin{pmatrix}
      1 & 1 & 0 \\
      0 & 3 & 3 \\
      0 & 1 & 1 \\
    \end{pmatrix}
    \sim
    \begin{pmatrix}
      1 & 1 & 0 \\
      0 & 1 & 1 \\
      0 & 0 & 0 \\
    \end{pmatrix}
    \sim
    \begin{pmatrix}
      1 & 0 & -1 \\
      0 & 1 & 1  \\
      0 & 0 & 0  \\
    \end{pmatrix}
  $$
  $$
    N\left( A+1\cdot I \right)=\sp{
      \begin{pmatrix}
        1  \\
        -1 \\
        1  \\
      \end{pmatrix}
    }
  $$
  $$
    A-1\cdot I=
    \begin{pmatrix}
      -1 & 1 & 0 \\
      1  & 2 & 3 \\
      -1 & 4 & 3 \\
    \end{pmatrix}
    \sim
    \begin{pmatrix}
      1 & -1 & 0 \\
      0 & 3  & 3 \\
      0 & 6  & 6 \\
    \end{pmatrix}
    \sim
    \begin{pmatrix}
      1 & -1 & 0 \\
      0 & 1  & 1 \\
      0 & 0  & 0 \\
    \end{pmatrix}
    \sim
    \begin{pmatrix}
      1 & 0 & 1 \\
      0 & 1 & 1 \\
      0 & 0 & 0 \\
    \end{pmatrix}
  $$
  $$
    N\left( A-1\cdot I \right)=\sp{
      \begin{pmatrix}
        -1 \\
        -1 \\
        1  \\
      \end{pmatrix}
    }
  $$
  La matrice modale è:
  $$
    \begin{pmatrix}
      1 & 1  & -1 \\
      7 & -1 & -1 \\
      9 & 1  & 1  \\
    \end{pmatrix}
  $$
  La matrice diagonale è:
  $$
    \begin{pmatrix}
      7 & 0  & 0 \\
      0 & -1 & 0 \\
      0 & 0  & 1 \\
    \end{pmatrix} 
  $$
\end{solution}

\begin{exercise}
  Data la matrice:
  $$
    A=
    \begin{pmatrix}
      3  & 0 & 0  \\
      3  & 3 & 6  \\
      -4 & 0 & -5 \\
    \end{pmatrix}
  $$
  Verificare che $A$ sia diagonalizzabile. Calcolare una base $B$ formata da autovettori di $A$. Calcolare la matrice rappresentativa di $T_A$ nella base $B$.
\end{exercise}
\begin{solution}
  $$
    p_A\left( t \right)=
    \begin{vmatrix}
      3-t & 0   & 0    \\
      3   & 3-t & 6    \\
      -4  & 0   & -5-t \\
    \end{vmatrix}
    =
    \left( 3-t \right)
    \begin{vmatrix}
      3-t & 6    \\
      0   & -5-t \\
    \end{vmatrix}
    =
    \left( 3-t \right)\left( 3-t \right)\left( -5-t \right)
  $$
  $$p_A\left( t \right)=0\iff t\in\left\{ 3,-5 \right\}$$
  $$m_a\left( 3 \right)=2$$
  $$m_a\left( -5 \right)=1$$
  $$
    m_g\left( 3 \right)=3-\rk
    \begin{pmatrix}
      0  & 0 & 0  \\
      3  & 0 & 6  \\
      -4 & 0 & -8 \\
    \end{pmatrix}
    =3-1=2
  $$
  $$
    m_g\left( -5 \right)=3-\rk
    \begin{pmatrix}
      8  & 0 & 0 \\
      3  & 8 & 6 \\
      -4 & 0 & 0 \\
    \end{pmatrix}
    =3-2=1
  $$
  $A$ è diagonalizzabile per il secondo criterio di diagonalizzabilità.
  $$
    A-3I=
    \begin{pmatrix}
      0  & 0 & 0  \\
      3  & 0 & 6  \\
      -4 & 0 & -8 \\
    \end{pmatrix}
    \sim
    \begin{pmatrix}
      1 & 0 & 2 \\
      0 & 0 & 0 \\
      0 & 0 & 0 \\
    \end{pmatrix}
  $$
  $$
    N\left( A-3I \right)=\sp{
      \begin{pmatrix}
        0 \\
        1 \\
        0 \\
      \end{pmatrix}
      ,
      \begin{pmatrix}
        -2 \\
        0  \\
        1  \\
      \end{pmatrix}
    }
  $$
  $$
    A+5I=
    \begin{pmatrix}
      8  & 0 & 0 \\
      3  & 8 & 6 \\
      -4 & 0 & 0 \\
    \end{pmatrix}
    \sim
    \begin{pmatrix}
      1 & 0 & 0 \\
      3 & 8 & 6 \\
      0 & 0 & 0 \\
    \end{pmatrix}
    \sim
    \begin{pmatrix}
      1 & 0 & 0 \\
      0 & 8 & 6 \\
      0 & 0 & 0 \\
    \end{pmatrix}
    \sim
    \begin{pmatrix}
      1 & 0 & 0               \\
      0 & 1 & \nicefrac{3}{4} \\
      0 & 0 & 0               \\
    \end{pmatrix}
  $$
  $$
    N\left( A+5I \right)=\sp{
      \begin{pmatrix}
        0  \\
        -3 \\
        4  \\  
      \end{pmatrix}
    }
  $$
  $$
    B=\left\{ 
    \begin{pmatrix}
      0 \\
      1 \\
      0 \\
    \end{pmatrix}
    ,
    \begin{pmatrix}
      -2 \\
      0  \\
      1  \\
    \end{pmatrix}
    ,
    \begin{pmatrix}
      0  \\
      -3 \\
      4  \\  
    \end{pmatrix}
    \right\}
  $$
  La matrice rappresentativa di $T_A$ nella base $B$ non è altro che la matrice diagonale:
  $$
    \begin{pmatrix}
      3 & 0 & 0  \\
      0 & 3 & 0  \\
      0 & 0 & -5 \\
    \end{pmatrix}
  $$
\end{solution}

\begin{exercise}
  Dato $k\in\reals$, sia:
  $$
    A_k=
    \begin{pmatrix}
      1 & k & 0 \\
      k & 1 & k \\
      0 & 1 & 1 \\
    \end{pmatrix}
  $$
  \begin{enumerate}
    \item verificare che $\left( 1,0,-1 \right)$ è una autovettore di $A_k\ \forall k$ e calcolarne il relativo autovalore;
    \item determinare se esiste una base $B$ di $\reals^3$ formata da autovettori di $A_1$ e in tal caso calcolarla;
    \item determinare per quali valori di $k$ la matrice $A_k$ non è diagonalizzabile.
  \end{enumerate}
\end{exercise}
\begin{solution}
  $$
    \begin{pmatrix}
      1 & k & 0 \\
      k & 1 & k \\
      0 & 1 & 1 \\
    \end{pmatrix}
    \begin{pmatrix}
      1  \\
      0  \\
      -1 \\
    \end{pmatrix}
    =
    \begin{pmatrix}
      1  \\
      0  \\
      -1 \\
    \end{pmatrix}
    =
    \l
    \begin{pmatrix}
      1  \\
      0  \\
      -1 \\
    \end{pmatrix}
    =
    \begin{pmatrix}
      \l  \\
      0   \\
      -\l \\
    \end{pmatrix}
  $$
  $$
    \begin{pmatrix}
      1  \\
      0  \\
      -1 \\
    \end{pmatrix}
    =
    \begin{pmatrix}
      \l  \\
      0   \\
      -\l \\
    \end{pmatrix}
    \iff \l=1
  $$
  $\left( 1,0,-1 \right)$ è un autovettore di $A_k\ \forall k$, con relativo autovalore $1$.
  $$
    A_1=
    \begin{pmatrix}
      1 & 1 & 0 \\
      1 & 1 & 1 \\
      0 & 1 & 1 \\
    \end{pmatrix}
  $$
  \begin{align*}
    p_{A_1}\left( t \right)
     & = 
    \begin{vmatrix}
      1-t & 1   & 0   \\
      1   & 1-t & 1   \\
      0   & 1   & 1-t \\
    \end{vmatrix}
    \\
     & = 
    \left( 1-t \right)
    \begin{vmatrix}
      1-t & 1   \\
      1   & 1-t \\
    \end{vmatrix}
    -
    \begin{vmatrix}
      1 & 1   \\
      0 & 1-t \\
    \end{vmatrix}
    \\
     & = 
    \left( 1-t \right)\left( t^2+2t \right)-\left( 1-t \right)
    \\
     & = 
    \left( 1-t \right)\left( t^2+2t-1 \right)
  \end{align*}
  $$p_{A_1}\left( t \right)=0\iff t\in\left\{ 1,1\pm\sqrt{2} \right\}$$
  $$
    A_1-I=
    \begin{pmatrix}
      0 & 1 & 0 \\
      1 & 0 & 1 \\
      0 & 1 & 0 \\
    \end{pmatrix}
    \sim
    \begin{pmatrix}
      1 & 0 & 1 \\
      0 & 1 & 0 \\
      0 & 0 & 0 \\
    \end{pmatrix}
  $$
  $$
    N\left( A_1-I \right)=\sp{
      \begin{pmatrix}
        -1 \\
        0  \\
        1  \\
      \end{pmatrix}
    }
  $$
  $$
    A_1-\left( 1-\sqrt{2} \right)I=
    \begin{pmatrix}
      \sqrt{2} & 1        & 0        \\
      1        & \sqrt{2} & 1        \\
      0        & 1        & \sqrt{2} \\
    \end{pmatrix}
    \sim
    \begin{pmatrix}
      2 & \sqrt{2} & 0        \\
      1 & \sqrt{2} & 1        \\
      0 & 1        & \sqrt{2} \\
    \end{pmatrix}
    \sim
    \begin{pmatrix}
      1 & 0        & -1       \\
      1 & \sqrt{2} & 1        \\
      0 & 1        & \sqrt{2} \\
    \end{pmatrix}
    \sim
  $$
  $$
    \sim
    \begin{pmatrix}
      1 & 0        & -1       \\
      0 & \sqrt{2} & 2        \\
      0 & 1        & \sqrt{2} \\
    \end{pmatrix}
    \sim
    \begin{pmatrix}
      1 & 0 & -1        \\
      0 & 2 & 2\sqrt{2} \\
      0 & 1 & \sqrt{2}  \\
    \end{pmatrix}
    \sim
    \begin{pmatrix}
      1 & 0 & -1       \\
      0 & 1 & \sqrt{2} \\
      0 & 0 & 0        \\
    \end{pmatrix}
  $$
  $$
    N\left( A_1-\left( 1-\sqrt{2} \right)I \right)=\sp{
      \begin{pmatrix}
        1         \\
        -\sqrt{2} \\
        1         \\
      \end{pmatrix}
    }
  $$
  $$
    A-\left( 1+\sqrt{2} \right)I=
    \begin{pmatrix}
      -\sqrt{2} & 1         & 0         \\
      1         & -\sqrt{2} & 1         \\
      0         & 1         & -\sqrt{2} \\
    \end{pmatrix}
    \sim
    \begin{pmatrix}
      1  & -\sqrt{2} & 1         \\
      -2 & \sqrt{2}  & 0         \\
      0  & 1         & -\sqrt{2} \\
    \end{pmatrix}
    \sim
    \begin{pmatrix}
      1 & -\sqrt{2} & 1         \\
      0 & -\sqrt{2} & 2         \\
      0 & 1         & -\sqrt{2} \\
    \end{pmatrix}
    \sim
  $$
  $$
    \sim
    \begin{pmatrix}
      1 & -\sqrt{2} & 1         \\
      0 & -2        & 2\sqrt{2} \\
      0 & 1         & -\sqrt{2} \\
    \end{pmatrix}
    \sim
    \begin{pmatrix}
      1 & -\sqrt{2} & 1         \\
      0 & 1         & -\sqrt{2} \\
      0 & 0         & 0         \\
    \end{pmatrix}
    \sim
    \begin{pmatrix}
      1 & -\sqrt{2} & 1  \\
      0 & \sqrt{2}  & -2 \\
      0 & 0         & 0  \\
    \end{pmatrix}
    \sim
    \begin{pmatrix}
      1 & 0 & -1        \\
      0 & 1 & -\sqrt{2} \\
      0 & 0 & 0         \\
    \end{pmatrix}
  $$
  $$
    N\left( A-\left( 1+\sqrt{2} \right)I \right)=\sp{
      \begin{pmatrix}
        1        \\
        \sqrt{2} \\
        1        \\
      \end{pmatrix}
    }
  $$
  $$
    B=\left\{ 
    \begin{pmatrix}
      -1 \\
      0  \\
      1  \\
    \end{pmatrix}
    ,
    \begin{pmatrix}
      1         \\
      -\sqrt{2} \\
      1         \\
    \end{pmatrix}
    ,
    \begin{pmatrix}
      1        \\
      \sqrt{2} \\
      1        \\
    \end{pmatrix}
    \right\}
  $$
  $$
    \reals^3=\sp{
      \begin{pmatrix}
        -1 \\
        0  \\
        1  \\
      \end{pmatrix}
      ,
      \begin{pmatrix}
        1         \\
        -\sqrt{2} \\
        1         \\
      \end{pmatrix}
      ,
      \begin{pmatrix}
        1        \\
        \sqrt{2} \\
        1        \\
      \end{pmatrix}
    }
  $$
  \begin{align*}
    p_{A_k}\left( t \right)
     & = 
    \begin{vmatrix}
      1-t & k   & 0   \\
      k   & 1-t & k   \\
      0   & 1   & 1-t \\
    \end{vmatrix}
    \\
     & = 
    \left( 1-t \right)
    \begin{vmatrix}
      1-t & k   \\
      1   & 1-t \\
    \end{vmatrix}
    -k
    \begin{vmatrix}
      k & k   \\
      0 & 1-t \\
    \end{vmatrix}
    \\
     & = 
    \left( 1-t \right)\left( 1+t^2-2t-k \right)-k\left( k-kt \right)
    \\
     & = 
    \left( 1-t \right)\left( t^2-2t+1-k \right)-k^2\left( 1-t \right)
    \\
     & = 
    \left( 1-t \right)\left( t^2-2t+1-k -k^2\right)
  \end{align*}
  $t^2-2t+1-k-k^2=0\iff \left( t-1 \right)^2-k\left( 1+k \right)=0\iff \left( t-1 \right)^2=k\left( 1+k \right)\iff \left( t-1 \right)^2= k+k^2 \iff t-1=\pm\sqrt{k+k^2}\iff t=1\pm\sqrt{k+k^2}$
  $$p_{A_k}\left( t \right)=0\iff t\in\left\{ 1,1\pm\sqrt{k+k^2} \right\}$$
  $$k+k^2=k(1+k)=0\iff k\in\left\{ -1,0 \right\}$$
  $$k+k^2<0\iff -1<k<0$$
  La matrice $A_k$ non è diagonalizzabile per $-1<k<0$.
  $$k=0\impl m_a\left( 1 \right)=3 \wedge m_g\left( 1 \right)=2$$
  $$k=-1\impl m_a\left( 1 \right)=3 \wedge m_g\left( 1 \right)=1$$
  La matrice $A_k$ non è diagonalizzabile per $k=-1\vee k=0$.
\end{solution}

\begin{exercise}
  Dato $h\in\reals$, sia:
  $$
    A_h=
    \begin{pmatrix}
      1 & 1 & h \\
      1 & 1 & 1 \\
      h & 1 & 1 \\
    \end{pmatrix}
  $$
  \begin{enumerate}
    \item verificare che $\left( -1,0,1 \right)$ è un autovettore di $A_h$, e in tal caso calcolarne il relativo autovalore;
    \item calcolare una matrice $M$ tale che $M^{-1}A_1M$ è diagonale;
    \item verificare che l'insieme $B$ formato dalle colonne di $M$ è una base di $\reals^3$ e calcolare la matrice $T_{A_1}$ in $B$.
  \end{enumerate}
\end{exercise}
\begin{solution}
  $$
    \begin{pmatrix}
      1 & 1 & h \\
      1 & 1 & 1 \\
      h & 1 & 1 \\
    \end{pmatrix}
    \begin{pmatrix}
      -1 \\
      0  \\
      1  \\
    \end{pmatrix}
    =
    \begin{pmatrix}
      h-1 \\
      0   \\
      1-h \\
    \end{pmatrix}
    =
    \l
    \begin{pmatrix}
      -1 \\
      0  \\
      1  \\
    \end{pmatrix}
    \iff
    \l=1-h
  $$
  $\l=1-h$ è l'autovalore di $A_h$ relativo all'autovettore $\left( -1,0,1 \right)$.
  $$
    A_1=
    \begin{pmatrix}
      1 & 1 & 1 \\
      1 & 1 & 1 \\
      1 & 1 & 1 \\
    \end{pmatrix}
  $$
  \begin{align*}
    p_{A_1}\left( t \right)
     & = 
    \begin{vmatrix}
      1-t & 1   & 1   \\
      1   & 1-t & 1   \\
      1   & 1   & 1-t \\
    \end{vmatrix}
    \\
     & = 
    \left( 1-t \right)
    \begin{vmatrix}
      1-t & 1   \\
      1   & 1-t \\
    \end{vmatrix}
    -
    \begin{vmatrix}
      1 & 1   \\
      1 & 1-t \\
    \end{vmatrix}
    +
    \begin{vmatrix}
      1 & 1-t \\
      1 & 1   \\
    \end{vmatrix}
    \\
     & = 
    \left( 1-t \right)\left( t^2-2t \right)-\left( -t \right)+ t 
    \\
     & = 
    t\left( 1-t \right)\left( t-2 \right)+2t
    \\
     & = 
    t\left( \left( 1-t \right)\left( t-2 \right)+2 \right)
    \\
     & = 
    t\left( 3t-t^2 \right)
    \\
     & = 
    t^2\left( 3-t \right)
  \end{align*}
  $$p_{A_1}\left( t \right)=0\iff t\in\left\{ 0,3 \right\}$$
  $$
    A_1-0\cdot I=
    \begin{pmatrix}
      1 & 1 & 1 \\
      1 & 1 & 1 \\
      1 & 1 & 1 \\
    \end{pmatrix}
    \sim
    \begin{pmatrix}
      1 & 1 & 1 \\
      0 & 0 & 0 \\
      0 & 0 & 0 \\
    \end{pmatrix}
  $$
  $$
    N\left( A_1-0\cdot I \right)=\sp{
      \begin{pmatrix}
        -1 \\
        1  \\
        0  \\
      \end{pmatrix}
      ,
      \begin{pmatrix}
        -1 \\
        0  \\
        1  \\
      \end{pmatrix}
    }
  $$
  $$
    A_1-3I=
    \begin{pmatrix}
      -2 & 1  & 1  \\
      1  & -2 & 1  \\
      1  & 1  & -2 \\
    \end{pmatrix}
    \sim
    \begin{pmatrix}
      1 & 1  & -2 \\
      0 & -3 & 3  \\
      0 & 3  & -3 \\
    \end{pmatrix}
    \sim
    \begin{pmatrix}
      1 & 1 & -2 \\
      0 & 1 & -1 \\
      0 & 0 & 0  \\
    \end{pmatrix}
    \sim
    \begin{pmatrix}
      1 & 0 & -1 \\
      0 & 1 & -1 \\
      0 & 0 & 0  \\
    \end{pmatrix}
  $$
  $$
    N\left( A_1-3I \right)=\sp{
      \begin{pmatrix}
        1 \\
        1 \\
        1 \\
      \end{pmatrix}
    }
  $$
  $$
    M=
    \begin{pmatrix}
      -1 & -1 & 1 \\
      1  & 0  & 1 \\
      0  & 1  & 1 \\
    \end{pmatrix}
  $$
  $$
    M^{-1}A_1M=
    \begin{pmatrix}
      0 & 0 & 0 \\
      0 & 0 & 0 \\
      0 & 0 & 3 \\
    \end{pmatrix}
  $$
  $$
    \det M=
    \begin{vmatrix}
      -1 & -1 & 1 \\
      1  & 0  & 1 \\
      0  & 1  & 1 \\
    \end{vmatrix}
    =
    -
    \begin{vmatrix}
      -1 & 1 \\
      1  & 1 \\
    \end{vmatrix}
    +
    \begin{vmatrix}
      -1 & -1 \\
      1  & 0  \\
    \end{vmatrix}
    =
    1+1+1=3
    \impl
    \rk M=3
    \impl
    R\left( M \right)=\reals^3
  $$
  $$
    B=\left\{ 
    \begin{pmatrix}
      -1 \\
      1  \\
      0  \\
    \end{pmatrix}
    ,
    \begin{pmatrix}
      -1 \\
      0  \\
      1  \\
    \end{pmatrix}
    ,
    \begin{pmatrix}
      1 \\
      1 \\
      1 \\
    \end{pmatrix}
    \right\}
  $$
  $$
    \begin{pmatrix}
      1 & 1 & 1 \\
      1 & 1 & 1 \\
      1 & 1 & 1 \\
    \end{pmatrix}
    \begin{pmatrix}
      -1 \\
      1  \\
      0  \\
    \end{pmatrix}
    =
    \begin{pmatrix}
      0 \\
      0 \\
      0 \\
    \end{pmatrix}
  $$
  $$
    \begin{pmatrix}
      -1 & -1 & 1 & 0 \\
      1  & 0  & 1 & 0 \\
      0  & 1  & 1 & 0 \\
    \end{pmatrix}
    \sim
    \begin{pmatrix}
      1 & 0  & 1 & 0 \\
      0 & -1 & 2 & 0 \\
      0 & 1  & 1 & 0 \\
    \end{pmatrix}
    \sim
    \begin{pmatrix}
      1 & 0 & 1 & 0 \\
      0 & 1 & 1 & 0 \\
      0 & 0 & 3 & 0 \\
    \end{pmatrix}
    \sim
    \begin{pmatrix}
      1 & 0 & 0 & 0 \\
      0 & 1 & 0 & 0 \\
      0 & 0 & 1 & 0 \\
    \end{pmatrix}
  $$
  $$
    \begin{pmatrix}
      1 & 1 & 1 \\
      1 & 1 & 1 \\
      1 & 1 & 1 \\
    \end{pmatrix}
    \begin{pmatrix}
      -1 \\
      0  \\
      1  \\
    \end{pmatrix}
    =
    \begin{pmatrix}
      0 \\
      0 \\
      0 \\
    \end{pmatrix}
  $$
  $$
    \begin{pmatrix}
      -1 & -1 & 1 & 0 \\
      1  & 0  & 1 & 0 \\
      0  & 1  & 1 & 0 \\
    \end{pmatrix}
    \sim
    \begin{pmatrix}
      1 & 0  & 1 & 0 \\
      0 & -1 & 2 & 0 \\
      0 & 1  & 1 & 0 \\
    \end{pmatrix}
    \sim
    \begin{pmatrix}
      1 & 0 & 1 & 0 \\
      0 & 1 & 1 & 0 \\
      0 & 0 & 3 & 0 \\
    \end{pmatrix}
    \sim
    \begin{pmatrix}
      1 & 0 & 0 & 0 \\
      0 & 1 & 0 & 0 \\
      0 & 0 & 1 & 0 \\
    \end{pmatrix}
  $$
  $$
    \begin{pmatrix}
      1 & 1 & 1 \\
      1 & 1 & 1 \\
      1 & 1 & 1 \\
    \end{pmatrix}
    \begin{pmatrix}
      1 \\
      1 \\
      1 \\
    \end{pmatrix}
    =
    \begin{pmatrix}
      3 \\
      3 \\
      3 \\
    \end{pmatrix}
  $$
  $$
    \begin{pmatrix}
      -1 & -1 & 1 & 3 \\
      1  & 0  & 1 & 3 \\
      0  & 1  & 1 & 3 \\
    \end{pmatrix}
    \sim
    \begin{pmatrix}
      1 & 0  & 1 & 3 \\
      0 & -1 & 2 & 6 \\
      0 & 1  & 1 & 3 \\
    \end{pmatrix}
    \sim
    \begin{pmatrix}
      1 & 0 & 1 & 3 \\
      0 & 1 & 1 & 3 \\
      0 & 0 & 3 & 9 \\
    \end{pmatrix}
    \sim
    \begin{pmatrix}
      1 & 0 & 1 & 3 \\
      0 & 1 & 1 & 3 \\
      0 & 0 & 1 & 3 \\
    \end{pmatrix}
    \sim
    \begin{pmatrix}
      1 & 0 & 0 & 0 \\
      0 & 1 & 0 & 0 \\
      0 & 0 & 1 & 3 \\
    \end{pmatrix}
  $$
  La matrice rappresentativa di $T_{A_1}$ nella base $B$ è:
  $$
    \begin{pmatrix}
      0 & 0 & 0 \\
      0 & 0 & 0 \\
      0 & 0 & 3 \\
    \end{pmatrix}
  $$
\end{solution}

\end{document}