\documentclass[a4paper,12pt]{article}

\usepackage[italian]{babel}
\usepackage[utf8]{inputenc}
\usepackage[a4paper, left=18mm, right=18mm, top=20mm, bottom=20mm]{geometry}
\usepackage{amssymb}
\usepackage{mathtools}
\usepackage{interval}
\usepackage{amsthm}
\usepackage{thmtools}
\usepackage{cancel}
\usepackage{hyperref}
\usepackage{tikz}
\usepackage{pgfplots}
\usepackage{nicefrac}
\usepackage{enumitem}
\usepackage{verbatim}
\usepackage{tabularray}
\usepackage{algorithm}
\usepackage[noend]{algpseudocode}
\usepackage{fontspec}
\usepackage{bold-extra}

\setmonofont[Contextuals={Alternate},SizeFeatures={Size=9}]{Fira Code}

\usetikzlibrary{shapes.geometric, arrows}
\tikzstyle{startstop} = [rectangle, rounded corners, minimum width=3cm, minimum height=1cm,text centered, draw=black, fill=magenta!30]
\tikzstyle{input} = [trapezium, trapezium left angle=70, trapezium right angle=110, minimum width=2cm, minimum height=.5cm, text centered, draw=black, fill=cyan!30]
\tikzstyle{output} = [trapezium, trapezium left angle=110, trapezium right angle=70, minimum width=2cm, minimum height=.5cm, text centered, draw=black, fill=cyan!30]
\tikzstyle{process} = [rectangle, minimum width=3cm, minimum height=1cm, text centered, draw=black, fill=orange!30]
\tikzstyle{decision} = [diamond, minimum width=3cm, minimum height=1cm, text centered, draw=black, fill=green!30, aspect=1.618]
\tikzstyle{dummy} = [rectangle, minimum width=0cm, minimum height=0cm, inner sep=0pt]
\tikzstyle{arrow} = [thick,->,>=stealth]

\hypersetup{
  colorlinks=true,
  linkcolor=black,
    filecolor=magenta,      
    urlcolor=cyan,
    pdftitle={Fondamenti di informatica},
    % bookmarks=true,
    bookmarksopen=true,
    pdfpagemode=UseOutlines,
    pdfauthor={Amato Michele Pasquale},
}

\newtheorem{definition}{Definizione}
\renewcommand{\(}{\left(}
\renewcommand{\)}{\right)}
\newcommand{\impl}{\Rightarrow}
\DeclareMathOperator{\sgn}{sgn}
\renewcommand{\epsilon}{\varepsilon}

\title{\huge Fondamenti di informatica}
\author{Amato Michele Pasquale}
\date{\today}

\pgfplotsset{compat = newest}
\makeatletter
\renewcommand\l@subsection{\@dottedtocline{2}{1.5em}{3em}}
\makeatother
\setitemize{noitemsep,topsep=3pt,parsep=0pt,partopsep=0pt}
\setenumerate{noitemsep,topsep=3pt,parsep=0pt,partopsep=0pt}

\newcommand{\abs}[1]{\left\lvert #1 \right\rvert}
\newcommand{\ceil}[1]{\left\lceil #1 \right\rceil}

\newenvironment{example}
{\par\noindent{\small\bf Esempio}\hspace{0.5em}}
{\hfill$\nsim$}

\begin{document}
\pagenumbering{gobble}
\maketitle
\vfill
\begin{center}
  \emph{Efficienza è fare le cose bene. \\ Efficacia è fare le cose giuste.}
\end{center}
\newpage
\pagenumbering{roman}
\tableofcontents
\newpage
\pagenumbering{arabic}
\part{La rappresentazione dell'informazione}
% % lezione del 12/09/2022

Il problema che ci si pone è trovare un modo opportuno per rappresentare all'interno di un sistema di calcolo le informazioni in modo efficiente, rispetto alla realtà fisica del sistema e alla loro manipolazione.

\begin{definition}[Alfabeto]
  Si definisce \textbf{alfabeto} un insieme di simboli utilizzabili e, pertanto, distinguibili tra loro.
\end{definition}

\begin{definition}[Codice]
  Si definisce \textbf{codice} l'insieme delle sequenze di simboli o delle regole per definire le combinazioni ammissibili.
\end{definition}

Dati l'insieme degli elementi da rappresentare e l'insieme delle configurazioni ammissibili, il codice ne definisce la relazione biunivoca.
Le configurazioni ammissibili hanno tutte egual dimensione. Tale dimensione dipende sia dall'alfabeto, sia dalla quantità di elementi da rappresentare: siano $S$ l'alfabeto di riferimento e $\abs{S}$ la sua cardinalità (ossia il numero di simboli che lo compone), se si vogliono rappresentare $n$ elementi, ogni elemento avrà dimensione:
$$k=\ceil{\log_{\abs{S}} n}$$
Al contrario, se gli elementi di un codice hanno lunghezza $k$, le combinazioni ammissibili sono:
$$n=\abs{S}^k$$

\section{Rappresentazione binaria}

I componenti elettronici che costituiscono il sistema di calcolo sono caratterizzati da una realtà costituita da due stati (condensatore carico/scarico, tensione alta/bassa, etc...). Si effettua, quindi, una mappatura diretta con un sistema costituito da \textbf{due simboli} che, pertanto, si chiama \textbf{binario}. L'alfabeto di riferimento diventa $\left\{ 0,1 \right\}$.

La cifra della codifica (0 o 1) prende il nome di \textbf{bit}, dall'inglese \emph{\textbf{bi}nary digi\textbf{t}}. L'insieme ordinato di 8 bit prende il nome di \textbf{byte}. Come per le cifre decimali, si ha una nomenclatura per le potenze della base:
\begin{center}
  \begin{tblr}{colspec={c|c}, cells={c,m}, columns={20mm}}
    \textbf{Nome} & \textbf{Quantità} \\
    \hline
    KB            & $2^{10}$          \\
    MB            & $2^{20}$          \\
    GB            & $2^{30}$          \\
    TB            & $2^{40}$          \\
  \end{tblr}
\end{center}

\begin{example}
  Se si vogliono rappresentare i giorni della settimana usando l'alfabeto binario, si calcola la dimensione del singolo elemento:
  $$k=\ceil{\log_27}=3$$
  e si assegna ad ogni combinazione di 3 bit un giorno della settimana distinto:
  \begin{center}
    \begin{tblr}{colspec={c|c|c|c|c|c|c}, cells={c,m}, columns={18mm}}
      Lunedì & Martedì & Mercoledì & Giovedì & Venerdì & Sabato & Domenica \\
      \hline
      000    & 001     & 010       & 011     & 100     & 101    & 110      
    \end{tblr}
  \end{center}
  Da notare che non viene utilizzata la combinazione 111, in quanto le combinazioni ammissibili sono $2^3=8$ ma per le necessità del caso ne servono solo 7.
\end{example}

Nella scelta della codifica da adottare, bisogna tenere a mente alcuni aspetti:
\begin{itemize}
  \item l'insieme degli elementi da rappresentare;
  \item il grado di semplificazione delle operazioni più eseguite;
  \item il grado di conservazione delle proprietà dell'insieme originale.
\end{itemize}

L'informazione può essere, per comodità, suddivisa in aree:

\begin{center}
  \begin{tblr}{colspec={c|c|c|c|c|c}, cells={c,m}, columns={20mm}}
    \SetCell[c=6]{c} Informazione                                                                       \\ \hline
    \SetCell[c=3]{c} Numerica &          &           & \SetCell[c=3]{c} Non numerica                    \\ \hline
    Naturali                  & Relativi & Razionali & Testi                         & Audio & Immagini 
  \end{tblr}
\end{center}

% lezione del 13/09/2022

La notazione che si usa in base 10 è una \textbf{notazione posizionale pesata}, vale a dire che ogni cifra vale in base alla posizione che essa occupa all'interno del numero. Il valore del numero, infatti, è dato da
$$\text{valore}=\sum_{0}^nc_ib^i$$
dove $c_i$ è la cifra in posizione $i$ e $b$ è la base di riferimento.

\begin{example}
  $$315_{10}=3\cdot10^2+1\cdot10^1+5\cdot10^0$$
\end{example}

Nella notazione in base 2, si definisce un codice che associa al valore numerico una configurazione, in cui la cifra più a destra è la cifra meno significativa (\textbf{LSB}, Least Significant Bit), mentre quella più a sinistra è quella più significativa (\textbf{MSB}, Most Significant Bit).

\section{Conversione di base}
Idealmente, ogni qualvolta si vuole rappresentare una quantità in una certa base, si vogliono implicitamente rappresentare in un'altra base tutte le quantità minori uguali a quella di partenza, non avrebbe senso altrimenti\footnote{se me ne serve solo uno, uso il primo valore disponibile}.

Se $n$ è il valore da rappresentare, significa che in base 2 si avrà bisogno di $k$ bits:
$$k=\ceil{\log_2\(n+1\)}$$
\paragraph*{Osservazione} L'argomento del logaritmo è $n+1$ in quanto bisogna anche considerare lo 0.

Per passare da una base $a$ ad una base $b$, con $a<b$, si procede nel seguente modo: si moltiplica ogni coefficiente per la base elevata alla sua posizione e poi si sommano tutti i prodotti ottenuti.

\begin{example}
  $$1010_2=1\cdot2^3+0\cdot2^2+1\cdot2^1+0\cdot2^0=8+0+2+0=10_{10}$$
\end{example}

Per passare da una base $a$ ad una base $b$, con $a>b$, si procede nel seguente modo:

\begin{enumerate}
  \item si divide il numero per la base;
  \item si prende il resto;
  \item si ripete il processo con il quoziente ottenuto.
\end{enumerate}

Il processo si conclude quando il quoziente diventa 0. Il risultato non è altro che la sequenza ordinata dei resti ottenuti, letta al contrario. 

\begin{example}
  \begin{center}
    \begin{tblr}{c|c}
      17 & 1 \\
      8  & 0 \\
      4  & 0 \\
      2  & 0 \\
      1  & 1 
    \end{tblr}
    $$\impl 17_{10}=10001_2$$
  \end{center}
\end{example}

Le modalità con cui si trasformano le quantità fra le diverse basi sono interscambiabili, pertanto si dicono metodi.

\section{Modulo e segno}

Nella notazione decimale si utilizza la forma ``modulo e segno'' per esprimere i numeri relativi:
$$-11\;\;+4$$

Tale ragionamento non si può estendere alla rappresentazione binaria, perché richiederebbe l'introduzione un nuovo simbolo per `$+$' o `$-$'. Pertanto, per convenzione, il primo bit indica il segno, che può essere 0 se è positivo, 1 se è negativo:
$$+17_{10\text{MS}}=010001_{2\text{MS}}$$
$$-17_{10\text{MS}}=110001_{2\text{MS}}$$
$$-23_{10\text{MS}}=110111_{2\text{MS}}$$

Si devono eseguire le operazioni aritmetiche usando la notazione modulo e segno.
\begin{center}
  \begin{tblr}{c|c}
    $
      \begin{aligned}
        x & =+11_{10\text{MS}} \\
        y & =+8_{10\text{MS}}  \\
        z & =-7_{10\text{MS}}  \\
      \end{aligned}
    $
     & 
    $
      \begin{aligned}
        x+y & =+19_{10\text{MS}} \\
        x+z & =+4_{10\text{MS}}  \\
      \end{aligned}
    $
  \end{tblr}
\end{center}

Nella rappresentazione decimale si eseguono i passaggi descritti nel seguente algoritmo:

\begin{algorithm}[H]
  \caption{Somma di due valori in notazione modulo e segno}\label{algo:sommams}
  \begin{algorithmic}[1]
    \If{$\sgn x= \sgn y$}
    \State $\abs{n}=\abs{x}+\abs{y}$
    \State $\sgn n=\sgn x$
    \Else
    \State $a = \max \(\abs{x},\abs{y}\)$
    \State $b = \min \(\abs{x},\abs{y}\)$
    \State $\abs{n}=a-b$
    \If{$\abs{x}=a$}
    \State $\sgn n=\sgn a$
    \Else
    \State $\sgn n=\sgn b$
    \EndIf
    \EndIf
  \end{algorithmic}
\end{algorithm}

Tuttavia, per un calcolatore un tal numero di operazioni renderebbe il calcolo proibitivamente lento. Per questo, si introduce una nuova notazione, più efficiente di quella di modulo e segno, incentrata sull'agilità dell'elaborazione delle informazioni.

Ovviamente, la finestra dei valori che si vuole rappresentare si estende anche al negativo, per cui se si vuole rappresentare il $17$, si vuole rappresentare tutti i valori da $-17$ a $17$.

Si descrivono le possibili combinazioni avendo a disposizione 4 bit:
\begin{center}
  \begin{tblr}{c|c||c|c}
    $0000_2$ & $0_{10}$ & $1000_2$ & $-0_{10}$ \\
    $0001_2$ & $1_{10}$ & $1001_2$ & $-1_{10}$ \\
    $0010_2$ & $2_{10}$ & $1010_2$ & $-2_{10}$ \\
    $0011_2$ & $3_{10}$ & $1011_2$ & $-3_{10}$ \\ \hline
    $0100_2$ & $4_{10}$ & $1100_2$ & $-4_{10}$ \\
    $0101_2$ & $5_{10}$ & $1101_2$ & $-5_{10}$ \\
    $0110_2$ & $6_{10}$ & $1110_2$ & $-6_{10}$ \\
    $0111_2$ & $7_{10}$ & $1111_2$ & $-7_{10}$ \\
  \end{tblr}
\end{center}

Si nota subito che lo 0 ha una codifica ridondante:
$$0000_{2\text{MS}}=1000_{2\text{MS}}=0_{10}$$

Si pone l'attenzione sull'identità:
$$x+\(-x\)=0$$

Volendo rispettare l'identità con un valore $x=3_{10}$ si ha:

\begin{center}
  \begin{tblr}{ccccc}
    $^10$ & $^10$ & $^11$ & $1$ & $+$ \\
    $1$   & $1$   & $0$   & $1$ & $=$ \\
    \hline
    $0$   & $0$   & $0$   & $0$       
  \end{tblr}
\end{center}

quindi risulta che $1101_2$ è la codifica di $0011_2$ al negativo. Da questo ragionamento, esteso agli altri numeri risulta che:
\begin{center}
  \begin{tblr}{c|c||c|c}
    $0000_2$ & $0_{10}$ & $1000_2$ & $\pm8_{10}$ \\
    $0001_2$ & $1_{10}$ & $1001_2$ & $-7_{10}$   \\
    $0010_2$ & $2_{10}$ & $1010_2$ & $-6_{10}$   \\
    $0011_2$ & $3_{10}$ & $1011_2$ & $-5_{10}$   \\ \hline
    $0100_2$ & $4_{10}$ & $1100_2$ & $-4_{10}$   \\
    $0101_2$ & $5_{10}$ & $1101_2$ & $-3_{10}$   \\
    $0110_2$ & $6_{10}$ & $1110_2$ & $-2_{10}$   \\
    $0111_2$ & $7_{10}$ & $1111_2$ & $-1_{10}$   \\
  \end{tblr}
\end{center}

Esiste anche un metodo algebrico per trovare l'opposto:
$$-x=\sim x+1$$
oppure un metodo grafico:
\emph{tutti i bit che dall'LSB al primo bit 1 rimangono invariati, mentre tutti gli altri si cambiano}.

L'efficacia di questa notazione sta nella facilità con cui si eseguono le operazioni:
$$2_{10}+4_{10}=0010_2+0100_2=0110_2=6_{10}$$
$$-3_{10}+\(-1_{10}\)=1101_2+1111_2=\cancel{1}1100_2=-4_{10}$$
$$5_{10}+\(-3_{10}\)=0101_2+1101_2=\cancel{1}0010_2=2_{10}$$

Questa notazione prende il nome di complemento in base 2.
Si chiama complemento in quanto i valori negativi si completano a quelli positivi.

Per passare dalla base 10 alla base 2 in complemento a 2, si effettuano i seguenti passaggi:
\begin{algorithm}[H]
  \caption{Conversione da 10MS a 2C2}\label{algo:10to2c2}
  \begin{algorithmic}[1]
    \State $n=\abs{x}$
    \If{$x<0$}
    \State $n=-n$
    \EndIf
  \end{algorithmic}
\end{algorithm}
\begin{example}
  $$x=-13_{10}$$
  $$\abs{x}=13_{10}=1101_2=01101_{2\text{MS}}\equiv 01101_{2\text{C}2}$$
  $$x=-\abs{x}=10011_{2\text{C}2}$$
\end{example}

Si presentano casi ambigui in cui apparentemente sembra che l'aritmetica non funzioni:
$$5_{10}+3_{10}=0101_2+0011_2=1000_2=-8_{10}$$
$$-2_{10}+\(-6_{10}\)=1110_2+1010_2=\cancel{1}1000_2=-8_{10}$$
Si nota che tutti i numeri positivi cominciano con 0 mentre quelli negativi cominciano con 1.
Convenzionalmente, in virtù di quanto appena detto, la combinazione $1000_2$ assume il valore $-8_{10}$.

Usando la notazione modulo e segno il range dei valori ammissibili è $\interval{-7}{7}$, mentre in complemento a 2 il range è $\interval{-8}{7}$.

Considerando che la dimensione dell'informazione è sempre fissa e determinata, ci si ritrova in casi particolari detti di \textbf{overflow} (\emph{traboccamento}):
$$6_{10}+4_{10}=0110_2+0100_2=1010_2=-6_{10}$$
L'overflow è facilmente risolvibile con un aumento dello spazio a disposizione:
$$6_{10}+4_{10}=00110_2+00100_2=01010_2=10_{10}$$
Si ha overflow quando due valori concordi generano un valore discorde dai primi due.
Quando gli operandi sono discordi è impossibile generare overflow.

\paragraph*{Osservazione} L'estensione di un valore in complemento a 2 si esegue ripetendo il MSB quante volte se ne ha bisogno:

\begin{example}
  $$5_{10}=0101_{2\text{C}2}=000000101_{2\text{C}2}$$
  $$-5_{10}=1011_{2\text{C}2}=111111011_{2\text{C}2}$$
\end{example}

\begin{example}
  $$x=+12_{10\text{MS}}=01100_{2\text{C}2}$$
  $$y=-3_{10\text{MS}}=101_{2\text{C}2}=11101_{2\text{C}2}$$
  $$x+y=01100_2+11101_2=\cancel{1}01001_2$$
  $$x-y=x+\(-y\)=01100_2+00011_2=01111_2$$
\end{example}

% lezione del 15/09/2022

Per passare dalla base 2 in complemento a 2 alla base 10, si possono verificare due scenari:
\begin{enumerate}
  \item il numero è positivo, pertanto lo si converte usando la formula dei pesi;
  \item il numero è negativo: in questo caso si calcola l'opposto, lo si converte e gli si cambia il segno.
\end{enumerate}
\begin{example}
  $$010110_{2\text{C}2}=22_{10\text{MS}}$$
  $$1011101_{2\text{C}2}=-35_{10\text{MS}}$$
\end{example}

\section{Rappresentazione esadecimale}
Il problema principale con la rappresentazione binaria è l'enorme spazio richiesto. Inoltre, in virtù dell'uso sui calcolatori, la base da cercare deve essere una potenza di 2.
Per anni è stata utilizzata la base 8, ma ben presto è stata resa obsoleta, in favore della base 16.

Il codice esadecimale si crea associando le cifre esadecimali e combinazioni di 4 bit del codice binario.
\begin{center}
  \begin{tblr}{c|c|c}
    \textbf{esadecimale} & \textbf{binario} & \textbf{decimale} \\\hline
    0                    & 0000             & 0                 \\
    1                    & 0001             & 1                 \\
    2                    & 0010             & 2                 \\
    3                    & 0011             & 3                 \\
    4                    & 0100             & 4                 \\
    5                    & 0101             & 5                 \\
    6                    & 0110             & 6                 \\
    7                    & 0111             & 7                 \\
    8                    & 1000             & 8                 \\
    9                    & 1001             & 9                 \\
    A                    & 1010             & 10                \\
    B                    & 1011             & 11                \\
    C                    & 1100             & 12                \\
    D                    & 1101             & 13                \\
    E                    & 1110             & 14                \\
    F                    & 1111             & 15                \\
  \end{tblr}
\end{center}

Per effettuare la conversione dalla base 2 alla base 16 basta considerare ogni quadrupla e convertirla in loco:
\begin{center}
  \begin{tblr}{cccc}
    0101 & 0101 & 1101 & 1010 \\ \hline
    5    & 5    & D    & A    
  \end{tblr}
\end{center}
Banalmente, la stessa cosa avviene per il processo inverso.

Al fine di rappresentare attraverso la notazione modulo e segno un numero esadecimale, gli si prepone il segno, facendo la conversione dal binario.

\section{Numeri razionali}

Si vuole rappresentare i valori espressi da:
$$\frac{m}{n},\; m\in\mathbb{Z},n\in\mathbb{N}/\left\{ 0 \right\}$$

Nella notazione decimale si usa scrivere prima la parte intera, un separatore decimale e poi la parte frazionaria.
In base 2 si può fare la stessa cosa:
$$101.01_2=2^2+2^0+2^{-2}=5.25_{10}$$
Avendo a disposizione una certa quantità bit, si sceglie la parte di essi che contiene la parte frazionaria e quella che contiene la parte intera. 

\begin{example}
  Si considerano 3 bit per la parte intera e 2 per quella frazionaria. Si ha che il range massimo esprimibile è $\interval{0}{7.75}$, con salti di $0.25$.
  
  Tuttavia, se si vuole rappresentare il valore $6.3$, non si può. Al limite, si può esprimere una sua approssimazione, che in questo caso è $6.25$.
\end{example}

Il fatto che non si possa rappresentare un valore con \textbf{precisione} non è dovuto ad una cattiva gestione dei bit, ma bensì al loro stato di finitezza.

La notazione utilizza si chiama a virgola fissa: viene stabilita la dimensione a priori sia per la parte intera sia per la parte frazionaria.
In una notazione del genere l'\textbf{errore assoluto} $\epsilon_A$ è costante: si sbaglia sempre della stessa quantità.

Esiste, tuttavia, una notazione alternativa che consente di variare l'errore assoluto e mantenere costante l'errore relativo: la notazione in \textbf{virgola mobile} (\emph{floating point} in inglese).
L'errore relativo $$\epsilon_R=\frac{\epsilon_A}{\text{valore}}$$

Per rappresentare un numero razionale in base 2, si procede come al solito per la parte intera, ovvero si divide per 2 segnando il resto, ma al contrario per la parte frazionaria, ovvero si moltiplica per 2 segnando l'unità. Un'altra differenza fondamentale è il verso di lettura delle cifre: mentre per la parte intera si procede dal basso verso l'alto, per la parte frazionaria si procede dall'alto verso il basso.

\begin{example}
  Si vuole rappresentare il numero $13.75$.
  \begin{center}
    \begin{minipage}{0.2\linewidth}
      \begin{center}
        \begin{tblr}{c|c}
          13 & 1 \\
          6  & 0 \\
          3  & 1 \\
          1  & 1 \\
          0  &   
        \end{tblr}
      \end{center}
    \end{minipage}
    \begin{minipage}{0.2\linewidth}
      \begin{center}
        \begin{tblr}{c|c}
          0.75 &   \\
          1.5  & 1 \\
          1.0  & 1 \\
          0    &   \\
        \end{tblr}
      \end{center}
    \end{minipage}
  \end{center}
  $$13.75_{10}=1101.11_2$$
\end{example}

\begin{example}
  Si vuole rappresentare il numero $7.32$.
  \begin{center}
    \begin{minipage}{0.2\linewidth}
      \begin{center}
        \begin{tblr}{c|c}
          7 & 1 \\
          3 & 1 \\
          1 & 1 \\
          0 &   
        \end{tblr}
      \end{center}
    \end{minipage}
    \begin{minipage}{0.2\linewidth}
      \begin{center}
        \begin{tblr}{c|c}
          0.32     &          \\
          0.64     & 0        \\
          1.28     & 1        \\
          0.56     & 0        \\
          1.12     & 1        \\
          0.24     & 0        \\
          0.48     & 0        \\
          0.96     & 0        \\
          1.92     & 1        \\
          1.84     & 1        \\
          $\cdots$ & $\cdots$ 
        \end{tblr}
      \end{center}
    \end{minipage}
  \end{center}
  $$7.32_{10}=111.010100011...$$
\end{example}

Se si considerano i razionali da una diversa prospettiva, si nota che:
$$13.75_{10}=1.375\cdot 10^1$$
$$7.32_{10}=7.32\cdot 10^0$$
Analogamente:
$$1101.11_2=1.10111\cdot2^3$$
$$111.010100011_2=1.11010100011\cdot2^2$$

La notazione in virgola mobile, quindi, ha:
\begin{itemize}
  \item un bit per il \textbf{segno};
  \item una parte per l'\textbf{esponente}, a cui viene aggiunto un numero tale che il più piccolo esponente possibile sia 0;
  \item una parte per la \textbf{mantissa}, ossia la parte frazionaria della notazione scientifica\footnote{si prende la parte frazionaria in quanto la parte intera sarà sempre 1, quindi è inutile sprecare un bit}.
\end{itemize}
Lo standard che regola la notazione in virgola mobile è lo \textbf{IEEE 754}. Lo standard prevede 3 versioni:
\begin{itemize}
  \item 32 bit: 1 per il segno, 8 per l'esponente, 23 per la mantissa;
  \item 64 bit: 1 per il segno, 11 per l'esponente, 52 per la mantissa;
  \item 128 bit: 1 per il segno, 15 per l'esponente, 112 per la mantissa.
\end{itemize}

Il valore di un numero espresso attraverso lo standard IEEE 754 è dato da:
$$\text{valore}=\(-1\)^S\(1+M\)\cdot2^E$$
dove $S$ è il segno, $M$ è la mantissa e $E$ è l'esponente

\paragraph*{Nota bene}
La costante che bisogna aggiungere all'esponente si chiama ``eccesso'' ed è:
\begin{itemize}
  \item 127 per la singola precisione;
  \item 1023 per la doppia precisione;
  \item 16383 per la quadrupla precisione.
\end{itemize}

\begin{example}
  $$+5.65_{10\text{MS}}$$
  $$+101.10{1001}_{2\text{MS}}=1.01101001\cdot2^2=01000000101101001100110011001100$$
  $$-0.028_{10\text{MS}}$$
  $$-0.00000111001010_2=-1.1100101\cdot2^{-6}=101001111110101...$$
\end{example}

Lo standard, tuttavia, presenta delle falle che vengono compensate dall'introduzione di combinazioni di bit speciali che, proprio per questo, vengono dette \textbf{denormarlizzate}.

\paragraph*{Forma denormarlizzata generale}
Quando tutti i bit dell'esponente sono 0, la mantissa non è sommata ad 1 ma bensì a 0.

\paragraph*{Zero}
Quando tutti i bit (indifferentemente dal primo) sono 0, il valore è 0.

\paragraph*{Infinito}
Quando tutti i bit dell'esponente sono 1 e quelli della mantissa sono tutti 0, il valore è infinito, che può essere sia negativo che positivo.

\paragraph*{NaN}
Quando tutti i bit dell'esponente sono 1 e almeno uno di quelli della mantissa è 1, il valore non è un numero (\textbf{NaN}, \emph{Not a Number}).

% lezione del 16/09/2022

\section{L'informazione non numerica}

Per ovvi motivi, non si può rappresentare solo l'informazione numerica. L'obiettivo è, quindi, cercare di definire una codifica per ogni possibile carattere rappresentabile. Inizialmente si hanno:
\begin{itemize}
  \item alfabeto base (a...z, A...Z);
  \item caratteri numerici (0...9);
  \item caratteri di interpunzione (.,;:);
  \item caratteri speciali.
\end{itemize}

Tutti questi elementi sono circa 120, per cui sono necessari 7 bit.
Per la rappresentazione di tali simboli si utilizza il \textbf{codice ASCII} (\emph{American Standard Code for Information Interchange}): il codice dispone di 7 bit, per cui ha 128 combinazioni.
Per far fronte alle necessità sorte nel tempo, si è aggiunto un bit al codice ASCII, creando il codice \textbf{ASCII esteso}: esso dispone di 8 bit, e comprende caratteri nazionali, simboli e cornici.


\section{Successioni}

\begin{definition}[Successione]
  Una successione in un dato insieme $X$ è una funzione $f:D\to X$ con dominio $D$ numerabile (il più delle volte $D\subseteq \mathbb{N}$). La si indica con $\left\{ x_n \right\}_{n\in D}\subseteq X$.
\end{definition}

\subsection{Limite}

\begin{definition}[Limite di successione]
  Una successione $\left\{ x_n \right\}_{n\in\mathbb{N}}\subseteq\mathbb{Q}$ converge al numero $\l\in\mathbb{Q}$, in simboli:
  $$\lim_{n\to+\infty} x_n=\l\ \mathrm{oppure}\ x_n\xrightarrow{n\to+\infty}\l$$
  se:
  $$\forall\epsilon>0\ \exists\ N\(\epsilon\)\in\mathbb{N}:n\ge N\(\epsilon\)\impl \abs{x_n-\l}<\epsilon$$
  cioè:
  $$\forall\epsilon>0\ \exists\ N\(\epsilon\)\in\mathbb{N}:n\ge N\(\epsilon\)\impl \l-\epsilon<x_n<\l+\epsilon$$
  cioè:
  $$\forall\epsilon>0\ \exists\ N\(\epsilon\)\in\mathbb{N}:n\ge N\(\epsilon\)\impl x_n\in\ointv{\l-\epsilon}{\l+\epsilon}$$
  \begin{center}
    \begin{tikzpicture}[scale=1.5]
      \draw[-stealth] (0,0) -- (6,0) node [right] {$\mathbb{Q}$};
      \draw (3,0) node [label=below:$\l$] {};
      \draw (3,0) circle (1.5pt);
      \draw (2.5,0) node [label=$x_n$] {};
      \fill (2.5,0) circle (1.5pt);
      \draw (2,0) node [label=below:$\l-\epsilon$] {$\vert$};
      \draw (4,0) node [label=below:$\l+\epsilon$] {$\vert$};
    \end{tikzpicture}
  \end{center}
\end{definition}

\begin{example}
  $$\lim_{n\to+\infty}\frac{1}{n}=0$$
  \begin{center}
    \begin{tikzpicture}[scale=5]
      \draw[-stealth] (-0.3,0) -- (1.3,0) node [right] {$\mathbb{Q}$};
      \draw (0,0) node [label=below:$0$] {$\vert$};
      \foreach \n in {1,...,5} {
          \draw (1/\n,0) node {$\vert$};
          \node[below=8pt] at (1/\n,0) {$\frac{1}{\n}$};
        }
    \end{tikzpicture}
  \end{center}
  Fissato $\epsilon>0$, si ha:
  $$0-\epsilon<\frac{1}{n}<0+\epsilon$$
  $$-\epsilon<\frac{1}{n}<\epsilon\iff \frac{1}{n}<\epsilon$$
  $$n>\frac{1}{\epsilon}$$
  Si sceglie $N\(\epsilon\)$ come il più piccolo numero naturale maggiore di $\nicefrac{1}{\epsilon}$, in modo tale da avere:
  $$n\ge N\(\epsilon\)>\frac{1}{\epsilon}$$
  e, quindi, verificare la definizione di limite.
\end{example}

\begin{example}
  $$\lim_{n\to+\infty}n=+\infty$$
  \begin{center}
    \begin{tikzpicture}[scale=1.5]
      \draw[-stealth] (0,0) -- (5,0) node [right] {$\mathbb{Q}$};
      \draw (1,0) node [label=below:$a_{n}$] {};
      \fill (1,0) circle (1.5pt);
      \draw (4,0) node [label=below:$a_{n+1}$] {};
      \fill (4,0) circle (1.5pt);
      \draw (2,0) node [label=below:$\l$] {};
      \draw (2,0) circle (1.5pt);
    \end{tikzpicture}
  \end{center}
  Si dice che la successione tende a $+\infty$, ossia il limite esiste ma non è un numero $\in\mathbb{Q}$.
\end{example}

\subsection{Proprietà}

\begin{lemma}
  Se $\exists \lim_{n\to+\infty}a_n\in\mathbb{Q}$, allora tale limite è unico.
\end{lemma}
\begin{proof}
  Siano $a_n\xrightarrow{n\to+\infty}\l_1$ e $a_n\xrightarrow{n\to+\infty}\l_2$. Si ha:
  $$\forall\epsilon>0\ \exists\ N_1\(\epsilon\)\in\mathbb{N}:n\ge N_1\(\epsilon\)\impl \abs{a_n-\l_1}<\epsilon$$
  $$\forall\epsilon>0\ \exists\ N_2\(\epsilon\)\in\mathbb{N}:n\ge N_2\(\epsilon\)\impl \abs{a_n-\l_2}<\epsilon$$
  Fissato $\epsilon>0$, se $n\ge N\(\epsilon\)\walrus\max\left\{ N_1\(\epsilon\),N_2\(\epsilon\) \right\}$, allora $n\ge N_1\(\epsilon\)\wedge n\ge N_2\(\epsilon\)$ e:
  $$0\le \abs{\l_1-\l_2}<\abs{\l_1-a_n+a_n-\l_2}\le\abs{\l_1-a_n}+\abs{a_n-\l_2}\le\epsilon+\epsilon=2\epsilon$$
  $$0\le\abs{\l_1-\l_2}\le2\epsilon\impl \abs{\l_1-\l_2}=0\iff \l_1=\l_2$$
\end{proof}
\begin{lemma}
  Se $\exists \lim_{n\to+\infty}a_n\in\mathbb{Q}$, allora $\left\{ a_n \right\}$ è limitata.
\end{lemma}
\begin{proof}
  $$\l\walrus\lim_{n\to+\infty}a_n$$
  $$\forall\epsilon>0\ \exists\ N\(\epsilon\)\in\mathbb{N}:n\ge N\(\epsilon\)\impl a_n\in\ointv{\l-\epsilon}{\l+\epsilon}$$
  Allora $\left\{ a_n \right\}_{n\ge N\(\epsilon\)}\subset \ointv{\l-\epsilon}{\l+\epsilon}$, ed è quindi limitata, e $\left\{ a_n \right\}_{n< N\(\epsilon\)}$ è limitata in quanto insieme finito. Quindi:
  $$\left\{ a_n \right\}_{n\ge0}=\left\{ a_n \right\}_{n< N\(\epsilon\)}\cup \left\{ a_n \right\}_{n\ge N\(\epsilon\)}$$
  è limitata, in poiché unione di insiemi limitati.
\end{proof}
\begin{lemma}
  $$
    \begin{cases}
      \exists\ \l_1\walrus\lim_{n\to+\infty}a_n \\
      \exists\ \l_2\walrus\lim_{n\to+\infty}b_n \\
    \end{cases}
    \impl
    \lim_{n\to+\infty}\(a_n+b_n\)=\lim_{n\to+\infty}a_n+\lim_{n\to+\infty}b_n
  $$
\end{lemma}
\begin{proof}
  Dalle ipotesi, si ha:
  $$0\le\abs{\(a_n+b_n\)-\(\l_1+\l_2\)}=\abs{\(a_n-\l_1\)-\(b_n-\l_2\)}\le\abs{a_n-\l_1}+\abs{b_n-\l_2}<\epsilon+\epsilon=2\epsilon$$
  Dato che si può scegliere $\epsilon$ arbitrariamente piccolo, si ha:
  $$\abs{\(a_n+b_n\)-\(\l_1+\l_2\)}=0\iff \(a_n+b_n\)-\(\l_1+\l_2\)=0\iff a_n+b_n=\l_1+\l_2$$
\end{proof}
\begin{lemma}
  $$
    \begin{cases}
      \exists\ \l_1\walrus\lim_{n\to+\infty}a_n \\
      \exists\ \l_2\walrus\lim_{n\to+\infty}b_n \\
    \end{cases}
    \impl
    \lim_{n\to+\infty}\(a_n\cdot b_n\)=\(\lim_{n\to+\infty}a_n\)\cdot\(\lim_{n\to+\infty}b_n\)
  $$
\end{lemma}
\begin{proof}
  Dalle ipotesi, si ha:
  \begin{align*}
    0 & \le \abs{a_nb_n-\l_1\l_2}                         \\
      & =\abs{a_nb_n-\l_1b_n+\l_1b_n-\l_1\l_2}            \\
      & =\abs{b_n\(a_n-\l_1\)+\l_1\(b_n-\l_2\)}           \\
      & \le\abs{b_n\(a_n-\l_1\)}+\abs{\l_1\(b_n-\l_2\)}   \\
      & =\abs{b_n}\abs{a_n-\l_1}+\abs{\l_1}\abs{b_n-\l_2} 
  \end{align*}
  Dato che $\left\{ b_n \right\}$ converge, allora è anche limitata, per cui:
  $$\exists\ M\ge0:\abs{b_n}<M\ \forall n\ge0$$
  Da cui:
  $$0\le \abs{a_nb_n-\l_1\l_2} \le M\abs{a_n-\l_1}+\abs{\l_1}\abs{b_n-\l_2}<M\epsilon+\abs{\l_1}\epsilon=\epsilon\(M+\abs{\l_1}\)$$
  Dato che si può scegliere $\epsilon$ arbitrariamente piccolo, si ha:
  $$\abs{a_nb_n-\l_1\l_2}=0\iff a_nb_n-\l_1\l_2=0\iff a_nb_n=\l_1\l_2$$
\end{proof}
\begin{lemma}
  $$\lim_{n\to+\infty}\frac{a_n}{b_n}=\frac{\lim_{n\to+\infty}a_n}{\lim_{n\to+\infty}b_n}$$
\end{lemma}

\begin{lemma}[Convergenza delle successioni costanti]
  Sia $\left\{ c \right\}$, con $c\in\mathbb{Q}$ fissato.
  $$\lim_{n\to+\infty} c=c$$
\end{lemma}
\begin{proof}
  $$\forall \epsilon>0\ \abs{a_n-\l}=\abs{c-c}=0<\epsilon$$
\end{proof}
\begin{observation}
  Se $\exists\ \l\walrus\lim_{n\to+\infty}a_n$, si verifica il seguente fenomeno:
  \begin{center}
    \begin{tikzpicture}[scale=1.5]
      \draw[-stealth] (0,0) -- (6,0) node [right] {$\mathbb{Q}$};
      \draw[dashdotted] (3,-.3) node [below] {$N\(\epsilon\)$} -- (3,3.3);
      \draw[] (0,1.5) node [left] {$\l$} -- (6,1.5);
      \draw[dashed] (0,0.9) node [left] {$\l-\epsilon$} -- (6,0.9);
      \draw[dashed] (0,2.1) node [left] {$\l+\epsilon$} -- (6,2.1);
      \foreach \i in {1,...,8} {
          \pgfmathrandominteger{\x}{1}{15}
          \fill (\i / 3,\x / 5) circle (1.5pt);
        }
      \foreach \i in {6,...,11} {
          \pgfmathrandominteger{\x}{1}{5}
          \fill (\i / 2,\x / 5 + 1) circle (1.5pt);
        }
    \end{tikzpicture}
  \end{center}
\end{observation}

\begin{example}
  \begin{align*}
    \lim_{n\to+\infty} \frac{n}{n+1} & =\lim_{n\to+\infty}\frac{n+1-1}{n+1}                               \\
                                     & =\lim_{n\to+\infty}\frac{n+1}{n+1}-\lim_{n\to+\infty}\frac{1}{n+1} \\
                                     & =\lim_{n\to+\infty}1-\lim_{n\to+\infty}\frac{1}{n+1}               \\
                                     & =1-0=1                                                             
  \end{align*}
\end{example}

\begin{theorem}[Permanenza del segno]
  $$a_n\ge0\wedge\exists\ \l\walrus\lim_{n\to+\infty}a_n\impl\l\ge0$$
\end{theorem}
\begin{proof}
  Per assurdo, sia $\l<0$. Sia $\epsilon\walrus-\nicefrac{\l}{2}>0$. Poiché esiste il limite, si ha:
  $$\exists\ N\(\epsilon\)\ge0:n\ge N\(\epsilon\)\impl \l-\epsilon<a_n<\l+\epsilon$$
  da cui:
  $$\l+\nicefrac{\l}{2}<a_n<\l-\nicefrac{\l}{2}$$
  $$\nicefrac{3}{2}\l<a_n<\nicefrac{1}{2}\l<0$$
  per cui si ha l'assurdo.
\end{proof}

\begin{theorem}[Monotonia]
  $$
    \begin{cases}
      a_n\le b_n                                \\
      \exists\ \l_1\walrus\lim_{n\to+\infty}a_n \\
      \exists\ \l_2\walrus\lim_{n\to+\infty}b_n \\
    \end{cases}
    \impl
    \l_1\le\l_2
  $$
\end{theorem}

\begin{theorem}
  $$\lim_{n\to+\infty}a_n=0\wedge \left\{ b_n \right\}\subseteq\ointv{a}{b}\impl\exists\ \lim_{n\to+\infty}a_nb_n=0$$  
\end{theorem}

\begin{definition}
  $$\lim_{n\to-\infty}a_n=\l\iff \forall\epsilon>0\ \exists\ N\(\epsilon\)\le0:n\le N\(\epsilon\)\impl\abs{a_n-\l}<\epsilon$$
\end{definition}

\begin{definition}
  $$\lim_{n\to+\infty}a_n=+\infty\iff \forall M\in\mathbb{Q}\ \exists\ N\(M\)\ge0:n\ge N\(M\)\impl a_n\ge M$$
\end{definition}

Tutte le proprietà dei limiti continuano a valere anche nel caso in cui $\left\{ a_n \right\}$ e $\left\{ b_n \right\}$ convergano a $\pm\infty$, tranne nei casi seguenti, detti di indecisione:
\begin{itemize}
  \item $+\infty-\infty$
  \item $0\cdot \infty$
  \item $\nicefrac{\infty}{\infty}$
  \item $\nicefrac{0}{0}$
\end{itemize}
Per convenzione $\infty\cdot\infty=\infty$.

\begin{example}
  $$a_n\walrus\(1+\frac{1}{n}\)^n$$
  \begin{align*}
    a_n & =\(1+\frac{1}{n}\)^n                                                               \\
        & =\sum_{k=0}^n\binom{n}{k}1^{n-k}\(\frac{1}{n}\)^k                                  \\
        & =\sum_{k=0}^n\frac{n!}{k!\(n-k\)!}\frac{1}{n^k}                                    \\
        & =\frac{n!}{0!n!n^0}+\frac{n!}{1!\(n-1\)!n^1}+\sum_{k=2}^n\frac{n!}{n^kk!\(n-k\)!}  \\
        & =2+\sum_{k=2}^n\frac{n\(n-1\)\cdots\(n-k+1\)\(n-k\)!}{n^kk!\(n-k\)!}               \\
        & =2+\sum_{k=2}^n\frac{n\(n-1\)\cdots\(n-k+1\)}{n^kk!}                               \\
        & =2+\sum_{k=2}^n\frac{1}{k!}\cdot\frac{n}{n}\cdot\frac{n-1}{n}\cdots\frac{n-k+1}{n} \\
        & =2+\sum_{k=2}^n\frac{1}{k!}\(1-\frac{1}{n}\)\cdots\(1-\frac{k-1}{n}\)              
  \end{align*}
  \begin{align*}
    a_{n+1} & =2+\sum_{k=2}^{n+1}\frac{1}{k!}\(1-\frac{1}{n+1}\)\cdots\(1-\frac{k-1}{n+1}\) \\
            & \ge2+\sum_{k=2}^{n}\frac{1}{k!}\(1-\frac{1}{n+1}\)\cdots\(1-\frac{k-1}{n+1}\) \\
            & \ge2+\sum_{k=2}^{n}\frac{1}{k!}\(1-\frac{1}{n}\)\cdots\(1-\frac{k-1}{n}\)     \\
            & =a_n                                                                          
  \end{align*}
  $$a_{n+1}\ge a_n$$
  Essendo crescente, $\left\{ a_n \right\}$ è inferiormente limitata:
  $$a_n\ge a_1=\(1+\frac{1}{1}\)^1=2$$
  Inoltre:
  \begin{align*}
    a_n & <2+\sum_{k=2}^n\frac{1}{k!}                           \\
        & \le 2+\sum_{k=2}^n\frac{1}{2^{k-1}}                   \\
        & =2+\sum_{k=0}^n\frac{1}{2^{k-1}}-2-1                  \\
        & =\sum_{k=0}^n\frac{1}{2^{k-1}}-1                      \\
        & =\sum_{k=0}^n\frac{2}{2^{k}}                          \\
        & =2\sum_{k=0}^n\frac{1}{2^k}-1                         \\
        & =2\(\frac{1-\(\frac{1}{2}\)^{n+1}}{1-\frac{1}{2}}\)-1 \\
        & =4\(1-\frac{1}{2^{n+1}}\)-1                           
  \end{align*}
  Passando al limite:
  $$
    \lim_na_n<\lim_n 4\(1-\frac{1}{2^{n+1}}\)-1 = 4-1=3
  $$
  In conclusione, $\forall n\ge1\ a_n\in\rintv{2}{3}$.
\end{example}

\subsection{Costruzione di $\reals$}

\begin{theorem}[Proprietà di Cauchy]
  Se $\exists\ \lim_{n}a_n\in\mathbb{Q}$, allora:
  $$\forall\epsilon>0\ \exists\ N\(\epsilon\)\ge0:n,m\ge N\(\epsilon\)\impl \abs{a_n-a_m}<\epsilon$$
\end{theorem}
\begin{proof}
  Sia $\l\walrus\lim_na_n$. Allora:
  $$\forall\epsilon>0\ \exists\ N\(\epsilon\)\ge0:n\ge N\(\epsilon\)\impl \abs{x_n-\l}<\epsilon$$
  Da cui, se $n,m\ge N\(\epsilon\)$:
  $$0\le\abs{a_n-a_m}=\abs{\(a_n-\l\)+\(\l-a_m\)}\le\abs{a_n-\l}+\abs{a_m-\l}\le 2\epsilon$$
  Perciò:
  $$\abs{a_n-a_m}<2\epsilon$$
\end{proof}


\begin{definition}[Relazione di equivalenza]
  Sia $X$ un insieme. Una \textbf{relazione di equivalenza} su $X$ è un insieme $R\subset X\times X$, avente le seguenti proprietà:
  \begin{itemize}
    \item riflessività: $\(x,x\)\in R\ \forall x\in X$ oppure $x\stackrel{R}{\sim}x$
    \item simmetria: $\(x,y\)\in R\impl \(y,x\)\in R$ oppure $x\stackrel{R}{\sim}y\impl y\stackrel{R}{\sim}x$
    \item transitività: $\(x,y\),\(y,z\)\in R\impl \(x,z\)\in R$ oppure $x\stackrel{R}{\sim}y\wedge y\stackrel{R}{\sim}z\impl x\stackrel{R}{\sim}z$
  \end{itemize}
\end{definition}

\begin{definition}[Insieme quoziente]
  Sia $\left[ x \right]_R\walrus\left\{ y\in X:x\stackrel{R}{\sim}y \right\}\subset X$. L'\textbf{insieme quozionte} è definito come:
  $$\nicefrac{X}{R}\walrus\left\{ \left[ x \right]_R:x\in X \right\}$$
\end{definition}

Sia $X\walrus\left\{ \text{successioni di Cauchy in }\mathbb{Q} \right\}$. Le successioni $\left\{ a_n \right\},\left\{ b_n \right\}$ sono in relazione $R$ se:
$$\lim_{n\to+\infty}\(a_n-b_n\)=0$$
Si dimostra che che $R$ è una relazione di equivalenza.

\begin{definition}[Numeri reali]
  L'insieme dei \textbf{numeri reali} è definito come:
  $$\reals\walrus \nicefrac{X}{R}$$
\end{definition}

Si definiscono le seguenti operazioni su $\reals$.

\begin{definition}[Somma in $\reals$]
  Siano $\left[ \left\{ a_n \right\} \right]_R,\left[ \left\{ b_n \right\} \right]_R\in\reals$. La loro somma è:
  $$\left[ \left\{ a_n \right\} \right]_R+\left[ \left\{ b_n \right\} \right]_R\walrus\left[ \left\{ a_n+b_n \right\} \right]_R$$
\end{definition}

\begin{definition}[Prodotto in $\reals$]
  Siano $\left[ \left\{ a_n \right\} \right]_R,\left[ \left\{ b_n \right\} \right]_R\in\reals$. Il loro prodotto è:
  $$\left[ \left\{ a_n \right\} \right]_R\cdot\left[ \left\{ b_n \right\} \right]_R\walrus\left[ \left\{ a_n\cdot b_n \right\} \right]_R$$
\end{definition}

\begin{observation}
  Queste operazioni godono delle ``solite'' proprietà: associatività, distributività, elementi neutri, etc...
\end{observation}

\begin{definition}[Positività]
  Si dirà che $\left[ \left\{ a_n \right\} \right]_R\in\reals$ è positivo se:
  $$\exists\ q>0\in\mathbb{Q},\exists\ N\ge0:n\ge N\impl a_n\ge q$$
\end{definition}

\begin{definition}[Ordine in $\reals$]
  Siano $\left[ \left\{ a_n \right\} \right]_R,\left[ \left\{ b_n \right\} \right]_R\in\reals$. Si stabilisce l'ordine:
  $$\left[ \left\{ a_n \right\} \right]_R<\left[ \left\{ b_n \right\} \right]_R\iff \left[ \left\{ b_n \right\} \right]_R-\left[ \left\{ a_n \right\} \right]_R>0$$
\end{definition}

\begin{observation}
  Si può ``immergere'' $\mathbb{Q}$ in $\reals$, poiché esiste una funzione iniettiva $j:\mathbb{Q}\to\reals$:
  $$j\(x\)\walrus\left[ \left\{ x \right\} \right]_R\in\reals$$
  $$j\(x\)+j\(y\)=j\(x+y\)$$
  $$j\(x\)\cdot j\(y\)=j\(x\cdot y\)$$
\end{observation}

\begin{definition}[Intervalli in $\reals$]
  Siano $a,b\in\reals:a<b$:
  \begin{itemize}
    \item l'insieme $\ointv{a}{b}\walrus\left\{ x\in\reals:a<x<b \right\}$ è detto \textbf{intervallo aperto};
    \item l'insieme $\intv{a}{b}\walrus\left\{ x\in\reals:a\le x\le b \right\}$ è detto \textbf{intervallo chiuso};
    \item l'insieme $\lintv{a}{b}\walrus\left\{ x\in\reals:a<x\le b \right\}$ è detto \textbf{intervallo semiaperto a sinistra};
    \item l'insieme $\rintv{a}{b}\walrus\left\{ x\in\reals:a\le x<b \right\}$ è detto \textbf{intervallo semiaperto a destra}.
  \end{itemize}
\end{definition}

\begin{definition}[Limite di successione]
  Una successione $\left\{ x_n \right\}\subseteq\reals$ converge al numero $\l\in\reals$, in simboli:
  $$\lim_{n\to+\infty} x_n=\l\ \mathrm{oppure}\ x_n\xrightarrow{n\to+\infty}\l$$
  se:
  $$\forall\epsilon>0\ \exists\ N\(\epsilon\)\in\mathbb{N}:n\ge N\(\epsilon\)\impl \abs{x_n-\l}<\epsilon$$
\end{definition}

\begin{observation}
  Tutte le proprietà dei limiti di successioni razionali restano valide anche per limiti di successioni di numeri reali.
\end{observation}

\begin{theorem}[Teorema fondamentale di completezza di $\reals$]
  La condizione di Cauchy per successioni di numeri reali è necessaria e sufficiente per la convergenza in $\reals$.
\end{theorem}

\begin{theorem}[Convergenza delle successioni monotone limitate]
  Sia $\left\{ x_n \right\}\subset\reals$ una successione monotona crescente e superiormente limitata:
  $$x_n\le x_{n+1}$$
  $$\exists\ M\in\reals:x_n\le M$$
  Allora $\left\{ x_n \right\}\subset\reals$ è di Cauchy e quindi convergente:
  $$\exists\ \lim_{n\to+\infty}x_n\in\reals$$
\end{theorem}
\begin{proof}
  Per la completezza di $\reals$ è sufficiente dimostrare che $\left\{ x_n \right\}$ è di Cauchy, ossia:
  $$\forall\epsilon>0\ \exists\ N\(\epsilon\)\ge1:n,m\ge N\(\epsilon\)\impl \abs{x_n-x_m}<\epsilon$$
  Per assurdo, se $\left\{ x_n \right\}$ non è di Cauchy, allora:
  $$\exists\epsilon>0\ \forall N\(\epsilon\)\ge1:\exists\ n,m\ge N\(\epsilon\)\impl \abs{x_n-x_m}\ge\epsilon$$
  Senza ledere la generalità della tesi, si assume che $x_n>x_m$. In questo modo:
  $$x_n-x_m=\abs{x_n-x_m}\ge\epsilon\iff x_n\ge\epsilon+x_m$$
  $$N\(\epsilon\)=1\impl\exists\ n_1>m_1:x_{n_1}\ge\epsilon+x_{m_1}$$
  $$N\(\epsilon\)=n_1\impl \exists\ n_2>m_2\ge N\(\epsilon\) \ge n_1 >m_1:x_{n_2}\ge\epsilon+x_{m_2}\ge\epsilon+x_{n_1}\ge 2\epsilon+x_{m_1}$$
  $$N\(\epsilon\)=n_k>m_k\impl\exists\ n_{k+1}>m_{k+1}\ge N\(\epsilon\)\ge n_k>m_k$$
  Per la monotonia del limite, si ha:
  $$\lim_{k\to+\infty}x_{n_{k+1}}\ge \lim_{k\to+\infty}k\epsilon+x_{m_1}=\epsilon\(+\infty\)+x_{m_1}=+\infty$$
  Ma ciò contraddice l'ipotesi di limitatezza della successione, per cui $\left\{ x_n \right\}$ è di Cauchy e pertanto converge in $\reals$.
\end{proof}

\begin{definition}[Maggiorante e minorante]
  Sia $A\subseteq\reals$.
  $x_0\in\reals$ è detto \textbf{maggiorante} di $A$ se $x\in A\impl x<x_0$.
  Analogamente, $x_0$ è detto \textbf{minorante} di $A$ se $x\in A\impl x>x_0$.
\end{definition}

\begin{definition}[Estremi superiore ed inferiore]
  Sia $A\subseteq\reals$.
  $x_0$ è detto \textbf{estremo superiore} di $A$, indicato con $\sup A$, se $x_0$ è maggiorante di $A$ ed è il suo più piccolo maggiorante, vale a dire che $\forall\epsilon>0\ x_0-\epsilon$ non è maggiorante di $A$.
  Analogamente, $x_0$ è detto \textbf{estremo inferiore} di $A$, indicato con $\inf A$, se $x_0$ è minorante di $A$ ed è il suo più grande minorante, vale a dire che $\forall\epsilon>0\ x_0+\epsilon$ non è minorante di $A$.
\end{definition}

\begin{definition}[Massimo e minimo]
  Sia $A\subseteq\reals$.
  Se $\sup A\in A$, allora $\sup A$ è detto \textbf{massimo} di $A$.
  Se $\inf A\in A$, allora $\inf A$ è detto \textbf{minimo} di $A$.
\end{definition}

\begin{theorem}[Esistenza di $\sup A$ e/o $\inf A$]
  Se $A\subseteq\reals$ è superiormente limitato, ossia ha almeno un maggiorante, allora $\exists\ \sup A\in\reals$.
\end{theorem}
\begin{proof}
  Poiché $A$ è superiormente limitato, $\exists\ b_0\in\reals:x\in A\impl x\le b_0$. Sia $a_0\in A:\intv{a_0}{b_0}\cap A\neq\emptyset$. Sia $c$ il punto medio dell'intervallo: $$c\walrus\frac{a_0+b_0}{2}$$
  In almeno uno degli intervalli $\intv{a_0}{c}$ e $\intv{c}{b_0}$ vi sono punti di $A$. Senza ledere la generalità della dimostrazione, si considera l'ultimo intervallo e lo si indica con $\intv{a_1}{b_1}$:
  $$\intv{a_1}{b_1}\subset\intv{a_0}{b_0}$$
  $$\intv{a_1}{b_1}\cap A\neq \emptyset$$
  $$x\in A\impl x\le b_1$$
  Iterando questo processo (che prende il nome di dicotomia), si ottiene una successione decrescente di intervalli:
  $$\intv{a_{n+1}}{b_{n+1}}\subset\intv{a_n}{b_n}\subset\cdots\subset\intv{a_0}{b_0}$$
  per cui si verifica che $\forall n\ge0\ b_n$ è un maggiorante di $A$.
  Si ha che $\left\{ a_n \right\}$ è crescente e superiormente limitata da $b_0$ e $\left\{ b_n \right\}$ è decrescente e inferiormente limitata da $a_0$.
  Per il teorema di convergenza delle successioni monotone limitate:
  $$\exists\ \lim_{n\to+\infty}a_n\quad\wedge\quad\exists\ \lim_{n\to+\infty}b_n$$
  Inoltre:
  $$\lim_{n\to+\infty}\(b_n-a_n\)=\lim_{n\to+\infty}\frac{b_0-a_0}{2^n}=0$$
  da cui:
  $$\reals\ni\bar{x}\walrus \lim_{n\to+\infty}a_n=\lim_{n\to+\infty}b_n$$
  $$\forall x\in A\impl x\le b_n$$
  $$\forall x\in A\impl \lim_{n\to+\infty}x\le \lim_{n\to+\infty}b_n$$
  $$\forall x\in A\impl x\le \bar{x}$$
  ossia $\bar{x}$ è un maggiorante di $A$.
  Sia $\epsilon>0$.
  $$\bar{x}=\lim_{n\to+\infty}a_n\impl\exists\ n\ge1:\bar{x}-\epsilon<a_n\le\bar{x}$$
  $$\intv{a_n}{b_n}\cap A\neq\emptyset\impl \exists\ x\in\intv{a_n}{b_n}\cap A\impl x\ge a_n$$
  $$\bar{x}-\epsilon<a_n\le x\le b_n$$
  pertanto $\bar{x}-\epsilon$ non è un maggiorante.
  Quindi $\bar{x}\in R$ è il più piccolo maggiorante, ossia $\bar{x}=\sup A$.
\end{proof}

\begin{definition}[Punti di accumulazione]
  Sia $E\subseteq\reals$. Un punto $\bar{x}\in\reals$ è detto \textbf{punto di accumulazione} di $E$ se:
  $$\forall\epsilon>0\ \(E\setminus\left\{ \bar{x} \right\}\)\cap\ointv{\bar{x}-\epsilon}{\bar{x}+\epsilon}\neq\emptyset$$
  ovvero:
  $$\forall\epsilon>0\ \exists\ x\in E:x\neq \bar{x}\wedge \abs{x-\bar{x}}<\epsilon$$
  ovvero:
  $$\forall\epsilon>0\ \#\(E\cap\ointv{\bar{x}-\epsilon}{\bar{x}+\epsilon}\)=+\infty$$
  Se $\bar{x}\in E$ non è di accumulazione, è detto \textbf{isolato}.
\end{definition}
\begin{observation}
  Non è detto che $\bar{x}\in E$.
\end{observation}

\begin{definition}[Insieme derivato]
  In generale, si denota con $E'$ l'insieme dei punti di accumulazione di $E$, detto \textbf{insieme derivato} di $E$.
\end{definition}

\begin{theorem}[Proprietà di Bolzano--Weierstrass]
  Sia $E\subseteq \reals$ limitato e infinito. Allora $\exists\ \bar{x}\in E'$.
\end{theorem}
\begin{proof}
  Poiché limitato, $E\subseteq\intv{a_0}{b_0}$. Sia $c$ il punto medio dell'intervallo:
  $$c\walrus\frac{a_0+b_0}{2}$$
  In almeno uno fra gli intervalli $\intv{a_0}{c}$ e $\intv{c}{b_0}$ vi sono infiniti punti di $E$, dato che $E\subseteq\intv{a_0}{b_0}=\intv{a_0}{c}\cup\intv{c}{b_0}$ e $\#\(E\)=+\infty$. Considerato tale intervallo, e indicandolo con $\intv{a_1}{b_1}$, si ha:
  $$\#\(E\cap\intv{a_1}{b_1}\)=+\infty$$
  Iterando la dicotomia, si ottiene una successione decrescente di intervalli:
  $$\intv{a_{n+1}}{b_{n+1}}\subset\intv{a_n}{b_n}\subset\cdots\subset\intv{a_0}{b_0}$$
  da cui:
  $$\#\(E\cap\intv{a_n}{b_n}\)=+\infty$$
  Si ha che $\left\{ a_n \right\}$ è crescente e superiormente limitata da $b_0$ e $\left\{ b_n \right\}$ è decrescente e inferiormente limitata da $a_0$.
  Per il teorema di convergenza delle successioni monotone limitate:
  $$\exists\ \lim_{n\to+\infty}a_n\quad\wedge\quad\exists\ \lim_{n\to+\infty}b_n$$
  Inoltre:
  $$\lim_{n\to+\infty}\(b_n-a_n\)=\lim_{n\to+\infty}\frac{b_0-a_0}{2^n}=0$$
  da cui:
  $$\reals\ni\bar{x}\walrus \lim_{n\to+\infty}a_n=\lim_{n\to+\infty}b_n$$
  Sia $x_n\in E\cap\intv{a_n}{b_n}:x_n\neq\bar{x}$. Si ha:
  $$a_n\le x_n\le b_n$$
  $$\bar{x}=\lim_{n\to+\infty}a_n\le \lim_{n\to+\infty}x_n\le \lim_{n\to+\infty}b_n=\bar{x}$$
  $$\bar{x}=\lim_{n\to+\infty}x_n$$
  Pertanto, $\bar{x}$ è di accumulazione per $E$: $$\bar{x}\in E'$$
\end{proof}

\end{document}