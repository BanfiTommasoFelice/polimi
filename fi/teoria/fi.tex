\documentclass[a4paper,12pt]{article}

\usepackage[italian]{babel}
\usepackage[utf8]{inputenc}
\usepackage[a4paper, left=18mm, right=18mm, top=20mm, bottom=20mm]{geometry}
\usepackage{amssymb}
\usepackage{mathtools}
\usepackage{interval}
\usepackage{amsthm}
\usepackage{thmtools}
\usepackage{cancel}
\usepackage{hyperref}
\usepackage{tikz}
\usepackage{pgfplots}
\usepackage{nicefrac}
\usepackage{enumitem}
\usepackage{verbatim}
\usepackage{tabularray}
\usepackage{algorithm}
\usepackage[noend]{algpseudocode}
\usepackage{fontspec}
\usepackage{bold-extra}

\setmonofont[Contextuals={Alternate},SizeFeatures={Size=9}]{Fira Code}

\usetikzlibrary{shapes.geometric, arrows}
\tikzstyle{startstop} = [rectangle, rounded corners, minimum width=3cm, minimum height=1cm,text centered, draw=black, fill=magenta!30]
\tikzstyle{input} = [trapezium, trapezium left angle=70, trapezium right angle=110, minimum width=2cm, minimum height=.5cm, text centered, draw=black, fill=cyan!30]
\tikzstyle{output} = [trapezium, trapezium left angle=110, trapezium right angle=70, minimum width=2cm, minimum height=.5cm, text centered, draw=black, fill=cyan!30]
\tikzstyle{process} = [rectangle, minimum width=3cm, minimum height=1cm, text centered, draw=black, fill=orange!30]
\tikzstyle{decision} = [diamond, minimum width=3cm, minimum height=1cm, text centered, draw=black, fill=green!30, aspect=1.618]
\tikzstyle{dummy} = [rectangle, minimum width=0cm, minimum height=0cm, inner sep=0pt]
\tikzstyle{arrow} = [thick,->,>=stealth]

\hypersetup{
  colorlinks=true,
  linkcolor=black,
    filecolor=magenta,      
    urlcolor=cyan,
    pdftitle={Fondamenti di informatica},
    % bookmarks=true,
    bookmarksopen=true,
    pdfpagemode=UseOutlines,
    pdfauthor={Amato Michele Pasquale},
}

\newtheorem{definition}{Definizione}
\renewcommand{\(}{\left(}
\renewcommand{\)}{\right)}
\newcommand{\impl}{\Rightarrow}
\DeclareMathOperator{\sgn}{sgn}
\renewcommand{\epsilon}{\varepsilon}

\title{\huge Fondamenti di informatica}
\author{Amato Michele Pasquale}
\date{\today}

\pgfplotsset{compat = newest}
\makeatletter
\renewcommand\l@subsection{\@dottedtocline{2}{1.5em}{3em}}
\makeatother
\setitemize{noitemsep,topsep=3pt,parsep=0pt,partopsep=0pt}
\setenumerate{noitemsep,topsep=3pt,parsep=0pt,partopsep=0pt}

\newcommand{\abs}[1]{\left\lvert #1 \right\rvert}
\newcommand{\ceil}[1]{\left\lceil #1 \right\rceil}

\newenvironment{example}
{\par\noindent{\small\bf Esempio}\hspace{0.5em}}
{\hfill$\nsim$}

\begin{document}
\pagenumbering{gobble}
\maketitle
\vfill
\begin{center}
  \emph{Efficienza è fare le cose bene. \\ Efficacia è fare le cose giuste.}
\end{center}
\newpage
\pagenumbering{roman}
\tableofcontents
\newpage
\pagenumbering{arabic}
\part{La rappresentazione dell'informazione}
% % lezione del 12/09/2022

Il problema che ci si pone è trovare un modo opportuno per rappresentare all'interno di un sistema di calcolo le informazioni in modo efficiente, rispetto alla realtà fisica del sistema e alla loro manipolazione.

\begin{definition}[Alfabeto]
  Si definisce \textbf{alfabeto} un insieme di simboli utilizzabili e, pertanto, distinguibili tra loro.
\end{definition}

\begin{definition}[Codice]
  Si definisce \textbf{codice} l'insieme delle sequenze di simboli o delle regole per definire le combinazioni ammissibili.
\end{definition}

Dati l'insieme degli elementi da rappresentare e l'insieme delle configurazioni ammissibili, il codice ne definisce la relazione biunivoca.
Le configurazioni ammissibili hanno tutte egual dimensione. Tale dimensione dipende sia dall'alfabeto, sia dalla quantità di elementi da rappresentare: siano $S$ l'alfabeto di riferimento e $\abs{S}$ la sua cardinalità (ossia il numero di simboli che lo compone), se si vogliono rappresentare $n$ elementi, ogni elemento avrà dimensione:
$$k=\ceil{\log_{\abs{S}} n}$$
Al contrario, se gli elementi di un codice hanno lunghezza $k$, le combinazioni ammissibili sono:
$$n=\abs{S}^k$$

\section{Rappresentazione binaria}

I componenti elettronici che costituiscono il sistema di calcolo sono caratterizzati da una realtà costituita da due stati (condensatore carico/scarico, tensione alta/bassa, etc...). Si effettua, quindi, una mappatura diretta con un sistema costituito da \textbf{due simboli} che, pertanto, si chiama \textbf{binario}. L'alfabeto di riferimento diventa $\left\{ 0,1 \right\}$.

La cifra della codifica (0 o 1) prende il nome di \textbf{bit}, dall'inglese \emph{\textbf{bi}nary digi\textbf{t}}. L'insieme ordinato di 8 bit prende il nome di \textbf{byte}. Come per le cifre decimali, si ha una nomenclatura per le potenze della base:
\begin{center}
  \begin{tblr}{colspec={c|c}, cells={c,m}, columns={20mm}}
    \textbf{Nome} & \textbf{Quantità} \\
    \hline
    KB            & $2^{10}$          \\
    MB            & $2^{20}$          \\
    GB            & $2^{30}$          \\
    TB            & $2^{40}$          \\
  \end{tblr}
\end{center}

\begin{example}
  Se si vogliono rappresentare i giorni della settimana usando l'alfabeto binario, si calcola la dimensione del singolo elemento:
  $$k=\ceil{\log_27}=3$$
  e si assegna ad ogni combinazione di 3 bit un giorno della settimana distinto:
  \begin{center}
    \begin{tblr}{colspec={c|c|c|c|c|c|c}, cells={c,m}, columns={18mm}}
      Lunedì & Martedì & Mercoledì & Giovedì & Venerdì & Sabato & Domenica \\
      \hline
      000    & 001     & 010       & 011     & 100     & 101    & 110      
    \end{tblr}
  \end{center}
  Da notare che non viene utilizzata la combinazione 111, in quanto le combinazioni ammissibili sono $2^3=8$ ma per le necessità del caso ne servono solo 7.
\end{example}

Nella scelta della codifica da adottare, bisogna tenere a mente alcuni aspetti:
\begin{itemize}
  \item l'insieme degli elementi da rappresentare;
  \item il grado di semplificazione delle operazioni più eseguite;
  \item il grado di conservazione delle proprietà dell'insieme originale.
\end{itemize}

L'informazione può essere, per comodità, suddivisa in aree:

\begin{center}
  \begin{tblr}{colspec={c|c|c|c|c|c}, cells={c,m}, columns={20mm}}
    \SetCell[c=6]{c} Informazione                                                                       \\ \hline
    \SetCell[c=3]{c} Numerica &          &           & \SetCell[c=3]{c} Non numerica                    \\ \hline
    Naturali                  & Relativi & Razionali & Testi                         & Audio & Immagini 
  \end{tblr}
\end{center}

% lezione del 13/09/2022

La notazione che si usa in base 10 è una \textbf{notazione posizionale pesata}, vale a dire che ogni cifra vale in base alla posizione che essa occupa all'interno del numero. Il valore del numero, infatti, è dato da
$$\text{valore}=\sum_{0}^nc_ib^i$$
dove $c_i$ è la cifra in posizione $i$ e $b$ è la base di riferimento.

\begin{example}
  $$315_{10}=3\cdot10^2+1\cdot10^1+5\cdot10^0$$
\end{example}

Nella notazione in base 2, si definisce un codice che associa al valore numerico una configurazione, in cui la cifra più a destra è la cifra meno significativa (\textbf{LSB}, Least Significant Bit), mentre quella più a sinistra è quella più significativa (\textbf{MSB}, Most Significant Bit).

\section{Conversione di base}
Idealmente, ogni qualvolta si vuole rappresentare una quantità in una certa base, si vogliono implicitamente rappresentare in un'altra base tutte le quantità minori uguali a quella di partenza, non avrebbe senso altrimenti\footnote{se me ne serve solo uno, uso il primo valore disponibile}.

Se $n$ è il valore da rappresentare, significa che in base 2 si avrà bisogno di $k$ bits:
$$k=\ceil{\log_2\(n+1\)}$$
\paragraph*{Osservazione} L'argomento del logaritmo è $n+1$ in quanto bisogna anche considerare lo 0.

Per passare da una base $a$ ad una base $b$, con $a<b$, si procede nel seguente modo: si moltiplica ogni coefficiente per la base elevata alla sua posizione e poi si sommano tutti i prodotti ottenuti.

\begin{example}
  $$1010_2=1\cdot2^3+0\cdot2^2+1\cdot2^1+0\cdot2^0=8+0+2+0=10_{10}$$
\end{example}

Per passare da una base $a$ ad una base $b$, con $a>b$, si procede nel seguente modo:

\begin{enumerate}
  \item si divide il numero per la base;
  \item si prende il resto;
  \item si ripete il processo con il quoziente ottenuto.
\end{enumerate}

Il processo si conclude quando il quoziente diventa 0. Il risultato non è altro che la sequenza ordinata dei resti ottenuti, letta al contrario. 

\begin{example}
  \begin{center}
    \begin{tblr}{c|c}
      17 & 1 \\
      8  & 0 \\
      4  & 0 \\
      2  & 0 \\
      1  & 1 
    \end{tblr}
    $$\impl 17_{10}=10001_2$$
  \end{center}
\end{example}

Le modalità con cui si trasformano le quantità fra le diverse basi sono interscambiabili, pertanto si dicono metodi.

\section{Modulo e segno}

Nella notazione decimale si utilizza la forma ``modulo e segno'' per esprimere i numeri relativi:
$$-11\;\;+4$$

Tale ragionamento non si può estendere alla rappresentazione binaria, perché richiederebbe l'introduzione un nuovo simbolo per `$+$' o `$-$'. Pertanto, per convenzione, il primo bit indica il segno, che può essere 0 se è positivo, 1 se è negativo:
$$+17_{10\text{MS}}=010001_{2\text{MS}}$$
$$-17_{10\text{MS}}=110001_{2\text{MS}}$$
$$-23_{10\text{MS}}=110111_{2\text{MS}}$$

Si devono eseguire le operazioni aritmetiche usando la notazione modulo e segno.
\begin{center}
  \begin{tblr}{c|c}
    $
      \begin{aligned}
        x & =+11_{10\text{MS}} \\
        y & =+8_{10\text{MS}}  \\
        z & =-7_{10\text{MS}}  \\
      \end{aligned}
    $
     & 
    $
      \begin{aligned}
        x+y & =+19_{10\text{MS}} \\
        x+z & =+4_{10\text{MS}}  \\
      \end{aligned}
    $
  \end{tblr}
\end{center}

Nella rappresentazione decimale si eseguono i passaggi descritti nel seguente algoritmo:

\begin{algorithm}[H]
  \caption{Somma di due valori in notazione modulo e segno}\label{algo:sommams}
  \begin{algorithmic}[1]
    \If{$\sgn x= \sgn y$}
    \State $\abs{n}=\abs{x}+\abs{y}$
    \State $\sgn n=\sgn x$
    \Else
    \State $a = \max \(\abs{x},\abs{y}\)$
    \State $b = \min \(\abs{x},\abs{y}\)$
    \State $\abs{n}=a-b$
    \If{$\abs{x}=a$}
    \State $\sgn n=\sgn a$
    \Else
    \State $\sgn n=\sgn b$
    \EndIf
    \EndIf
  \end{algorithmic}
\end{algorithm}

Tuttavia, per un calcolatore un tal numero di operazioni renderebbe il calcolo proibitivamente lento. Per questo, si introduce una nuova notazione, più efficiente di quella di modulo e segno, incentrata sull'agilità dell'elaborazione delle informazioni.

Ovviamente, la finestra dei valori che si vuole rappresentare si estende anche al negativo, per cui se si vuole rappresentare il $17$, si vuole rappresentare tutti i valori da $-17$ a $17$.

Si descrivono le possibili combinazioni avendo a disposizione 4 bit:
\begin{center}
  \begin{tblr}{c|c||c|c}
    $0000_2$ & $0_{10}$ & $1000_2$ & $-0_{10}$ \\
    $0001_2$ & $1_{10}$ & $1001_2$ & $-1_{10}$ \\
    $0010_2$ & $2_{10}$ & $1010_2$ & $-2_{10}$ \\
    $0011_2$ & $3_{10}$ & $1011_2$ & $-3_{10}$ \\ \hline
    $0100_2$ & $4_{10}$ & $1100_2$ & $-4_{10}$ \\
    $0101_2$ & $5_{10}$ & $1101_2$ & $-5_{10}$ \\
    $0110_2$ & $6_{10}$ & $1110_2$ & $-6_{10}$ \\
    $0111_2$ & $7_{10}$ & $1111_2$ & $-7_{10}$ \\
  \end{tblr}
\end{center}

Si nota subito che lo 0 ha una codifica ridondante:
$$0000_{2\text{MS}}=1000_{2\text{MS}}=0_{10}$$

Si pone l'attenzione sull'identità:
$$x+\(-x\)=0$$

Volendo rispettare l'identità con un valore $x=3_{10}$ si ha:

\begin{center}
  \begin{tblr}{ccccc}
    $^10$ & $^10$ & $^11$ & $1$ & $+$ \\
    $1$   & $1$   & $0$   & $1$ & $=$ \\
    \hline
    $0$   & $0$   & $0$   & $0$       
  \end{tblr}
\end{center}

quindi risulta che $1101_2$ è la codifica di $0011_2$ al negativo. Da questo ragionamento, esteso agli altri numeri risulta che:
\begin{center}
  \begin{tblr}{c|c||c|c}
    $0000_2$ & $0_{10}$ & $1000_2$ & $\pm8_{10}$ \\
    $0001_2$ & $1_{10}$ & $1001_2$ & $-7_{10}$   \\
    $0010_2$ & $2_{10}$ & $1010_2$ & $-6_{10}$   \\
    $0011_2$ & $3_{10}$ & $1011_2$ & $-5_{10}$   \\ \hline
    $0100_2$ & $4_{10}$ & $1100_2$ & $-4_{10}$   \\
    $0101_2$ & $5_{10}$ & $1101_2$ & $-3_{10}$   \\
    $0110_2$ & $6_{10}$ & $1110_2$ & $-2_{10}$   \\
    $0111_2$ & $7_{10}$ & $1111_2$ & $-1_{10}$   \\
  \end{tblr}
\end{center}

Esiste anche un metodo algebrico per trovare l'opposto:
$$-x=\sim x+1$$
oppure un metodo grafico:
\emph{tutti i bit che dall'LSB al primo bit 1 rimangono invariati, mentre tutti gli altri si cambiano}.

L'efficacia di questa notazione sta nella facilità con cui si eseguono le operazioni:
$$2_{10}+4_{10}=0010_2+0100_2=0110_2=6_{10}$$
$$-3_{10}+\(-1_{10}\)=1101_2+1111_2=\cancel{1}1100_2=-4_{10}$$
$$5_{10}+\(-3_{10}\)=0101_2+1101_2=\cancel{1}0010_2=2_{10}$$

Questa notazione prende il nome di complemento in base 2.
Si chiama complemento in quanto i valori negativi si completano a quelli positivi.

Per passare dalla base 10 alla base 2 in complemento a 2, si effettuano i seguenti passaggi:
\begin{algorithm}[H]
  \caption{Conversione da 10MS a 2C2}\label{algo:10to2c2}
  \begin{algorithmic}[1]
    \State $n=\abs{x}$
    \If{$x<0$}
    \State $n=-n$
    \EndIf
  \end{algorithmic}
\end{algorithm}
\begin{example}
  $$x=-13_{10}$$
  $$\abs{x}=13_{10}=1101_2=01101_{2\text{MS}}\equiv 01101_{2\text{C}2}$$
  $$x=-\abs{x}=10011_{2\text{C}2}$$
\end{example}

Si presentano casi ambigui in cui apparentemente sembra che l'aritmetica non funzioni:
$$5_{10}+3_{10}=0101_2+0011_2=1000_2=-8_{10}$$
$$-2_{10}+\(-6_{10}\)=1110_2+1010_2=\cancel{1}1000_2=-8_{10}$$
Si nota che tutti i numeri positivi cominciano con 0 mentre quelli negativi cominciano con 1.
Convenzionalmente, in virtù di quanto appena detto, la combinazione $1000_2$ assume il valore $-8_{10}$.

Usando la notazione modulo e segno il range dei valori ammissibili è $\interval{-7}{7}$, mentre in complemento a 2 il range è $\interval{-8}{7}$.

Considerando che la dimensione dell'informazione è sempre fissa e determinata, ci si ritrova in casi particolari detti di \textbf{overflow} (\emph{traboccamento}):
$$6_{10}+4_{10}=0110_2+0100_2=1010_2=-6_{10}$$
L'overflow è facilmente risolvibile con un aumento dello spazio a disposizione:
$$6_{10}+4_{10}=00110_2+00100_2=01010_2=10_{10}$$
Si ha overflow quando due valori concordi generano un valore discorde dai primi due.
Quando gli operandi sono discordi è impossibile generare overflow.

\paragraph*{Osservazione} L'estensione di un valore in complemento a 2 si esegue ripetendo il MSB quante volte se ne ha bisogno:

\begin{example}
  $$5_{10}=0101_{2\text{C}2}=000000101_{2\text{C}2}$$
  $$-5_{10}=1011_{2\text{C}2}=111111011_{2\text{C}2}$$
\end{example}

\begin{example}
  $$x=+12_{10\text{MS}}=01100_{2\text{C}2}$$
  $$y=-3_{10\text{MS}}=101_{2\text{C}2}=11101_{2\text{C}2}$$
  $$x+y=01100_2+11101_2=\cancel{1}01001_2$$
  $$x-y=x+\(-y\)=01100_2+00011_2=01111_2$$
\end{example}

% lezione del 15/09/2022

Per passare dalla base 2 in complemento a 2 alla base 10, si possono verificare due scenari:
\begin{enumerate}
  \item il numero è positivo, pertanto lo si converte usando la formula dei pesi;
  \item il numero è negativo: in questo caso si calcola l'opposto, lo si converte e gli si cambia il segno.
\end{enumerate}
\begin{example}
  $$010110_{2\text{C}2}=22_{10\text{MS}}$$
  $$1011101_{2\text{C}2}=-35_{10\text{MS}}$$
\end{example}

\section{Rappresentazione esadecimale}
Il problema principale con la rappresentazione binaria è l'enorme spazio richiesto. Inoltre, in virtù dell'uso sui calcolatori, la base da cercare deve essere una potenza di 2.
Per anni è stata utilizzata la base 8, ma ben presto è stata resa obsoleta, in favore della base 16.

Il codice esadecimale si crea associando le cifre esadecimali e combinazioni di 4 bit del codice binario.
\begin{center}
  \begin{tblr}{c|c|c}
    \textbf{esadecimale} & \textbf{binario} & \textbf{decimale} \\\hline
    0                    & 0000             & 0                 \\
    1                    & 0001             & 1                 \\
    2                    & 0010             & 2                 \\
    3                    & 0011             & 3                 \\
    4                    & 0100             & 4                 \\
    5                    & 0101             & 5                 \\
    6                    & 0110             & 6                 \\
    7                    & 0111             & 7                 \\
    8                    & 1000             & 8                 \\
    9                    & 1001             & 9                 \\
    A                    & 1010             & 10                \\
    B                    & 1011             & 11                \\
    C                    & 1100             & 12                \\
    D                    & 1101             & 13                \\
    E                    & 1110             & 14                \\
    F                    & 1111             & 15                \\
  \end{tblr}
\end{center}

Per effettuare la conversione dalla base 2 alla base 16 basta considerare ogni quadrupla e convertirla in loco:
\begin{center}
  \begin{tblr}{cccc}
    0101 & 0101 & 1101 & 1010 \\ \hline
    5    & 5    & D    & A    
  \end{tblr}
\end{center}
Banalmente, la stessa cosa avviene per il processo inverso.

Al fine di rappresentare attraverso la notazione modulo e segno un numero esadecimale, gli si prepone il segno, facendo la conversione dal binario.

\section{Numeri razionali}

Si vuole rappresentare i valori espressi da:
$$\frac{m}{n},\; m\in\mathbb{Z},n\in\mathbb{N}/\left\{ 0 \right\}$$

Nella notazione decimale si usa scrivere prima la parte intera, un separatore decimale e poi la parte frazionaria.
In base 2 si può fare la stessa cosa:
$$101.01_2=2^2+2^0+2^{-2}=5.25_{10}$$
Avendo a disposizione una certa quantità bit, si sceglie la parte di essi che contiene la parte frazionaria e quella che contiene la parte intera. 

\begin{example}
  Si considerano 3 bit per la parte intera e 2 per quella frazionaria. Si ha che il range massimo esprimibile è $\interval{0}{7.75}$, con salti di $0.25$.
  
  Tuttavia, se si vuole rappresentare il valore $6.3$, non si può. Al limite, si può esprimere una sua approssimazione, che in questo caso è $6.25$.
\end{example}

Il fatto che non si possa rappresentare un valore con \textbf{precisione} non è dovuto ad una cattiva gestione dei bit, ma bensì al loro stato di finitezza.

La notazione utilizza si chiama a virgola fissa: viene stabilita la dimensione a priori sia per la parte intera sia per la parte frazionaria.
In una notazione del genere l'\textbf{errore assoluto} $\epsilon_A$ è costante: si sbaglia sempre della stessa quantità.

Esiste, tuttavia, una notazione alternativa che consente di variare l'errore assoluto e mantenere costante l'errore relativo: la notazione in \textbf{virgola mobile} (\emph{floating point} in inglese).
L'errore relativo $$\epsilon_R=\frac{\epsilon_A}{\text{valore}}$$

Per rappresentare un numero razionale in base 2, si procede come al solito per la parte intera, ovvero si divide per 2 segnando il resto, ma al contrario per la parte frazionaria, ovvero si moltiplica per 2 segnando l'unità. Un'altra differenza fondamentale è il verso di lettura delle cifre: mentre per la parte intera si procede dal basso verso l'alto, per la parte frazionaria si procede dall'alto verso il basso.

\begin{example}
  Si vuole rappresentare il numero $13.75$.
  \begin{center}
    \begin{minipage}{0.2\linewidth}
      \begin{center}
        \begin{tblr}{c|c}
          13 & 1 \\
          6  & 0 \\
          3  & 1 \\
          1  & 1 \\
          0  &   
        \end{tblr}
      \end{center}
    \end{minipage}
    \begin{minipage}{0.2\linewidth}
      \begin{center}
        \begin{tblr}{c|c}
          0.75 &   \\
          1.5  & 1 \\
          1.0  & 1 \\
          0    &   \\
        \end{tblr}
      \end{center}
    \end{minipage}
  \end{center}
  $$13.75_{10}=1101.11_2$$
\end{example}

\begin{example}
  Si vuole rappresentare il numero $7.32$.
  \begin{center}
    \begin{minipage}{0.2\linewidth}
      \begin{center}
        \begin{tblr}{c|c}
          7 & 1 \\
          3 & 1 \\
          1 & 1 \\
          0 &   
        \end{tblr}
      \end{center}
    \end{minipage}
    \begin{minipage}{0.2\linewidth}
      \begin{center}
        \begin{tblr}{c|c}
          0.32     &          \\
          0.64     & 0        \\
          1.28     & 1        \\
          0.56     & 0        \\
          1.12     & 1        \\
          0.24     & 0        \\
          0.48     & 0        \\
          0.96     & 0        \\
          1.92     & 1        \\
          1.84     & 1        \\
          $\cdots$ & $\cdots$ 
        \end{tblr}
      \end{center}
    \end{minipage}
  \end{center}
  $$7.32_{10}=111.010100011...$$
\end{example}

Se si considerano i razionali da una diversa prospettiva, si nota che:
$$13.75_{10}=1.375\cdot 10^1$$
$$7.32_{10}=7.32\cdot 10^0$$
Analogamente:
$$1101.11_2=1.10111\cdot2^3$$
$$111.010100011_2=1.11010100011\cdot2^2$$

La notazione in virgola mobile, quindi, ha:
\begin{itemize}
  \item un bit per il \textbf{segno};
  \item una parte per l'\textbf{esponente}, a cui viene aggiunto un numero tale che il più piccolo esponente possibile sia 0;
  \item una parte per la \textbf{mantissa}, ossia la parte frazionaria della notazione scientifica\footnote{si prende la parte frazionaria in quanto la parte intera sarà sempre 1, quindi è inutile sprecare un bit}.
\end{itemize}
Lo standard che regola la notazione in virgola mobile è lo \textbf{IEEE 754}. Lo standard prevede 3 versioni:
\begin{itemize}
  \item 32 bit: 1 per il segno, 8 per l'esponente, 23 per la mantissa;
  \item 64 bit: 1 per il segno, 11 per l'esponente, 52 per la mantissa;
  \item 128 bit: 1 per il segno, 15 per l'esponente, 112 per la mantissa.
\end{itemize}

Il valore di un numero espresso attraverso lo standard IEEE 754 è dato da:
$$\text{valore}=\(-1\)^S\(1+M\)\cdot2^E$$
dove $S$ è il segno, $M$ è la mantissa e $E$ è l'esponente

\paragraph*{Nota bene}
La costante che bisogna aggiungere all'esponente si chiama ``eccesso'' ed è:
\begin{itemize}
  \item 127 per la singola precisione;
  \item 1023 per la doppia precisione;
  \item 16383 per la quadrupla precisione.
\end{itemize}

\begin{example}
  $$+5.65_{10\text{MS}}$$
  $$+101.10{1001}_{2\text{MS}}=1.01101001\cdot2^2=01000000101101001100110011001100$$
  $$-0.028_{10\text{MS}}$$
  $$-0.00000111001010_2=-1.1100101\cdot2^{-6}=101001111110101...$$
\end{example}

Lo standard, tuttavia, presenta delle falle che vengono compensate dall'introduzione di combinazioni di bit speciali che, proprio per questo, vengono dette \textbf{denormarlizzate}.

\paragraph*{Forma denormarlizzata generale}
Quando tutti i bit dell'esponente sono 0, la mantissa non è sommata ad 1 ma bensì a 0.

\paragraph*{Zero}
Quando tutti i bit (indifferentemente dal primo) sono 0, il valore è 0.

\paragraph*{Infinito}
Quando tutti i bit dell'esponente sono 1 e quelli della mantissa sono tutti 0, il valore è infinito, che può essere sia negativo che positivo.

\paragraph*{NaN}
Quando tutti i bit dell'esponente sono 1 e almeno uno di quelli della mantissa è 1, il valore non è un numero (\textbf{NaN}, \emph{Not a Number}).

% lezione del 16/09/2022

\section{L'informazione non numerica}

Per ovvi motivi, non si può rappresentare solo l'informazione numerica. L'obiettivo è, quindi, cercare di definire una codifica per ogni possibile carattere rappresentabile. Inizialmente si hanno:
\begin{itemize}
  \item alfabeto base (a...z, A...Z);
  \item caratteri numerici (0...9);
  \item caratteri di interpunzione (.,;:);
  \item caratteri speciali.
\end{itemize}

Tutti questi elementi sono circa 120, per cui sono necessari 7 bit.
Per la rappresentazione di tali simboli si utilizza il \textbf{codice ASCII} (\emph{American Standard Code for Information Interchange}): il codice dispone di 7 bit, per cui ha 128 combinazioni.
Per far fronte alle necessità sorte nel tempo, si è aggiunto un bit al codice ASCII, creando il codice \textbf{ASCII esteso}: esso dispone di 8 bit, e comprende caratteri nazionali, simboli e cornici.


\part{Elaborazione dell'informazione}

\section{Algoritmi}

Un algoritmo è un procedimento, ovvero una sequenza di passi, che a partire da un insieme di dati (\emph{input}), attraverso un numero finito di passi, genera un insieme di dati (\emph{output}).

Ogni algoritmo ha un punto d'inizio e un punto di fine.
Un algoritmo che non accetta dati di input produce sempre lo stesso output.

Ogni passo dell'algoritmo si serve di operatori, ossia simboli che indicano un'operazione da effettuare:
\begin{itemize}
  \item operatore assegnamento \verb|=|: è un operatore binario e tutto ciò che è alla sua destra, una volta calcolato, viene assegnato alla variabile che si trova alla sua sinistra;
  \item operatori aritmetici \verb|+ - * / %|: sono operatori binari che effettuano le principali operazioni aritmetiche tra due operandi (rispettivamente somma, differenza, prodotto, quoziente, resto), che possono essere variabili o costanti;
  \item operatori logici \verb|&&|, \verb!||!, \verb|!|: sono operatori binari, tranne l'ultimo che è unario, che esprimono rispettivamente le verità date dalle proposizioni: $a\wedge b$, $a\vee b$, $\lnot a$;
  \item operatori relazionali \verb|>|, \verb|<|, \verb|>=|, \verb|<=|, \verb|==|, \verb|!=|: sono operatori binari che esprimono l'uguaglianza e la disuguaglianza degli operandi. 
\end{itemize}

\begin{example}
  Algoritmo che calcola e visualizza, dato un numero strettamente positivo, l'area e il perimetro di un cerchio avente come raggio il numero dato.
  \begin{center}
    \begin{tikzpicture}[node distance=1.5cm]
      \node (start) [startstop] {\scshape Inizio};
      \node (in1) [input, below of=start] {\verb|r|};
      \node (pro1) [process, below of=in1] {\verb|p = 2 * PI * r|};
      \node (pro2) [process, below of=pro1] {\verb|a = PI * r * r|};
      \node (out1) [output, below of=pro2] {\verb|p, a|};
      \node (stop) [startstop, below of=out1] {\scshape Fine};
      \draw [arrow] (start) -- (in1);
      \draw [arrow] (in1) -- (pro1);
      \draw [arrow] (pro1) -- (pro2);
      \draw [arrow] (pro2) -- (out1);
      \draw [arrow] (out1) -- (stop);
    \end{tikzpicture}
  \end{center}
\end{example}

\begin{example}
  Algoritmo che, dato un numero $s$ intero positivo di secondi, calcola e visualizza le ore, i minuti e i secondi che $s$ quantifica.
  \begin{center}
    \begin{tikzpicture}[node distance=1.5cm]
      \node (start) [startstop] {\scshape Inizio};
      \node (pro1) [process, below of=start] {\verb|SECS_PER_MIN = 60|};
      \node (pro2) [process, below of=pro1] {\verb|SECS_PER_HOUR = SECS_PER_MIN * 60|};
      \node (in) [input, below of=pro2] {\verb|s|};
      \node (pro3) [process, below of=in] {\verb|h = s / SECS_PER_HOUR|};
      \node (pro4) [process, below of=pro3] {\verb|s = s % SECS_PER_HOUR|};
      \node (pro5) [process, below of=pro4] {\verb|m = s / SECS_PER_MIN|};
      \node (pro6) [process, below of=pro5] {\verb|s = s % SECS_PER_MIN|};
      \node (out1) [output, below of=pro6] {\verb|h, m, s|};
      \node (stop) [startstop, below of=out1] {\scshape Fine};
      \draw [arrow] (start) -- (pro1);
      \draw [arrow] (pro1) -- (pro2);
      \draw [arrow] (pro2) -- (in);
      \draw [arrow] (in) -- (pro3);
      \draw [arrow] (pro3) -- (pro4);
      \draw [arrow] (pro4) -- (pro5);
      \draw [arrow] (pro5) -- (pro6);
      \draw [arrow] (pro6) -- (out1);
      \draw [arrow] (out1) -- (stop);
    \end{tikzpicture}
  \end{center}
\end{example}

\begin{example}
  Algoritmo che, dato un numero $x$, calcola e visualizza il suo valore assoluto.
  \begin{center}
    \begin{tikzpicture}[node distance=1.8cm]
      \node (start) [startstop] {\scshape Inizio};
      \node (in) [input, below of=start] {\verb|x|};
      \node (d1) [decision, below of=in] {\verb|x < 0|};
      \node (p1) [process, left of=d1, xshift=-2cm] {\verb|y = -x|};
      \node (p2) [process, right of=d1, xshift=2cm] {\verb|y = x|};
      \node (out) [output, below of=d1] {\verb|y|};
      \node (stop) [startstop, below of=out] {\scshape Fine};
      \draw [arrow] (d1) -- node[anchor=north] {sì} (p1);
      \draw [arrow] (d1) -- node[anchor=north] {no} (p2);
      \draw [arrow] (start) -- (in);
      \draw [arrow] (in) -- (d1);
      \draw [arrow] (out) -- (stop);
      \draw [arrow] (p1) |- (out);
      \draw [arrow] (p2) |- (out);
    \end{tikzpicture}
  \end{center}
\end{example}

\begin{example}
  Algoritmo che, dato un numero $r$ come raggio di una circonferenza, ne calcola l'area e il perimetro, solo se $r$ è positivo, altrimenti attende un nuovo valore.
  \begin{center}
    \begin{tikzpicture}[node distance=1.8cm]
      \node (start) [startstop] {\scshape Inizio};
      \node (in) [input, below of=start] {\verb|r|};
      \node (d1) [decision, below of=in] {\verb|r < 0|};
      \node (p1) [process, below of=d1] {\verb|p = 2 * PI * r|};
      \node (p2) [process, below of=p1] {\verb|a = PI * r * r|};
      \node (o1) [output, below of=p2] {\verb|p, a|};
      \node (stop) [startstop, below of=o1] {\scshape Fine};
      \draw [arrow] (start) -- (in);
      \draw [arrow] (in) -- (d1);
      \draw [arrow] (d1) -- node[anchor=east] {no} (p1);
      \draw [arrow] (d1.west) |- node[anchor=east] {sì} (in);
      \draw [arrow] (p1) -- (p2);
      \draw [arrow] (p2) -- (o1);
      \draw [arrow] (o1) -- (stop);
    \end{tikzpicture}
  \end{center}
\end{example}

\begin{example}
  Algoritmo che, dati tre numeri, visualizza 1 se i tre numeri compongono una terna pitagorica, 0 altrimenti.
  \begin{center}
    \begin{tikzpicture}[node distance=2cm]
      \node (start) [startstop] {\scshape Inizio};
      \node (in) [input, below of=start] {\verb|a, b, c|};
      \node (d1) [decision, below of=in] {\verb|a > b|};
      \node (d2) [decision, below left of=d1, xshift=-2cm] {\verb|a > c|};
      \node (d3) [decision, below right of=d1, xshift=2cm] {\verb|b > c|};
      \node (p1) [process, below of=d2] {\verb|swap(a, c)|};
      \node (p2) [process, below of=d3] {\verb|swap(b, c)|};
      \node (d4) [decision, below of=d1, yshift=-3.5cm] {\verb|a * a + b * b == c * c|};
      \node (o1) [output, below left of=d4, xshift=-2cm, yshift=-1cm] {\verb|1|};
      \node (o2) [output, below right of=d4, xshift=2cm, yshift=-1cm] {\verb|0|};
      \node (stop) [startstop, below of=d4, yshift=-2cm] {\scshape Fine};
      \draw [arrow] (start) -- (in);
      \draw [arrow] (in) -- (d1);
      \draw [arrow] (d1) -| node[anchor=east] {sì} (d2);
      \draw [arrow] (d1) -| node[anchor=west] {no} (d3);
      \draw [arrow] (d2) -- node[anchor=east] {sì} (p1);
      \draw [arrow] (d2) -| node[anchor=south] {no} (d4);
      \draw [arrow] (d3) -- node[anchor=west] {sì} (p2);
      \draw [arrow] (d3) -| (d4);
      \draw [arrow] (p1.east) -| (d4.north);
      \draw [arrow] (p2.west) -| (d4.north);
      \draw [arrow] (d4.west) -| node[anchor=east] {sì} (o1);
      \draw [arrow] (d4.east) -| node[anchor=west] {no} (o2);
      \draw [arrow] (o1) |- (stop);
      \draw [arrow] (o2) |- (stop);
    \end{tikzpicture}
  \end{center}
\end{example}

\begin{example}
  Algoritmo che, dati 10 valori interi, calcola e visualizza il massimo. 
  \begin{center}
    \begin{tikzpicture}[node distance=1.5cm]
      \node (start) [startstop] {\scshape Inizio};
      \node (p1) [process, below of=start] {\verb|i = 0|};
      \node (p2) [process, below of=p1] {\verb|m = -INF|};
      \node (d1) [decision, below of=p2] {\verb|i < 10|};
      \node (in) [input, below left of=d1, xshift=-2cm] {\verb|x|};
      \node (d2) [decision, below of=in] {\verb|x > m|};
      \node (p3) [process, below of=d2] {\verb|m = x|};
      \node (p4) [process, below of=p3] {\verb|i = i + 1|};
      \node (f) [dummy, left of=p4, xshift=-1cm] {};
      \node (g) [dummy, right of=d2, xshift=1cm] {};
      \node (out) [output, below right of=d1, xshift=2cm] {\verb|m|};
      \node (stop) [startstop, below of=out] {\scshape Fine};
      \draw [arrow] (start) -- (p1);
      \draw [arrow] (p1) -- (p2);
      \draw [arrow] (p2) -- (d1);
      \draw [arrow] (d1) -| node[anchor=south] {sì} (in);
      \draw [arrow] (in) -- (d2);
      \draw [arrow] (d2) -- node[anchor=east] {sì} (p3);
      \draw [arrow] (d2) -- node[anchor=south] {no} (g) |- (p4);
      \draw [arrow] (p3) -- (p4);
      \draw [arrow] (d1) -| node[anchor=south] {no} (out);
      \draw [arrow] (out) -- (stop);
      \draw [arrow] (p4) -- (f) |- (d1.north);
    \end{tikzpicture}
  \end{center}
\end{example}

\begin{example}
  Algoritmo che, dati 10 valori interi, calcola e visualizza il massimo e il minimo. 
  \begin{center}
    \begin{tikzpicture}[node distance=1.5cm]
      \node (start) [startstop] {\scshape Inizio};
      \node (p1) [process, below of=start] {\verb|i = 0|};
      \node (p2) [process, below of=p1] {\verb|M = -INF|};
      \node (p3) [process, below of=p2] {\verb|m = INF|};
      \node (d1) [decision, below of=p3] {\verb|i < 10|};
      \node (in) [input, below left of=d1, xshift=-2cm] {\verb|x|};
      \node (d2) [decision, below of=in] {\verb|x < m|};
      \node (p4) [process, below of=d2] {\verb|m = x|};
      \node (d3) [decision, below of=p4, yshift=-1cm] {\verb|x > M|};
      \node (p5) [process, below of=d3] {\verb|M = x|};
      \node (p6) [process, below of=p5] {\verb|i = i + 1|};
      \node (f) [dummy, right of=d3, xshift=1cm] {};
      \node (j) [dummy, above of=d3] {};
      \node (g) [dummy, right of=d2, xshift=1.4cm] {};
      \node (h) [dummy, left of=p6, xshift=-1cm] {};
      \node (k) [dummy, right of=p4, xshift=1.2cm] {};
      \node (out) [output, below right of=d1, xshift=2cm] {\verb|M, m|};
      \node (stop) [startstop, below of=out] {\scshape Fine};
      \draw [arrow] (start) -- (p1);
      \draw [arrow] (p1) -- (p2);
      \draw [arrow] (p2) -- (p3);
      \draw [arrow] (p3) -- (d1);
      \draw [arrow] (d1) -| node[anchor=south] {sì} (in);
      \draw [arrow] (in) -- (d2);
      \draw [arrow] (d2) -- node[anchor=east] {sì} (p4);
      \draw [arrow] (d2) -- node[anchor=south] {no} (g) |- (j) -- (d3.north);
      \draw [arrow] (p4) -- (k) |- (p6);
      \draw [arrow] (d3) -- node[anchor=east] {sì} (p5);
      \draw [arrow] (p5) -- (p6);
      \draw [arrow] (p6) -- (h) |- (d1.north);
      \draw [arrow] (d3) -- node[anchor=south] {no} (f) |- (p6);
      \draw [arrow] (d1) -| node[anchor=south] {no} (out);
      \draw [arrow] (out) -- (stop);
    \end{tikzpicture}
  \end{center}
\end{example}

\newpage

\section{Il linguaggio C}

Ogni programma scritto in C, comincia la propria esecuzione dalla funzione \verb|main|:

\begin{verbatim}
  int main(int argc, char *argv[]) {
    // codice
    return 0;
  }
\end{verbatim}

Il linguaggio C è un {\bf linguaggio staticamente tipizzato}, ossia per ogni variabile deve essere esplicitamente indicato il tipo.
La ragione principale di questa scelta è stata la gestione della memoria.

La {\bf dichiarazione} di una variabile si compone di due parti: un tipo e un nome.
I tipi elementari del C sono:
\begin{itemize}
  \item \verb|char|: numero intero a 8 bit, normalmente rappresenta un carattere ASCII;
  \item \verb|short|: numero intero a 16 bit;
  \item \verb|int|: numero intero a 32 bit;
  \item \verb|long|: numero intero a 64 bit;
  \item \verb|float|: numero razionale a 32 bit;
  \item \verb|double|: numero razionale a 64 bit;
  \item \verb|long double|: numero razionale a 128 bit;
\end{itemize}
Ad ogni tipo intero può essere preposto la keyword \verb|unsigned|, che trasforma il tipo in naturale.

I nomi delle variabili seguono alcune regole di base:
\begin{itemize}
  \item non possono cominciare con una cifra;
  \item non possono contenere operatori aritmetici;
  \item non possono contenere spazi o simboli di interpunzione.
\end{itemize}

La dichiarazione di una variabile è un'{\bf istruzione}.
Ogni istruzione termina con il carattere `\verb|;|'. Difatti, questo metodo di separazione delle istruzioni consente di comprimere il codice elidendo gli spazi e gli a capo.
L'istruzione ``\verb|;|'' è un'istruzione valida, ma che non produce effetti.

Il gruppo di istruzioni:
\begin{verbatim}
  int a;
  int b;
\end{verbatim}
è equivalente all'unica istruzione:
\begin{verbatim}
  int a, b;
\end{verbatim}

L'istruzione di {\bf assegnamento} è composta da 3 parti: l'assegnando, l'operatore di assegnazione e l'espressione da assegnare
\begin{verbatim}
  a = 1;      // assegnamento di un valore costante
  a = b;      // assegnamento di una variabile
  a = b * 2;  // assegnamento di un'espressione
\end{verbatim}

In C, le operazioni tra {\bf variabili} e {\bf costanti} seguono alcune regole di valutazione:
\begin{verbatim}
  int   op int   -> int
  float op float -> float
  int   op float -> float
\end{verbatim}
In generale, qualora due operandi non siano omogenei, il risultato dell'operazione ha il tipo dell'operando che contiene più informazione.

L'operazione che cambia il tipo di una variabile si chiama \textbf{\emph{cast}} (promozione). Nel caso in cui il cast venga fatto automaticamente, viene detto \textbf{implicito}. Altrimenti, viene detto \textbf{esplicito}, e avviene mediante l'istruzione:
\begin{verbatim}
  (tipo)variabile
\end{verbatim}

L'istruzione che consente di acquisire da input i dati è
\begin{verbatim}
  scanf(formato, ...);
\end{verbatim}
il formato è una stringa che contiene le specifiche del formato di acquisizione ed è caratterizzato da sequenze specifiche di caratteri che indicano i formati e le posizioni:
\begin{itemize}
  \item \verb|%d|, \verb|%i|: \verb|int|;
  \item \verb|%f|, \verb|%g|: \verb|float|;
  \item \verb|%lf|: \verb|double|;
  \item \verb|%c|: \verb|char|;
  \item \verb|%s|: \verb|char*|;
\end{itemize}
I vari argomenti della funzione saranno le destinazioni dei vari elementi del formato.

\begin{example}
\begin{verbatim}
    int a, b;
    float c;
    scanf("%d", &a);
    scanf("%f", &c);
    scanf("%d %d", &a, &b);\end{verbatim}
\end{example}

L'istruzione che consente di visualizzare l'output è
\begin{verbatim}
  printf(formato, ...);
\end{verbatim}
Il formato e gli argomenti seguono le stesse regole della \verb|scanf|.

Le stringhe in C possono avere al loro interno caratteri ASCII non stampabili, come a capo o la tabulazione. Questi caratteri sono rappresentabili attraverso le formule di escape, e sono:
\begin{itemize}
  \item \verb|\n|: a capo;
  \item \verb|\t|: tabulazione;
  \item \verb|\a|: campanella;
  \item \verb|\r|: ritorno;
  \item \verb|\0|: terminatore;
\end{itemize}

\begin{example}
  \begin{verbatim}
    int a
    float b;
    printf("risultato: %d (%f%%)\n", a, b);\end{verbatim}
\end{example}


Nel linguaggio C esistono le {\bf direttive}, ossia delle istruzioni che non vengono eseguite, ma di cui il compilatore tiene conto. Tra queste le più importanti sono:
\begin{itemize}
  \item \verb|#include|: serve a inserire nel codice altro codice contenuto nel file indicato;
  \item \verb|#define|: consente di indicare degli alias per una sequenza di caratteri.
\end{itemize}

\begin{example}
  \begin{verbatim}
    #include "stdio.h"
    #include <stdlib.h>
    
    #define PI 3.14
    #define F(X) ((X) - 1)\end{verbatim}
\end{example}

\begin{example}
  Scrivere un programma che, dato in input un numero intero strettamente positivo, calcola e visualizza la circonferenza e l'area.
  \begin{verbatim}
    #include "stdio.h"
    #include "math.h"
    int main(int argc, char **argv) {
      unsigned int r;
      scanf("%u", &r);
      double p, a;
      p = 2 * M_PI * r;
      a = M_PI * r * r;
      printf("%lf %lf", p, a);
      return 0;
    }\end{verbatim}
\end{example}

\begin{example}
  Scrivere un programma che, dato in input un numero razionale, ne calcola e visualizza l'arrotondamento per difetto.
  \begin{verbatim}
    #include "stdio.h"
    int main(int argc, char **argv) {
      float x;
      scanf("%f", &x);
      int y;
      y = (int)x;
      printf("%d", y);
      return 0;
    }\end{verbatim}
\end{example}

Nei programmi C è solito utilizzare l'istruzione \verb|n = n + 1|. Questa operazione è equivalente all'istruzione \verb|n++|. Analogamente, l'istruzione \verb|n = n - 1| è equivalente a \verb|n--|.
Inoltre, le istruzioni del tipo \verb|n = n op x| sono equivalenti alle istruzioni \verb|n op= x|.

\begin{example}
  Scrivere un programma che, dato in input un numero intero, ne calcola e visualizza il valore assoluto.
  \begin{verbatim}
    #include <stdio.h>
    int main(int argc, char **argv) {
      int x;
      scanf("%d", &x);
      if (x < 0) 
        x *= -1;
      printf("%d", x);
      return 0;
    }\end{verbatim}
\end{example}

Il costrutto di selezione in C è nella forma:
\begin{verbatim}
  if (expr)
    code_true;
  else
    code_false;
\end{verbatim}

La parola chiave \verb|if| valuta l'espressione \verb|expr| e, se vera, esegue l'istruzione \verb|code_true|, altrimenti esegue l'istruzione \verb|code_false|.
Si può, altresì, eseguire più di un'istruzione, usando i blocchi, ossia sequenze ordinate di istruzioni, che si delimitano dalle parentesi graffe \verb|{ }|: agli occhi della \verb|if|, il blocco appare come un'unica istruzione.
Si precisa che la parte \verb|else ...| è opzionale, per cui si può omettere quando non serve.

Le espressioni che esprimono un valore di verità sono dette booleane. Esse si servono dei costrutti logici \verb|&&| (congiunzione), \verb!||! (disgiunzione) e \verb|!| (negazione).

\begin{example}
  Scrivere un programma che, dato in input un numero intero, visualizza `\verb|+|' se il valore è positivo, `\verb|-|' se è negativo e `\verb| |' se è nullo.
  \begin{verbatim}
    #include <stdio.h>

    #define POSITIVE '+'
    #define NEGATIVE '-'
    #define ZERO ' '

    int main(int argc, char **argv) {
      int x;
      char sign;
      scanf("%d", &x);
      if (x > 0) sign = POSITIVE;
      else if (x < 0) sign = NEGATIVE;
      else sign = ZERO;
      printf("%c", sign);
      return 0;
    }\end{verbatim}
\end{example}

In C, un'espressione è comunque booleana, ed è vera se il suo valore numerico è diverso da 0.
Alcuni esempi:
\begin{itemize}
  \item \verb|x|: \verb|x!=0|;
  \item \verb|!a|: \verb|a==0|;
  \item \verb|5 = a|: errore nella compilazione;
  \item \verb|a = 5|: sempre vero, dato che ad \verb|a| viene assegnato il valore 5 e il risultato di un'operazione di assegnamento è il valore assegnato.
\end{itemize}

Il costrutto di iterazione si presenta sia nella forma a condizione iniziale, sia nella forma a condizione finale:
\begin{verbatim}
  while (expr)
    code;
\end{verbatim}
La parola chiave \verb|while| valuta l'espressione \verb|expr| e, finché vera, esegue l'istruzione \verb|code|, reiterandone l'esecuzione.
\begin{verbatim}
  do
    code;
  while (expr);
\end{verbatim}
La parola chiave \verb|do| esegue l'istruzione \verb|code| e successivamente valuta l'espressione \verb|expr| attraverso \verb|while|: finché vera, l'esecuzione di \verb|code| è reiterata.

In entrambi i casi, analogamente al costrutto di selezione, \verb|code| può essere un blocco di istruzioni, consentendo l'esecuzione ripetuta di più istruzioni consecutive.

\begin{example}
  Scrivere un programma che, dati in input 20 numeri interi, ne calcola e visualizza il valore massimo.
  \begin{verbatim}
    #define N 20
    #include <stdio.h>
    int main(int argc, char** argv) {
      unsigned i = N;
      int n, m = 0x80000000;
      while (i--) {
        scanf("%d", &n);
        if (n > m) m = n;
      }
      printf("%d", m);
      return 0;
    }\end{verbatim}
\end{example}

\begin{example}
  Scrivere un programma che chiede un numero intero (finché esso non è positivo) e ne calcola e visualizza il numero di cifre.
  \begin{verbatim}
    #include <stdio.h>
    int main(int argc, char** argv) {
      int n, m, tmp;
      do scanf("%d", &n);
      while (n <= 0);
      m   = 0;
      tmp = n;
      while (tmp) {
        tmp /= 10;
        m++;
      }
      printf("%d", m);
      return 0;
    }\end{verbatim}
\end{example}

Il linguaggio C consente l'uso di una struttura dati particolare: l'array. L'array è una struttura dati che contiene dati omogenei, la cui cardinalità è costante e nota a priori. Lo si può immaginare come una cassettiera, dove ogni cassetto contiene un elemento.
La dichiarazione di un array è nella seguente forma:
\begin{verbatim}
  tipo nome[dimensione];
\end{verbatim}
Per accedere ad un elemento specifico dell'array, si scrive:
\begin{verbatim}
  nome[i]
\end{verbatim}
dove \verb|i| è l'indice, ossia un valore numerico che va da 0 a \verb|dimensione - 1|.
Ai fini della scrittura di codice, l'espressione \verb|nome[i]| è assimilabile ad una variabile e ne eredita tutte le proprietà.

\begin{example}
  Scrivere un programma che acquisisce 50 valori interi e visualizza i valori superiori al valor medio dei valori acquisiti.
  \begin{verbatim}
    #define N 50
    #include <stdio.h>
    int main(int argc, char** argv) {
      int v[N], s, i;
      float avg;
      i = 0;
      s = 0;
      while (i < N) {
        scanf("%d", &v[i]);
        s += v[i++];
      }
      avg = (float)s / N;
      i   = 0;
      while (i < N) {
        if (v[i] > avg) printf("%d ", v[i]);
        i++;
      }
      return 0;
    }\end{verbatim}
\end{example}

Strettamente legato agli array è il costrutto \verb|for|:
\begin{verbatim}
  for (pre; condition; post)
    code;
\end{verbatim}
Esso non è altro che l'abbreviazione di:
\begin{verbatim}
  pre;
  while (condition) {
    code;
    post;
  }
\end{verbatim}

Il linguaggio C consente di creare array di cardinalità $n$--dimensionale: monodimensionale è un vettore, bidimensionale è una matrice, e così via.
La dichiarazione di un array bidimensionale è nella forma:
\begin{verbatim}
  tipo nome[dim1][dim2];
\end{verbatim}
tenendo bene a mente che \verb|dim1| è il numero delle righe e \verb|dim2| è il numero delle colonne.
Valgono per essi le stesse proprietà degli array monodimensionale, tra cui l'accesso ad un singolo elemento che è:
\begin{verbatim}
  nome[i][j]
\end{verbatim}

\begin{example}
  Scrivere un programma che acquisisce i valori interi di una matrice $5\times 5$ e calcola e visualizza 1 se si tratta di una matrice identità, 0 altrimenti.
  \begin{verbatim}
    #define NR           5
    #define NC           5
    #define IDENTITY     1
    #define NON_IDENTITY 0
    int main(int argc, char** argv) {
      int m[NR][NC], s, i, j;
      for (i = 0; i < NR; i++)
        for (j = 0; j < NC; j++) scanf("%d", &m[i][j]);
      s = (NR == NC ? IDENTITY : NON_IDENTITY);
      for (i = 0; s == IDENTITY && i < NR; i++)
        for (j = 0; s == IDENTITY && j < NC; j++)
          if (i == j && m[i][j] != 1)
            s = NON_IDENTITY;
          else if (i != j && m[i][j] != 0)
            s = NON_IDENTITY;
      printf("%d", s);
      return 0;
    }\end{verbatim}
\end{example}

\paragraph*{Linearizzazione della memoria}

Essendo la memoria contigua, un array bidimensionale non può essere contenuto nella sua forma naturale in memoria. Tuttavia, il segmento di memoria occupato dall'array deve essere contiguo e perciò si ricorre alla linearizzazione dell'array, ossia dell'appiattimento virtuale che consente di considerare una matrice come un vettore. Così, una matrice di dimensione $n\times m$ diventa un vettore $n\cdot m$ e, all'elemento \verb|x[i][j]| corrisponderà l'elemento \verb|y[i * C + j]|, dove \verb|C| è il numero di colonne dichiarato.

Il linguaggio C prevede la dichiarazione di un nuovo tipo, contenente altri tipi di natura non necessariamente omogenea. Questa nuova tipologia di informazione è resa possibile dalla keyword \verb|struct|:
\begin{verbatim}
  struct nome_s {
    tipo v1;
    tipo v2;
  };
\end{verbatim}
Ogni qualvolta si vuole utilizzare il nuovo tipo, si procede nel seguente modo:
\begin{verbatim}
  struct nome_s nome;
\end{verbatim}
Per utilizzare i campi della nuova struttura si usa la notazione punto, ossia:
\begin{verbatim}
  nome.campo
\end{verbatim}
\begin{example}
  \begin{verbatim}
    struct data_s {
      int giorno;
      int mese;
      int anno;
    };
    ...
    struct data_s oggi;
    oggi.giorno = 21;
  \end{verbatim}
\end{example}

Per evitare l'uso continuo di \verb|struct|, si fa uso della keyword \verb|typedef|, che si usa nel seguente modo:
\begin{verbatim}
  typedef nome_s nome_t;
\end{verbatim}

\begin{example}
  \begin{verbatim}
    typedef struct data_s {
      int giorno;
      int mese;
      int anno;
    } data_t;
    ...
    data_t oggi;
    oggi.giorno = 21;
  \end{verbatim}
\end{example}

La libreria standard del C prevede una specializzazione degli array di caratteri: le stringhe. Esse sono array di caratteri, il cui contenuto è terminato dal carattere terminatore \verb|\0| (valore ASCII 0). Nelle funzioni di input/output, le stringhe hanno il segnaposto \verb|%s|:
\begin{verbatim}
  char s[32];
  scanf("%s", s); /* equivalente: scanf("%s", &s[0]) */
  printf("%s", s); /* equivalente: printf("%s", &s[0]) */
\end{verbatim}

La \verb|scanf|, però, ha un limite: non considera gli spazi come caratteri di una stringa. Pertanto, al fine di ottenere stringhe ``complete'', si usa la funzione \verb|gets| contenuta nell'header \verb|stdio.h|:
\begin{verbatim}
  #include <stdio.h>
  ...
  char s[32];
  gets(s);
\end{verbatim}

Ovviamente, il carattere terminatore deve essere in memoria e quindi deve occupare uno spazio all'interno dell'array. A tal fine, considerando \verb|n| come dimensione massima della stringa, in fase di dichiarazione la lunghezza da assegnare sarà \verb|n + 1|:
\begin{verbatim}
  #define N 50
  ...
  char s[N + 1];
\end{verbatim}
\end{document}