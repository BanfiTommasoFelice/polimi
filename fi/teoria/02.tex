\part{Elaborazione dell'informazione}

\section{Algoritmi}

Un algoritmo è un procedimento, ovvero una sequenza di passi, che a partire da un insieme di dati (\emph{input}), attraverso un numero finito di passi, genera un insieme di dati (\emph{output}).

Ogni algoritmo ha un punto d'inizio e un punto di fine.
Un algoritmo che non accetta dati di input produce sempre lo stesso output.

Ogni passo dell'algoritmo si serve di operatori, ossia simboli che indicano un'operazione da effettuare:
\begin{itemize}
  \item operatore assegnamento \verb|=|: è un operatore binario e tutto ciò che è alla sua destra, una volta calcolato, viene assegnato alla variabile che si trova alla sua sinistra;
  \item operatori aritmetici \verb|+ - * / %|: sono operatori binari che effettuano le principali operazioni aritmetiche tra due operandi (rispettivamente somma, differenza, prodotto, quoziente, resto), che possono essere variabili o costanti;
  \item operatori logici \verb|&&|, \verb!||!, \verb|!|: sono operatori binari, tranne l'ultimo che è unario, che esprimono rispettivamente le verità date dalle proposizioni: $a\wedge b$, $a\vee b$, $\lnot a$;
  \item operatori relazionali \verb|>|, \verb|<|, \verb|>=|, \verb|<=|, \verb|==|, \verb|!=|: sono operatori binari che esprimono l'uguaglianza e la disuguaglianza degli operandi. 
\end{itemize}

\begin{example}
  Algoritmo che calcola e visualizza, dato un numero strettamente positivo, l'area e il perimetro di un cerchio avente come raggio il numero dato.
  \begin{center}
    \begin{tikzpicture}[node distance=1.5cm]
      \node (start) [startstop] {\scshape Inizio};
      \node (in1) [input, below of=start] {\verb|r|};
      \node (pro1) [process, below of=in1] {\verb|p = 2 * PI * r|};
      \node (pro2) [process, below of=pro1] {\verb|a = PI * r * r|};
      \node (out1) [output, below of=pro2] {\verb|p, a|};
      \node (stop) [startstop, below of=out1] {\scshape Fine};
      \draw [arrow] (start) -- (in1);
      \draw [arrow] (in1) -- (pro1);
      \draw [arrow] (pro1) -- (pro2);
      \draw [arrow] (pro2) -- (out1);
      \draw [arrow] (out1) -- (stop);
    \end{tikzpicture}
  \end{center}
\end{example}

\begin{example}
  Algoritmo che, dato un numero $s$ intero positivo di secondi, calcola e visualizza le ore, i minuti e i secondi che $s$ quantifica.
  \begin{center}
    \begin{tikzpicture}[node distance=1.5cm]
      \node (start) [startstop] {\scshape Inizio};
      \node (pro1) [process, below of=start] {\verb|SECS_PER_MIN = 60|};
      \node (pro2) [process, below of=pro1] {\verb|SECS_PER_HOUR = SECS_PER_MIN * 60|};
      \node (in) [input, below of=pro2] {\verb|s|};
      \node (pro3) [process, below of=in] {\verb|h = s / SECS_PER_HOUR|};
      \node (pro4) [process, below of=pro3] {\verb|s = s % SECS_PER_HOUR|};
      \node (pro5) [process, below of=pro4] {\verb|m = s / SECS_PER_MIN|};
      \node (pro6) [process, below of=pro5] {\verb|s = s % SECS_PER_MIN|};
      \node (out1) [output, below of=pro6] {\verb|h, m, s|};
      \node (stop) [startstop, below of=out1] {\scshape Fine};
      \draw [arrow] (start) -- (pro1);
      \draw [arrow] (pro1) -- (pro2);
      \draw [arrow] (pro2) -- (in);
      \draw [arrow] (in) -- (pro3);
      \draw [arrow] (pro3) -- (pro4);
      \draw [arrow] (pro4) -- (pro5);
      \draw [arrow] (pro5) -- (pro6);
      \draw [arrow] (pro6) -- (out1);
      \draw [arrow] (out1) -- (stop);
    \end{tikzpicture}
  \end{center}
\end{example}

\begin{example}
  Algoritmo che, dato un numero $x$, calcola e visualizza il suo valore assoluto.
  \begin{center}
    \begin{tikzpicture}[node distance=1.8cm]
      \node (start) [startstop] {\scshape Inizio};
      \node (in) [input, below of=start] {\verb|x|};
      \node (d1) [decision, below of=in] {\verb|x < 0|};
      \node (p1) [process, left of=d1, xshift=-2cm] {\verb|y = -x|};
      \node (p2) [process, right of=d1, xshift=2cm] {\verb|y = x|};
      \node (out) [output, below of=d1] {\verb|y|};
      \node (stop) [startstop, below of=out] {\scshape Fine};
      \draw [arrow] (d1) -- node[anchor=north] {sì} (p1);
      \draw [arrow] (d1) -- node[anchor=north] {no} (p2);
      \draw [arrow] (start) -- (in);
      \draw [arrow] (in) -- (d1);
      \draw [arrow] (out) -- (stop);
      \draw [arrow] (p1) |- (out);
      \draw [arrow] (p2) |- (out);
    \end{tikzpicture}
  \end{center}
\end{example}

\begin{example}
  Algoritmo che, dato un numero $r$ come raggio di una circonferenza, ne calcola l'area e il perimetro, solo se $r$ è positivo, altrimenti attende un nuovo valore.
  \begin{center}
    \begin{tikzpicture}[node distance=1.8cm]
      \node (start) [startstop] {\scshape Inizio};
      \node (in) [input, below of=start] {\verb|r|};
      \node (d1) [decision, below of=in] {\verb|r < 0|};
      \node (p1) [process, below of=d1] {\verb|p = 2 * PI * r|};
      \node (p2) [process, below of=p1] {\verb|a = PI * r * r|};
      \node (o1) [output, below of=p2] {\verb|p, a|};
      \node (stop) [startstop, below of=o1] {\scshape Fine};
      \draw [arrow] (start) -- (in);
      \draw [arrow] (in) -- (d1);
      \draw [arrow] (d1) -- node[anchor=east] {no} (p1);
      \draw [arrow] (d1.west) |- node[anchor=east] {sì} (in);
      \draw [arrow] (p1) -- (p2);
      \draw [arrow] (p2) -- (o1);
      \draw [arrow] (o1) -- (stop);
    \end{tikzpicture}
  \end{center}
\end{example}

\begin{example}
  Algoritmo che, dati tre numeri, visualizza 1 se i tre numeri compongono una terna pitagorica, 0 altrimenti.
  \begin{center}
    \begin{tikzpicture}[node distance=2cm]
      \node (start) [startstop] {\scshape Inizio};
      \node (in) [input, below of=start] {\verb|a, b, c|};
      \node (d1) [decision, below of=in] {\verb|a > b|};
      \node (d2) [decision, below left of=d1, xshift=-2cm] {\verb|a > c|};
      \node (d3) [decision, below right of=d1, xshift=2cm] {\verb|b > c|};
      \node (p1) [process, below of=d2] {\verb|swap(a, c)|};
      \node (p2) [process, below of=d3] {\verb|swap(b, c)|};
      \node (d4) [decision, below of=d1, yshift=-3.5cm] {\verb|a * a + b * b == c * c|};
      \node (o1) [output, below left of=d4, xshift=-2cm, yshift=-1cm] {\verb|1|};
      \node (o2) [output, below right of=d4, xshift=2cm, yshift=-1cm] {\verb|0|};
      \node (stop) [startstop, below of=d4, yshift=-2cm] {\scshape Fine};
      \draw [arrow] (start) -- (in);
      \draw [arrow] (in) -- (d1);
      \draw [arrow] (d1) -| node[anchor=east] {sì} (d2);
      \draw [arrow] (d1) -| node[anchor=west] {no} (d3);
      \draw [arrow] (d2) -- node[anchor=east] {sì} (p1);
      \draw [arrow] (d2) -| node[anchor=south] {no} (d4);
      \draw [arrow] (d3) -- node[anchor=west] {sì} (p2);
      \draw [arrow] (d3) -| (d4);
      \draw [arrow] (p1.east) -| (d4.north);
      \draw [arrow] (p2.west) -| (d4.north);
      \draw [arrow] (d4.west) -| node[anchor=east] {sì} (o1);
      \draw [arrow] (d4.east) -| node[anchor=west] {no} (o2);
      \draw [arrow] (o1) |- (stop);
      \draw [arrow] (o2) |- (stop);
    \end{tikzpicture}
  \end{center}
\end{example}

\begin{example}
  Algoritmo che, dati 10 valori interi, calcola e visualizza il massimo. 
  \begin{center}
    \begin{tikzpicture}[node distance=1.5cm]
      \node (start) [startstop] {\scshape Inizio};
      \node (p1) [process, below of=start] {\verb|i = 0|};
      \node (p2) [process, below of=p1] {\verb|m = -INF|};
      \node (d1) [decision, below of=p2] {\verb|i < 10|};
      \node (in) [input, below left of=d1, xshift=-2cm] {\verb|x|};
      \node (d2) [decision, below of=in] {\verb|x > m|};
      \node (p3) [process, below of=d2] {\verb|m = x|};
      \node (p4) [process, below of=p3] {\verb|i = i + 1|};
      \node (f) [dummy, left of=p4, xshift=-1cm] {};
      \node (g) [dummy, right of=d2, xshift=1cm] {};
      \node (out) [output, below right of=d1, xshift=2cm] {\verb|m|};
      \node (stop) [startstop, below of=out] {\scshape Fine};
      \draw [arrow] (start) -- (p1);
      \draw [arrow] (p1) -- (p2);
      \draw [arrow] (p2) -- (d1);
      \draw [arrow] (d1) -| node[anchor=south] {sì} (in);
      \draw [arrow] (in) -- (d2);
      \draw [arrow] (d2) -- node[anchor=east] {sì} (p3);
      \draw [arrow] (d2) -- node[anchor=south] {no} (g) |- (p4);
      \draw [arrow] (p3) -- (p4);
      \draw [arrow] (d1) -| node[anchor=south] {no} (out);
      \draw [arrow] (out) -- (stop);
      \draw [arrow] (p4) -- (f) |- (d1.north);
    \end{tikzpicture}
  \end{center}
\end{example}

\begin{example}
  Algoritmo che, dati 10 valori interi, calcola e visualizza il massimo e il minimo. 
  \begin{center}
    \begin{tikzpicture}[node distance=1.5cm]
      \node (start) [startstop] {\scshape Inizio};
      \node (p1) [process, below of=start] {\verb|i = 0|};
      \node (p2) [process, below of=p1] {\verb|M = -INF|};
      \node (p3) [process, below of=p2] {\verb|m = INF|};
      \node (d1) [decision, below of=p3] {\verb|i < 10|};
      \node (in) [input, below left of=d1, xshift=-2cm] {\verb|x|};
      \node (d2) [decision, below of=in] {\verb|x < m|};
      \node (p4) [process, below of=d2] {\verb|m = x|};
      \node (d3) [decision, below of=p4, yshift=-1cm] {\verb|x > M|};
      \node (p5) [process, below of=d3] {\verb|M = x|};
      \node (p6) [process, below of=p5] {\verb|i = i + 1|};
      \node (f) [dummy, right of=d3, xshift=1cm] {};
      \node (j) [dummy, above of=d3] {};
      \node (g) [dummy, right of=d2, xshift=1.4cm] {};
      \node (h) [dummy, left of=p6, xshift=-1cm] {};
      \node (k) [dummy, right of=p4, xshift=1.2cm] {};
      \node (out) [output, below right of=d1, xshift=2cm] {\verb|M, m|};
      \node (stop) [startstop, below of=out] {\scshape Fine};
      \draw [arrow] (start) -- (p1);
      \draw [arrow] (p1) -- (p2);
      \draw [arrow] (p2) -- (p3);
      \draw [arrow] (p3) -- (d1);
      \draw [arrow] (d1) -| node[anchor=south] {sì} (in);
      \draw [arrow] (in) -- (d2);
      \draw [arrow] (d2) -- node[anchor=east] {sì} (p4);
      \draw [arrow] (d2) -- node[anchor=south] {no} (g) |- (j) -- (d3.north);
      \draw [arrow] (p4) -- (k) |- (p6);
      \draw [arrow] (d3) -- node[anchor=east] {sì} (p5);
      \draw [arrow] (p5) -- (p6);
      \draw [arrow] (p6) -- (h) |- (d1.north);
      \draw [arrow] (d3) -- node[anchor=south] {no} (f) |- (p6);
      \draw [arrow] (d1) -| node[anchor=south] {no} (out);
      \draw [arrow] (out) -- (stop);
    \end{tikzpicture}
  \end{center}
\end{example}

\newpage

\section{Il linguaggio C}

Ogni programma scritto in C, comincia la propria esecuzione dalla funzione \verb|main|:

\begin{verbatim}
  int main(int argc, char *argv[]) {
    // codice
    return 0;
  }
\end{verbatim}

Il linguaggio C è un {\bf linguaggio staticamente tipizzato}, ossia per ogni variabile deve essere esplicitamente indicato il tipo.
La ragione principale di questa scelta è stata la gestione della memoria.

La {\bf dichiarazione} di una variabile si compone di due parti: un tipo e un nome.
I tipi elementari del C sono:
\begin{itemize}
  \item \verb|char|: numero intero a 8 bit, normalmente rappresenta un carattere ASCII;
  \item \verb|short|: numero intero a 16 bit;
  \item \verb|int|: numero intero a 32 bit;
  \item \verb|long|: numero intero a 64 bit;
  \item \verb|float|: numero razionale a 32 bit;
  \item \verb|double|: numero razionale a 64 bit;
  \item \verb|long double|: numero razionale a 128 bit;
\end{itemize}
Ad ogni tipo intero può essere preposto la keyword \verb|unsigned|, che trasforma il tipo in naturale.

I nomi delle variabili seguono alcune regole di base:
\begin{itemize}
  \item non possono cominciare con una cifra;
  \item non possono contenere operatori aritmetici;
  \item non possono contenere spazi o simboli di interpunzione.
\end{itemize}

La dichiarazione di una variabile è un'{\bf istruzione}.
Ogni istruzione termina con il carattere `\verb|;|'. Difatti, questo metodo di separazione delle istruzioni consente di comprimere il codice elidendo gli spazi e gli a capo.
L'istruzione ``\verb|;|'' è un'istruzione valida, ma che non produce effetti.

Il gruppo di istruzioni:
\begin{verbatim}
  int a;
  int b;
\end{verbatim}
è equivalente all'unica istruzione:
\begin{verbatim}
  int a, b;
\end{verbatim}

L'istruzione di {\bf assegnamento} è composta da 3 parti: l'assegnando, l'operatore di assegnazione e l'espressione da assegnare
\begin{verbatim}
  a = 1;      // assegnamento di un valore costante
  a = b;      // assegnamento di una variabile
  a = b * 2;  // assegnamento di un'espressione
\end{verbatim}

In C, le operazioni tra {\bf variabili} e {\bf costanti} seguono alcune regole di valutazione:
\begin{verbatim}
  int   op int   -> int
  float op float -> float
  int   op float -> float
\end{verbatim}
In generale, qualora due operandi non siano omogenei, il risultato dell'operazione ha il tipo dell'operando che contiene più informazione.

L'operazione che cambia il tipo di una variabile si chiama \textbf{\emph{cast}} (promozione). Nel caso in cui il cast venga fatto automaticamente, viene detto \textbf{implicito}. Altrimenti, viene detto \textbf{esplicito}, e avviene mediante l'istruzione:
\begin{verbatim}
  (tipo)variabile
\end{verbatim}

L'istruzione che consente di acquisire da input i dati è
\begin{verbatim}
  scanf(formato, ...);
\end{verbatim}
il formato è una stringa che contiene le specifiche del formato di acquisizione ed è caratterizzato da sequenze specifiche di caratteri che indicano i formati e le posizioni:
\begin{itemize}
  \item \verb|%d|, \verb|%i|: \verb|int|;
  \item \verb|%f|, \verb|%g|: \verb|float|;
  \item \verb|%lf|: \verb|double|;
  \item \verb|%c|: \verb|char|;
  \item \verb|%s|: \verb|char*|;
\end{itemize}
I vari argomenti della funzione saranno le destinazioni dei vari elementi del formato.

\begin{example}
\begin{verbatim}
    int a, b;
    float c;
    scanf("%d", &a);
    scanf("%f", &c);
    scanf("%d %d", &a, &b);\end{verbatim}
\end{example}

L'istruzione che consente di visualizzare l'output è
\begin{verbatim}
  printf(formato, ...);
\end{verbatim}
Il formato e gli argomenti seguono le stesse regole della \verb|scanf|.

Le stringhe in C possono avere al loro interno caratteri ASCII non stampabili, come a capo o la tabulazione. Questi caratteri sono rappresentabili attraverso le formule di escape, e sono:
\begin{itemize}
  \item \verb|\n|: a capo;
  \item \verb|\t|: tabulazione;
  \item \verb|\a|: campanella;
  \item \verb|\r|: ritorno;
  \item \verb|\0|: terminatore;
\end{itemize}

\begin{example}
  \begin{verbatim}
    int a
    float b;
    printf("risultato: %d (%f%%)\n", a, b);\end{verbatim}
\end{example}


Nel linguaggio C esistono le {\bf direttive}, ossia delle istruzioni che non vengono eseguite, ma di cui il compilatore tiene conto. Tra queste le più importanti sono:
\begin{itemize}
  \item \verb|#include|: serve a inserire nel codice altro codice contenuto nel file indicato;
  \item \verb|#define|: consente di indicare degli alias per una sequenza di caratteri.
\end{itemize}

\begin{example}
  \begin{verbatim}
    #include "stdio.h"
    #include <stdlib.h>
    
    #define PI 3.14
    #define F(X) ((X) - 1)\end{verbatim}
\end{example}

\begin{example}
  Scrivere un programma che, dato in input un numero intero strettamente positivo, calcola e visualizza la circonferenza e l'area.
  \begin{verbatim}
    #include "stdio.h"
    #include "math.h"
    int main(int argc, char **argv) {
      unsigned int r;
      scanf("%u", &r);
      double p, a;
      p = 2 * M_PI * r;
      a = M_PI * r * r;
      printf("%lf %lf", p, a);
      return 0;
    }\end{verbatim}
\end{example}

\begin{example}
  Scrivere un programma che, dato in input un numero razionale, ne calcola e visualizza l'arrotondamento per difetto.
  \begin{verbatim}
    #include "stdio.h"
    int main(int argc, char **argv) {
      float x;
      scanf("%f", &x);
      int y;
      y = (int)x;
      printf("%d", y);
      return 0;
    }\end{verbatim}
\end{example}

Nei programmi C è solito utilizzare l'istruzione \verb|n = n + 1|. Questa operazione è equivalente all'istruzione \verb|n++|. Analogamente, l'istruzione \verb|n = n - 1| è equivalente a \verb|n--|.
Inoltre, le istruzioni del tipo \verb|n = n op x| sono equivalenti alle istruzioni \verb|n op= x|.

\begin{example}
  Scrivere un programma che, dato in input un numero intero, ne calcola e visualizza il valore assoluto.
  \begin{verbatim}
    #include <stdio.h>
    int main(int argc, char **argv) {
      int x;
      scanf("%d", &x);
      if (x < 0) 
        x *= -1;
      printf("%d", x);
      return 0;
    }\end{verbatim}
\end{example}

Il costrutto di selezione in C è nella forma:
\begin{verbatim}
  if (expr)
    code_true;
  else
    code_false;
\end{verbatim}

La parola chiave \verb|if| valuta l'espressione \verb|expr| e, se vera, esegue l'istruzione \verb|code_true|, altrimenti esegue l'istruzione \verb|code_false|.
Si può, altresì, eseguire più di un'istruzione, usando i blocchi, ossia sequenze ordinate di istruzioni, che si delimitano dalle parentesi graffe \verb|{ }|: agli occhi della \verb|if|, il blocco appare come un'unica istruzione.
Si precisa che la parte \verb|else ...| è opzionale, per cui si può omettere quando non serve.

Le espressioni che esprimono un valore di verità sono dette booleane. Esse si servono dei costrutti logici \verb|&&| (congiunzione), \verb!||! (disgiunzione) e \verb|!| (negazione).

\begin{example}
  Scrivere un programma che, dato in input un numero intero, visualizza `\verb|+|' se il valore è positivo, `\verb|-|' se è negativo e `\verb| |' se è nullo.
  \begin{verbatim}
    #include <stdio.h>

    #define POSITIVE '+'
    #define NEGATIVE '-'
    #define ZERO ' '

    int main(int argc, char **argv) {
      int x;
      char sign;
      scanf("%d", &x);
      if (x > 0) sign = POSITIVE;
      else if (x < 0) sign = NEGATIVE;
      else sign = ZERO;
      printf("%c", sign);
      return 0;
    }\end{verbatim}
\end{example}

In C, un'espressione è comunque booleana, ed è vera se il suo valore numerico è diverso da 0.
Alcuni esempi:
\begin{itemize}
  \item \verb|x|: \verb|x!=0|;
  \item \verb|!a|: \verb|a==0|;
  \item \verb|5 = a|: errore nella compilazione;
  \item \verb|a = 5|: sempre vero, dato che ad \verb|a| viene assegnato il valore 5 e il risultato di un'operazione di assegnamento è il valore assegnato.
\end{itemize}

Il costrutto di iterazione si presenta sia nella forma a condizione iniziale, sia nella forma a condizione finale:
\begin{verbatim}
  while (expr)
    code;
\end{verbatim}
La parola chiave \verb|while| valuta l'espressione \verb|expr| e, finché vera, esegue l'istruzione \verb|code|, reiterandone l'esecuzione.
\begin{verbatim}
  do
    code;
  while (expr);
\end{verbatim}
La parola chiave \verb|do| esegue l'istruzione \verb|code| e successivamente valuta l'espressione \verb|expr| attraverso \verb|while|: finché vera, l'esecuzione di \verb|code| è reiterata.

In entrambi i casi, analogamente al costrutto di selezione, \verb|code| può essere un blocco di istruzioni, consentendo l'esecuzione ripetuta di più istruzioni consecutive.

\begin{example}
  Scrivere un programma che, dati in input 20 numeri interi, ne calcola e visualizza il valore massimo.
  \begin{verbatim}
    #define N 20
    #include <stdio.h>
    int main(int argc, char** argv) {
      unsigned i = N;
      int n, m = 0x80000000;
      while (i--) {
        scanf("%d", &n);
        if (n > m) m = n;
      }
      printf("%d", m);
      return 0;
    }\end{verbatim}
\end{example}

\begin{example}
  Scrivere un programma che chiede un numero intero (finché esso non è positivo) e ne calcola e visualizza il numero di cifre.
  \begin{verbatim}
    #include <stdio.h>
    int main(int argc, char** argv) {
      int n, m, tmp;
      do scanf("%d", &n);
      while (n <= 0);
      m   = 0;
      tmp = n;
      while (tmp) {
        tmp /= 10;
        m++;
      }
      printf("%d", m);
      return 0;
    }\end{verbatim}
\end{example}

Il linguaggio C consente l'uso di una struttura dati particolare: l'array. L'array è una struttura dati che contiene dati omogenei, la cui cardinalità è costante e nota a priori. Lo si può immaginare come una cassettiera, dove ogni cassetto contiene un elemento.
La dichiarazione di un array è nella seguente forma:
\begin{verbatim}
  tipo nome[dimensione];
\end{verbatim}
Per accedere ad un elemento specifico dell'array, si scrive:
\begin{verbatim}
  nome[i]
\end{verbatim}
dove \verb|i| è l'indice, ossia un valore numerico che va da 0 a \verb|dimensione - 1|.
Ai fini della scrittura di codice, l'espressione \verb|nome[i]| è assimilabile ad una variabile e ne eredita tutte le proprietà.

\begin{example}
  Scrivere un programma che acquisisce 50 valori interi e visualizza i valori superiori al valor medio dei valori acquisiti.
  \begin{verbatim}
    #define N 50
    #include <stdio.h>
    int main(int argc, char** argv) {
      int v[N], s, i;
      float avg;
      i = 0;
      s = 0;
      while (i < N) {
        scanf("%d", &v[i]);
        s += v[i++];
      }
      avg = (float)s / N;
      i   = 0;
      while (i < N) {
        if (v[i] > avg) printf("%d ", v[i]);
        i++;
      }
      return 0;
    }\end{verbatim}
\end{example}

Strettamente legato agli array è il costrutto \verb|for|:
\begin{verbatim}
  for (pre; condition; post)
    code;
\end{verbatim}
Esso non è altro che l'abbreviazione di:
\begin{verbatim}
  pre;
  while (condition) {
    code;
    post;
  }
\end{verbatim}

Il linguaggio C consente di creare array di cardinalità $n$--dimensionale: monodimensionale è un vettore, bidimensionale è una matrice, e così via.
La dichiarazione di un array bidimensionale è nella forma:
\begin{verbatim}
  tipo nome[dim1][dim2];
\end{verbatim}
tenendo bene a mente che \verb|dim1| è il numero delle righe e \verb|dim2| è il numero delle colonne.
Valgono per essi le stesse proprietà degli array monodimensionale, tra cui l'accesso ad un singolo elemento che è:
\begin{verbatim}
  nome[i][j]
\end{verbatim}

\begin{example}
  Scrivere un programma che acquisisce i valori interi di una matrice $5\times 5$ e calcola e visualizza 1 se si tratta di una matrice identità, 0 altrimenti.
  \begin{verbatim}
    #define NR           5
    #define NC           5
    #define IDENTITY     1
    #define NON_IDENTITY 0
    int main(int argc, char** argv) {
      int m[NR][NC], s, i, j;
      for (i = 0; i < NR; i++)
        for (j = 0; j < NC; j++) scanf("%d", &m[i][j]);
      s = (NR == NC ? IDENTITY : NON_IDENTITY);
      for (i = 0; s == IDENTITY && i < NR; i++)
        for (j = 0; s == IDENTITY && j < NC; j++)
          if (i == j && m[i][j] != 1)
            s = NON_IDENTITY;
          else if (i != j && m[i][j] != 0)
            s = NON_IDENTITY;
      printf("%d", s);
      return 0;
    }\end{verbatim}
\end{example}

\paragraph*{Linearizzazione della memoria}

Essendo la memoria contigua, un array bidimensionale non può essere contenuto nella sua forma naturale in memoria. Tuttavia, il segmento di memoria occupato dall'array deve essere contiguo e perciò si ricorre alla linearizzazione dell'array, ossia dell'appiattimento virtuale che consente di considerare una matrice come un vettore. Così, una matrice di dimensione $n\times m$ diventa un vettore $n\cdot m$ e, all'elemento \verb|x[i][j]| corrisponderà l'elemento \verb|y[i * C + j]|, dove \verb|C| è il numero di colonne dichiarato.

Il linguaggio C prevede la dichiarazione di un nuovo tipo, contenente altri tipi di natura non necessariamente omogenea. Questa nuova tipologia di informazione è resa possibile dalla keyword \verb|struct|:
\begin{verbatim}
  struct nome_s {
    tipo v1;
    tipo v2;
  };
\end{verbatim}
Ogni qualvolta si vuole utilizzare il nuovo tipo, si procede nel seguente modo:
\begin{verbatim}
  struct nome_s nome;
\end{verbatim}
Per utilizzare i campi della nuova struttura si usa la notazione punto, ossia:
\begin{verbatim}
  nome.campo
\end{verbatim}
\begin{example}
  \begin{verbatim}
    struct data_s {
      int giorno;
      int mese;
      int anno;
    };
    ...
    struct data_s oggi;
    oggi.giorno = 21;
  \end{verbatim}
\end{example}

Per evitare l'uso continuo di \verb|struct|, si fa uso della keyword \verb|typedef|, che si usa nel seguente modo:
\begin{verbatim}
  typedef nome_s nome_t;
\end{verbatim}

\begin{example}
  \begin{verbatim}
    typedef struct data_s {
      int giorno;
      int mese;
      int anno;
    } data_t;
    ...
    data_t oggi;
    oggi.giorno = 21;
  \end{verbatim}
\end{example}

La libreria standard del C prevede una specializzazione degli array di caratteri: le stringhe. Esse sono array di caratteri, il cui contenuto è terminato dal carattere terminatore \verb|\0| (valore ASCII 0). Nelle funzioni di input/output, le stringhe hanno il segnaposto \verb|%s|:
\begin{verbatim}
  char s[32];
  scanf("%s", s); /* equivalente: scanf("%s", &s[0]) */
  printf("%s", s); /* equivalente: printf("%s", &s[0]) */
\end{verbatim}

La \verb|scanf|, però, ha un limite: non considera gli spazi come caratteri di una stringa. Pertanto, al fine di ottenere stringhe ``complete'', si usa la funzione \verb|gets| contenuta nell'header \verb|stdio.h|:
\begin{verbatim}
  #include <stdio.h>
  ...
  char s[32];
  gets(s);
\end{verbatim}

Ovviamente, il carattere terminatore deve essere in memoria e quindi deve occupare uno spazio all'interno dell'array. A tal fine, considerando \verb|n| come dimensione massima della stringa, in fase di dichiarazione la lunghezza da assegnare sarà \verb|n + 1|:
\begin{verbatim}
  #define N 50
  ...
  char s[N + 1];
\end{verbatim}

Un sottoprogramma può essere definito come un programma indipendente, che però opera in una prospettiva più ridotta rispetto a quella globale del \verb|main|.
Un sottoprogramma produce effetti, altrimenti la sua chiamata sarebbe inutile.
Un sottoprogramma può avere a disposizione una quantità di informazioni date in input dal chiamante e, a sua volta, restituisce come output una certa quantità di informazioni.

La dichiarazione di un sottoprogramma avviene mediante la seguente scrittura:
\begin{verbatim}
  tipo nome(parametri...) {
    codice;
    return ...;
  }
\end{verbatim}

Nel caso più semplice, ossia senza input e senza output, essa diventa:
\begin{verbatim}
  void f(void) {
    ...
  }
\end{verbatim}

Il tipo \verb|void| è un tipo particolare: indica, infatti, l'assenza di un tipo preciso.
Come tipo di ritorno, il \verb|void| indica che il sottoprogramma non restituisce alcuna informazione. Come argomento, esso indica che non vi è necessario alcun parametro, e, peraltro, rende illegale il passaggio di parametri al sottoprogramma.

Il caso senza input, ma con un singolo output è:
\begin{verbatim}
  int f() {
    ...
    return x;
  }
\end{verbatim}

Nel caso in cui sia presente l'input e sia presente l'output:
\begin{verbatim}
  int f(int n) {
    ...
    return x;
  }
\end{verbatim}

La chiamata di un sottoprogramma avviene mediante il costrutto:
\begin{verbatim}
  void f() { ... }
  int g() { ... }
  ...
  f();
  x = g();
\end{verbatim}

\begin{example}
  Sottoprogramma che calcola e restituisce 1 se un numero è primo, 0 altrimenti.
  \begin{verbatim}
    int prime(int n) {
      int i, p;
      p = 1;
      if (n <= 0) p = -1;
      else if (n < 4) p = n > 1;
      else if (n % 2 == 0) p = 0;
      else if (n % 3 == 0) p = 0;
      for (i = 5; p && i * i <= n; i += 6)
        if (n % i == 0 || n % (i + 2) == 0)
          p = 0;
      return p;
    }
  \end{verbatim}
\end{example}

La disposizione dei vari sottoprogrammi all'interno del sorgente, avviene secondo lo schema:
\begin{verbatim}
  #define ...
  #include <...>
  typedef ...

  /* prototipi */
  void f(void);
  int g(int);

  int main(int argc, char** argv) {
    ...
    return 0;
  }

  /* implementazioni */
  
  void f() {
    ...
  }

  int g(int n) {
    ...
    return x;
  }
\end{verbatim}

Qualora fosse necessario passare come input un array, si procede nel seguente modo:
\begin{verbatim}
  int f(int v[]);
\end{verbatim}
Analogamente, per passare un array bidimensionale, si scrive:
\begin{verbatim}
  #define C 10
  int f(int m[][C]);
\end{verbatim}
\paragraph*{Osservazione}
Nella dichiarazione di un parametro bidimensionale, si rende necessaria la dichiarazione del numero di colonne della matrice. Questo è dovuto al fatto che la memoria è lineare, ed, per accedere ad un elemento della matrice, si deve necessariamente conoscere il numero delle colonne.

Inoltre, si rende necessario l'utilizzo di un parametro per indicare la lunghezza del vettore o le dimensioni della matrice. Per cui, le dichiarazioni diventano:
\begin{verbatim}
  #define C 10
  int f(int n, int v[]);
  int g(int n, int m, int m[][C]);
\end{verbatim}

La libreria standard del C fornisce molte funzioni utili:
\begin{itemize}
  \item \verb|size_t strlen(const char *)|: calcola la lunghezza di una stringa;
  \item \verb|int strcmp(const char *, const char *)|: calcola la lunghezza di una stringa;
  \item 
\end{itemize}

Le variabili fino ad ora dichiarate sono solo copie di quelle originali (passaggio parametri per copia valore). Pertanto, qualora si modifichi< tale variabile, la variabile originale non subirebbe alcuna modifica. Per ovviare a questa problematica, si ricorre ai puntatori (passaggio parametri per copia riferimento). Un puntatore è una variabile che contiene un indirizzo di memoria. L'operatore che consente, a partire da una variabile, di ottenerne l'indirizzo è \verb|&|. Nella dichiarazione di una funzione si scrive:
\begin{verbatim}
  void f(int* p);
\end{verbatim}
L'utilizzo della variabile originale, invece, avviene mediante l'operatore di derefenziazione \verb|*|:
\begin{verbatim}
  *p
\end{verbatim}
La chiamata alla funzione, invece, è:
\begin{verbatim}
  int n;
  f(&n);
\end{verbatim}
\paragraph*{Osservazione}
Quando si passa un array, poiché le forme \verb|v| e \verb|&v[0]| sono equivalenti, si capisce bene che, in realtà, si passa l'indirizzo del primo elemento. Questo permette un risparmio di memoria (non bisogna fare una copia dell'array) e un'indicizzazione coerente dell'array (\verb|v[i]|, ossia \verb|*(&v[0 + i])|).

Il passaggio per copia valore introduce la tematica dello scope, o visibilità. In C, ogni variabile è visibile nel blocco di appartenenza. Le variabili globali, essendo al di fuori di qualsiasi blocco, sono sempre visibili, a patto che la loro dichiarazione sia precedente al loro utilizzo.

\paragraph*{Lo stack}

Sullo stack viene allocato tutto lo spazio necessario al funzionamento del programma.
Viene creato il cosiddetto record di attivazione, ossia una zona di memoria riservata ai parametri passati alla funzione e al valore di ritorno.
Inoltre, è necessario avere l'indirizzo da dove il chiamante chiama, per poter ritornare dopo l'esecuzione della funzione.

Ci sono poi alcune informazioni di servizio e lo spazio dedicato alle variabili locali.

Quindi, la struttura è la seguente:
\begin{itemize}
  \item variabili locali
  \item informazioni di servizio
  \item indirizzo di ritorno
  \item parametri attuali
  \item valore di ritorno
\end{itemize}
% TODO: fare una grafica per la struttura dello stack

Da un punto di vista macroscopico, ogni volta che si chiama una funzione, lo stack costruisce un blocco logico costituito dagli elementi cui sopra, che vengono impilati nello stack:

% TODO: fare due blocchi, equivalenti a due funzioni

\subsection{Ricorsione}

La ricorsione è una metodologia risolutiva che consiste nello scrivere un sottoprogramma che chiama se stesso.
Per ovvi motivi, è necessaria una condizione che consente alla funzione di chiudere il ciclo della ricorsione.

\begin{example}
  \begin{verbatim}
    int factorial(int n) {
      int f;
      if (n < 2) f = 1;
      else f = n * factorial(n - 1);
      return f;
    }
  \end{verbatim}
\end{example}

\subsection{Argomenti da riga di comando}

Il valore numerico \verb|argc| contiene il numero di elementi dell'array \verb|argv|. L'array \verb|argv| contiene le stringhe degli argomenti passati al programma da riga di comando:
\begin{verbatim}
  gcc test.c -o test -ansi
\end{verbatim}
Al programma \verb|gcc| vengono passati come argomenti \verb|argc| pari a 5 e \verb|argv| che contiene le stringhe \verb|{ "gcc", "test.c", "-o", "test", "-ansi" }|.

Risultano utili, al fine di utilizzare gli argomenti da riga di comando, le funzioni della libreria standard:
\begin{itemize}
  \item \verb|int atoi(const char*)|
  \item \verb|long atol(const char*)|
  \item \verb|double atof(const char*)|
\end{itemize}