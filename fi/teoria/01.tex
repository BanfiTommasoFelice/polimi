% lezione del 12/09/2022

Il problema che ci si pone è trovare un modo opportuno per rappresentare all'interno di un sistema di calcolo le informazioni in modo efficiente, rispetto alla realtà fisica del sistema e alla loro manipolazione.

\begin{definition}[Alfabeto]
  Si definisce \textbf{alfabeto} un insieme di simboli utilizzabili e, pertanto, distinguibili tra loro.
\end{definition}

\begin{definition}[Codice]
  Si definisce \textbf{codice} l'insieme delle sequenze di simboli o delle regole per definire le combinazioni ammissibili.
\end{definition}

Dati l'insieme degli elementi da rappresentare e l'insieme delle configurazioni ammissibili, il codice ne definisce la relazione biunivoca.
Le configurazioni ammissibili hanno tutte egual dimensione. Tale dimensione dipende sia dall'alfabeto, sia dalla quantità di elementi da rappresentare: siano $S$ l'alfabeto di riferimento e $\abs{S}$ la sua cardinalità (ossia il numero di simboli che lo compone), se si vogliono rappresentare $n$ elementi, ogni elemento avrà dimensione:
$$k=\ceil{\log_{\abs{S}} n}$$
Al contrario, se gli elementi di un codice hanno lunghezza $k$, le combinazioni ammissibili sono:
$$n=\abs{S}^k$$

\section{Rappresentazione binaria}

I componenti elettronici che costituiscono il sistema di calcolo sono caratterizzati da una realtà costituita da due stati (condensatore carico/scarico, tensione alta/bassa, etc...). Si effettua, quindi, una mappatura diretta con un sistema costituito da \textbf{due simboli} che, pertanto, si chiama \textbf{binario}. L'alfabeto di riferimento diventa $\left\{ 0,1 \right\}$.

La cifra della codifica (0 o 1) prende il nome di \textbf{bit}, dall'inglese \emph{\textbf{bi}nary digi\textbf{t}}. L'insieme ordinato di 8 bit prende il nome di \textbf{byte}. Come per le cifre decimali, si ha una nomenclatura per le potenze della base:
\begin{center}
  \begin{tblr}{colspec={c|c}, cells={c,m}, columns={20mm}}
    \textbf{Nome} & \textbf{Quantità} \\
    \hline
    KB            & $2^{10}$          \\
    MB            & $2^{20}$          \\
    GB            & $2^{30}$          \\
    TB            & $2^{40}$          \\
  \end{tblr}
\end{center}

\begin{example}
  Se si vogliono rappresentare i giorni della settimana usando l'alfabeto binario, si calcola la dimensione del singolo elemento:
  $$k=\ceil{\log_27}=3$$
  e si assegna ad ogni combinazione di 3 bit un giorno della settimana distinto:
  \begin{center}
    \begin{tblr}{colspec={c|c|c|c|c|c|c}, cells={c,m}, columns={18mm}}
      Lunedì & Martedì & Mercoledì & Giovedì & Venerdì & Sabato & Domenica \\
      \hline
      000    & 001     & 010       & 011     & 100     & 101    & 110      
    \end{tblr}
  \end{center}
  Da notare che non viene utilizzata la combinazione 111, in quanto le combinazioni ammissibili sono $2^3=8$ ma per le necessità del caso ne servono solo 7.
\end{example}

Nella scelta della codifica da adottare, bisogna tenere a mente alcuni aspetti:
\begin{itemize}
  \item l'insieme degli elementi da rappresentare;
  \item il grado di semplificazione delle operazioni più eseguite;
  \item il grado di conservazione delle proprietà dell'insieme originale.
\end{itemize}

L'informazione può essere, per comodità, suddivisa in aree:

\begin{center}
  \begin{tblr}{colspec={c|c|c|c|c|c}, cells={c,m}, columns={20mm}}
    \SetCell[c=6]{c} Informazione                                                                       \\ \hline
    \SetCell[c=3]{c} Numerica &          &           & \SetCell[c=3]{c} Non numerica                    \\ \hline
    Naturali                  & Relativi & Razionali & Testi                         & Audio & Immagini 
  \end{tblr}
\end{center}

% lezione del 13/09/2022

La notazione che si usa in base 10 è una \textbf{notazione posizionale pesata}, vale a dire che ogni cifra vale in base alla posizione che essa occupa all'interno del numero. Il valore del numero, infatti, è dato da
$$\text{valore}=\sum_{0}^nc_ib^i$$
dove $c_i$ è la cifra in posizione $i$ e $b$ è la base di riferimento.

\begin{example}
  $$315_{10}=3\cdot10^2+1\cdot10^1+5\cdot10^0$$
\end{example}

Nella notazione in base 2, si definisce un codice che associa al valore numerico una configurazione, in cui la cifra più a destra è la cifra meno significativa (\textbf{LSB}, Least Significant Bit), mentre quella più a sinistra è quella più significativa (\textbf{MSB}, Most Significant Bit).

\section{Conversione di base}
Idealmente, ogni qualvolta si vuole rappresentare una quantità in una certa base, si vogliono implicitamente rappresentare in un'altra base tutte le quantità minori uguali a quella di partenza, non avrebbe senso altrimenti\footnote{se me ne serve solo uno, uso il primo valore disponibile}.

Se $n$ è il valore da rappresentare, significa che in base 2 si avrà bisogno di $k$ bits:
$$k=\ceil{\log_2\(n+1\)}$$
\paragraph*{Osservazione} L'argomento del logaritmo è $n+1$ in quanto bisogna anche considerare lo 0.

Per passare da una base $a$ ad una base $b$, con $a<b$, si procede nel seguente modo: si moltiplica ogni coefficiente per la base elevata alla sua posizione e poi si sommano tutti i prodotti ottenuti.

\begin{example}
  $$1010_2=1\cdot2^3+0\cdot2^2+1\cdot2^1+0\cdot2^0=8+0+2+0=10_{10}$$
\end{example}

Per passare da una base $a$ ad una base $b$, con $a>b$, si procede nel seguente modo:

\begin{enumerate}
  \item si divide il numero per la base;
  \item si prende il resto;
  \item si ripete il processo con il quoziente ottenuto.
\end{enumerate}

Il processo si conclude quando il quoziente diventa 0. Il risultato non è altro che la sequenza ordinata dei resti ottenuti, letta al contrario. 

\begin{example}
  \begin{center}
    \begin{tblr}{c|c}
      17 & 1 \\
      8  & 0 \\
      4  & 0 \\
      2  & 0 \\
      1  & 1 
    \end{tblr}
    $$\impl 17_{10}=10001_2$$
  \end{center}
\end{example}

Le modalità con cui si trasformano le quantità fra le diverse basi sono interscambiabili, pertanto si dicono metodi.

\section{Modulo e segno}

Nella notazione decimale si utilizza la forma ``modulo e segno'' per esprimere i numeri relativi:
$$-11\;\;+4$$

Tale ragionamento non si può estendere alla rappresentazione binaria, perché richiederebbe l'introduzione un nuovo simbolo per `$+$' o `$-$'. Pertanto, per convenzione, il primo bit indica il segno, che può essere 0 se è positivo, 1 se è negativo:
$$+17_{10\text{MS}}=010001_{2\text{MS}}$$
$$-17_{10\text{MS}}=110001_{2\text{MS}}$$
$$-23_{10\text{MS}}=110111_{2\text{MS}}$$

Si devono eseguire le operazioni aritmetiche usando la notazione modulo e segno.
\begin{center}
  \begin{tblr}{c|c}
    $
      \begin{aligned}
        x & =+11_{10\text{MS}} \\
        y & =+8_{10\text{MS}}  \\
        z & =-7_{10\text{MS}}  \\
      \end{aligned}
    $
     & 
    $
      \begin{aligned}
        x+y & =+19_{10\text{MS}} \\
        x+z & =+4_{10\text{MS}}  \\
      \end{aligned}
    $
  \end{tblr}
\end{center}

Nella rappresentazione decimale si eseguono i passaggi descritti nel seguente algoritmo:

\begin{algorithm}[H]
  \caption{Somma di due valori in notazione modulo e segno}\label{algo:sommams}
  \begin{algorithmic}[1]
    \If{$\sgn x= \sgn y$}
    \State $\abs{n}=\abs{x}+\abs{y}$
    \State $\sgn n=\sgn x$
    \Else
    \State $a = \max \(\abs{x},\abs{y}\)$
    \State $b = \min \(\abs{x},\abs{y}\)$
    \State $\abs{n}=a-b$
    \If{$\abs{x}=a$}
    \State $\sgn n=\sgn a$
    \Else
    \State $\sgn n=\sgn b$
    \EndIf
    \EndIf
  \end{algorithmic}
\end{algorithm}

Tuttavia, per un calcolatore un tal numero di operazioni renderebbe il calcolo proibitivamente lento. Per questo, si introduce una nuova notazione, più efficiente di quella di modulo e segno, incentrata sull'agilità dell'elaborazione delle informazioni.

Ovviamente, la finestra dei valori che si vuole rappresentare si estende anche al negativo, per cui se si vuole rappresentare il $17$, si vuole rappresentare tutti i valori da $-17$ a $17$.

Si descrivono le possibili combinazioni avendo a disposizione 4 bit:
\begin{center}
  \begin{tblr}{c|c||c|c}
    $0000_2$ & $0_{10}$ & $1000_2$ & $-0_{10}$ \\
    $0001_2$ & $1_{10}$ & $1001_2$ & $-1_{10}$ \\
    $0010_2$ & $2_{10}$ & $1010_2$ & $-2_{10}$ \\
    $0011_2$ & $3_{10}$ & $1011_2$ & $-3_{10}$ \\ \hline
    $0100_2$ & $4_{10}$ & $1100_2$ & $-4_{10}$ \\
    $0101_2$ & $5_{10}$ & $1101_2$ & $-5_{10}$ \\
    $0110_2$ & $6_{10}$ & $1110_2$ & $-6_{10}$ \\
    $0111_2$ & $7_{10}$ & $1111_2$ & $-7_{10}$ \\
  \end{tblr}
\end{center}

Si nota subito che lo 0 ha una codifica ridondante:
$$0000_{2\text{MS}}=1000_{2\text{MS}}=0_{10}$$

Si pone l'attenzione sull'identità:
$$x+\(-x\)=0$$

Volendo rispettare l'identità con un valore $x=3_{10}$ si ha:

\begin{center}
  \begin{tblr}{ccccc}
    $^10$ & $^10$ & $^11$ & $1$ & $+$ \\
    $1$   & $1$   & $0$   & $1$ & $=$ \\
    \hline
    $0$   & $0$   & $0$   & $0$       
  \end{tblr}
\end{center}

quindi risulta che $1101_2$ è la codifica di $0011_2$ al negativo. Da questo ragionamento, esteso agli altri numeri risulta che:
\begin{center}
  \begin{tblr}{c|c||c|c}
    $0000_2$ & $0_{10}$ & $1000_2$ & $\pm8_{10}$ \\
    $0001_2$ & $1_{10}$ & $1001_2$ & $-7_{10}$   \\
    $0010_2$ & $2_{10}$ & $1010_2$ & $-6_{10}$   \\
    $0011_2$ & $3_{10}$ & $1011_2$ & $-5_{10}$   \\ \hline
    $0100_2$ & $4_{10}$ & $1100_2$ & $-4_{10}$   \\
    $0101_2$ & $5_{10}$ & $1101_2$ & $-3_{10}$   \\
    $0110_2$ & $6_{10}$ & $1110_2$ & $-2_{10}$   \\
    $0111_2$ & $7_{10}$ & $1111_2$ & $-1_{10}$   \\
  \end{tblr}
\end{center}

Esiste anche un metodo algebrico per trovare l'opposto:
$$-x=\sim x+1$$
oppure un metodo grafico:
\emph{tutti i bit che dall'LSB al primo bit 1 rimangono invariati, mentre tutti gli altri si cambiano}.

L'efficacia di questa notazione sta nella facilità con cui si eseguono le operazioni:
$$2_{10}+4_{10}=0010_2+0100_2=0110_2=6_{10}$$
$$-3_{10}+\(-1_{10}\)=1101_2+1111_2=\cancel{1}1100_2=-4_{10}$$
$$5_{10}+\(-3_{10}\)=0101_2+1101_2=\cancel{1}0010_2=2_{10}$$

Questa notazione prende il nome di complemento in base 2.
Si chiama complemento in quanto i valori negativi si completano a quelli positivi.

Per passare dalla base 10 alla base 2 in complemento a 2, si effettuano i seguenti passaggi:
\begin{algorithm}[H]
  \caption{Conversione da 10MS a 2C2}\label{algo:10to2c2}
  \begin{algorithmic}[1]
    \State $n=\abs{x}$
    \If{$x<0$}
    \State $n=-n$
    \EndIf
  \end{algorithmic}
\end{algorithm}
\begin{example}
  $$x=-13_{10}$$
  $$\abs{x}=13_{10}=1101_2=01101_{2\text{MS}}\equiv 01101_{2\text{C}2}$$
  $$x=-\abs{x}=10011_{2\text{C}2}$$
\end{example}

Si presentano casi ambigui in cui apparentemente sembra che l'aritmetica non funzioni:
$$5_{10}+3_{10}=0101_2+0011_2=1000_2=-8_{10}$$
$$-2_{10}+\(-6_{10}\)=1110_2+1010_2=\cancel{1}1000_2=-8_{10}$$
Si nota che tutti i numeri positivi cominciano con 0 mentre quelli negativi cominciano con 1.
Convenzionalmente, in virtù di quanto appena detto, la combinazione $1000_2$ assume il valore $-8_{10}$.

Usando la notazione modulo e segno il range dei valori ammissibili è $\interval{-7}{7}$, mentre in complemento a 2 il range è $\interval{-8}{7}$.

Considerando che la dimensione dell'informazione è sempre fissa e determinata, ci si ritrova in casi particolari detti di \textbf{overflow} (\emph{traboccamento}):
$$6_{10}+4_{10}=0110_2+0100_2=1010_2=-6_{10}$$
L'overflow è facilmente risolvibile con un aumento dello spazio a disposizione:
$$6_{10}+4_{10}=00110_2+00100_2=01010_2=10_{10}$$
Si ha overflow quando due valori concordi generano un valore discorde dai primi due.
Quando gli operandi sono discordi è impossibile generare overflow.

\paragraph*{Osservazione} L'estensione di un valore in complemento a 2 si esegue ripetendo il MSB quante volte se ne ha bisogno:

\begin{example}
  $$5_{10}=0101_{2\text{C}2}=000000101_{2\text{C}2}$$
  $$-5_{10}=1011_{2\text{C}2}=111111011_{2\text{C}2}$$
\end{example}

\begin{example}
  $$x=+12_{10\text{MS}}=01100_{2\text{C}2}$$
  $$y=-3_{10\text{MS}}=101_{2\text{C}2}=11101_{2\text{C}2}$$
  $$x+y=01100_2+11101_2=\cancel{1}01001_2$$
  $$x-y=x+\(-y\)=01100_2+00011_2=01111_2$$
\end{example}

% lezione del 15/09/2022

Per passare dalla base 2 in complemento a 2 alla base 10, si possono verificare due scenari:
\begin{enumerate}
  \item il numero è positivo, pertanto lo si converte usando la formula dei pesi;
  \item il numero è negativo: in questo caso si calcola l'opposto, lo si converte e gli si cambia il segno.
\end{enumerate}
\begin{example}
  $$010110_{2\text{C}2}=22_{10\text{MS}}$$
  $$1011101_{2\text{C}2}=-35_{10\text{MS}}$$
\end{example}

\section{Rappresentazione esadecimale}
Il problema principale con la rappresentazione binaria è l'enorme spazio richiesto. Inoltre, in virtù dell'uso sui calcolatori, la base da cercare deve essere una potenza di 2.
Per anni è stata utilizzata la base 8, ma ben presto è stata resa obsoleta, in favore della base 16.

Il codice esadecimale si crea associando le cifre esadecimali e combinazioni di 4 bit del codice binario.
\begin{center}
  \begin{tblr}{c|c|c}
    \textbf{esadecimale} & \textbf{binario} & \textbf{decimale} \\\hline
    0                    & 0000             & 0                 \\
    1                    & 0001             & 1                 \\
    2                    & 0010             & 2                 \\
    3                    & 0011             & 3                 \\
    4                    & 0100             & 4                 \\
    5                    & 0101             & 5                 \\
    6                    & 0110             & 6                 \\
    7                    & 0111             & 7                 \\
    8                    & 1000             & 8                 \\
    9                    & 1001             & 9                 \\
    A                    & 1010             & 10                \\
    B                    & 1011             & 11                \\
    C                    & 1100             & 12                \\
    D                    & 1101             & 13                \\
    E                    & 1110             & 14                \\
    F                    & 1111             & 15                \\
  \end{tblr}
\end{center}

Per effettuare la conversione dalla base 2 alla base 16 basta considerare ogni quadrupla e convertirla in loco:
\begin{center}
  \begin{tblr}{cccc}
    0101 & 0101 & 1101 & 1010 \\ \hline
    5    & 5    & D    & A    
  \end{tblr}
\end{center}
Banalmente, la stessa cosa avviene per il processo inverso.

Al fine di rappresentare attraverso la notazione modulo e segno un numero esadecimale, gli si prepone il segno, facendo la conversione dal binario.

\section{Numeri razionali}

Si vuole rappresentare i valori espressi da:
$$\frac{m}{n},\; m\in\mathbb{Z},n\in\mathbb{N}/\left\{ 0 \right\}$$

Nella notazione decimale si usa scrivere prima la parte intera, un separatore decimale e poi la parte frazionaria.
In base 2 si può fare la stessa cosa:
$$101.01_2=2^2+2^0+2^{-2}=5.25_{10}$$
Avendo a disposizione una certa quantità bit, si sceglie la parte di essi che contiene la parte frazionaria e quella che contiene la parte intera. 

\begin{example}
  Si considerano 3 bit per la parte intera e 2 per quella frazionaria. Si ha che il range massimo esprimibile è $\interval{0}{7.75}$, con salti di $0.25$.
  
  Tuttavia, se si vuole rappresentare il valore $6.3$, non si può. Al limite, si può esprimere una sua approssimazione, che in questo caso è $6.25$.
\end{example}

Il fatto che non si possa rappresentare un valore con \textbf{precisione} non è dovuto ad una cattiva gestione dei bit, ma bensì al loro stato di finitezza.

La notazione utilizza si chiama a virgola fissa: viene stabilita la dimensione a priori sia per la parte intera sia per la parte frazionaria.
In una notazione del genere l'\textbf{errore assoluto} $\epsilon_A$ è costante: si sbaglia sempre della stessa quantità.

Esiste, tuttavia, una notazione alternativa che consente di variare l'errore assoluto e mantenere costante l'errore relativo: la notazione in \textbf{virgola mobile} (\emph{floating point} in inglese).
L'errore relativo $$\epsilon_R=\frac{\epsilon_A}{\text{valore}}$$

Per rappresentare un numero razionale in base 2, si procede come al solito per la parte intera, ovvero si divide per 2 segnando il resto, ma al contrario per la parte frazionaria, ovvero si moltiplica per 2 segnando l'unità. Un'altra differenza fondamentale è il verso di lettura delle cifre: mentre per la parte intera si procede dal basso verso l'alto, per la parte frazionaria si procede dall'alto verso il basso.

\begin{example}
  Si vuole rappresentare il numero $13.75$.
  \begin{center}
    \begin{minipage}{0.2\linewidth}
      \begin{center}
        \begin{tblr}{c|c}
          13 & 1 \\
          6  & 0 \\
          3  & 1 \\
          1  & 1 \\
          0  &   
        \end{tblr}
      \end{center}
    \end{minipage}
    \begin{minipage}{0.2\linewidth}
      \begin{center}
        \begin{tblr}{c|c}
          0.75 &   \\
          1.5  & 1 \\
          1.0  & 1 \\
          0    &   \\
        \end{tblr}
      \end{center}
    \end{minipage}
  \end{center}
  $$13.75_{10}=1101.11_2$$
\end{example}

\begin{example}
  Si vuole rappresentare il numero $7.32$.
  \begin{center}
    \begin{minipage}{0.2\linewidth}
      \begin{center}
        \begin{tblr}{c|c}
          7 & 1 \\
          3 & 1 \\
          1 & 1 \\
          0 &   
        \end{tblr}
      \end{center}
    \end{minipage}
    \begin{minipage}{0.2\linewidth}
      \begin{center}
        \begin{tblr}{c|c}
          0.32     &          \\
          0.64     & 0        \\
          1.28     & 1        \\
          0.56     & 0        \\
          1.12     & 1        \\
          0.24     & 0        \\
          0.48     & 0        \\
          0.96     & 0        \\
          1.92     & 1        \\
          1.84     & 1        \\
          $\cdots$ & $\cdots$ 
        \end{tblr}
      \end{center}
    \end{minipage}
  \end{center}
  $$7.32_{10}=111.010100011...$$
\end{example}

Se si considerano i razionali da una diversa prospettiva, si nota che:
$$13.75_{10}=1.375\cdot 10^1$$
$$7.32_{10}=7.32\cdot 10^0$$
Analogamente:
$$1101.11_2=1.10111\cdot2^3$$
$$111.010100011_2=1.11010100011\cdot2^2$$

La notazione in virgola mobile, quindi, ha:
\begin{itemize}
  \item un bit per il \textbf{segno};
  \item una parte per l'\textbf{esponente}, a cui viene aggiunto un numero tale che il più piccolo esponente possibile sia 0;
  \item una parte per la \textbf{mantissa}, ossia la parte frazionaria della notazione scientifica\footnote{si prende la parte frazionaria in quanto la parte intera sarà sempre 1, quindi è inutile sprecare un bit}.
\end{itemize}
Lo standard che regola la notazione in virgola mobile è lo \textbf{IEEE 754}. Lo standard prevede 3 versioni:
\begin{itemize}
  \item 32 bit: 1 per il segno, 8 per l'esponente, 23 per la mantissa;
  \item 64 bit: 1 per il segno, 11 per l'esponente, 52 per la mantissa;
  \item 128 bit: 1 per il segno, 15 per l'esponente, 112 per la mantissa.
\end{itemize}

Il valore di un numero espresso attraverso lo standard IEEE 754 è dato da:
$$\text{valore}=\(-1\)^S\(1+M\)\cdot2^E$$
dove $S$ è il segno, $M$ è la mantissa e $E$ è l'esponente

\paragraph*{Nota bene}
La costante che bisogna aggiungere all'esponente si chiama ``eccesso'' ed è:
\begin{itemize}
  \item 127 per la singola precisione;
  \item 1023 per la doppia precisione;
  \item 16383 per la quadrupla precisione.
\end{itemize}

\begin{example}
  $$+5.65_{10\text{MS}}$$
  $$+101.10{1001}_{2\text{MS}}=1.01101001\cdot2^2=01000000101101001100110011001100$$
  $$-0.028_{10\text{MS}}$$
  $$-0.00000111001010_2=-1.1100101\cdot2^{-6}=101001111110101...$$
\end{example}

Lo standard, tuttavia, presenta delle falle che vengono compensate dall'introduzione di combinazioni di bit speciali che, proprio per questo, vengono dette \textbf{denormarlizzate}.

\paragraph*{Forma denormarlizzata generale}
Quando tutti i bit dell'esponente sono 0, la mantissa non è sommata ad 1 ma bensì a 0.

\paragraph*{Zero}
Quando tutti i bit (indifferentemente dal primo) sono 0, il valore è 0.

\paragraph*{Infinito}
Quando tutti i bit dell'esponente sono 1 e quelli della mantissa sono tutti 0, il valore è infinito, che può essere sia negativo che positivo.

\paragraph*{NaN}
Quando tutti i bit dell'esponente sono 1 e almeno uno di quelli della mantissa è 1, il valore non è un numero (\textbf{NaN}, \emph{Not a Number}).
